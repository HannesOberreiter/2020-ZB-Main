\section{Methodik}

Antikörper sind die Schlüsselreagenzien für die Etablierung der ELISA. Nach Virusanzucht in Bienenpuppen und Präparation der Virionen mittels Ultrazentrifugation können die Viruspartikel von ABPV, DWV und SBV gereinigt werden. Zusätzlich werden außerdem die Strukturproteine der Viren rekombinant in \textit{E. Coli} exprimiert und gereinigt. Nach Immunisierung von Mäusen mit den rekombinanten Antigenen bzw. den hochreinen Viruspräparationen sollen neue Hybridomzellklone generiert, monoklonale Antikörper produziert und charakterisiert werden. Hochspezifische Antikörper werden in Virusneutralisations-Experimenten, Westernblot und Immunfluoreszenztest validiert und in der Kreuzkompetition getestet. Durch rekombinante Expression werden außerdem geeignete Boost-Antigene und Positivkontrollen generiert. Geeignete Antikörper werden in großen Mengen produziert, gereinigt und untersucht, ob sie zur Anwendung in einem Sandwich-ELISA geeignet sind. Gleichzeitig werden die Antikörper sequenziert, um als chimäre Moleküle für breite ELISA-Anwendungen zur Verfügung zu stehen. Zur Generierung der Antikörper sind im Modul V Tierversuche erforderlich. In diesen Versuchen werden Mäuse mit gereinigten Antigenen immunisiert („Impfung“) und der Erfolg der Immunisierung anhand der Serokonversion geprüft (Blutentnahme). Die Versuchstiere werden nach erfolgreicher Immunisierung euthanasiert und B-Lymphozyten aus ihrer Milz gewonnen. Die Tierversuche werden im Rahmen eines genehmigten Antrages zur Generierung von monoklonalen Antikörpern gegen Bienenviren durchgeführt (BMWF-68.205/0107-II/3b/2013). Die neu generierten Reagenzien und bereits charakterisierte Antikörper gegen DWV werden zur Etablierung diagnostischer ELISA genutzt. Dabei werden Probenlysis (Puffer, Zerkleinerung, Probenmaterial) und verschiedene ELISA-Systeme untersucht. Mit Hilfe von Feldproben, die vom Kooperationspartner AGES (Modul A) bereitgestellt werden sollen, werden die Verfahren optimiert und validiert.

\subsection{Ausgangslage der Studie}

In den letzten Jahren wurden substantielle Vorarbeiten zur serologischen Diagnostik von Bienenviren an der Vetmeduni geleistet. Das Institut für Virologie der Vetmeduni hat langjährige Erfahrung in der Diagnostik von Virusinfektionen. Es wurden bereits rekombinante Antigene und serologische Reagenzien gegen Bienenviren (DWV) hergestellt und serologische Nachweisverfahren für den Einsatz in der Forschung etabliert. Neben dem ersten molekularen Klon des Flügeldeformationsvirus wurden auch die ersten monoklonalen Antikörper gegen Strukturproteine von DWV präsentiert \citep{Lamp2016}. Ein Forschungsschwerpunkt bildet die molekulare Pathologie von Virusinfektionen bei Honigbienen. 
\\
\\
Die vorhandenen monoklonalen Antikörper gegen DWV VP1 und 3c-Protease wurden bereits charakterisiert, die Bindung an die viralen Proteine untersucht und eine orientierende Epitopbestimmung durchgeführt. Im hier vorgestellten Projekt wird die praktische Anwendung dieser Antikörper in der Diagnostik untersucht, um belastbare Assays zum Nachweis von DWV zu entwickeln. Zur rekombinanten Expression von ABPV Strukturproteinen und der Produktion von ABPV wurden bereits Vorarbeiten geleistet.
