% In diesen Datei werden die Haupt Sektionen der Projekte Geschrieben. Jede Sektion hat eine eigene Datei erhalten welche im Chapter.tex zusammengeführt wird. Mit dem Prozentzeichen % können Notizen geschrieben werden die nicht im PDF aufscheinen. Der Befehl \blindtext kann gelöscht werden, er erstellt nur einen Muster Text

% Hauptüberschrift
\section{Kurzzusammenfassung des bisherigen Verlaufs der Studie}

% Unterüberschrift
\subsection{Muster Unterüberschrift}

Muster Zeilenumbruch \\ Umbruch.

Muster Referenz \citep{morawetzHealthStatusHoney2019}.

%% Es können so auch PDF Grafiken eingebunden werden
Muster Grafik
\myfig{figures/logos/zukunft_biene_logo_rechteck} % Pfad
  {width=0.1\textwidth} % Größe Relativ zu Text Breite
  {Text unterhalb der Grafik.} % Text unterhalb der Grafik
  {Optionaler Kurz Titel} % Optional Kurz Überschrift
  {fig:example1} % Label zum Verweisen im Text
  
Zum Verweis auf die Grafik (\cref{fig:example})
  
Muster Tabelle:

\begin{table}[htp]
    \centering
    \begin{tabular}{|l|r|} 
      \hline
      Spalte1 & Spalte2 \\
      \hline
      Erste Zelle & Zweite Zelle \\
      \hline
      Dritte Zelle & Vierte Zelle \\
      \hline
    \end{tabular}
    \caption{Table to test captions and labels}
    \label{tab:example}
\end{table}

Verschiedene Typographien: \textbf{ich bin ein fetter Text}, \textit{ich bin ein kursiver Text}

Prozent schreiben: 90\%

Sample: \sample{10}

Confidenz Intervall: \confi{11.3}{98}{9.2}{12.0} % MW, Intervall, Lower, Upper 

Confidenz Intervall ohne Befehl: 11\% (95\%~KI:~9.2-12.0\%)
