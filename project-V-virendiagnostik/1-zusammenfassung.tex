Im Modul V sollen Antikörper zum Nachweis viraler Genprodukte generiert und „Enzyme linked immuno sorbent assays (ELISA)“ zum schnellen und kostengünstigen diagnostischen Nachweis von Virusinfektionen bei Honigbienen im Labor und im Feld entwickelt werden. Als Projektdauer wurden drei bis vier Jahre veranschlagt, wobei ein planmäßiger Projektbeginn ab Mai 2018 nach Eingang der Finanzierung eingehalten werden konnte. Im ersten Schritt wurden geeignete Antigen-Präparationen für die Immunisierungen und Tests produziert werden. Daher wurden Antigene des Deformed wing virus (DWV-A), Varroa destructor Virus (VDV/DWV-B), Sackbrutvirus (SBV) sowie des Virus der akuten Bienenparalyse (ABPV) durch gentechnische Methoden in Bakterien produziert. Durch Ultrazentrifugation wurden zudem Viruspartikel angereichert, die in Screening- und Charakterisierungsversuchen eingesetzt werden können. Die Strukturproteine VP1, VP2 und VP3 von ABPV und das Strukturprotein VP1 des SBV wurden bakteriell exprimiert und chromatographisch gereinigt. Letzteres diente bereits zur Immunisierungen von Mäusen und mehrere Hybridomzellklone, die monoklonale Antikörper produzieren, wurden isoliert. Die Antigenpräparationen von ABPV stehen für die Immunisierung von Mäusen zur Verfügung. Als Ergänzung zu den gentechnisch erzeugten Antigenen wurden große Mengen von SBV, ABPV und DWV durch Infektion von Bienenlarven erzeugt und sollen als hochgereinigte Viruspräparationen für die Testentwicklung herangezogen werden.