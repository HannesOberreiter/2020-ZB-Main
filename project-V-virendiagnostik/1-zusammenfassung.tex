Im Modul V sollen Antikörper zum Nachweis viraler Antigene erzeugt und „Enzyme linked immuno sorbent assays (ELISA)“ zum schnellen und kostengünstigen diagnostischen Nachweis von Virusinfektionen bei Honigbienen im Labor und im Feld entwickelt werden. Der geplante Test soll ähnlich wie der Schnelltest der aktuellen COVID-19 Diagnostik  ablaufen und dem Imker binnen Minuten ohne apparativen Aufwand ein Ergebnis liefern. Als Projektdauer wurden drei bis vier Jahre veranschlagt, wobei ein planmäßiger Projektbeginn nach Eingang der Finanzierung eingehalten werden konnte. Im ersten Schritt wurden geeignete Antigen-Präparationen für die Immunisierungen und Tests produziert. Dafür wurden Antigene des Deformed wing virus (DWV-A), Varroa destructor Virus (VDV/DWV-B), Sackbrutvirus (SBV) sowie des Virus der akuten Bienenparalyse (ABPV) durch gentechnische Methoden in Bakterien produziert. Die Strukturproteine VP1, VP2 und VP3 von ABPV und das Strukturprotein VP1 des SBV wurden bakteriell exprimiert und chromatographisch gereinigt. Beide Antigene wurden zur Immunisierung von Mäusen verwendet. Für das VP1 des SBV konnten mehrere Hybridomzellklone, die monoklonale Antikörper produzieren, isoliert werden. Die Hybridome der rekombinant hergestellten Antigene des ABPV befinden sich derzeit noch in Selektion. Als Ergänzung zu den gentechnisch erzeugten Antigenen wurden große Mengen von SBV, ABPV und DWV durch Infektion von Bienenlarven erzeugt und mittels Dichtegradientenzentrifugation gereinigt. Die hochreinen Viruspräparationen von ABPV und SBV wurden wiederum zur Immunisierung von Versuchsmäusen herangezogen. Bei allen Versuchstieren konnte die Serokonversion bestätigt werden und die B-Lymphozyten der mit ABPV immunisierten Mäuse wurden bereits fusioniert.