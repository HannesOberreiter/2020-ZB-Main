\section{Ergebnisse}

\subsection{Antigene}

Bislang wurden Virusstocks von DWV, SBV und ABPV produziert, um die saisonale Bruttätigkeit der Honigbienen möglichst gut auszunutzen. Außerdem wurden rekombinante Proteine der verwendeten SBV und ABPV Stämme in bakteriellen Expressionssystemen (\textit{E. Coli} Stamm: BL21 Rosetta) generiert. Um die Viren aus Feldproben möglichst effizient detektieren zu können, wurden die viralen Strukturproteine für die Proteinexpression ausgewählt. 
Das Strukturprotein VP1 des SBV sowie die Strukturproteine VP1, VP2 und VP3 des ABPV konnten erfolgreich in Bakterien exprimiert und mittels Nickel-Affinitätschromatographie aufgereinigt werden. Das gereinigte Antigen VP1 des SBV wurde im Anschluss dazu verwendet, Mäuse zu immunisieren. Da es in diesem Versuch nicht gelang geeignete Antikörper Kandidaten zu identifizieren, wurde das Expressionskonstrukt modifiziert. Um ein stabileres und löslicheres Protein zu erhalten, wurde ein Fluoreszenzprotein (\textit{mCherry}) an das virale Strukturprotein angeknüpft (\cref{fig:v:ProteinexpressionSBV}).




\myfig{project-V-virendiagnostik/figures/sbv_neu}
%% filename in given folder
{width=0.6\textwidth,height=0.6\textheight}
%% maximum width/height, aspect ratio will be kept
{Proteinexpression des rekombinanten Fusionsproteins mCherry-VP1 des Sackbrutvirus (SBV).}%% caption
{}%% optional (short) caption for table of figures
{fig:v:ProteinexpressionSBV}%% label

Das modifizierte Antigen konnte erfolgreich exprimiert, gereinigt und zur Immunisierung von zwei Mäusen herangezogen werden (s.u.). 

Für ABPV wurden insgesamt drei Strukturproteine (VP1, VP2 und VP3) exprimiert (\cref{fig:v:ProteinexpressionABPV}). Diese Proteine wurden unter denaturierenden Bedingungen (8M Urea) chromatographisch aufgereinigt. Mittels Dialyse wurden sie in einen, für die Immunisierung geeigneten, Puffer (PBS) verbracht und dabei langsam partiell renaturiert (\cref{fig:v:ProteinprepABPV}). Im Anschluss wurden die gereinigten Antigene zur Immunisierung zweier Versuchsmäuse herangezogen.

\myfig{project-V-virendiagnostik/figures/abpv_neu}
%% filename in given folder
{width=0.7\textwidth,height=1\textheight}
%% maximum width/height, aspect ratio will be kept
{Proteinexpression der rekombinanten Strukturproteine VP1, VP2 und VP3 des akuten Bienenparalysevirus (ABPV); 1 – nicht induzierte Bakterienkultur, 2 – induzierte Bakterienkultur nach 2.5h, 3 – induzierte Bakterienkultur nach 3.5h
}%% caption
{}%% optional (short) caption for table of figures
{fig:v:ProteinexpressionABPV}%% label

\myfig{project-V-virendiagnostik/figures/abpv_sps}
%% filename in given folder
{width=0.5\textwidth,height=1\textheight}
%% maximum width/height, aspect ratio will be kept
{SDS-PAGE-Proteingel der rekombinanten Strukturproteine VP1, VP2, VP3 des ABPV, gefärbt mit Coomassie-Brilliant-Blau; 1 - Strukturproteine in 8M Urea, 2 - Strukturproteine in 2M Urea, 3 - Strukturproteine in PBS
}%% caption
{}%% optional (short) caption for table of figures
{fig:v:ProteinprepABPV}%% label


Als Ergänzung zu den rekombinanten Proteinen wurde ein weiterer Ansatz für die Präparation der Virionen und das nachfolgende Screening der Antikörper gewählt. Dadurch können hochspezifische Antikörper gegen Bienenviren produziert werden, die ohne Denaturierung des Antigens in der Bienenprobe funktionieren sollen. Dieser Ansatz hat den Vorteil, dass die Probenaufarbeitung für den Imker vereinfacht wird. Das Problem bei früheren Viruspräparationen mit kontaminierenden Proteine aus den Bienenlysaten wird dabei durch eine andere Art der Virusaufreinigung gelöst.

Dafür wurden neue Virusstocks von DWV, SBV und ABPV durch Infektion von Bienenbrut erzeugt (jeweils ca. 400 Bienenpuppen). 96h nach der Infektion wurden die Larven und Puppen mechanisch lysiert, zentrifugiert und das Lysat sterilfiltriert. Im Anschluss wurden die Viren durch fraktionierte Pelletierung und Dichtegradientenzentrifugation angereichert (\cref{fig:v:csclabpv}). Dabei trennen sich die jeweiligen Moleküle nach ihrer Schwebdichte auf und können im Anschluss analysiert werden. Dazu wurden die sichtbaren Fraktionen einzeln mit einer Kanüle aus dem Zentrifugenröhrchen abgenommen und auf ein SDS-PAGE-Proteingel aufgetragen und mit Coomassie-Brilliant-Blau angefärbt (\cref{fig:v:SelfformingGradientABPV}).
Die gereinigten Virusantigene wurden im Anschluss zur Immunisierung von Mäusen herangezogen (s.u.) und sollen außerdem als Testantigene Verwendung finden. 

\myfig{project-V-virendiagnostik/figures/cscl_self_abpv2}
%% filename in given folder
{width=0.5\textwidth,height=1.0\textheight}
%% maximum width/height, aspect ratio will be kept
{Präparation von ABPV mit Hilfe eines selbstformierenden Cäsiumchlorid Gradienten (1,4g/ml, 14h, 55.000 rpm). Die einzelnen Banden wurden im Anschluss abgenommen und getrennt auf ein SDS-PAGE-Proteingel aufgetragen und mit Coomassie-Brilliant-Blau angefärbt. Dabei entsprechen die Zahlen in Klammern (2-5) den Spalten in \cref{fig:v:SelfformingGradientABPV}.}%% caption
{}%% optional (short) caption for table of figures
{fig:v:csclabpv}%% label

\myfig{project-V-virendiagnostik/figures/ABPV_SelfformingCsCl_}
%% filename in given folder
{width=0.7\textwidth,height=1.2\textheight}
%% maximum width/height, aspect ratio will be kept
{Präparation von ABPV, aufgetragen auf einem SDS-PAGE-Gel, gefärbt mit Coomassie-Brilliant-Blau. 
1 - Viruspräparation nach einer ersten Zentrifugation. Diese wurde anschließend mittels selbstformierenden CsCl-Gradienten weiter aufgereinigt und die einzelnen Fraktionen auf das Gel aufgetragen (2-9). Die Virusproteine (VP1, VP2, VP3) sind deutlich zu erkennen, außerdem kann man auch das kleinste Strukturprotein (VP4) erkennen (6-8). }%% caption
{}%% optional (short) caption for table of figures
{fig:v:SelfformingGradientABPV}%% label

\subsection{Immunisierungen}
\subsubsection{Sackbrutvirus (SBV)}

Eine Maus wurden mit gereinigten, viralen Strukturproteinen von SBV immunisiert, um Vergleichsseren zu generieren. Die Antigenpräparationen wurden mittels chromatographischer Verfahren aus inaktivierten Viruspräparationen hergestellt. Dabei trat eine sehr starke Reaktion der Antiseren gegen kontaminierende hochimmunogene Bienenproteine (v.a. Hexamerin) auf, die eine Präparation monoklonaler Antikörper aus den Plasmazellen der entsprechenden Maus nicht ratsam erscheinen ließ. Das Serum kann allerdings für Kontrollversuche genutzt werden, um zu zeigen, dass die in \textit{E. coli} erzeugten rekombinanten Antigene in Virusproteinpräparationen vorkommen.

Außerdem wurden zwei Mäuse mit dem rekombinanten, gereinigten mCherry-VP1 des SBV über einen Zeitraum von sechs Wochen immunisiert. Die Serokonversion wurde mittels Blutentnahme im Westernblot bestätigt. Im Anschluss wurden die B-Lymphozyten der Mäuse mit Myelomzellen (sp2/0) fusioniert und Hybridomzellklone gewonnen.

Zusätzlich wurden zwei Mäuse mit der Viruspräparation von SBV immunisiert. Eine Serokonversion konnte nach vierwöchiger Immunisierung mit Antigen und Adjuvans im Westernblot bestätigt werden. Somit stehen diese Mäuse in Kürze zur finalen Boosterung und Fusion bereit.

\subsubsection{Akutes Bienenparalysevirus (ABPV)}

Zwei Mäuse wurden mit der Viruspräparation von ABPV (siehe \cref{fig:v:csclabpv} sowie \cref{fig:v:SelfformingGradientABPV}) immunisiert. Die Immunisierung erfolgte über einen Zeitraum von vier Wochen mit dem Antigen und einem kommerziell erhältlichem Adjuvans. Im Anschluss wurde die Serokonversion mittels Blutentnahme im Westernblot bestätigt. Nach einer täglichen Boosterung der Mäuse in der sechsten Woche über insgesamt drei Tage, wurden die B-Lymphozyten der zwei Mäuse mit sp2/0-Zellen fusioniert und stehen für das Screening bereit.

Zusätzlich wurden zwei weitere Mäuse mit den rekombinanten, gereinigten ABPV Antigenen (VP1, VP2, VP3) über einen Zeitraum von vier Wochen immunisiert. Die Serokonversion konnte auch hier mittels Blutentnahme im Westernblot bestätigt werden (siehe  \cref{fig:v:westernblotserumabpvrek}). Die Mäuse wurden mit sp2/0 Zellen fusioniert und befinden sich derzeit in Selektion.

\myfig{project-V-virendiagnostik/figures/abpv_maus_pks97-99}
%% filename in given folder
{width=0.5\textwidth,height=1.2\textheight}
%% maximum width/height, aspect ratio will be kept
{Westernblot mit dem Serum (1:2000 verdünnt) einer der mit rekombinanten VP1, VP2 und VP3 immunisierten Mäuse; 1: rekombinante Proteine (VP1, VP2, VP3) 2: Lysat aus \textit{E. coli} 3: Negativkontrolle 4: gereinigte ABPV Viruspräparation; insbesondere die Reaktivität des Serums gegen VP1 (Pfeil) zeigt ein besonders deutliches Signal.}%% caption
{}%% optional (short) caption for table of figures
{fig:v:westernblotserumabpvrek}%% label




\subsection{Antikörper}
\subsubsection{Flügeldeformationsvirus (DWV)}
Eine Testung und Produktion der vielversprechendsten anti-DWV Immunglobuline wurde gestartet. Die entsprechenden Hybridome wurden in Kultur genommen und es wurde begonnen, serumfreie Zellkulturen dieser Hybridome zu etablieren. Zur Reinigung der betreffenden IgG1 Antikörper der Maus mittels Protein-G-Affinitätschromatographie ist eine serumfreie Kultur Voraussetzung, da ansonsten kontaminierende IgGs aus dem Kälberserum mit gereinigt werden. Gleichzeitig wurden die Nukleotid-Sequenzen der Immunglobuline zweier Hybridome (DWV-VP1A1 und DWV-VP1B1) bestimmt, um eine rekombinante Expression der Antikörper und eine Manipulation der Moleküle zu ermöglichen. Zur Etablierung der gewünschten Sandwich-ELISA sollen rekombinante Moleküle mit anderen FC-Regionen erzeugt werden, um kostengünstige, kommerziell erhältliche Sekundärantikörper verwenden zu können. Erste bakterielle Plasmide, die chimäre Immunglobuline kodieren, wurden bereits generiert. Erste Ergebnisse zur Reaktivität der chimären Antikörper zeigen eine hohe Reaktivität mit den authentischen Virusproteinen von DWV im Westernblot und in Immunfluoreszenz-Assays. Eine präparative Expression und Reinigung der Antikörper von den betreffenden Klonen (anti-DWV VP1A1 und anti-DWV VP1B1) wurden vorbereitet.

\subsubsection{Sackbrutvirus (SBV)}
Das initiale Screening mittels ELISA auf Reaktivität gegen das rekombinante Antigen (mCherry-VP1) ergab insgesamt etwa 55 stark reaktive Zellklone pro Maus. 
Die weitere Charakterisierung der Hybridomüberstände umfasste Westernblots mit dem rekombinanten Antigen (mCherry-VP1) sowie Virus aus SBV infizierten Bienenpuppen. Dabei wurden 17 Antikörper identifiziert, die das rekombinante Antigen (mCherry-VP1) erkennen. Jene Hybridomzellklone, welche eine ausreichende Sensitivität sowie Spezifität gegen das Antigen (mCherry-VP1) gezeigt haben, wurden mittels der so genannten „limiting dilution method“ reselektiert, um eine homogene (klonale) Hybridomzellpopulation mit stabiler Antikörperproduktion aufzubauen. Sie werden derzeit auf -150°C gelagert und nach Aufreinigung des o.g. SBV Virusstocks weiter charakterisiert.

\subsubsection{Akutes Bienenparalysevirus (ABPV)}
\subsubsubsection {Antikörper gegen das hochreine Virus}
Die Fusion der B-Lymphozyten zweier Mäuse mit Myelomzellen (sp2/0) wurde planmäßig durchgeführt. Dabei konnten insgesamt 11 96-well Zellkulturplatten mit jeweils mindestens einem Hybridomzellklon hochgezogen werden. 
Aufgrund der Immunisierung mit der gereinigten Viruspräparation kann das Screening nicht mittels ELISA erfolgen, da es unter Umständen erneut zu unerwünschten Kreuzreaktionen (und damit falsch positiven Zellklonen) mit Bienenproteinen kommen könnte. Außerdem bindet das Virus nur unzureichend an ELISA-Platten. \\

Daher wird für das Screening ein anderes Verfahren eingesetzt. Das Grundprinzip des Screenings basiert auf dem spezifischen Fangen von Virus mithilfe immobilisierter Antikörper aus den Hybridomüberständen und die anschließende Analyse der gebundenen Virusmenge mittels RT-qPCR. Dazu werden die Antikörper aus dem Hybridomüberstand an magnetische Beads, die mit rekombinanten Protein A/G gekoppelt sind, gebunden. Die gebundenen Antikörper werden mit Virus inkubiert, gewaschen und der Antikörper-Antigen-Komplex von den Beads eluiert. Im Anschluss wird die gebundene Virusmenge mittels RT-qPCR analysiert. \\

Die grundsätzliche Durchführbarkeit ("proof of principle") dieser Methode wurde bereits mit dem Serum der immunisierten Mäuse erfolgreich getestet und die Hybridomzellen werden in Kürze gescreent. Da Protein A/G die Eigenschaft hat, Immunglobulin G verschiedener Tierarten zu binden, wurden die Hybridome bereits an ein Zellkulturmedium, das fetales Kälberserum (FKS) mit einem extrem niedrigen bovinen IgG-Gehalt enthält, adaptiert. Damit kann beim Screening die Konkurrenz um die Bindungsstellen verringert werden.


\subsubsubsection {Antikörper gegen die rekombinanten Strukturproteine VP1, VP2 und VP3}
Zwei Mäuse wurden über einen Zeitraum von 4 Wochen mit den rekombinanten Antigenen und Adjuvans immunisiert und die Serokonversion wurde mittels Blutentnahme im Westernblot bestätigt. Anschließend erfolgte die Fusion mit den Myelomzellen und die Hybridome befinden sich derzeit im Selektionsmedium und können in Kürze initial mit ELISA gescreent werden.

\subsection{ELISA-Tests mit definierten Proben}
Es wurden noch keine ELISA-Tests mit im Labor erzeugten, Antigen-definierten Proben aus Virusinfektionen durchgeführt. Allerdings wurden die entsprechenden, definierten Proben erzeugt (s.o.).

\subsection{ELISA-Tests mit Feldproben}
Es wurden noch keine ELISA-Tests mit Feldproben durchgeführt.

















