\section{Ergebnisse}

\subsection{Antigene}

Bislang wurden Virusstocks von DWV, SBV und ABPV produziert, um die saisonale Bruttätigkeit der Honigbienen möglichst gut auszunutzen. Außerdem wurden rekombinante Proteine der verwendeten SBV und ABPV Stämme in bakteriellen Expressionssystemen generiert. Um die Viren aus Feldproben möglichst effizient detektieren zu können, wurden die viralen Strukturproteine für die Proteinexpression ausgewählt. 
Das Strukturprotein VP1 des SBV sowie die Strukturproteine VP1, VP2 und VP3 des ABPV konnten erfolgreich in Bakterien exprimiert und mittels Nickel-Affinitätschromatographie aufgereinigt werden. Das gereinigte Antigen VP1 des SBV wurde im Anschluss dazu verwendet, Mäuse zu immunisieren. Da es in diesem Versuch nicht gelang geeignete Antikörper Kandidaten zu identifizieren, wurde das Expressionskonstrukt modifiziert. Um ein stabileres und löslicheres Protein zu erhalten, wurde ein kleines, lösliches Fluoreszenzprotein (mCherry) an das virale Strukturprotein angeknüpft (Abb. 1.1).

\myfig{project-V-virendiagnostik/figures/sbv_neu}
%% filename in given folder
{width=0.6\textwidth,height=0.6\textheight}
%% maximum width/height, aspect ratio will be kept
{Proteinexpression des rekombinanten Fusionsproteins mCherry-VP1 des Sackbrutvirus (SBV).}%% caption
{}%% optional (short) caption for table of figures
{fig:Proteinexpression SBV}%% label

Das modifizierte Antigen konnte erfolgreich exprimiert, gereinigt und bereits zur Immunisierung von zwei Mäusen herangezogen werden (s.u.). 

Für ABPV wurden insgesamt drei Strukturproteine (VP1, VP2 und VP3) exprimiert (Abb. 1.2). Diese Proteine wurden unter denaturierenden Bedingungen (8M Urea) aufgereinigt. Mittels Dialyse wurden sie in einen, für die Immunisierung geeigneten, Puffer verbracht und langsam partiell renaturiert. Die Antigene stehen somit zur Immunisierung zur Verfügung.

\myfig{project-V-virendiagnostik/figures/abpv_neu}
%% filename in given folder
{width=0.8\textwidth,height=1\textheight}
%% maximum width/height, aspect ratio will be kept
{Proteinexpression der rekombinanten Strukturproteine VP1, VP2 und VP3 des akuten Bienenparalysevirus (ABPV); 1 – nicht induzierte Bakterienkultur, 2 – induzierte Bakterienkultur nach 2.5h, 3 – induzierte Bakterienkultur nach 3.5h
}%% caption
{}%% optional (short) caption for table of figures
{fig:Proteinexpression ABPV}%% label

Zusätzlich wurden Virusstocks von DWV, SBV und ABPV durch Infektion von Bienenbrut erzeugt (jeweils ca. 400 Bienenpuppen). 96h nach der Infektion wurden die Larven und Puppen lysiert und die Viren durch fraktionierte Pelletierung und Dichtegradientenzentrifugation angereichert. Gereinigte Virusantigene dienen zur Immunisierung von Mäusen und sollen als Testantigene Verwendung finden. 

\subsection{Immunisierungen}
Eine Maus wurden mit gereinigten, viralen Strukturproteinen von SBV immunisiert, um Vergleichsseren zu generieren. Die Antigenpräparationen wurden mittels chromatographischer Verfahren aus inaktivierten Viruspräparationen hergestellt. Dabei trat eine sehr starke Reaktion der Antiseren gegen kontaminierende hochimmunogene Bienenproteine (v.a. Hexamerin) auf, die eine Präparation monoklonaler Antikörper aus den Plasmazellen der entsprechenden Maus nicht ratsam erscheinen ließ. Das Serum kann allerdings für Kontrollversuche genutzt werden, um zu zeigen, dass die in \textit{E. coli} erzeugten rekombinanten Antigene in Virusproteinpräparationen vorkommen. 
Außerdem wurden zwei Mäuse mit dem rekombinanten, gereinigten mCherry-VP1 des SBV über einen Zeitraum von sechs Wochen immunisiert. Die Serokonversion wurde mittels Blutentnahme im Westernblot bestätigt. Im Anschluss wurden die B-Lymphozyten der Mäuse mit Myelomzellen (sp2/0) fusioniert und Hybridomzellklone gewonnen. 

\subsection{Antikörper}
\subsubsection{Flügeldeformationsvirus (DWV)}
Eine Testung und Produktion der vielversprechendsten anti-DWV Immunglobuline wurde gestartet. Die entsprechenden Hybridome wurden in Kultur genommen und es wurde begonnen, serumfreie Zellkulturen dieser Hybridome zu etablieren. Zur Reinigung der betreffenden IgG1 Antikörper der Maus mittels Protein-G-Affinitätschromatographie ist eine serumfreie Kultur Voraussetzung, da ansonsten kontaminierende IgGs aus dem Kälberserum mit gereinigt werden. Gleichzeitig wurden die Nukleotid-Sequenzen der Immunglobuline zweier Hybridome (DWV-VP1A1 und DWV-VP1B1) bestimmt, um eine rekombinante Expression der Antikörper und eine Manipulation der Moleküle zu ermöglichen. Zur Etablierung der gewünschten Sandwich-ELISA sollen rekombinante Moleküle mit anderen FC-Regionen erzeugt werden, um kostengünstige, kommerziell erhältliche Sekundärantikörper verwenden zu können. Erste bakterielle Plasmide, die chimäre Immunglobuline kodieren, wurden bereits generiert. Erste Ergebnisse zur Reaktivität der chimären Antikörper zeigen eine hohe Reaktivität mit den authentischen Virusproteinen von DWV im Westernblot und in Immunfluoreszenz-Assays. Eine präparative Expression und Reinigung der Antikörper von den betreffenden Klonen (anti-DWV VP1A1 und anti-DWV VP1B1) wurden vorbereitet.
\subsubsection{Sackbrutvirus (SBV)}
Das initiale Screening mittels ELISA auf Reaktivität gegen das rekombinante Antigen (mCherry-VP1) ergab insgesamt etwa 55 stark reaktive Zellklone pro Maus. 
Die weitere Charakterisierung der Hybridomüberstände umfasste Westernblots mit dem rekombinanten Antigen (mCherry-VP1) sowie Virus aus SBV infizierten Bienenpuppen. Dabei konnten 17 Antikörper identifiziert werden, die das rekombinante Antigen (mCherry-VP1) erkennen können. Jene Hybridomzellklone, welche eine ausreichende Sensitivität sowie Spezifität gegen das Antigen (mCherry-VP1) gezeigt haben, wurden mittels der so genannten „limiting dilution method“ reselektiert, um eine homogene (klonale) Hybridomzellpopulation mit stabiler Antikörperproduktion aufzubauen. Sie werden derzeit auf -150°C gelagert und nach Aufreinigung des o.g. SBV Virusstocks weiter charakterisiert. 

\subsection{ELISA-Tests mit definierten Proben}
Es wurden noch keine ELISA-Tests mit im Labor erzeugten, Antigen-definierten Proben aus Virusinfektionen durchgeführt. Allerdings wurden die entsprechenden, definierten Proben erzeugt (s.o.).

\subsection{ELISA-Tests mit Feldproben}
Es wurden noch keine ELISA-Tests mit Feldproben durchgeführt.

















