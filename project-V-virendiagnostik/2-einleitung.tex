
% Zitate müssen noch umformatiert werden zB \cite{carreck2010, genersch2010, dainat2012}
%Zitate müssen außerdem noch hinterlegt werden
% kursiv: \textit{Varroa destructor}
%Stichpunkte: \begin{itemize}
% \item 
% \end{itemize}
% Die Unterpunkte müssen noch entsprechend formatiert werden
% für Zitate zweier Autoren &: \&
%Durch dieses Projekt soll die Diagnostik von Virusinfektionen bei Honigbienen vereinfacht werden. Die ektoparasitische Varroamilbe \textit{(Varroa destructor)} und virale Krankheitserreger werden für einen Großteil der Kolonieverluste vor \cite{(Dainat et al., 2012)} 
\section{Einleitung}
Durch dieses Projekt soll die Diagnostik von Virusinfektionen bei Honigbienen vereinfacht werden. Die ektoparasitische Varroamilbe \textit{(Varroa destructor)} und virale Krankheitserreger werden für einen Großteil der Kolonieverluste vor \citep{dainat2012} und während der Überwinterung verantwortlich gemacht \citep{Kielmanowicz2015}. Untersuchungen aus verschiedenen Ländern belegen einen direkten Einfluss von Iflaviren (SBV, DWV) und Dicistroviren (ABPV) auf die Vitalität von Bienenvölkern. Diese viralen Erreger der Bienen sind unbehüllte RNA Viren, aus der Ordnung Picornavirales. Sie teilen sich einen grundsätzlich ähnlichen Aufbau, zeigen aber zueinander nur eine geringe Ähnlichkeit und haben eine variable Genomsequenz. Im Gegensatz zu hochspezifischen RT-PCR-Protokollen, die in den meisten Fällen nur bestimmte Varianten bzw. Virusspezies detektieren können und sehr teuer sind, sind serologische Assays häufig in der Lage, größere Erregergruppen diagnostisch zu erfassen . In diesem Projekt sollen bereits vorhandene serologische Reagenzien gegen das Flügeldeformationsvirus (DWV) zur Etablierung von Hochdurchsatz-Tests im Labor sowie „Point of care“ Tests am Bienenstock genutzt werden. Gleichzeitig sollen neue Reagenzien gegen Sackbrutvirus (SBV) und Dicistroviren (Virus der akuten Bienenparalyse) produziert werden, um auch diese Pathogengruppen abdecken zu können. 
Viele frühe wissenschaftliche Studien zu Viruserkrankungen der Honigbienen nutzten zum Nachweis der Erreger serologische Reagenzien \citep{Anderson1984}. Serologische Nachweisverfahren wurden früher auch erfolgreich zur Diagnostik von Feldinfektionen eingesetzt. In allen Studien wurden polyvalente Seren verwendet, die durch Immunisierung von Versuchstieren mit Virusextrakten gewonnen wurden und eine komplexe Mixtur aus unterschiedlichen Immunglobulinen darstellten. Da es für Honigbienen weder Zellkultursysteme noch klonale Virusstämme gab, war die Spezifität der Antiseren beschränkt. Neben Hintergrundreaktionen mit Bienenproteinen aus der Virusproduktion waren auch unerwünschte Reaktionen mit kontaminierenden Viren nie auszuschließen. Obwohl man Reinigungsprotokolle für diese Reagenzien entwickelte (z. B. Adsorption der Seren gegen Bienenlysate), wurden die etablierten ELISA-Systeme in der Diagnostik wegen der limitierten Verfügbarkeit der Reagenzien bald durch RT-PCR Protokolle ersetzt. In jüngster Vergangenheit sollte im Programm „Bees in Europe and the Decline Of honeybee Colonies (BEE DOC)” im europäischen „Research Framework 7“ ein Bienenviren-ELISA entwickelt werden. Das Problem der unspezifischen Reaktionen mit Bienenproteinen sollte dabei durch Immunisierungen mit synthetischen viralen Peptiden umgangen werden. Die Immunisierungen mit Peptiden verliefen aber nicht erfolgreich, so dass kein Assay entwickelt werden konnte. Monoklonale Antikörper können dagegen durch Immunisierung mit authentischen Proteinen generiert werden, wobei durch Isolierung einzelner B-Zellen und damit IgGs eine hohe Spezifität und Sensitivität der Reagenzien gewährleistet werden kann. Durch den Einsatz monoklonaler Antikörper könnten die Nachteile der etablierten serologischen Nachweissysteme, wie mangelhafte Spezifität und begrenzte Verfügbarkeit der verwendeten Reagenzien, aufgehoben werden und kostengünstige, quantifizierende Testsysteme etabliert werden \citep{Usuda1999}.
Die Spezifität und Sensitivität der ELISA wird vorrangig durch die verwendeten serologischen Reagenzien bestimmt. In diesem Projekt werden monoklonale Antikörper verwendet bzw. generiert, die nach Charakterisierung der Reaktivität in unbeschränkter Menge zur Verfügung stehen. Die neuen Testverfahren sollen eine kostengünstige Diagnostik von Virusinfektionen bei Honigbienen in Österreich ermöglichen. ELISA Systeme geben schnell ein zuverlässiges quantitatives Ergebnis und können vor Ort durchgeführt werden („On-site-Tests“). Mit Platten-basierten ELISAs können in Laboratorien kostengünstige Massentests („high throughput systeme“) durchgeführt werden, während Lateral-Flow-Tests (bekannt aus Schwangerschaftstests) zu einem etwas höheren Preis direkt vor Ort angewendet werden können. Ein Lateral-Flow-Test zum Nachweis der Amerikanischen Faulbrut wurde bereits für die Imkerei entwickelt und wird weltweit vertrieben (Vita Europe Ltd.).




