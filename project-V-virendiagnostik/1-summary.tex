\begin{otherlanguage}{english}
The aim of Module V is to generate monoclonal antibodies for the detection of viral gene products and for the development of enzyme linked immunosorbent assays (ELISA). The ELISA will be used for rapid, low cost detection of viral infections in honey bee colonies in the laboratory and under field conditions. The project will run for three to four years and started as planned in May 2018. The first step of this module is the production of antigen preparations that can be used for vaccination of experimental animals and antigen tests. Due to the favorable project start time, the seasonal breeding activity of our honey bees in 2018 could be used to produce authentic antigens of different virus species (DWV, ABPV and SBV) through infection experiments. In this first phase of the project, large scale preparations of different virus species were generated and studied to determine their purity and infectivity. Highly purified virus stocks have been obtained by ultracentrifugation and will serve for immunizations, screening and test development. Using recombinant DNA techniques specific antigens of Deformed Wing Virus (DWV-A), Varroa destructor Virus (VDV/DWV-B), Sacbrood Virus (SBV) and Acute Bee Paralysis Virus (ABPV) were produced. So far, capsid proteins of SBV, DWV and APBV were cloned and successfully expressed in bacteria. Purified antigens were used for the immunization of mice. Monoclonal antibodies that detect VP1 of SBV and DWV have been successfully prepared, immunizations of mice with ABPV antigens are in progress.  For the intended use in diagnostic tests, antibodies will be further characterized to determine the highest possible sensitivity and specificity.
The project part is therefore on schedule.

\end{otherlanguage}
