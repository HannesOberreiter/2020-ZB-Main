\section{Diskussion}

Nach gegenwärtigem Stand der Forschung sind für die epidemisch auftretenden Völkerverluste bei Honigbienen in Europa und Nordamerika im Herbst und Winter der ubiquitäre Varroabefall in Kombination mit hohen Infektionsbelastungen mit DWV, SBV oder ABPV verantwortlich. Die Belastung mit diesen Varroa-assoziierten Viren lässt sich daher auch als ein prädiktiver Marker für den Zusammenbruch eines Bienenvolkes verwenden. Die Virusdiagnostik basiert auf sehr sensitiven molekularbiologischen RT-qPCR Methoden, bei denen jedoch hohe Laborkosten von mindestens 20 € für die Nukleinsäure-Extraktion und 15-20 € je Virusnachweis entstehen (je nach Sensitivität). Eine Untersuchung auf die drei Viren würde also mindestens 65 bis 80 € kosten. Trotz des vergleichsweise hohen Wertes eines Bienenvolkes von mehr als 200 € bei starken Kolonien mit Reinzuchtkönigin ist eine Untersuchung der Virenbelastung daher unwirtschaftlich. In Massentests zur Diagnostik in der Humanmedizin und auch zur Tierseuchenbekämpfung werden meist serologische Verfahren genutzt, da sie unempfindlich gegenüber der Probenlagerung, sehr sensitiv und wesentlich kostengünstiger sind. 
\\
\\
In dem hier vorgestellten Projekt werden bereits vorhandene Reagenzien gegen Struktur- und Nichtstrukturproteine von DWV genutzt und neue Reagenzien gegen SBV und ABPV generiert, um serologische Testverfahren zu etablieren. Ziel des Projektes ist die Entwicklung plattenbasierter Hochdurchsatztests, die im Massentest Analysen zu einem Preis zwischen 5 und 10 € ermöglichen. Die Entwicklung handelsüblicher „On-site“ Tests im Lateral-Flow-Format (z.B. COVID-19 Schnelltest, Schwangerschaftstest) soll dem Imker die Möglichkeit eröffnen, die Kolonien im Verdachtsfall vor Ort zu testen. Da bei der Diagnostik am Stock Versand- und Transportaufwand entfallen und der Imker unmittelbar vor Ort handeln kann, werden solche Testverfahren trotz höherer Assaykosten (ca. 20-30 € pro Schnelltest) sehr hilfreich sein. Erste Versuche mit den bereits vorhandenen Reagenzien gegen DWV ergaben, dass einige der Antikörper in Immun-Fluoreszenzmarkierungen und Westernblot-Applikationen die viralen Antigene hochspezifisch binden \citep{Lamp2016}. Die neuen Reagenzien und Testverfahren sollen helfen, die Verluste der Imkereien einzudämmen, da positive Untersuchungsergebnisse im Spätsommer eine Sanierung (z. B. Kunstschwarmbildung mit neuer Königin) oder eine Auflösung der Völker ermöglichen, bevor sichtbare Schäden auftreten.
