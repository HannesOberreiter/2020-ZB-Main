\section{Diskussion}

Die Erhebung der Winterverluste von Bienenvölkern wird seit 2008 vom Institut für Biologie (vormals: Zoologie) der Universität Graz durchgeführt. In diesem Bericht werden die Ergebnisse des Untersuchungsjahres 2019/20 zusammengefasst und zum Teil mit denen der Vorjahre verglichen. Die im Winter 2019/20 in Österreich erhobene Verlustrate von Bienenvölkern betrug \confi{12,6}{95}{11,9}{13,3}, bei einer Teilnahme von 1539 Imkereien, was einer prozentualen Beteiligung von 5,1\% aller österreichischen Imkereibetriebe entspricht. Diese stellten insgesamt Information über 30.724 eingewinterte Bienenvölker zur Verfügung, was wiederum 7,9\% aller in Österreich gehaltenen Bienenvölkern entspricht. Die Verlustrate liegt, verglichen mit den bisherigen Ergebnissen, im mittleren Bereich. Die laufende mittlere Verlustrate aller in Österreich erhobenen Winterverlustraten liegt bei 16,1\%. Die bisher niedrigste in dieser Erhebung gemessene Verlustrate lag bei \confi{8,1}{95}{7,4}{8,8} (2015/16) und die bisher höchste gemessene Verlustrate bei \confi{28,4}{95}{27,0}{29,9} (2014/15). Diese großen Schwankungen in den Verlustraten, deren genaue Ursachen größtenteils noch unverstanden sind, machen die große Bedeutung des Bienen-Monitorings und der anschließenden Ursachenforschung deutlich. Eine internationale Vergleichbarkeit der erhobenen Daten wird durch die vom Forschungsnetzwerk COLOSS festgelegten Fragen ermöglicht \citep{brodschneider2016,brodschneider2018,vanderzee2013}. Diese Art der Untersuchung ist ein Hilfsmittel, den Ursachen hoher Winterverluste auf den Grund zu gehen. Die Trends und Erkenntnisse der freiwilligen Umfrage sollen zusätzlich durch gezielte Probenentnahmen ergänzt werden \citep{vanderzee2015}.

\subsection{Populationsdynamik in Österreich}

Wie auch in den Jahren zuvor wird aus den Winterverlustraten und dem Völkerstand im Frühjahr davor eine hypothetische Populationsdynamik der Bienenvölker vom Frühjahr 2013 bis zum Frühjahr 2020 modelliert (\cref{fig:u:population}). Dabei zeigt sich, dass eine Vermehrung über den Sommer, etwa durch eigene Nachschaffung oder Zukäufe, eine konstante, sogar steigende, Bienenpopulation in Österreich erlaubt.
\newline
Im Jahr 2014/15 mit sehr hohen Winterverlusten waren die ImkerInnen in der Lage die verlorenen Bienenvölker über den Sommer wieder zu kompensieren. In den folgenden Jahren mit geringeren Winterverlusten, sind die ImkerInnen in der Lage, diese Verluste im Sommer auszugleichen und auch eine größere Anzahl an Bienenvölkern im folgenden Herbst einzuwintern.
\newline
Unser Modell ist aber nur begrenzt gültig --- je nach Höhe der Winterverluste sind von den ImkerInnen dementsprechend Zeit und Geld zu investieren, um eine zum Vorjahr vergleichbare Population  aufzubauen beziehungsweise zu erhalten. Die Arbeit und Kosten für die Nachschaffung werden dabei vorwiegend von Kleinimkereien, welche den Großteil der Bienenpopulation in Österreich betreuen, getragen. Ökonomische Abschätzungen dieser Leistungen zum Erhalt einer starken Bienenpopulation sollten für eine nachhaltige Entwicklung des Imkereisektors berücksichtigt werden.

\subsection{Repräsentativität und Anonymität}

Die in diesem Bericht präsentierten Ergebnisse beruhen auf den Angaben der österreichischen ImkerInnen. Um eine möglichst große Beteiligung der ImkerInnen zu erreichen, wurde sowohl online, in der Fachzeitschrift \enquote{Bienen Aktuell}, bei Veranstaltungen als auch über bereits bestehende Kontakte (z.B.: nicht anonyme Teilnehmer vorheriger Studien) zu einer Teilnahme an unserer Untersuchung aufgerufen. Diese über Jahre etablierten Kontakte helfen, eine nahezu vollständige Abdeckung an Rückmeldungen von einzelnen Gemeinden und Imkervereinen zu erhalten. Um möglichst klare und eindeutige Antworten zu erhalten, wurden die Fragen bewusst einfach gestaltet. Den Großteil der Antworten, ca. 88,5\%, haben wir online erhalten, wohingegen der Anteil derer, die den Fragebogen aus der Zeitschrift \enquote{Bienen Aktuell} retournieren über die letzten Jahre immer geringer geworden ist. Dieses Jahr gab es im Vergleich zum Vorjahr eine erhöhte Teilnahme mittels Kurzfragebogen der Zeitschrift (+1,5\%) welche in Summe 5,6\% der TeilnehmerInnen in unserer Umfrage ausmacht. Eine mögliche Erklärung für die höhere Teilnahme mittels Zeitschrift könnte ein Nebeneffekt durch die Corona bedingte Quarantäne im Frühjahr gewesen sein. Teilnehmer hatten vielleicht mehr Interesse und Zeit für die Zeitschrift und dabei auch dem Fragebogen mehr Beachtung geschenkt.
\newline
In diesem Jahr gab es einen signifikanten Unterschied der Verlustraten bei den Erfassungsmodi, hierbei hatten TeilnehmerInnen mittels Kurzfragebogen aus der Zeitung eine erhöhte Wahrscheinlichkeit mehr Winterverluste zu erfahren im Gegensatz zu den Online-Teilnehmern. Da wir mit dem Umfragebogen in der Zeitschrift auch Personengruppen ohne Internetzugang ansprechen wollen, könnte dies auf ein Problem in der Repräsentativität hindeuten. Die langjährige Erfahrung und auch Stichprobenuntersuchungen von anderen Studien zeigen aber, dass unsere jährlichen Verlustanalysen mit dem gewählten Konfidenzintervall bis dato immer alle Fälle abgedeckt haben. Da im letzten Winter dieser Effekt noch nicht zu sehen war können wir noch nicht genauer darauf eingehen, werden es aber im Auge behalten. Es zeigt die Wichtigkeit verschiedene Teilnahmemodi anzubieten und weiterhin den verkürzten Fragebogen in der Zeitschrift beizubehalten. Die Verlustraten zwischen anonymer und nicht-anonymer Teilnahme war wie auch im Vorjahr ohne signifikanten Unterschied. Im Untersuchungsjahr 2019/20 haben 67,3\% der TeilnehmerInnen freiwillig eine Kontaktmöglichkeit genannt, was zum einen von Vertrauen in die Untersuchung zeugt, uns aber auch die Möglichkeit gibt bei Unklarheiten nachzufragen.
\newline
Die durchschnittliche Völkeranzahl der an dieser Erhebung teilnehmenden Imkereien lag 2019/20 bei 20 eingewinterten Völkern, was etwas mehr ist als der in Österreich aufgrund der Angaben der Biene Österreich erwartete Mittelwert (13 Völker/Imkerei). Dazu trägt die Teilnahme von 19 großen Imkereien mit jeweils über 150 eingewinterten Völkern bei. Der Median der eingewinterten Völkerzahlen liegt jedoch bei 10 eingewinterten Völkern, was der erwarteten österreichischen Imkerei-Demographie durchaus entspricht. Auch daraus und durch die Verteilung der teilnehmenden Imkereien über ganz Österreich (\cref{fig:u:overviewmap}) folgern wir, dass die dieser Auswertung zugrundeliegenden Daten einer durchmischten und annähernd repräsentativen Gruppe der österreichischen Imkereien entstammen.

\subsection{Bundesländer und Bezirke}

Wie auch die Jahre davor sind die Winterverluste in Österreich nicht gleich verteilt. Dieses Jahr wurde besonders in Wien eine signifikant höhere Verlustrate beobachtet als in den anderen Bundesländern (\cref{fig:u:states}). Diese Variation zwischen den Bundesländern konnte auch in vorherigen Winterverlust-Analysen festgestellt werden \citep{crailsheim2018, brodschneider2018a, brodschneider2019b, oberreiter2020}. Hierbei handelt es sich nicht um ein österreichisches Phänomen sondern findet sich auch in anderen Ländern wie Tschechien oder den USA wieder, wo verschiedenen Regionen abweichende Verlustraten zeigen \citep{brodschneider2019,vanengelsdorp2008,vanengelsdorp2010}. Eine mögliche Erklärung hierfür zeigen zwei österreichischen Studien in denen ein Zusammenhang von Wetter \citep{switanek2017} und Landnutzung \citep{kuchling2018} mit Winterverlusten festgestellt wurde. Die Landnutzung betreffend wurde festgestellt, dass semi-natürliche Regionen, Weiden und Nadelwälder einen positiven Effekt auf die Überlebenswahrscheinlichkeit der Bienenvölker hatte.
\newline
Die österreichweite Gesamtverlustrate von \confi{12,6}{95}{11,9}{13,3} im Jahr 2019/20 liegt unter dem jährlichen Durchschnitt und spiegelt sich größtenteils auch in den Bezirksergebnissen wider. 
\newline
Durch die Analyse der Unterschiedlichen Bezirke zeigen sich teilweise aber auch Gebiete mit potentiell sehr hohen Winterverlusten (\cref{fig:u:map:loss:district}). Aus Gründen des Datenschutzes und der Repräsentativität werden nur jene Bezirke aufgelistet, bei denen mindestens Daten von fünf Imkereien zur Verfügung stehen. Hierbei zeigten sich besonders hohe Verluste mit einer Verlustwahrscheinlichkeit von über 20\% für Wien (20,07\%, \sample{92}), Neunkirchen (NÖ) (29,22\%, \sample{16}), Wels-Land (OÖ) (21,43\%, \sample{14}) und Lienz (Tirol) (23,48\%, \sample{9}). Niedrige Verlustwahrscheinlichkeiten von unter 10\% konnten in 18 Bezirken festgestellt werden, von denen die Bezirke Salzburg (5,88\%, \sample{5}), Hartberg-Fürstenfeld (Stmk) (5,96\%, \sample{16}) und Villach (Kärnten) (6,08\%, \sample{9}) die geringsten Verluste aufwiesen.
\newline
Die Bezirksergebnisse sollten aber mit Vorsicht angesehen werden, da durch die geringe Stichprobenanzahl auch die Konfidenzintervalle sehr breit werden und unser Modell zur Berechnung hierfür nicht optimal ist. Anhand der Bezirkkarte (\cref{fig:u:map:loss:district}) kann man aber Regionen erkennen in denen mehrere angrenzende Bezirke hohe bzw. niedrige Verluste erfahren hatten. Dies könnte, wie bereits oben angeführt, durch unterschiedliche Landnutzung erklärt werden \citep{kuchling2018}.

\subsubsection{Seehöhe}

Ein weiterer Faktor der die unterschiedlichen Regionen widerspiegeln könnte,  ist die Seehöhe. Hier konnte dieses Jahr wieder ein epidemiologischer Zusammenhang zwischen dem Überwinterungserfolg von Bienenvölkern und der Seehöhe des Haupt-Überwinterungsbienenstandes nachgewiesen werden, wie ebenfalls im Jahr zuvor \citep{oberreiter2020}. Es zeigten sich signifikant höhere Verlustwahrscheinlichkeiten für TeilnehmerInnen im Bereich zwischen 201-400\si{\meter} verglichen mit ImkerInnen zwischen 401-600\si{\meter} und ImkerInnen die über 800\si{\meter} überwintert haben (\cref{fig:u:elevation}).
\newline
Ein Möglicher Grund könnten die klimatischen Unterschiede sein, wie auch eine österreichische Studie aufgezeigt hat, bei der geringere Überwinterungsverluste zu erkennen waren je kälter die mittleren Temperaturen im September waren \citep{switanek2017}. Kältere Temperaturen und Klima haben Auswirkungen auf die Vegetation und machen sich in höheren Lagen durch einen verspäteten Brutbeginn und früherer Brutfreiheit bemerkbar. Dadurch verkürzt sich die Zeitspanne, in der sich die Varroamilbe in der Brut vermehren kann. Weitere Gründe könnten die unterschiedliche Landnutzungen sein \citep{kuchling2018} und die geringere Dichte an Bienenvölkern, was wiederum zu einer geringeren Verbreitung von Krankheitserregern führt \citep{seeley2015, forfert2016, dynes2019}. 

%\subsection{TODO Symptome}

%Den wichtigsten Teil der Winterverlusterhebungen bildet die Auswertung der Symptome der in Österreich verstorbenen Bienenvölker, über die wir anhand dieser Untersuchung detaillierte Informationen erhalten haben. Durch diese Daten können die Ursachen für die Winterverluste, wenn auch nicht klar einer einzelnen Ursache zugeordnet, so zumindest doch eingegrenzt und näher erforscht werden. Der Symptomkatalog wurde dabei bewusst einfach gewählt, um den ImkerInnen eine eindeutige Zuordnung ohne Hilfsmittel wie Laboruntersuchungen zu erleichtern. Auch bei demselben Schadbild können noch immer unterschiedliche Gründe hinter einem Völkerverlust stecken. Außerdem gilt es zu berücksichtigen, dass zwischen dem Auftreten des Verlustes und der Symptombeschreibung durch ImkerInnen ein beträchtlicher Zeitraum liegen kann, in dem Spuren, wie tote Bienen vor dem Volk, verschwinden können und daher nicht mehr eindeutig zuordenbar sind.
%\newline
%Im Untersuchungsjahr 2019/20 war das häufigste Symptom vor dem Verlust von Völkern \enquote{keine oder nur wenige tote Bienen im oder vor dem Volk}. Dieses Symptom ist relativ unspezifisch, und kann unterschiedlichste Ursachen haben. Dieses Schadbild wurde auch als charakteristisch für CCD beschrieben und in den USA aber auch in Europa \citep{dainat2012} in den letzten Jahren häufig beobachtet \citep{steinhauer2014,vanengelsdorp2009,williams2010}. Auch in Österreich wird dieses Schadbild Jahr für Jahr am häufigsten berichtet \citep{crailsheim2018}.
%\newline
%Eines der relativ eindeutig einer Ursache zuordenbare Symptom ist das Vorhandensein von toten Bienen, die mit dem Kopf voran in Zellen stecken. Bei gleichzeitigem Mangel an Futter im Volk kann dieses Schadbild relativ sicher als Verhungern gedeutet werden. Dabei gilt es zwei unterschiedliche Formen zu unterscheiden: Zum einen der bereits erwähnte Futtermangel, zum anderen aber auch noch verbliebene Futterreserven, die den Bienen von der Wintertraube aus, wegen widrigen Umständen, nicht zugänglich ist. Dieser auch als Futterabriss bezeichnete Winterverlust trat 2019/20 häufiger auf als das Verhungern.

\subsection{Betriebsgröße}

Der Großteil an ImkerInnen betreibt die Imkerei als reines Hobby mit durchschnittlich 1-50 Völkern. Es gibt aber auch eine geringe Anzahl an professionellen Betrieben die einen großen Anteil der Völker in Österreich besitzen (\cref{fig:u:betrieb}). Das spiegelt sich in in einer nicht Normalverteilung im Durchschnitt der Völker/Betrieb aus. Der Faktor Betriebsgröße in der Risikoanalyse ist als Sammelfaktor zu verstehen, der unterschiedlichste, im Detail nicht näher abgefragte, Parameter wie Professionalität, Erfahrung in der Imkerei etc. beinhaltet. Dieses Ergebnis spiegelt sich auch in einer anderen Studie aus Österreich wieder \citep{morawetz2019}.
\newline
Wie auch in den Jahren zuvor zeigen kleinere Betriebe mit weniger als 50 Völker eine Wahrscheinlichkeit für signifikant höhere Verlustraten (\cref{fig:u:factor:operationsize}) als größere Betriebe mit mehr als 50 Völkern. Hierbei handelt es sich mittlerweile um konstante, wiederholt aufgetretene,  Ergebnisse in Österreich wie im europäischen Vergleich \citep{brodschneider2016,brodschneider2018,crailsheim2018,oberreiter2020,vanderzee2014}. International, zum Beispiel in den USA, sind in bestimmten aber nicht allen Untersuchungsjahren bei steigender Betriebsgröße geringere Winterverluste festgestellt worden. Die an der Untersuchung teilnehmenden ImkerInnen wurden dort anhand der Anzahl ihrer Völker in drei Kategorien eingeteilt: \enquote{backyard beekeepers} (<50 Völker), \enquote{sideline beekeepers} (51-500 Völker) und \enquote{commercial beekeepers} (>500 Völker). Als Ursache für die Unterschiede in den Verlustraten werden mehrere Faktoren diskutiert. Es wird zum Beispiel ein Einfluss der Betriebsgröße auf die Betreuungsqualität der Völker angenommen: \enquote{Commercial beekeepers} zeigen oft mehr technisches Wissen in der Krankheits- und Schädlingsbekämpfung --- insbesondere bei der Bekämpfung der Varroamilbe --- als \enquote{backyard beekeepers}, welche die Bienenhaltung nicht gewerbsmäßig betreiben \citep{lee2015}. Auch die Wahl der Methode und die Qualität der verwendeten Bekämpfungsmaßnahmen könnte in Abhängigkeit von der Betriebsgröße eine andere sein \citep{underwood2019, thoms2019a}. Diese Argumente lassen sich möglicherweise eingeschränkt auch auf die österreichischen Imkereien übertragen, auch wenn hier Betriebe mit über 500 Völkern eher die Ausnahme bilden.

\subsection{Betriebsweisen}

Seit 2016/17 wird auf Wunsch der ImkerInnen auch der Einfluss der verschiedenen Betriebsweisen (zum Beispiel: Bio-Imkerei, kleine Brutzellen, Beutenisolierung) auf die Winterverlustrate getestet. Die höchste Anzahl an \enquote{Unsicher} und \enquote{keine Angaben} gab es bei der Frage \enquote{Kleine Brutzellen (5,1mm oder weniger)}. Am ausgeglichensten zwischen \enquote{Ja} und \enquote{Nein} waren die Fragen \enquote{Offener Gitterboden im Winter} und \enquote{Kaufe Wachs zu (kein eigener Wachskreislauf)} (\cref{fig:u:operational:hist}).

\subsubsection{Wanderimker}

In diesem Jahr hatten ImkerInnen die mit ihren Völkern zu Trachtangeboten wanderten eine signifikante Wahrscheinlichkeit geringere Verluste im Winter zu erfahren (\cref{fig:u:operational:loss}-B). Dieser Faktor ist nicht jedes Jahr zu sehen,  wurde aber dennoch schon öfters beobachtet, auch im europäischen Vergleich \citep{brodschneider2010, crailsheim2018, brodschneider2018, oberreiter2020}. Der geringere Verlust könnte mit der besseren Erfahrung der WanderimkerInnen im Gegensatz zu StandimkerInnen erklärt werden. Aber auch durch eine verbesserte Nahrungsversorgung über das Jahr oder bei Trachtlücken, obwohl der Transport von Bienenvölkern durchaus Stress für die Bienen bedeuten könnte \citep{vanderzee2014}. Da dieser Faktor nicht jedes Jahr einen statistischen Unterschied ausmacht, wäre es denkbar, dass hier andere Faktoren eine kombinierte Wirkung zeigen.

\subsubsection{Fremdwachs und Naturwabenbau}

Wie auch bereits im vorigen Winter zeigten Imkererein die Wachs zukauften eine signifikante Wahrscheinlichkeit für höhere Winterverluste als Imkererein mit einem eigenen Wachskreislauf (\cref{fig:u:operational:loss}-G) \citep{oberreiter2020}. Die Durchführung eines eigenen Wachskreislaufes kann für Professionalität des Imkereibetriebes sprechen. Zusätzlich ermöglicht das Verfahren aber auch eine bessere Kontrolle über Inhaltsstoffe des Wachses wie zum Beispiel Pestizidrückstände, welche in eingelagerte Pollen gelangen können und so wiederum einen negativen Einfluss auf die Wintersterblichkeit des Bienenvolkes haben können, da Wachs die am höchsten belastete Matrix im Bienenstock ist \citep{calatayud-vernich2018, harriet2017}. Versuche mit künstlich kontaminiertem Bienenwachs führten aber zu keiner erhöhter Sterblichkeitsrate in den Bienenvölkern \citep{payne2019}. Dennoch könnte es in Kombination mit anderen Faktoren zu negativen Auswirkungen führen. 
\newline
Hierbei ist zu erwähnen, dass zwischen ImkerInnen die Naturwabenbau eingesetzt haben oder nicht kein signifikanter Unterschied auftrat (\cref{fig:u:operational:loss}-H). 

\subsubsection{Varroatoleranz und kleine Brutzellen}

In diesem Jahr zeigten ImkerInnen mit Königinnen aus einer \enquote{Zucht auf Varroa-Toleranz} eine signifikante wahrscheinlichkeit für niedrigere Verlustraten im Vergleich zur Gruppe die über diese Frage \enquote{Unsicher} war. Dieser positive Effekt konnte in den vorherigen Umfragen noch nicht festgestellt werden. Der Vergleich zwischen \enquote{Ja} und \enquote{Unsicher} hat aber nur eine bedingte Aussagekraft (\cref{fig:u:operational:loss}-C).
\newline
Natürlicher Abwehrmechanismen von westlichen Honigbienen gegen die parasistierente Varroamilbe wurden bereits sehr früh gefunden \citep{ruttner1984}. Sowie erfolgreiche Feldversuche von solchen Zuchtlinien, die eine verminderte Menge an reproduzierenden Varroamilben in den Völkern haben, zeigen die Möglichkeit der selektiven Züchtung auf dieses Verhalten \citep{spivak2007}. Der große Erfolg ist aber bis jetzt ausgeblieben, was durch die komplizierte und mitunter unbekannte Ausbildung und Vererbung dieses Verhalten, Populationsdynamiken und Interaktion mit ökologischen Umgebungsfaktoren sowie andere noch unerforschte Faktoren zusammenhängen könnte \citep{fanny2020}. Kontraproduktiv in diesen Zusammenhang ist die in Europa häufig hohe Dichte an Bienenvölkern auf einzelnen Ständen und in einzelnen Regionen, welche wahrscheinlich zu einer erhöhten Übertragung von Krankheiten und Varroa führt \citep{seeley2015, forfert2016, dynes2019}. Um das fehlende Gleichgewicht zwischen Wirt und Parasit herzustellen, wie es unter natürlichen Umständen und in unberührten Regionen herrscht \citep{seeley2007}, wird von verschiedenen Autoren ein darwinistisches Prinzip propagiert um eine natürliche gerichtete Selektion wieder in den Vordergrund rücken zu lassen \citep{locke2016, neumann2017}. Dieses Konzept muss aber je nach Region und Bedingungen angepasst werden um Erfolgreich zu sein.
\newline
Interessanterweise zeigten TeilnehmerInnen die Angaben \enquote{kleine Brutzellen} verwendet zu haben eine signifikante Wahrscheinlichkeit weniger Winterverluste zu erfahren (\cref{fig:u:operational:loss}-I). Sollte dieser Betriebsfaktor auch in Zukunft wieder einen Unterschied machen, wäre hier eine genauere Analyse dieser zwei Gruppen wahrscheinlich Sinnvoll.

\subsubsection{Bauart des Bienenstockes}

Die Ausführung des Bienenstockes (Kunststoff-Beuten, Isolierte Beuten im Winter, offener Gitterboden im Winter) hatte keinen statistisch signifikanten Einfluss auf die Winterverluste (\cref{fig:u:operational:loss}-D,E,F). Die gleichen Ergebnisse sind auch in vorangegangen Umfragen zustande gekommen \citep{crailsheim2018, oberreiter2020}.

\subsubsection{Zertifizierte Bio-Imkerei}

Ob ein Imkereibetrieb in unserer Umfrage eine \enquote{Zertifizierter Bio-Imkerei} war oder nicht hatte keinen statistischen Einfluss auf den Überwinterungserfolg. Im ersten Untersuchungsjahr 2016/17, in welchem diese Frage gestellt wurde, hatten zertifizierte Imkereien eine signifikante Wahrscheinlichkeit weniger Völker über den Winter zu verlieren \citep{crailsheim2018}. In den Folgejahren konnte aber kein Unterschied zwischen den Gruppen mehr festgestellt werden \citep{brodschneider2018a, oberreiter2020}.
\newline
Als \enquote{Zertifizierter Bio-Imkerei} verstehen wir Imkereien die zumindest laut den Minimum-Standards der EU (European Organic Regulation, EC No. 834/2007, 889/2008) für Biokontrollstellen zertifiziert sind. Die Hauptunterschiede für Bio-Imkereien ist die Verpflichtung nur bio-zertifizierten Zucker/Sirup zum Füttern einzusetzen, das Verbot des Einsatzes von synthetischen Mitteln zur Schädlingsbekämpfung (zB Varroamilbe) sowie die Verpflichtung zum Einsatz von bio-zertifiziertem Wachs \citep{thrasyvoulou2015}. Die Umwelt und Landnutzung im Einzugsgebietes der Bienenstände könnte, wie bereits im Punkt Bundesländer und Bezirke erwähnt, einen Einfluss auf die Winterverluste haben \citep{kuchling2018}. Dieser Vorgabe ist aber nur unzureichend in der Regulierung ausgeführt und kann von den Biokontrollstellen unterschiedlich interpretiert werden. Dass gleiche gilt für Kontamination von Wachs \citep{thrasyvoulou2015}.
\newline
Anhand dieser Ausführungen, und da nur ein geringer Teil an TeilnehmerInnen angegeben hat synthetische Mittel gegen die Varroamilbe einzusetzen (\cref{tab:u:behandlungsmethoden}), können wir schlussfolgern das der Hauptunterschied zwischen \enquote{Zertifizierter Bio-Imkerei} und \enquote{Imkerei ohne Zertifikat} in unserer Umfrage die Fütterung und die Qualität des Wachses betrifft. Was jedoch keinen signifikanten Unterschied auf die Winterverlustraten in unserer Untersuchung zeigte, ausgenommen das Fremdwachs, wie bereits vorher diskutiert.

\subsubsection{Vereinigung von Völkern}

Eine neue Betriebsweisen-Frage in unserer Umfrage 2019/20 war die Frage ob weiselrichtige aber schwache Völker bereits vor dem Winter vereinigt wurden. Ingesamt haben 13,7\% der TeilnehmerInnen diese Praxis durchgeführt sie führte aber zu keinen statistisch signifikanten Unterschieden in den Winterverlustraten (\cref{fig:u:operational:loss}-J).

\subsection{Wabenhygiene}

Die relative Anzahl an ausgetauschten Waben zeigt keinen signifikanten Unterschied zwischen den Gruppen. In Vergleich zu vorangegangen Untersuchungen zeigt sich hier kein klares Bild \citep{crailsheim2018, oberreiter2020}, in den meisten Jahren ist hier aber kein Unterschied in den Verlustraten festzustellen.
\newline
\cite{berry2001} konnten zeigen, dass das Alter der Waben Einfluss auf das Wachstum und die Überlebensrate eines Bienenvolkes haben kann: Völker mit neuen Waben produzierten im Durchschnitt größere Bereiche mit Brut, größere Bereiche mit eingedeckelter Brut sowie Jungbienen mit höherem Indiviudalgewicht. Die Überlebensrate der Brut war in diesen Versuchen allerdings in Völkern mit älteren Waben signifikant höher, was an der Regulation des Mikroklimas in der Zelle liegen könnte. \cite{koenig1986} zeigten in ihrer Arbeit den Einfluss von verschiedenen Brutwabentypen auf das Vorkommen von Kalkbrut in Honigbienenvölkern, wobei hier alte Brutwaben die Entwicklung von Kalkbrut im Volk förderten.
\newline
Generell können Ansammlungen von Pestizidrückständen, wie auch von Krankheitserregern im Wachs, die Überlebensrate von Bienenvölkern vermindern, diese Rückstände werden durch einschmelzen der Altwaben aber nicht entfernt \citep{calatayud-vernich2018}. Eine vollständige Rückstandsfreiheit des neuen Wabenmaterials kann in unserer Untersuchung mangels Überprüfung allerdings nicht als gegeben angenommen werden. 

\subsection{Trachtangebot}

Rapstracht im Frühjahr 2019 zeigte einen negativen Einfluss auf die Überlebenswahrscheinlichkeit im folgenden Winter (\cref{fig:u:factor:yield}-A). In vorherigen und internationalen Studien finden sich zu Raps in Zusammenhang mit den Winterverlusten widersprüchliche Ergebnisse \citep{vanderzee2014, gray2019, oberreiter2020}, da in verschiedenen Länder sowohl positive wie auch negative Effekte beobachtet worden sind. Raps (\textit{Brassica napus}) liefert Nektar und Pollen in sehr hohen Mengen, aber potentiell vorkommende Pflanzenschutzmitteln könnten einen negativen Effekt auf die Entwicklung der Honigbienenvölker haben \citep{rundlof2015, goulson2015, rolke2016}. Da die Rapstracht allgemein sehr früh in der Honigsaison stattfindet, könnte neben der Tracht noch andere Faktoren für den negativen Effekt verantwortlich sein. Als weitere Ursache für hohe Winterverluste im Zusammenhang mit Trachtpflanzen wird eine einseitige Ernährung durch mangelndes Trachtangebot, welches durch Monokulturen zustande kommt, angenommen \citep{brodschneider2013, requieretal2017}. 
\newline
Diesen Winter hatte die Trachtpflanze Mais (\textit{Zea mays}) wieder einen negativen Effekt auf die Winterverlustrate (\cref{fig:u:factor:yield}-B). Obwohl Mais keinen Nektar produziert, kann der Pollen direkt von Honigbienen gesammelt werden. Auch in Österreich zeigten Untersuchungen, dass der Pollen aktiv von Bienen gesammelt wurde \citep{brodschneider2019a}. Es ist aber keine bevorzugte Quelle für Bienen \citep{hocherl2012, urbanowicz2019}. Als Ursachen für die negativen Auswirkungen werden der Einsatz von Pflanzenschutzmitteln und eine einseitige Ernährung diskutiert \citep{brodschneider2013, vanderzee2014, dipasquale2016}. \cite{hocherl2012} konnten zeigen, dass im Pollen von Maispflanzen zwar nur sehr geringe Mengen Histidin vorhanden sind, jedoch alle anderen essentiellen Aminosäuren in größerer Menge als in gemischten Pollen vorkommen. Dennoch hatten in dieser Studie Bienen, die mit einer ausschließlichen Mais-Pollen Diät gefüttert wurden, eine auffällig niedrigere Lebenserwartung als Bienen, die mit einem Pollengemisch gefüttert wurden. Bienen könnten auch indirekt über Guttationswasser mögliche Pestizide aufnehmen, was negative Auswirkungen auf das gesamte Volk haben könnte \citep{schmolke2018}. Die Auswirkungen von Pestiziden sind aber nicht immer leicht zu untersuchen. \cite{rundlof2015} zeigen in ihrer Studie aus Schweden negative Effekte von mit Neonicotinoiden behandelten Kulturen auf Wildbienen (solitäre Bienen, Hummeln). Die Untersuchung konnte aber keinen signifikanten Einfluss auf die Überwinterung von Honigbienenvölkern nachweisen. Ein solcher Beweis wurde von \cite{woodcock2017} in einer groß angelegten Feldstudie teilweise erbracht. 
\newline
Eine Sonnenblumentracht und spätblühende Zwischenfrüchte (zum Beispiel Ölrettich, Phacelia und Senf) zeigten dieses Jahr keine negativen Auswirkungen (\cref{fig:u:factor:yield}-C,D). Die spätblühenden Zwischenfrüchte befinden sich seit 2018 im COLOSS Fragebogen. Hier zeigt sich im Überwinterungsjahr 2017/18 \citep{gray2019} ebenfalls kein signifikanter Unterschied in Österreich. Im Jahr 2018/20  hatten ImkerInnen mit spätblühende Zwischenfrüchte eine signifikante Wahrscheinlichkeit für höhere Winterverluste \citep{oberreiter2020}. 
\newline
Bei der Waldtracht gibt es dieses Jahr ein sehr interessantes Ergebnis. Hier zeigt sich bei Vorhandensein dieser Tracht eine signifikante Wahrscheinlichkeit für niedrigere Überwinterungverluste (\cref{fig:u:factor:yield}). Bei der Waldtracht mit Melezitose (sogenannter Zementhonig), welcher die Brut- und Honigwaben mit schwerverdaulichem, kristallinem Honig füllt \citep{pechhacker1990}, wurde hingegen kein statistischer Unterschied festgestellt. Der Grund hierfür könnte der Zeitpunkt der Tracht sein. Eine frühe Waldtracht und Melezitose könnte wahrscheinlich weniger Einfluss auf das Volk haben als eine Tracht spät im Bienenjahr, da eine Überwinterung auf Melezitose zur Ruhr in den Völkern führen kann \citep{imdorf1985}. 
\newline
Aus epidemiologischer Sicht können wir also einen signifikanten Zusammenhang mit bestimmten Trachtquellen festhalten, wobei diese Trachtquellen auch stellvertretend für andere (nicht abgefragte oder bisher unbekannte) abträgliche Standortverhältnisse stehen können. Es können keine kausalen Gründe für die erhöhten Verluste beim Vorhandensein bestimmter Trachten genannt werden. Diese Trachten können aber als Indikator von für Bienenvölker nicht idealen Standorten gesehen werden, wie auch in der Untersuchung zum Einfluss der Landnutzung auf die Wintersterblichkeit des Vorläuferprojekts diskutiert wird \citep{kuchling2018}. Die Auswirkungen und Erforschung von Pflanzenschutzmitteln auf Bienenvölkern ist noch immer ein sehr aktuelles Thema und Bienenvölker sind durch ihre Sammelaktivität ein wichtiger Bioindikator für die Umwelt was durch neue Methoden immer besser erforscht werden kann \citep{murcia-morales2020}.

\subsection{Bekämpfung der Varroamilbe}

Im Kampf gegen hohe Winterverluste spielt die Behandlung der Völker gegen die Varroamilbe eine wichtige, wenn nicht sogar die zentrale Rolle. Der Parasit hat einen großen Einfluss auf den Überwinterungserfolg von Bienenvölkern \citep{dahle2010}, insbesondere auch dadurch, dass die Milbe als Vektor für andere Pathogene, wie Viren, dient \citep{rosenkranz2010, noel2020}. Deshalb waren nicht nur die Erhebungen über die verschiedenen Behandlungsmethoden, sondern auch über die Häufigkeit und den Zeitpunkt der Behandlungen wichtiger Bestandteil unserer Untersuchungen.

\subsubsection{Bestimmung des Varroabefalls}

Der Hauptteil der teilnehmenden ImkerInnen (91\%) hat eine Varroabefalls\-kontrolle durchgeführt und am Häufigsten war die Kontrolle in den Monaten Juli, August und September \cref{fig:u:treatment:varroa:overview}. Gruppiert nach Jahreszeit hat der Großteil im Sommer und Winter kontrolliert \cref{fig:u:treatment:varroa:combination}. Dieses Jahr konnte kein signifikanter Unterschied in den Winterverlustraten festgestellt werden zwischen ImkerInnen die eine Bestimmung durchgeführt haben und denen, die keine durchführten (\cref{fig:u:treatment:varroa:checked}). Auch die anspruchsvollere Auswertung auf Zeit und Dauer zeigt keine Unterschiede (\cref{fig:u:treatment:varroa:grouped,fig:u:treatment:varroa:combination}). Im Vergleich zum Winter 2018/19 zeigten TeilnehmerInnen die eine Bestimmung durchgeführt haben, eine signifikante Wahrscheinlichkeit weniger Völker über den Winter zu verlieren \citep{oberreiter2020}. Obwohl die Kontrolle alleine die Varroamilbe nicht vermindert, lässt sich dieser Effekt möglicherweise dadurch Erklären, dass ImkerInnen die über den Varroabefall ihrer Völker Bescheid wissen einerseits besser über den Gesundheitszustand ihrer Völker informiert sind, aber auch dementsprechend besser auf den Befall reagieren können. Deswegen sollte diese Praxis, welche bereits von der großen Mehrheit in unserer Umfrage praktiziert wird, weiterhin beworben werden.

\subsubsection{Behandlungsmethoden}

In Österreich sind im internationalen Vergleich verhältnismäßig nur wenige Akarizide zugelassen \citep{brodschneider2019}. In unserer Umfrage folgt der Großteil der Empfehlung der AGES \citep{moosbeckhofer2015} (\cref{tab:u:behandlungsmethoden}). Die Empfehlung ist eine Verdunstung von Ameisensäure im Sommer und eine Nachbehandlung im Winter bei Brutfreiheit mit Oxalsäure. Aber auch die Drohnenbrutentnahme als biotechnische Maßnahme ist noch immer eine weitverbreitete Methode. Um die Effektivität und Unterschiede zwischen den Behandlungsmethoden besser zu veranschaulichen wurden die Behandlungsmethoden zuerst zum Zeitpunkt ihrer Anwendung in die Gruppen \enquote{Frühling, Sommer, Winter} eingeteilt (\cref{fig:u:treatment:histogramm}).
\\
Die einzelnen Methoden und auch in Kombination werden in weiterer Folge näher betrachtet. 
Auf synthetischen Behandlungsmethoden und den Einsatz von Milchsäure wird nicht näher eingegangen, weil die Stichproben in diesen Gruppen für eine sinnvolle, statistische Auswertung zu klein waren.

\subsubsubsection{Biotechnische Maßnahmen}

Die Hyperthermie beruht auf dem Prinzip der unterschiedlichen Temperaturtoleranzen von \textit{V. destructor} und der Honigbienenbrut. Verdeckelte Brutwaben werden aus dem Bienenvolk entnommen und einer Wärmebehandlung unterzogen. Es konnte in diesem Untersuchungsjahr und auch in den letzten kein statistisch signifikanter Unterschied festgestellt werden wenn diese Methode verwendet wurde \citep{crailsheim2018, oberreiter2020} (\cref{fig:u:treatment:spring,fig:u:treatment:summer}-B). Hierbei sei aber auf die relativ geringe Anzahl der TeilnehmerInnen, welche die Methode verwendet haben, hingewiesen.
\newline
Drohnenbrutentnahme war die im Frühjahr am häufigsten eingesetzte Methode (\cref{fig:u:treatment:spring}-A). Hier zeigt sich für TeilnehmerInnen die diese Methode angewandt haben eine signifikante Wahrscheinlichkeit weniger Völker über den Winter zu verlieren. Wenn man nur die Drohnenbrutentnahme und alle möglichen Kombinationen betrachtet, erkennt man einen Trend für weniger Verluste wenn diese Methode im Sommer und Frühling eingesetzt wurde im Gegensatz zu nur im Sommer. Dies zeigt aber keinen statistischen Unterschied (\cref{fig:u:treatment:drone:combination}). Die Häufigkeit der Anwendung hatte auch keinen statistischen Einfluss auf die Überlebenswahrscheinlichkeit der Völker über den Winter (\cref{fig:u:treatment:drone:grouped}). Vorangegangene Untersuchungen haben gezeigt, dass sich eine Drohnenbrutentnahme im Frühjahr positiv auf den Überwinterungserfolg auswirkt \citep{brodschneider2013, oberreiter2020, crailsheim2018}. Feldexperimente in der USA, mehrmaliges entfernen der Drohnenbrut, und Schweiz, einmaliges entfernen im Frühjahr, haben eine Verminderung der Varroabelastung gezeigt ohne Einfluss auf Volksstärke und Honigertrag \citep{charriere1998, calderone2005}. Da die Drohnenbrutentnahme keine alleinige Behandlungsmethode darstellt, sei an dieser Stelle auf die oftmalige Verwendung mehrerer unterschiedlicher Bekämpfungsmaßnahmen in Kombination hingewiesen. Dieser Effekt wird durch den oben angeführten signifikanten positiven Effekt im Frühjahr sichtbar. Hierbei handelt es sich wahrscheinlich um TeilnehmerInnen die eine Drohnenbrutentnahme im Frühjahr mit einer weiteren erfolgreichen Strategie verbunden haben. 
\newline
Wurde eine \enquote{andere biotechnische Maßnahme} (exklusive Hypthermie und Drohnenbrutentnahme) im Sommer durchgeführt, zeigte sich eine signifikante Wahrscheinlichkeit weniger Winterverluste zu erleiden (\cref{fig:u:treatment:summer}-C). Solche biotechnische Maßnahmen zeigten bereits in den Wintern mit hohen Verlusten positive Effekte \citep{crailsheim2018}. Auch im letzten Jahr 2018/19 konnte ein signifikanter Unterschied festgestellt werden \citep{oberreiter2020}. Hierbei handelt es sich aber oft um sehr arbeitsaufwendige Methoden \citep{rosenkranz2010}. Eine Untersuchung aus Italien (Reggio Emilia, Po-Ebene) zeigt, dass solche Methoden auch im Frühjahr zu einer Verminderung der Varroabelastung führt ohne negativen Einfluss auf die Honigernte zu haben \citep{lodesani2019}. Dies ist besonders in Hinsicht auf die immer häufiger werdenden warmen Winter und somit verlängerte Brutperioden interessant und könnte auch in Österreich zu positiven Ergebnissen führen. Dies könnte auch für ImkerInnen die späte Trachten anstreben, welche sie ansonsten wegen zu starker Varroabelastung nicht ausnützen könnten, interessant sein.

\subsubsubsection{Ameisensäure}
\label{sss:AS:U}

Ameisensäure ist die meistverbreitete Behandlungsmethode in unserer Umfrage. Die Vorteile beim Einsatz der Ameisensäure (und anderer organischer Säuren) ist das geringe Risiko von Resistenzbildung, das geringere Riskio der Belastung mit Rückständen und eine bewiesene Effektivität gegen die Varroamilbe \citep{rosenkranz2010, noel2020}.
\newline
Dieses Jahr hatten Teilnehmer die im Sommer eine Langzeitbehandlung mit Ameisensäure durchgeführt haben, eine signifikante Wahrscheinlichkeit weniger Winterverluste zu erleiden (\cref{fig:u:treatment:summer}-E). Bei der Kurzzeit Behandlung konnte kein Unterschied festgestellt werden. Betrachtet man TeilnehmerInnen die nur eine Kurzzeitbehandlung im Sommer durchgeführt haben, erkennt man eine signifikant höhere Verlustrate als der österreichische Durchschnitt (\cref{fig:u:combination}-Q), hier sei aber der durch die geringe Stichprobenanzahl und möglichen großen Schwankungen der einzeln Verlustraten verursachte, breite Konfidenzintervall zu beachten. Diverse Kombinationen und die alleinige Anwendung von Ameisensäure in Kurzzeitbehandlung zeigten auch in der Untersuchung von 2018/19 potentiell höhere Verlustraten \citep{oberreiter2020}. Ein direkter Vergleich der Verlustraten zwischen Langzeitbehandlung und Kurzzeitbehandlung bei alleiniger Anwendung zeigt keinen signifikanten Unterschied (\cref{fig:u:combination}-Q,I).

\subsubsubsection{Oxalsäure}

Unsere Untersuchung hat gezeigt, dass ein Großteil der teilnehmenden ImkerInnen eine Form der Oxalsäurebehandlung im Winter zur sogenannten Restentmilbung eingesetzt haben (\cref{fig:u:treatment:winter}-B,C). Dabei kann die Oxalsäure verdampft, geträufelt, gesprüht oder als Fertigmischung (Hive-clean/Bienenwohl/Varromed) eingesetzt werden. Zwischen den Methoden geträufelt, gesprüht oder Fertigmischung hat es keinen signifikanten Unterschied gegeben und wurde deswegen in der Auswertung als \enquote{Oxalsäure - Träufeln} zusammengefasst (\cref{fig:u:treatment:oxalmix}).
\newline
Als Winterbehandlung zeigt sich für die TeilnehmerInnen, welche Oxalsäure sublimiert haben, eine signifikante Wahrscheinlichkeit weniger Winterverluste zu erfahren (\cref{fig:u:treatment:winter}-B). Beim direkten Vergleich der Verlustraten zwischen Träufeln und Sublimieren zeigt sich aber kein statistischer Unterschied (\cref{fig:u:treatment:winter}-B,C).
\newline
Durch die geringe Auswahl an Behandlungsmethoden im Winter gibt es keine verlässliche Kontrollgruppe von Imkereien, die nicht mit Oxalsäure arbeiten. Das unterstreicht die Bedeutung der Oxalsäure und die derzeitige Alternativlosigkeit in der Winterbehandlung. Oxalsäure wirkt ausschließlich auf die auf Bienen sitzenden Varroamilben, und ist damit nur bei Brutfreiheit der Völker effektiv \citep{rosenkranz2010}. Über das tatsächliche Vorherrschen von Brutfreiheit bei Anwendung der Oxalsäure kann im Rahmen dieser Untersuchung keine Aussage getroffen werden. Zusätzlich zielt eine Behandlung mit Oxalsäure auf eine Reduktion der Milbenpopulation für das Folgejahr ab. Deswegen ist eine direkte Auswirkung auf den Winterverluste im Umfragejahr nur bedingt zu erklären, da wir auch nicht wissen ob Teilnehmer die zum Beispiel Verdampfen diese Methode auch im letzten Jahr eingesetzt haben.

\subsubsubsection{Thymol}

TeilnehmerInnen die im Sommer Thymol angewandt haben, zeigten eine signifikante Wahrscheinlichkeit für höhere Winterverluste (\cref{fig:u:treatment:summer}-I). Hier zeigt eine nähere Betrachtung der Frequenz der Anwendung, dass ImkerInnen die nur in einem Monat eine Thymolbehandlung durchgeführt haben signifikant höhere Wahrscheinlichkeit für Verluste hatten als die Gruppe ohne Thymolbehandlung. Bei ImkerInnen die in mehreren Monaten Thymol eingesetzt haben ist kein statistisch signifikanter Unterschied mehr zu erkennen (\cref{fig:u:treatment:thymol:grouped}). Es sei hier angemerkt, dass es sich nicht zwangsweise um eine alleinige Behandlung mit Thymol handelt, sondern diese wahrscheinlich mit anderen Behandlungsmethoden kombiniert wurde. Die Gruppe an TeilnehmerInnen, welche ausschließlich Thymol als Behandlungsstrategie eingesetzt hat ist zu klein um eine statische sinnvolle Analyse durchzuführen.
\newline
Eine australische Studie die sich mit der Thymolbehandlung näher beschäftigt hat  zeigt einen vorteilhaften Effekt auf das Hygiene Verhalten der Bienen, wobei die Resultate sehr unterschiedlich waren  \citep{colin2019}. Vermutlich spielen für den erfolgreichen Einsatz von Thymol zusätzlich noch andere Faktoren eine Rolle, wie zum Beispiel verschiedene Umweltbedingungen und genetische Unterschiede.

\subsubsubsection{Kombinationen}

Um einen besseren Einblick in die angewandten Behandlungsmethoden in Kombination zu geben, haben wir zusätzlich verschiedene Behandlungskonzepte in Kombination direkt verglichen (\cref{fig:u:combination}). Durch die Aufteilung ergeben sich sehr viele unterschiedliche Gruppen wobei hier nur Kombinationen mit mindestens 15 TeilnehmerInnen analysiert wurden. Durch diese geringe Stichprobenanzahl und unterschiedliche Verlustergebnisse ergibt sich ein hoher Konfidenzintervall. Um die Stichprobenanzahl weiter zu erhöhen wurde die Drohnenbrutentnahme in dieser Auswertung nicht berücksichtigt, dh. TeilnehmerInnen in der Kombination könnten noch zusätzlich eine Drohnenbrutentahme gemacht haben.
\newline
Die mit Abstand am meisten genutzte Kombination in unserer Umfrage war eine Ameisensäure Langzeitbehandlung im Sommer gefolgt von Oxalsäure Sublimation oder Träufeln im Winter (\cref{fig:u:combination}-A,B). Ein statistischer Unterschied zwischen den zwei Kombination konnte nicht festgestellt werden. Die selben Ergebnisse wurden auch letztes Jahr beobachtet \citep{oberreiter2020}.
\newline
Kombinationen mit biotechnischen Methoden (exklusive Hyperthermie und Drohnenbrutentnahme) und Oxalsäurebehandlung im Sommer, gefolgt von einer Oxalsäurebehandlung im Winter zeigte Potenzial für geringere Winterverluste (\cref{fig:u:combination}-D,O,P). Durch die geringe Anzahl an Strichproben und die verschiedenen Möglichkeiten der biotechnischen Methoden ist eine genauere Analyse hier nicht möglich. 
\newline
Eine Anwendung einer Ameisensäure-Kurzzeitbehandlung im Sommer führte zu einer Wahrscheinlichkeit für höhere Winterverluste im Vergleich zum Österreichischen Durchschnitt (\cref{fig:u:combination}-Q). Dieser negative Effekt wurde im Abschnitt \ref{sss:AS:U} \nameref{sss:AS:U} diskutiert.

\subsection{Königinnen-Verluste}

Wie auch in den letzten Jahren ist die Verlustrate an Bienenvölkern über den Winter durch \enquote{unlösbaren Königinnenprobleme} unter 5\% (\cref{fig:u:queen:states}) und scheint stabil sowohl in Österreich als auch International zwischen ca. 3-5\% zu liegen \citep{brodschneider2019, gray2019, oberreiter2020}.

\subsubsection{Königinnenprobleme}

Insgesamt gaben 10,4\% der TeilnehmerInnen an, dass sie \enquote{Häufiger} Königinnenprobleme im Vorjahr beobachten konnten. Diese Gruppe hatte, im Vergleich zur Gruppe „Normal`` und „Seltener``, eine signifikant höhere Wahrscheinlichkeit Bienenvölker durch unlösbare Königinnenprobleme sowie auch ohne Königinnenprobleme die Völker über den Winter zu verlieren (\cref{fig:u:queen:problems}). Dieser Faktor hatte auch in der Umfrage 2018/19 einen negativen Einfluss \citep{oberreiter2020} und auch internationale Untersuchungen kamen zum selben Schluss \citep{vanderzee2014}. Königinnenprobleme sind auch in den USA ein signifikanter negativer Faktor für das Überleben der Bienenvölker \citep{vanengelsdorp2013}.

\subsubsection{Im Einwinterungsjahr begattete Königin (\enquote{junge Königin})}

Zudem stellten mehrere Studien fest, dass junge Königinnen, genauer im Vorjahr begattete Königinnen, einen positiven Einfluss auf die Überlebenswahrscheinlichkeit der Völker haben \citep{vanderzee2014, genersch2010, giacobino2016, morawetz2019}. Diesen Effekt können wir dieses Jahr wieder bestätigen. Es zeigt sich eine signifikante Wahrscheinlichkeit für geringere Verlustraten (exklusive Verluste durch Königinnenprobleme) wenn der Imkerbetrieb mehr als 25\% der alten Königinnen vor dem Winter ausgewechselt hat. Interessanterweise hatte es aber keinen statistischen Einfluss auf Verlustraten durch Königinnenprobleme (\cref{fig:u:queen:exchangerate}). Der positive Effekt auf die Überlebenswahrscheinlichkeit der Bienenvölker über den Winter durch junge Königinnen konnte in Österreich auch im Vorjahr beobachtet werden \citep{oberreiter2020}. 
\newline
Die Ergebnisse zeigen die Wichtigkeit einer gesunden und jungen Königin für ein erfolgreiches Überwintern des Bienenvolkes. Als Grund für die postiven Auswirkungen wird unter anderem die größere Menge an Brut angenommen. Genauere Ursachen für den gesteigerten Überwinterungserfolg konnten jedoch noch nicht gefunden werden \citep{genersch2010, amiri2017, ricigliano2018}. Eine weitere Ursache für einen verminderten Überwinterungserfolg von Völkern mit einer älteren Königin könnte in der gesundheitlichen Beeinträchtigung dieser Königin aufgrund von Neonicotinoiden \citep{williams2015} oder Viren \citep{amiri2020a} liegen. Das die Qualität von Königinnen wichtig ist und die Temperatur auch eine Rolle spielt zeigte sich aus Untersuchungen in denen höhere Temperaturen, zum Beispiel beim Versand von Königinnen, zu Problemen führten \citep{withrow2019, rousseau2020}.

\subsection{Verkrüppelte Flügel}

Bienenviren stehen im Fokus des Projektes \enquote{Zukunft Biene 2}. Wir haben deshalb nach einem Merkmal gefragt,
das mit dem Flügeldeformationsvirus (DWV = deformed wing virus) in Zusammenhang steht (siehe Modul A). Hierfür wurde das Auftreten von Arbeitsbienen mit verkrüppelten Flügeln abgefragt, dieses Symptom kann durch Kälteschaden aber auch durch die angeführte Virenschädigung entstehen. Folgende Kategorien standen zur Auswahl: „Häufig``, „Wenig``, „Überhaupt nicht`` und „Weiß nicht``. 
\newline
Insgesamt haben 29 Imkereien angegeben, solche verkrüppelten Flügel während der Bienensaison 2019 „Häufig`` beobachtet zu haben. Diese Betriebe hatten auch eine signifikant höhere Verlustrate an Bienenvölkern als  die Gruppen \enquote{Wenig} und \enquote{Weiß nicht} \cref{fig:u:queen:crippledbees}. Die gleichen Ergebnisse konnten auch in der Untersuchung 2018/19 festgestellt werden \citep{oberreiter2020}. Daraus kann man Schlussfolgern, dass ein vermehrtes Vorkommen von Bienen mit verkrüppelten Flügeln als Alarmsignal für hohe Winterverluste betrachtet werden kann \citep{morawetz2019}.

\subsection{Zusammenfassung}

Zusammenfassend zeigt sich, dass hohe Winterverluste von Bienenvölkern, wie sie 2014/15 aufgetreten sind, zumindest in einem bestimmten Ausmaß durch eine zeitgemäße Bekämpfung der Varroamilbe, wie zum Beispiel mit Hilfe biotechnischer Methoden, reduziert werden können.
\newline
Im Winter 2019/20 sind die Winterverluste im Vergleich zu den letzten Jahren eher als gering zu betrachten, obwohl es signifikante Unterschiede zwischen Bundesländern gibt. Die höchsten Winterverluste wurden beispielsweise in Wien verzeichnet. Bei den Betriebsweisen zeigten sich bereits bekannte Effekte vom Vorjahr, wie geringere Verluste für Wanderimker und der negative Effekt bei Verwendung von Fremdwachs. Es zeigte sich auch wieder das größere Betriebe eine geringere Wahrscheinlichkeit für Winterverluste haben.
\newline
Königinnenprobleme und das Alter der Königin hatte auch wieder einen negativen bzw. positiven Einfluss auf die Winterverluste. Des Weiteren konnten allerdings epidemiologisch auch andere Gründe für die hohen Verluste identifiziert werden, die jedoch außerhalb des direkten Einflussbereichs der Imkereien liegen. So konnten wir beispielsweise signifikante Einflüsse der Seehöhe des Bienenstands und der von Bienen laut ImkerInnen genutzten Trachtpflanzen nachweisen.
\newline
Weiterhin nicht vollständig geklärt bleibt die Frage, warum in manchen Jahren sehr viele und in anderen Jahren sehr wenige Bienenvölker den Winter nicht überleben. In Ansätzen scheinen betriebsweisenunabhängige Faktoren wie das Wetter \citep{switanek2017} oder, eingeschränkt, weil weniger stark schwankend, die Landnutzung \citep{kuchling2018} für diesen Effekt verantwortlich zu sein.
\newline
Die Belastung der Bienenvölker variiert von Jahr zu Jahr, wobei es in manchen Jahren schwieriger ist, die Völker erfolgreich zu überwintern als in anderen. Gerade aus Wintern mit hohen Verlustraten können wir wissenschaftlich belegte Empfehlungen über Anpassungen der Betriebsweisen (inklusive Varroa-Bekämpfungsstrategien) ablesen, wohingegen uns ein Winter mit geringen Verlusten wenig Gelegenheit bietet, die Spreu vom Weizen der Betriebsweisen zu trennen. Ein weiteres Argument bei der Interpretation stark schwankender Verlustraten ist der mögliche Einfluss von betrieblich bis überregional auftretenden Mehrjahresdynamiken, der noch weiterer Untersuchungen bedarf.
\newline
Unsere Studie zeigt wie divers die einzelnen Faktoren, welche die Völkerverluste beeinflussen, sind. Es zeigt sich auch wie schwierig es ist nur anhand eines einzelnen Faktors einen möglichen Rückschluss zu ziehen. Deswegen ist es essenzielle diese Studien jährlich zu wiederholen um Trends zu erkennen und diese auch mit Zahlen zu belegen. Damit möchten wir ImkerInnen unterstützten gewisse Entscheidungen auf der Grundlage von erhobenen Citizen Science Daten zu treffen und aus langjährigen Trends zu lernen. Ein immer deutlicherer Trend ist, dass mutmaßliche \enquote{professionellere Imkereien} eine statistisch signifikante Wahrscheinlichkeit haben weniger Völker über den Winter zu verlieren.