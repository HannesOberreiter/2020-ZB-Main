\begin{otherlanguage}{english}
In module U, risk factors for winter losses of bee colonies are monitored and examined. Thus far, this study has been carried out in Austria since the winter of 2007/08. The loss rates of overwintering colonies ranges from 8.1\% (95\%~CI:~7.4-8.8\%) (winter 2015/16) to 28.4\% (95\%~CI:~27.0-29.9\%) (winter 2014/15). 
\newline
The 1,539 answers concerning 30,724 wintered bee colonies of the 2020 study were checked for their representativeness and analyses were carried out on the geographical distribution of the losses, on the symptoms accompanying the winter losses, and on the beekeeping practices. An key factor was the analysis of treatment methods used to combat \textit{Varroa destructor} and the influence of these methods on winter mortality. 
\newline
The winter loss rate for the whole of Austria in 2019/20 was 12.6\% (95\%~CI:~11.9-13.3\%). The loss rate is close to the average of the past years, with 16,1\%. In comparison between the states, the loss rate for Vienna is noticeably higher than the rest of Austria with 20.1\% (95\%~CI:~16.0-24.8\%).
\newline
As in previous years factors which could allude for professionalism, experience of the beekeeping operation, etc. did show lower loss rates. These include factors like the size of the beekeeping operation (bigger smaller losses) as well as migratory beekeepers. Participants who did not purchase wax from outside their operation experienced also lower losses, which could also point to professionalism to a quality problem of the available wax.
\newline
Interestingly, this year participants which did use \enquote{Queens bred from Varroa tolerant/resistant stock} showed lower loss rates, but only compared to the group which did answer with \enquote{Unsure}. Additionally, the use of a \enquote{Small brood cell size (5.1 mm or less)} showed statistically lower loss rates.
\newline
The occurrence of maize or rapeseed forage did come with a statistically higher loss rate. Participants with honeydew harvest showed lower loss rates as the ones without. This year, in contrast to last year, no higher loss rate was associated with the occurrence of melezitose.
\newline
As for methods to combat the varroa mite, biotechnical methods applied in summer (total brood removal, queen confinement, etc.) demonstrated again the possibility for lower loss rates. The usage of thymol in summer caused a negative effect on colony survival over the winter. Exclusive use of formic acid - short term treatment in summer, without a spring or winter treatment displayed a negative potential for high winter losses.
\newline
More queen problems over the season in comparison to last year caused a negative effect on the winter loss rate, as in earlier analyses. Active exchange of \enquote{old queens} did improve the chance of survival of the colonies.
\newline
As a bridge to other modules of this project, we want to stress that the observation of bees with crippled wings during the bee’s active season, which poses a possible sign for viral diseases, results in significantly higher winter losses. 
\end{otherlanguage}