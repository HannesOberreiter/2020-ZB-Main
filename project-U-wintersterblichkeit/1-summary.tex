\begin{otherlanguage}{english}
In module U, risk factors for winter losses of bee colonies are monitored and examined. Thus far, this study has been carried out in Austria since the winter of 2007/08. The loss rates of overwintering colonies ranges from 8.1\% (95\%~CI:~7.4-8.8\%) (winter 2015/16) to 28.4\% (95\%~CI:~27.0-29.9\%) (winter 2014/15). 
\newline
The 1,539 answers concerning 30,724 wintered bee colonies of the 2020 study were checked for their representativeness and analyses were carried out on the geographical distribution of the losses, on the symptoms accompanying the winter losses, and on the beekeeping practices. A key factor was the analysis of treatment methods used to combat \textit{Varroa destructor} and the influence of these methods on winter mortality. 
\newline
The winter loss rate for the whole of Austria in 2019/20 was 12.6\% (95\%~CI:~11.9-13.3\%). The loss rate is close to the average of the past years, which is 16,1\%. A comparison between the provinces revealed that the loss rate for Vienna 20.1\% (95\%~CI:~16.0-24.8\%) is noticeably higher than for the rest of Austria. The most common reported symptom was \enquote{no or few dead bees in or in front of the colony} with a relative frequency of 39.9\%.
\newline
As in previous years factors which could allude for professionalism, experience of the beekeeping operation, etc. were associated with lower loss rates. These include factors like the size of the beekeeping operation (bigger have smaller losses) as well as migratory beekeeping. Participants who did not purchase wax from outside their operation experienced also lower losses, which could point to professionalism or a quality problem of the wax on the common market.
\newline
Interestingly, this year participants which did use \enquote{Queens bred from Varroa tolerant/resistant stock} showed lower loss rates, but only compared to the group which did answer with \enquote{Unsure}. Additionally, the use of a \enquote{Small brood cell size (5.1 mm or less)} led to statistically lower loss rates.
\newline
The report of beekeepers that bees foraged on maize or rapeseed was associated with a statistically higher loss rate. Participants with honeydew harvest showed lower loss rates compared to the ones without. This year, in contrast to last year, no higher loss rate was associated with the occurrence of melezitose.
\newline
As for methods to combat the varroa mite, biotechnical methods applied in summer (total brood removal, queen confinement, etc.) demonstrated again the possibility for lower loss rates. The usage of thymol in summer caused a negative effect on colony survival over the winter. Exclusive use of formic acid - short term treatment in summer, without a spring or winter treatment displayed a potential for high winter losses.
\newline
More observed queen problems over the season in comparison to last year caused a negative effect on the winter loss rate, as in earlier analyses. Active exchange of \enquote{old queens} did improve the chance of survival of the colonies.
\newline
As a link to the other two modules of this project, we want to stress that the observation of bees with crippled wings during the bees' active season, which is a possible sign for viral diseases, results in significantly higher winter losses. 
\end{otherlanguage}