\section{Einleitung}

In den letzten Jahren hat das Thema Bienensterben weltweit immer mehr an Bedeutung gewonnen, nicht zuletzt dadurch, dass auch die Zahl der natürlichen Bestäuber wie Wildbienen, Hummeln, Schmetterlinge und Schwebfliegen deutlich sichtbar und auch in Zahlen abnimmt \citep{hallmann2017}. Die Bedeutung der Honigbiene --- ökologisch und ökonomisch --- ist entscheidend und unumstritten. Sie bestäubt beim Sammeln von Pollen und Nektar einerseits viele Wildpflanzen, wodurch sie erheblich zur Erhaltung der Artenvielfalt beiträgt, und andererseits auch viele vom Menschen genutzte Pflanzen. Aufgrund der Stärke ihrer Völker, der Zucht- und Transportmöglichkeit an den Ort des gewünschten Bestäubungseinsatzes kann sie gezielt als Bestäuber von Kulturpflanzen eingesetzt werden. Ihre Bestäubungsleistung ist enorm, sie wird jährlich auf einen finanziellen Wert von 153 Milliarden Euro weltweit bzw. 14,2 Milliarden Euro in der EU geschätzt \citep{gallai2009}. Damit ist die Biene nach dem Rind und dem Schwein das drittwichtigste Tier für die Ernährung des Menschen \citep{kearns1998}.
\newline
Die größten Gefahren für Bienen, Hummeln, Schmetterlinge und andere Insekten sind von Menschen verursacht. Es wird vermutet, dass der Rückgang der natürlichen Bestäuber mit dem Verlust von Habitaten, dem Einsatz von Pestiziden, der Ausbreitung von Parasiten und Pathogenen sowie mit Umweltverschmutzung und dem Klimawandel zusammenhängt \citep{biesmeijer2006, cameron2011, vanengelsdorp2011, cornman2012, goulson2013, steinhauer2014, steinhauer2018, vanderzee2014, clermont2015, clermont2015, woodcock2016, belsky2019}. Für die ökonomisch wie auch ökologisch so wichtige Honigbiene \textit{Apis mellifera} konnte auch der Einfluss des Menschen durch die Betriebsweise (Krankheitsprophylaxe und -bekämpfung) auf Völkerverluste nachgewiesen werden \citep{jacques2017}.
\newline
Generell ist das Bienensterben kein neues Phänomen, im Gegenteil, schon seit Beginn der Bienenhaltung vor etwa 7.000 Jahren in Mesopotamien gibt es Überlieferungen von krankheitsbedingten Völkerverlusten \citep{flugel2015}. Bereits zu dieser Zeit gab es ausführlichere Schriften, aus Griechenland und dem Römischen Reich stammend, über die Haltung der Biene und Vorgehensweisen bei Erkrankungen. Im deutschsprachigen Raum sind bis ins 17. Jahrhundert keine Aufzeichnungen über Bienenverluste bekannt. Erst der Wunsch von Teilen der Bevölkerung, die Bienenhaltung zu verbessern, sorgte dafür, dass Bienenkrankheiten, wie etwa die Faulbrut oder Ruhr, dokumentiert wurden. Völkerverluste, die nicht auf Pathogene zurückzuführen sind, wurden während der industriellen Revolution durch die Optimierung der Beuten sowie Züchtung der Honigbiene weitestgehend reduziert. Dies, sowie der Fortschritt der Bakteriologie zu Beginn des 20. Jahrhunderts und etwas später die Möglichkeit Viren nachzuweisen, ermöglichte erstmals die Erforschung von Krankheitserregern, welche für hohe Bienenverluste verantwortlich sind. Im Laufe des 20. Jahrhunderts wurde eine Vielzahl von Massensterben von Bienenvölkern verzeichnet, bei welchen die genaue Ursache nicht bekannt war. Als erstes großes Massensterben wurde 1906 jenes auf der englischen Insel Wight (\enquote{Isle of Wight Disease}) dokumentiert \citep{neumann2009, flugel2015}.
\newline
Im letzten Jahrzehnt hatten besonders die USA immer wieder mit extremen Winterverlusten von Bienenvölkern zu kämpfen. Mehrere Jahre in Folge haben dort etwa 30\% der eingewinterten Völker nicht überlebt \citep{lee2015, steinhauer2014, vanengelsdorp2008, vanengelsdorp2007, vanengelsdorp2010, vanengelsdorp2011}. Auch Brasilien verzeichnete in den Jahren 2013-2017 sehr hohe Verlustraten, wobei hier Pestizide als Hauptursache für die Verluste vermutet werden \citep{castilhos2019}. Dennoch häufen sich die Fälle, in denen keine eindeutigen Gründe für das Massensterben festgestellt werden können. Es wird angenommen, dass die Ursachen für das Bienensterben komplex sind und dass der Auslöser für dieses Phänomen eine Kombination von mehreren Faktoren ist \citep{moritz2010, brodschneider2013, steinhauer2014, belsky2019}. In den USA spricht man von \enquote{colony collapse disorder}, kurz CCD \citep{vanengelsdorp2009, williams2010}. Im deutschsprachigen Raum wird CCD auch als \enquote{Bienen-Verschwindekrankheit} bezeichnet \citep{flugel2015}. Die Symptome sind: wenige adulte Bienen in den Völkern, verdeckelte Brut kann aber vorhanden sein, es finden sich aber keine toten Bienen in und um die Völker \citep{vanengelsdorp2009}.
\newline
Seit einigen Jahren werden auch in Europa immer wieder hohe Winterverluste von Bienenvölkern verzeichnet \citep{chauzat2016}. In Österreich ist ein Massensterben von Honigbienenvölkern im Winter mit derart hohen Verlusten von bis zu dreißig Prozent über mehrere Jahre in Folge --- wie in den USA --- bislang, mit Ausnahme der Winter 2011/12 und 2014/15, nicht aufgetreten. Im internationalen Vergleich waren die Verluste in Österreich in den vergangenen Jahren, mit Ausnahme der Winter 2011/12, 2014/15 und 2016/17, gering bis durchschnittlich \citep{vanderzee2012, vanderzee2014, brodschneider2016, brodschneider2018, brodschneider2019}. In den ersten vier Jahren der vom Zoologischen Institut (jetzt: Institut für Biologie) der Karl-Franzens-Universität Graz durchgeführten Untersuchungen lagen sie zwischen 9,3\% und 16\%. Verluste dieser Größenordnung können durch Nachzucht im Sommer kompensiert werden \citep{brodschneider2019}. Im Winter 2011/12 verloren die österreichischen ImkerInnen jedoch 25,9\% ihrer eingewinterten Völker \citep{brodschneider2013} und im Winter 2014/15 sogar 28,4\% \citep{crailsheim2018}. Als Ursachen für die hohe Wintersterblichkeit werden Parasiten und Pathogene, allen voran die Milbe \textit{Varroa destructor}, durch den Menschen ausgebrachte Pestizide, mangelhafte Ernährung durch ein einseitiges oder zeitlich verkürztes Trachtangebot, sowie die unzureichende Betreuung durch den Menschen, aber auch sozioökonomische Faktoren vermutet \citep{genersch2010, budge2015, budge2015a, goulson2015, lee2015, moritz2016, tsvetkov2017, woodcock2017, jacques2017}. Weitere Gründe können außerdem Verluste von Königinnen, Probleme mit Königinnen --- etwa aufgrund von Schädigungen durch Neonicotinoide \citep{williams2015, dussaubat2016, wu-smart2016,  siefert2020} --- oder schlichtweg das Verhungern von Völkern während des Winters sein. Auch eine Rolle spielen die Betriebsgröße der Imkerei, sprich die Anzahl der vorhandenen Völker und die damit einhergehende Professionalität im Bienenmanagement, sowie das Wandern mit Bienenstöcken zu verschiedenen Trachtquellen \citep{vanderzee2012,vanderzee2014,steinhauer2014,lee2015,gray2019,oberreiter2020}. Eine einfache, alleinige Ursache für die Winterverluste ist sehr selten auszumachen. Vielmehr sind die Ursachen auch in Österreich komplex und vielfältig und somit in ihrer Bedeutung schwer einzuschätzen \citep{brodschneider2010, moritz2010, potts2010, brodschneider2013, staveley2014, doke2015, goulson2015, oberreiter2020}.
\newline
Ein Parasit, welcher seit den 1980er Jahren für Völkerverluste in Österreich und Deutschland verantwortlich gemacht wird ist die aus Asien eingeschleppte, ektoparasitische Milbe \textit{Varroa destructor} \citep{rosenkranz2010, genersch2010, morawetz2019}. Die Analyse der von den ImkerInnen durchgeführten Behandlungen gegen diesen Risikofaktor ist daher von großer Bedeutung um Verluste zu reduzieren. Ursprünglich war nur die östliche Honigbiene \textit{Apis cerana} von diesem Parasiten befallen. Heute ist die Varroamilbe beinahe weltweit --- mit Ausnahme einiger Gebiete im Norden Europas und einigen Inseln --- verbreitet \citep{dahle2010,brodschneider2011}. \textit{Apis mellifera} hat mit dem ursprünglichen Wirt \textit{Apis cerana} nur das aggressive Verhalten als Abwehrstrategie gegenüber dem Parasiten gemein, wenngleich das bei der westlichen Honigbiene \textit{Apis mellifera} geringer ausgeprägt ist. Unterschiede zwischen den beiden Arten zeigen sich etwa im Hygieneverhalten und bei \textit{Apis cerana} der weitgehenden Limitierung des Befalls auf Drohnenbrut \citep{rosenkranz2010}.
\newline
Die Vermehrung von \textit{Varroa destructor} findet in verdeckelten Drohnen- und Arbeiterinnenbrutzellen statt. Als Nahrung der Milbe dient hauptsächlich der Fettkörper von Larven und adulten Bienen \citep{ramsey2019}. Nach dem Schlupf der Biene ernährt sich die weibliche Milbe parasitisch für mehrere Tage an Ammenbienen und schädigt damit auch erwachsene Bienen \citep{ramsey2019}. In dieser Phase betreibt die Milbe auch Phoresie. Das bedeutet, dass \textit{Varroa destructor} die adulte Biene als Transportmittel benutzt, um in neue Brutzellen oder gar entfernt liegende Bienenvölker zu gelangen. Die Parasitierung der Brut führt unter anderem zu einem Gewichtsverlust, welcher sich auf den späteren Paarungserfolg von Drohnen auswirkt, außerdem kann sich die Lebensspanne von Arbeiterinnen verkürzen. Der Befall von Sammlerinnen beeinflusst das Lernvermögen und das Heimflug-Verhalten \citep{kralj2007, kralj2006, rosenkranz2010, noel2020}. Die betroffenen Bienen leiden zudem unter Orientierungsschwierigkeiten, was möglicherweise der Verbreitung der Milbe dient. Zudem wird der Aufbau wichtiger Proteinreserven, die für den Überwinterungserfolg entscheidend sind, erschwert \citep{amdam2004}. Durch die Parasitierung mit der Varroamilbe wird durch eine Immunsuppression die Vermehrung von Viren bei Sekundärinfektionen erleichtert. Am besten bekannt ist die durch Varroose begünstigte Infektion mit dem Deformed Wing Virus (DWV, Flügeldeformationsvirus), welche sich durch die stark verkümmerten Flügel und verkürzten Abdomina (Hinterleibe) der Bienen auszeichnet \citep{rosenkranz2010}. Bienenvölker, in denen dieses Virus nachgewiesen wurde, waren schwächer, das heißt sie hatten weniger mit Bienen besetzte Waben und Brut als nicht befallene Völker \citep{budge2015}. Mit DWV infizierte adulte Bienen zeigen eine verkürzte Lebensspanne, ein jüngeres Sammelalter und eine verkürzte Sammelzeitspanne \citep{benaets2017}, aber auch die Übertragung anderer Viren steht in Zusammenhang mit der Varroamilbe \citep{traynor2016}. Ein hoher Parasitierungsgrad durch \textit{Varroa destructor} im Monat September hat neben anderen Faktoren wie dem Alter der Königinnen, dem Erfahrungsgrad der ImkerInnen, der Stärke der Völker im September, einen großen Einfluss auf das Ausmaß der Winterverluste \citep{morawetz2019}.
\newline
Fehlende oder mangelhaft vom Imker durchgeführte Behandlung gegen \textit{Varroa destructor} führt zumeist zu einer Schwächung oder sogar zum Verlust des Volkes innerhalb von zwei bis drei Jahren \citep{rosenkranz2010}. Die Bekämpfung des Parasiten basiert entweder auf biotechnischen Maßnahmen (zum Beispiel Entnahme von verdeckelter Drohnen- oder Arbeiterinnenbrut, Bannwabenverfahren, Brutunterbrechung), Einsatz zugelassener Tierarzneimittel auf Basis organischer Säuren (Ameisen-, Oxal-, Milchsäure), ätherischer Öle (Thymol, Eucalyptol, Menthol, Kampfer) und Akariziden aus verschiedenen Wirkstoffgruppen. Je nach chemischen Eigenschaften der eingesetzten Stoffe (fett- bzw. wasserlöslicher Wirkstoff) kann es dabei zu einer ungewollten Ansammlung der Stoffe in Honig und Wachs, bis hin zur Beeinflussung der Gesundheit des Volkes kommen \citep{rosenkranz2010, noel2020}. Entscheidend für den Erfolg der Behandlung ist der Zustand des Volkes (mit bzw. ohne verdeckelte Brut), die Art und der Zeitpunkt der jeweiligen Varroabehandlung \citep{brodschneider2013, vanderzee2014}.
\newline
Neben der Varroamilbe existieren in Österreich noch andere Schädlinge, die der Gesundheit der Völker zusetzen. Zu erwähnen ist unter anderem das Mikrosporidium \textit{Nosema} spp., das in den Arten \textit{N. ceranae} und \textit{N. apis} in Österreich nachgewiesen wurde. Dieser Einzeller befällt die Epithelzellen des Mitteldarms adulter Tiere und wirkt dort als intrazellulärer Parasit, der Dysenterie, eine Entzündung des Darms mit einhergehender Diarrhö, auslöst. Die durch \textit{Nosema} spp. ausgelöste Krankheit wird als Nosemose bezeichnet. Weitere Bedrohungen sind Bakterien, die Amerikanische Faulbrut (\textit{Paenibacillus larvae}) oder Europäische Faulbrut (\textit{Melissococcus plutonius}) auslösen, Pilze (\textit{Ascosphaera apis, Aspergillus flavus}), Amöben (\textit{Malpighamoeba mellificae}), weitere Arthropoden wie die Tracheenmilbe (\textit{Acarapis woodi}) und in geringem Ausmaß Innenschädlinge wie zum Beispiel die Große und Kleine Wachsmotte (\textit{Galleria mellonella, Achroia grisella}) oder der Totenkopfschwärmer (\textit{Acherontia atropos}) \citep{brodschneider2011}. Die Liste der vorkommenden Schädlingen in Österreich könnte in Zukunft noch durch den kleinen Bienenstockkäfer (\textit{Aethina tumida}) \citep{neumann2016} und der asiatischen Hornisse (\textit{Vespa velutina}) \citep{monceau2014} ergänzt werden.
\newline
Der Mensch fördert durch die räumliche Nähe vieler Bienenvölker an einem Bienenstand die horizontale Verbreitung der Krankheitserreger \citep{seeley2015, degrandi-hoffman2015, forfert2016}. Die Verbreitung von Krankheiten kann aber nicht nur durch eine hohe Völkerdichte an einem Standort, sondern auch durch Handel und Wanderimkerei über weite Entfernungen stattfinden. Internationale Untersuchungen von \cite{vanderzee2012, vanderzee2014} sowie Ergebnisse aus den USA \citep{steinhauer2014,lee2015} zeigen, dass größere Betriebe in manchen Jahren sogar geringere Winterverluste als kleinere Betriebe verzeichneten, was neben der hohen Völkerdichte auch auf andere Faktoren bei der Verbreitung schließen lässt. Auch eine vertikale Erregerübertragung mancher Krankheiten über Ei- oder Samenzellen ist möglich \citep{peng2015, yue2007}.
\newline
Immer mehr Studien widmen sich der Erforschung der synergistischen Wirkung verschiedener Stressfaktoren. Untersuchungen von \cite{pettis2012, diprisco2013} sowie von \cite{alburaki2017} machen auf die Zusammenhänge zwischen dem subletalen Einfluss von Pestiziden und gesteigertem Pathogenbefall bei Honigbienen aufmerksam. Zum Beispiel verursacht das Neonicotinoid Clothianidin eine Schwächung des Immunsystems, und damit eine stärkere Infektion mit dem Flügeldeformationsvirus \citep{diprisco2013}. Auch die Kombination von Neonicotinoiden und der Varroamilbe resultieren in höheren Verlustraten, insbesondere bei den langlebigen Winterbienen \citep{straub2019}. \cite{pettis2012} stellte zudem eine höhere Anzahl von \textit{Nosema} spp.-Sporen unter dem Einfluss von Imidacloprid fest, ein Hinweis für eine gesteigerte Anfälligkeit für den Darmparasit \textit{Nosema} spp. Erst kürzlich konnte gezeigt werden, dass Larven, die mit Amerikanischer Faulbrut infiziert und gleichzeitig subletalen Dosen bestimmter Pestizide ausgesetzt waren, eine signifikant höhere Mortalität aufwiesen als Larven die nur einem dieser beiden Stressoren ausgesetzt waren \citep{lopez2017}. Die Ergebnisse dieser Studien unterstützen somit die Annahme multifaktorieller Ursachen hoher Winterverlustraten. Diese Ergebnisse lassen vermuten, dass sich die Aufsummierung mehrerer Risikofaktoren, wie zum Beispiel Krankheitserreger, Mangelernährung und Pestizidkontamination, stärker als ein Faktor allein, und auch die Summe der einzelnen Schädigungen auf den Überwinterungserfolg auswirken kann \citep{goulson2015,barroso-arevalo2019}. Eine gezielte Einleitung von Gegenmaßnahmen wird dadurch erschwert \citep{brodschneider2013}.
\newline
Die Dokumentation der Winterverluste ist aus mehreren Gründen sehr wichtig: einerseits wegen der bereits erwähnten ökonomischen und ökologischen Bedeutung der Honigbiene und andererseits, um die Ursachen zu ergründen und entsprechend darauf reagieren zu können. Die Erhebung der Winterverluste in Österreich erfolgte seit 2008 durch das Institut für Biologie (vormals: Zoologie) der Karl-Franzens-Universität Graz im Rahmen des Forschungsnetzwerkes COLOSS (prevention of honey bee COlony LOSSes; \cite{brodschneider2010,brodschneider2013}). Seit der Überwinterungsperiode 2013/14 erfolgen diese Erhebungen im Rahmen des Projektes \enquote{Zukunft Biene} unter Einhaltung der von COLOSS etablierten und immer weiter angepassten Standards. Die Befragung erfolgte anhand des im Rahmen von COLOSS erarbeiteten Fragebogens: ImkerInnen in ganz Österreich werden auf freiwilliger Basis und auf Wunsch auch anonym zur Zahl ihrer eingewinterten Völker, deren Standort sowie zum Völkerverlust befragt. Zudem werden auch Fragen zur Betriebsweise, Behandlung der Völker gegen die Varroamilbe und zu ökologischen Faktoren (etwa zum Trachtangebot) gestellt. Diese standardisierte Vorgehensweise, auch als Citizen Science oder Bürgerbeteiligung bezeichnet, erlaubt die Gewinnung großer Datensätze über das Bienensterben im Winter und ermöglicht zusätzlich zur Analyse der Winterverluste in Österreich auch internationale Vergleiche \citep{brodschneider2016, brodschneider2018,brodschneider2019}. In Österreich konnte mithilfe dieser Daten ein Zusammenhang mit Wetter \citep{switanek2017} und Landnutzung \citep{kuchling2018} auf Winterverluste festgestellt werden. Die Befragung der ImkerInnen stellt aber nur eine der im Rahmen von \enquote{Zukunft Biene} getätigten Maßnahmen dar. In weiteren Untersuchungen im Rahmen von \enquote{Zukunft Biene} wurden ausgewählte Völker auch Inspektionen durch geschulte Experten unterzogen und aufwändige Untersuchungen von Probenmaterial durchgeführt. Die Umfragen bilden eine wichtige Datenbasis für vertiefende Untersuchungen und sind somit, gemeinsam mit weiteren Untersuchungen (zum Beispiel gezielte Probenahmen im Teilprojekt \enquote{Virenmonitoring}), der Grundstock im Kampf gegen hohe Winterverlustraten \citep{vanderzee2015}.