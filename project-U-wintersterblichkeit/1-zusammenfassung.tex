\section{Zusammenfassung}
Im Modul U werden die Winterverluste von Bienenvölkern erhoben und auf Risikofaktoren hin untersucht. In Österreich wird diese Untersuchung seit dem Winter 2007/08 durchgeführt. Die Verlustraten der eingewinterten Völker reichen dabei von \confi{8,1}{95}{7,4}{8,8} (Winter 2015/16) bis \confi{28,4}{95}{27,0}{29,9} (Winter 2014/15). 
\newline
Die 1.539 Antworten betreffend 30.724 eingewinterte Bienenvölker der Untersuchung 2020 wurden auf ihre Repräsentativität überprüft und Analysen zur geografischen Verteilung der Verluste, zu den begleitenden Symptomen, sowie zur Betriebsweise durchgeführt. Ein wichtiger Faktor war außerdem die Analyse der Behandlungsmethoden, welche zur Bekämpfung von \textit{Varroa destructor} eingesetzt wurden und der Einfluss dieser Methoden auf die Wintersterblichkeit.
\newline
Die Winterverlustrate für ganz Österreich lag 2019/20 bei \confi{12,6}{95}{11,9}{13,3}. Dieser Wert liegt im Jahresvergleich etwas unter dem laufenden langjährigen Durchschnitt von 12,6\%. Im Vergleich zwischen den Bundesländern zeigt sich ein deutlich höherer Verlust für Wien \confi{20,1}{95}{16,0}{24,8}.
\newline
Wie in den vorangegangen Jahren konnte bei Faktoren die auf Parameter wie Professionalität, Erfahrung in der Imkerei etc. hindeuten könnten signifikant niedrigere Verlustraten festgestellt werden. Hierzu zählen zB.: Betriebsgröße (größere Betriebe weniger Verluste) und Wanderimker mit geringeren Verlustraten. Auch konnte heuer wieder ein positiver Effekt mit eigenen Wachskreislauf identifiziert werden, was wieder auf die Professionalität oder auf die Qualitätsprobleme mit Wachs hindeuten könnte. 
\newline
Interessanterweise zeigte sich heuer zum ersten Mal eine Wahrscheinlichkeit für einen niedrigeren Winterverlust, wenn Königinnen aus \enquote{Zucht auf Varroa-Toleranz} gemeldet wurden aber nur im Vergleich zur Gruppe die \enquote{Unsicher} Angaben konnte diese niedrigere Verlustrate auch statistisch belegt werden. Zusätzlich zeigten TeilnehmerInnen die \enquote{kleine Brutzellen} verwendet haben deutlich geringere Verlustraten.
\newline
Bei den Trachten zeigte sich ein signifikanter höherer Verlust bei Raps und Mais. ImkerInnen mit Waldtracht hatten geringere Verluste als ohne. Dieses Jahr konnte kein signifikant höherer Verlust durch Melezitose erkannt werden, wie letztes Jahr. 
\newline
Bei der Varroabekämpfung zeigte sich ein weiteres Mal, dass biotechnische Methoden im Sommer (totale Brutentnahme, Königin käfigen etc.) ein probates Mittel zur Reduktion von Völkerverlusten sind. Eine Anwendung von Thymol im Sommer führte zu einem negativen Effekt und höheren Verlustraten über den Winter. Eine Anwendung von Ameisensäure Kurzzeitbehandlung ohne Frühjahr oder Winterbehandlung führte zu potentiell hohen Verlusten.
\newline
Wie bereits aus mehreren vorangegangenen Untersuchungen bekannt, konnten auch heuer wieder negative Effekte bei den im Vergleich zum Vorjahr häufiger aufgetretenen Königinnenproblemen während der Saison auf die Verlustrate durch Königinnenprobleme festgestellt werden. Ein aktiver Austausch mit jungen Königinnen hatte einen positiven Effekt auf die Überlebenswahrscheinlichkeit der Völker.
\newline
Als Brücke zu den anderen Modulen, wie bereits auch im vorherigen Jahr, sei auf das Ergebnis signifikant höherer Winterverluste bei Imkereien hingewiesen, die während der Saison verkrüppelte Flügel zeigten, unter anderem ein Symptom für Virenerkrankung, hingewiesen.

\begin{otherlanguage}{english}
\section{Summary}
In module U, winter losses of bee colonies are monitored and examined for risk factors. Thus far, this study has been carried out in Austria since the winter 2007/08. The loss rates of overwintering colonies range from 8.1\% (95\%~CI:~7.4-8.8\%) (winter 2015/16) to 28.4\% (95\%~CI:~27.0-29.9\%) (winter 2014/15). 
\newline
The 1,539 answers concerning 30,724 wintered bee colonies of the 2020 study were checked for their representativeness and analyses were carried out on the geographical distribution of the losses, on the symptoms accompanying the winter losses, and on the beekeeping practices. An key factor was the analysis of treatment methods used to combat \textit{Varroa destructor} and the influence of these methods on winter mortality. 
\newline
The winter loss rate for the whole of Austria in 2019/20 was 12.6\% (95\%~CI:~11.9-13.3\%). The loss rate is close to the average of the past years, with 12.6\%. In comparison between the states, the loss rate for Vienna is noticeably higher than the rest of Austria with 20.1\% (95\%~CI:~16.0-24.8\%).
\newline
As in previous years factors which could allude for professionalism, experience of the beekeeping operation, etc. did show lower loss rates. These include factors like size of beekeeping operation (bigger smaller losses) and migratory beekeepers had lower losses. Participants who did not purchase wax from outside their operation experienced also lower losses, which could also point to professionalism or could mean there is a quality problem of the available wax.
\newline
Interestingly, these year participants which did use \enquote{Queens bred from Varroa tolerant/resistant stock} showed lower loss rates, but only compared to the group which did answer with \enquote{Unsure}. In addition the use of \enquote{Small brood cell size (5.1 mm or less)} showed statistically lower loss rates.
\newline
The occurrence of maize or rapeseed forage did come with a statistically higher loss rate. Participants with honeydew harvest showed lower loss rates as the ones without. This year, in contrast to last year, no higher loss rate with melezitose was observed.
\newline
As for methods to combat the varroa mite, biotechnical methods in summer (total brood removal, queen confinement, etc.) demonstrated again the possibility for lower loss rates. The usage of thymol in summer caused a negative effect on colony survival over winter. The exclusive use of formic acid - short term treatment in summer, without a spring or winter treatment displayed a negative potential for high winter losses.
\newline
More queen problems over the season in comparison to last year(s) caused a negative effect on the winter loss rate, as in earlier analysis. Active exchange of \enquote{old queens} did improve the survival change of the colonies.
\newline
As a bridge to other modules of this project, we want to stress that the observation of bees with crippled wings during the bee’s active season, a possible sign for viral diseases, results in significantly higher winter losses. 
\end{otherlanguage}