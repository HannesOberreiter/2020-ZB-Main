Im Modul U werden die Winterverluste von Bienenvölkern erhoben und auf Risikofaktoren hin untersucht. In Österreich wird diese Untersuchung seit dem Winter 2007/08 durchgeführt. Die Verlustraten der eingewinterten Völker reichen dabei von \confi{8,1}{95}{7,4}{8,8} (Winter 2015/16) bis \confi{28,4}{95}{27,0}{29,9} (Winter 2014/15). 
\newline
Die 1.539 Antworten betreffend 30.724 eingewinterte Bienenvölker der Untersuchung 2020 wurden auf ihre Repräsentativität überprüft und Analysen zur geografischen Verteilung der Verluste, zu den begleitenden Symptomen, sowie zur Betriebsweise durchgeführt. Ein wichtiger Faktor war außerdem die Analyse der Behandlungsmethoden, welche zur Bekämpfung von \textit{Varroa destructor} eingesetzt wurden und deren Einfluss auf die Wintersterblichkeit.
\newline
Die Winterverlustrate für ganz Österreich lag 2019/20 bei \confi{12,6}{95}{11,9}{13,3}. Dieser Wert liegt im Jahresvergleich etwas unter dem laufenden langjährigen Durchschnitt von 16,1\%. Im Vergleich zwischen den Bundesländern zeigt sich ein deutlich höherer Verlust für Wien \confi{20,1}{95}{16,0}{24,8}.
\newline
Wie in den vorangegangen Jahren konnte bei Faktoren die auf Parameter wie Professionalität, Erfahrung in der Imkerei etc. hindeuten könnten signifikant niedrigere Verlustraten festgestellt werden. Hierzu zählen zB.: Betriebsgröße (größere Betriebe weniger Verluste) und WanderimkerInnen die geringere Verlustraten aufweisen. Auch konnte heuer wieder ein positiver Effekt für einen eigenen Wachskreislauf identifiziert werden, was wiederum auf die Professionalität der ImkerInnen oder auf generelle Qualitätsprobleme mit Wachs hindeuten könnte. 
\newline
Interessanterweise zeigte sich heuer zum ersten Mal eine Wahrscheinlichkeit für einen niedrigeren Winterverlust, wenn Königinnen aus \enquote{Zucht auf Varroa-Toleranz} gemeldet wurden. Diese niedrigere Verlustrate konnte aber nur im Vergleich zur Gruppe mit der Angabe \enquote{Unsicher} belegt werden. Zusätzlich zeigten TeilnehmerInnen die \enquote{kleine Brutzellen} verwendet haben deutlich geringere Verlustraten.
\newline
Bei den Trachten zeigte sich ein signifikant höherer Verlust bei Raps und Mais. ImkerInnen mit Waldtracht hatten geringere Verluste als die ohne. Im Vergleich zum letzten Jahr konnte dieses Jahr kein signifikant höherer Verlust durch Melezitose erkannt werden. 
\newline
Bei der Varroabekämpfung zeigte sich ein weiteres Mal, dass biotechnische Methoden im Sommer (totale Brutentnahme, Königin käfigen etc.) ein probates Mittel zur Reduktion von Völkerverlusten sind. Im Gegensatz führte eine Anwendung von Thymol im Sommer zu höheren Verlustraten über den Winter. Die Anwendung von Ameisensäure in Kurzzeitbehandlung ohne Frühjahr oder Winterbehandlung führte potentiell zu hohen Verlusten.
\newline
Wurden Königinnenprobleme während der Saison häufiger beobachtet als im Vorjahr, führte dies zu erhöhten Verlustraten über den Winter. Dahingegen hat ein aktiver Austausch mit jungen Königinnen einen positiven Effekt auf die Überlebenswahrscheinlichkeit der Völker.
\newline
Als Brücke zu den anderen Modulen, sei auf das Ergebnis signifikant höherer Winterverluste bei Imkereien hingewiesen, die während der Saison verkrüppelte Flügel beobachtet haben, was ein mögliches Symptom von Virenerkrankung sein kann. Dies konnte auch schon vorangegangenen Untersuchungen festgestellt werden.