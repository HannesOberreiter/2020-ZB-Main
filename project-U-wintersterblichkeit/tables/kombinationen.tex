\newpage
\begin{landscape}

\begin{table}[H]
    \centering
    \caption{Verlustraten und Stichprobenanzahl der Kombinationen. W = Winter, S = Sommer, Einzelne Monate sind nach Saison zusammengefasst, siehe \cref{fig:u:treatment:histogramm}.}
    \scriptsize
    \setlength{\tabcolsep}{0.5em} % for the horizontal padding
    \label{tab:u:kombinationen}
    \begin{tabular}{l|*{6}{l}|rr}
        %\hline
            \multicolumn{1}{c|}{Abkürzung} & 
            \multicolumn{6}{c|}{Methode} & 
            \multicolumn{1}{c}{\textit{n}} &
            \multicolumn{1}{c}{Verlust (95\% KI)}
            \\
        \hline
        (A) S-AS-LZ   \& W-Ox-Träu.             & S & Ameisensäure - Langzeit       & W & Oxalsäure - träufeln      &   &                       & 156 & 10,4 (8,8-12,3) \\
        (B) S-AS-LZ   \& W-Ox-Sub.              & S & Ameisensäure - Langzeit       & W & Oxalsäure - sub.          &   &                       &  95 & 10,9 (8,2-14,4) \\
        (C) S-AS-KZ   \& W-Ox-Träu.             & S & Ameisensäure - Kurzzeit       & W & Oxalsäure - träufeln      &   &                       &  64 & 13,5 (10,2-17,7) \\
        (D) S-Biot.   \& S-Ox-Sub. \& W-Ox-Sub. & S & Andere biotechnische Methode  & S & Oxalsäure - sub.          & W & Oxalsäure - sub.      &  57 & 10,7 (8,1-13,9) \\
        (E) S-AS-KZ   \& S-AS-LZ \& W-Ox-Träu.  & S & Ameisensäure - Kurzzeit       & S & Ameisensäure - Langzeit   & W & Oxalsäure - träufeln  &  51 & 15,0 (10,1-21,8) \\
        (F) S-Ox-Sub. \& W-Ox-Sub.              & S & Oxalsäure - sub.              & W & Oxalsäure - sub.          &   &                       &  39 & 13,1 (9,9-17,2) \\
        (G) S-AS-KZ   \& W-Ox-Sub.              & S & Ameisensäure - Kurzzeit       & W & Oxalsäure - sub.          &   &                       &  37 & 16,1 (11,3-22,3) \\
        (H) S-AS-LZ   \& S-Ox-Sub. \& W-Ox-Sub. & S & Ameisensäure - Langzeit       & S & Oxalsäure - sub.          & W & Oxalsäure - sub.      &  34 & 13,3 (9,5-18,5) \\
        (I) S-AS-LZ                             & S & Ameisensäure - Langzeit       &   &                           &   &                       &  29 & 15,3 (11,4-20,4) \\
        (J) S-AS-KZ \& S-AS-LZ \& W-Ox-Sub.     & S & Ameisensäure - Kurzzeit       & S & Ameisensäure - Langzeit   & W & Oxalsäure - sub.      &  29 & 13,3 (9,6-18,0) \\
        (K) S-AS-KZ \& S-Ox-Sub. \& W-Ox-Sub.   & S & Ameisensäure - Kurzzeit       & S & Oxalsäure - sub.          & W & Oxalsäure - sub.      &  27 & 11,7 (7,9-17,2) \\
        (L) S-AS-KZ \& S-Ox-Träu. \& W-Ox-Träu. & S & Ameisensäure - Kurzzeit       & S & Oxalsäure - träufeln      & W & Oxalsäure - träufeln  &  24 & 15,1 (10,4-21,4) \\
        (M) S-AS-LZ \& S-Ox-Träu. \& W-Ox-Träu. & S & Ameisensäure - Langzeit       & S & Oxalsäure - träufeln      & W & Oxalsäure - träufeln  &  23 & 12,7 (7,6-20,5) \\
        (N) S-Biot. \& S-AS-LZ \& W-Ox-Träu.    & S & Andere biotechnische Methode  & S & Ameisensäure - Langzeit   & W & Oxalsäure - träufeln  &  20 & 13,6 (9,1-19,9) \\
        (O) S-Biot. \& S-Ox-Träu. \& W-Ox-Träu. & S & Andere biotechnische Methode  & S & Oxalsäure - träufeln      & W & Oxalsäure - träufeln  &  19 &  9,1 (4,3-18,2) \\
        (P) S-Biot. \& S-Ox-Träu. \& W-Ox-Sub.  & S & Andere biotechnische Methode  & S & Oxalsäure - träufeln      & W & Oxalsäure - sub.      &  18 &  9,0 (5,6-14,4) \\
        (Q) S-AS-KZ                             & S & Ameisensäure - Kurzzeit       &   &                           &   &                       &  17 & 23,3 (13,8-36,5) \\
        (R) S-AS-KZ \& S-Ox-Träu. \& W-Ox-Sub.  & S & Ameisensäure - Kurzzeit       & S & Oxalsäure - träufeln      & W & Oxalsäure - sub.      &  16 & 15,1 (6,9-29,9) \\
        (S) S-AS-LZ \& S-Ox-Träu. \& W-Ox-Sub.  & S & Ameisensäure - Langzeit       & S & Oxalsäure - träufeln      & W & Oxalsäure - sub.      &  16 & 14,2 (8,1-23,7) \\
     \hline
    \end{tabular}
\end{table}

\end{landscape}
\newpage
