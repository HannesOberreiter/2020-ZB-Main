\begin{table}[H]
    \caption{Populationsdynamik der Subpopulation (Imkereien mit vollständigen Angaben) untersuchter österreichischer Bienenvölker vom Frühjahr 2013 bis zum Frühjahr 2020.}
    \label{tab:u:population}
    \scriptsize
    \begin{tabular}{c*{8}{r}}
        \toprule
        & & & 
        \multicolumn{3}{c}{Anzahl Bienenvölker} & & 
        \multirow{2}{*}{\shortstack{Vergleich \\ Frühjahr-\\Frühjahr[\%]}} & 
        \\
        \cmidrule{4-6}
        Jahr & 
        \makecell{Imkereien \\ [\textit{n}]} & 
        \makecell{Verlust- \\ rate [\%]$^1$} & 
        \makecell{Frühjahr [\#]$^2$} & 
        \makecell{Eingewintert \\ Herbst [\#]} & 
        \makecell{Ausgewintert \\ Frühjahr [\#]} & 
        \makecell{Vermehrung \\ Sommer [\%]} & 
        & 
        \makecell{Ausgleich \\ Verluste [\%]$^3$} 
        \\
        \midrule
        2013/14 &   973 & 12,9 & 14.319 & 17.816 & 15.518 & 24,4 &   8,4 & 14,8 \\
        2014/15 & 1.188 & 28,6 & 17.355 & 21.616 & 15.437 & 24,6 & -11,1 & 40,0 \\
        2015/16 & 1.195 &  7,9 & 15.102 & 21.800 & 20.070 & 44,4 &  32,9 &  8,6 \\
        2016/17 & 1.537 & 22,5 & 27.695 & 40.141 & 31.108 & 44,9 &  12,3 & 29,0 \\
        2017/18 & 1.285 & 11,6 & 18.983 & 25.670 & 22.695 & 35,2 &  19,6 & 13,1 \\
        2018/19 & 1.465 & 15,3 & 24.747 & 31.036 & 26.277 & 25,4 &   6,2 & 18,1 \\
        2019/20 & 1.478 & 12,5 & 23.802 & 29.484 & 25.802 & 23.9 &   8,4 & 14,3 \\
        \bottomrule
    \end{tabular}
    \scriptsize
    $^1$ Verlustrate der teilnehmende Imkereien mit vollständigen Angaben zur Anzahl der Völker im Frühjahr des Einwinterungsjahres.
    \newline
    $^2$ Völker im Frühjahr des Einwinterungsjahres.
    \newline
    $^3$ Erforderliche Netto-Zuwachsrate, um nach dem Winter wieder auf den Stand der Bienenpopulation im Herbst des Einwinterungsjahres zu kommen.
\end{table}

