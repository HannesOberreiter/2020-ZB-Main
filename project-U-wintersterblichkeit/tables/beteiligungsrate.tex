\begin{table}[H]
    \caption{Beteiligungsrate der österreichischen Imkereien an unserer Umfrage zu den Winterverlusten seit Winter 2013/14.}
    \label{tab:u:beteiligungsrate}
    \begin{tabular}{c|*{2}{r}r|*{3}{r}}
        %%\hline
        \multicolumn{1}{c}{}&
        \multicolumn{3}{c}{Imkereien} & 
        \multicolumn{3}{c}{Bienenvölker} \\
        \cline{2-7}
        \multicolumn{1}{c}{Jahr} & 
        \makecell{Gesamt$^1$ [\#]} &
        \makecell{Teilnehmer [\textit{n}]} &
        \makecell{Anteil [\%]} &
        \makecell{Gesamt$^1$ [\#]} &
        \makecell{Teilnehmer$^2$ [\textit{n}]} &
        \makecell{Anteil [\%]} \\ 
        \hline
        2013/14 & 25.492 & 1.023 & 4,0 & 382.638 & 18.794 &  4,9 \\
        2014/15 & 25.277 & 1.259 & 5,0 & 376.121 & 22.882 &  6,1 \\
        2015/16 & 26.063 & 1.289 & 4,9 & 347.128 & 23.418 &  6,7 \\
        2016/17 & 26.609 & 1.656 & 6,2 & 354.080 & 43.852 & 12,4 \\
        2017/18 & 27.580 & 1.391 & 5,0 & 353.267 & 28.373 &  8,0 \\
        2018/19 & 28.432 & 1.534 & 5,4 & 373.412 & 33.651 &  9,0 \\
        2019/20 & 30.237 & 1.539 & 5,1 & 390.607 & 30.724 &  7,9 \\
        \hline
    \end{tabular}
    \scriptsize
    $^1$Die angeführten Gesamtzahlen beziehen sich auf Imkereien und Bienenvölker in Österreich und beruhen auf Angaben der \enquote{Biene Österreich}. Diese Zahlen bilden die Grundlage für die Berechnung der Rückmeldungen. 
    \\
    $^2$Gesamtsumme der eingewinterten Bienenvölker der teilnehmenden Imkererein.
\end{table}

