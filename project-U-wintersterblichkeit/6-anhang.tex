\section{Anhang}

\begin{table}[H]
    \centering
    \caption{Burgenland - Jahresvergleich der Verlustraten in den Bezirken. Verlustrate in \%, (TeilnehmerInnen; eingewinterte Völker). -: weniger als fünf TeilnehmerInnen.}
    \scriptsize
    \setlength{\tabcolsep}{0.5em} % for the horizontal padding
    \label{tab:u:district-burgenland}
    \begin{tabular}{|c|*{5}{rr|}}
        \hline
        \multicolumn{11}{|c|}{Burgenland - Verlustrate \% (Teilnehmer, eingewinterte Völker)} \\    
        \hline
        \makecell{Jahre} & 
        \multicolumn{2}{c|}{Eisenstadt}    & 
        \multicolumn{2}{c|}{Eisenstadt-Umgebung}    & 
        \multicolumn{2}{c|}{Güssing} & 
        \multicolumn{2}{c|}{Jennersdorf}  &  
        \multicolumn{2}{c|}{Mattersburg} 
        \\
        \hline
        2013/24 & - &  &       - &         &       - &           &       - &         &       - &          \\
        2014/15 & - &  & 40,30\% & (6; 67) & 42,24\% & (13; 161) & 42,71\% & (8; 96) &       - &          \\
        2015/16 & - &  &       - &         &       - &           & 23,53\% & (5; 85) &       - &          \\
        2016/17 & - &  &       - &         & 31,46\% &   (6; 89) &       - &         & 28,93\% & (9; 121) \\
        2017/18 & - &  &       - &         &       - &           &       - &         &  9,49\% & (8; 137) \\
        2018/19 & - &  &       - &         &       - &           &       - &         & 16,30\% & (8; 135) \\
        2019/20 & - &  &       - &         &       - &           &       - &         &  8,05\% & (5;  87) \\
        \hline
        \makecell{Jahre} & 
        \multicolumn{2}{c|}{Neusiedl am See}    & 
        \multicolumn{2}{c|}{Oberpullendorf}    & 
        \multicolumn{2}{c|}{Oberwart} & 
        \multicolumn{2}{c|}{Rust}  & &  \\
        \hline
        2013/14 &       - &          &        - &          & 31,82\% &   (5; 88) & - &  &&\\
        2014/15 & 24,72\% & (7; 178) & 53,72\% & (12; 376) & 30,08\% & (17; 256) & - &  &&\\
        2015/16 &  3,55\% & (5; 169) & 20,51\% &   (7; 78) &  4,17\% &   (9;144) & - &  &&\\
        2016/17 & 12,22\% & (8; 311) & 20,95\% & (12; 253) & 27,57\% & (15; 185) & - &  &&\\
        2017/18 &  6,45\% &  (5; 62) &  5,00\% &   (5; 60) & 11,96\% &   (7; 92) & - &  &&\\
        2018/19 &  8,16\% & (7; 147) &       - &           &  5,58\% &  (9; 269) & - &  &&\\
        2019/20 &      -  &          & 15,09\% &   (7; 53) & 13,11\% &  (10;183) & - &  &&\\
        \hline
    \end{tabular}
\end{table}


\begin{table}[H]
    \centering
    \caption{Kärnten - Jahresvergleich der Verlustraten in den Bezirken. Verlustrate in \%, (TeilnehmerInnen; eingewinterte Völker). -: weniger als fünf TeilnehmerInnen.}
    \scriptsize
    \setlength{\tabcolsep}{0.5em} % for the horizontal padding
    \label{tab:u:district-kaernten}
    \begin{tabular}{|c|*{5}{rr|}}
        \hline
        \multicolumn{11}{|c|}{Kärnten - Verlustrate \% (Teilnehmer, eingewinterte Völker)} \\    
        \hline
        \makecell{Jahre} & 
        \multicolumn{2}{c|}{Feldkirchen}    & 
        \multicolumn{2}{c|}{Hermagor}    & 
        \multicolumn{2}{c|}{Klagenfurt am Wörthersee} & 
        \multicolumn{2}{c|}{Klagenfurt-Land}  &  
        \multicolumn{2}{c|}{Sankt Veit an der Glan} 
        \\
        \hline
        2013/14 & 11,39\% & (7; 202) &  5,48\% & (20; 365) & 12,66\% &   (5; 79) &  6,76\% & (10; 281) & 15,32\% & (15; 385) \\
        2014/15 & 32,46\% & (6; 191) & 56,41\% & (11; 195) & 54,55\% &  (10; 77) & 34,55\% & (18; 330) & 26,45\% & (25; 881) \\
        2015/16 &  3,86\% & (5; 233) &  5,93\% & (17; 337) & 12,10\% &  (9; 124) &  4,98\% & (15; 301) &  7,28\% & (12; 604) \\
        2016/17 & 40,11\% & (7; 187) & 19,95\% & (19; 436) & 52,43\% &  (9; 103) & 15,34\% & (10; 189) & 41,05\% & (16; 592) \\
        2017/18 & 13,99\% & (5; 193) & 14,51\% & (22; 448) & 25,79\% &  (8; 159) &  6,16\% &  (9; 276) & 20,48\% & (16; 420) \\
        2018/19 &  7,06\% & (6; 354) & 12,03\% & (18; 316) & 13,54\% & (10; 192) & 11,22\% & (16; 401) &  7,22\% & (16; 568) \\
        2019/20 &  8,81\% & (8; 318) & 11,24\% & (20; 436) &  8,86\% &   (5; 79) & 13,52\% & (21; 355) & 18,58\% & (13; 366) \\
        \hline
        \makecell{Jahre} & 
        \multicolumn{2}{c|}{Spittal an der Drau}    & 
        \multicolumn{2}{c|}{Villach}    & 
        \multicolumn{2}{c|}{Villach-Land } & 
        \multicolumn{2}{c|}{Völkermarkt}  &
        \multicolumn{2}{c|}{Wolfsberg}
        \\
        \hline
        2013/14 &  7,09\% &  (35; 705) & 12,32\% & (14; 138) &       - &           &  7,47\% & (13;  482) &      - &          \\
        2014/15 & 28,75\% &  (33; 574) & 25,84\% & (13; 178) & 32,36\% & (50; 615) & 21,78\% & (20;  652) &      - &          \\
        2015/16 &  5,81\% & (56; 1445) & 11,32\% &  (5; 106) &  5,60\% & (30; 393) &  6,60\% & (12;  303) &      - &          \\
        2016/17 & 17,34\% & (67; 1632) & 19,05\% &   (5; 63) & 20,64\% & (22; 344) & 11,76\% & (13;  561) & 8,37\% & (9; 203) \\
        2017/18 & 12,11\% &  (35; 950) & 12,82\% &   (5; 78) & 15,12\% & (25; 324) &  6,51\% & (15;  538) &      - &          \\
        2018/19 & 11,96\% &  (36; 836) & 12,17\% &  (9; 115) & 23,28\% & (20; 262) & 12,32\% & (11;  682) &      - &          \\
        2019/20 & 17,54\% &  (25; 593) &  6,08\% &  (9; 148) & 15,38\% & (28; 364) & 10,25\% & (17; 1015) &      - &          \\        
        \hline
    \end{tabular}
\end{table}


\begin{table}[H]
    \centering
    \caption{Niederösterreich - Jahresvergleich der Verlustraten in den Bezirken. Verlustrate in \%, (TeilnehmerInnen; eingewinterte Völker). -: weniger als fünf TeilnehmerInnen. **: Bezirksauflösung Wien-Umgebung 2017.}
    \scriptsize
    \setlength{\tabcolsep}{0.5em} % for the horizontal padding
    \label{tab:u:district-niederoesterreich}
    \begin{tabular}{|c|*{5}{rr|}}
        \hline
        \multicolumn{11}{|c|}{Niederösterreich - Verlustrate \% (Teilnehmer, eingewinterte Völker)} \\    
        \hline
        \makecell{Jahre} & 
        \multicolumn{2}{c|}{Amstetten}    & 
        \multicolumn{2}{c|}{Baden}    & 
        \multicolumn{2}{c|}{Bruck an der Leitha} & 
        \multicolumn{2}{c|}{Gänserndorf}  &  
        \multicolumn{2}{c|}{Gmünd} 
        \\
        \hline
        2013/24 & 15,75\% & (22; 419) &  7,92\% &  (8; 101) &       - &           & 27,78\% & (20; 198) &  7,66\% & (29; 444) \\
        2014/15 & 19,76\% & (24; 506) & 31,63\% &  (12; 98) & 47,78\% &   (8; 90) & 28,33\% & (23; 300) & 26,32\% &  (6; 133) \\
        2015/16 &  9,35\% & (38; 631) &  0,00\% &   (6; 54) &       - &           & 16,83\% & (23; 208) & 14,24\% & (24; 316) \\
        2016/17 & 37,09\% & (47; 647) & 16,36\% &   (6; 55) &  8,86\% &  (8; 158) & 19,53\% & (13; 379) & 21,04\% & (22; 461) \\
        2017/18 & 13,86\% & (39; 635) &       - &           & 16,42\% &  (10; 67) & 17,48\% & (25; 286) & 21,45\% & (18; 275) \\
        2018/19 & 24,74\% & (44; 663) &  8,43\% &   (8; 83) & 18,42\% &   (7; 76) & 10,84\% & (27; 821) & 11,43\% & (17; 420) \\
        2019/20 & 25,59\% & (35; 590) & 15,15\% &   (7; 66) & 13,59\% & (11; 103) & 11,52\% & (21; 903) & 11,62\% & (11; 198) \\
        \hline
        \makecell{Jahre} & 
        \multicolumn{2}{c|}{Hollabrunn}    & 
        \multicolumn{2}{c|}{Horn}    & 
        \multicolumn{2}{c|}{Korneuburg} & 
        \multicolumn{2}{c|}{Krems an der Donau}  & 
        \multicolumn{2}{c|}{Krems-Land}
        \\
        \hline
        2013/14 & 33,46\% &  (7; 254) & 18,15\% &  (17; 325) & 14,06\% & (14; 192) & - &  &       - &           \\
        2014/15 & 31,07\% & (12; 280) & 34,38\% &  (17; 349) & 43,64\% & (19; 236) & - &  & 25,19\% &  (9, 135) \\
        2015/16 & 12;00\% &  (8; 200) &  8,32\% &  (17; 505) &  8,42\% &  (17; 95) & - &  &  2,13\% &   (7; 47) \\
        2016/17 & 12,05\% &   (8; 83) & 14,57\% &  (22; 597) & 21,77\% & (22; 372) & - &  & 13,41\% &  (9; 179) \\
        2017/18 &  5,45\% &  (8; 110) & 19,07\% &  (19; 708) & 16,47\% & (18; 334) & - &  & 15,97\% & (12; 119) \\
        2018/19 & 28,69\% &  (7; 237) & 13,44\% &  (13; 491) & 11,95\% & (20; 728) & - &  & 14,59\% & (13; 233) \\
        2019/20 &  8,26\% & (13; 121) & 13,12\% & (55; 1151) & 14,62\% & (16; 513) & - &  & 19,71\% & (15; 279) \\
        \hline
        \makecell{Jahre} & 
        \multicolumn{2}{c|}{Lilienfeld}    & 
        \multicolumn{2}{c|}{Melk}    & 
        \multicolumn{2}{c|}{Mistelbach} & 
        \multicolumn{2}{c|}{Mödling}  & 
        \multicolumn{2}{c|}{Neunkirchen}
        \\
        \hline
        2013/14 & 14,89\% &   (5; 47) &  7,64\% & (16; 157) & 17,85\% &  (43; 521) & 16,56\% & (14; 151) & 11,34\% &   (9; 97) \\
        2014/15 & 10,58\% &  (5; 104) & 32,53\% & (26; 332) & 22,35\% &  (27; 671) & 29,08\% & (17; 141) & 44,83\% & (14; 145) \\
        2015/16 &       - &           & 15,17\% & (34; 422) &  9,47\% &  (29; 581) & 15,15\% &   (9; 66) & 29,38\% & (13; 160) \\
        2016/17 &  6,13\% & (15; 212) & 29,30\% & (19; 314) & 26,84\% &  (38; 991) & 25,23\% & (13; 107) & 25,41\% & (17; 303) \\
        2017/18 & 10,98\% & (11; 246) & 12,29\% & (21; 301) & 11,58\% & (41; 1408) & 14,50\% & (18; 262) & 10,19\% & (14; 157) \\
        2018/19 & 13,25\% & (15; 166) & 17,89\% & (33; 598) & 15,63\% &  (26; 416) & 15,24\% & (11; 164) & 17,24\% & (17; 174) \\
        2019/20 & 13,64\% &  (9; 132) & 16,32\% & (13; 190) & 15,20\% &  (19; 454) & 16,18\% & (12; 136) & 29,22\% & (16; 154) \\
        \hline
        \makecell{Jahre} & 
        \multicolumn{2}{c|}{Scheibbs}    & 
        \multicolumn{2}{c|}{St. Pölten}    & 
        \multicolumn{2}{c|}{St. Pölten-Land} & 
        \multicolumn{2}{c|}{Tulln}  & 
        \multicolumn{2}{c|}{Waidhofen an der Ybbs}
        \\
        \hline
        2013/14 &  7,74\% & (18; 594) & - &  & 12,15\% & (32; 288) & 18,71\% &  (9; 465) & - &                \\
        2014/15 & 14,48\% & (41; 808) & - &  & 24,62\% & (21; 260) & 12,63\% & (13; 372) & - &                \\
        2015/16 & 10,46\% & (29; 526) & - &  &       - &           &  7,53\% &   (5; 93) & - &                \\
        2016/17 & 44,31\% & (42; 686) & - &  & 25,55\% & (26; 274) & 13,07\% & (20; 153) & - &                \\
        2017/18 &  8,38\% & (37; 752) & - &  & 13,38\% & (25; 284) & 13,83\% &  (10; 94) & - &                \\
        2018/19 & 19,65\% & (27; 692) & - &  &  8,14\% & (25; 258) & 20,86\% & (13; 465) & 65,33\% & (7; 150) \\
        2019/20 &   9,2\% & (21; 424) & - &  & 12,95\% & (34; 533) & 12,71\% & (14; 118) & - &                \\
        \hline
        \makecell{Jahre} & 
        \multicolumn{2}{c|}{Waidhofen an der Thaya}    & 
        \multicolumn{2}{c|}{Wiener Neustadt}    & 
        \multicolumn{2}{c|}{Wiener Neustadt-Land} & 
        \multicolumn{2}{c|}{Wien-Umgebung}  & 
        \multicolumn{2}{c|}{Zwettl}
        \\
        \hline
        2013/14 & 19,61\% & (20; 311) & - &  & 10,08\% &  (8; 129) & 20,63\% & (14; 160) &  2,17\% &  (7; 138) \\
        2014/15 &       - &           & - &  & 46,37\% & (12; 317) & 31,99\% & (20; 372) & 20,93\% &  (9; 172) \\
        2015/16 & 13,45\% & (36; 394) & - &  & 17,93\% &  (9; 184) & 28,99\% &  (14; 69) &  8,77\% & (11; 171) \\
        2016/17 & 28,21\% & (34; 560) & - &  & 13,66\% & (10, 205) &      ** &           & 30,96\% & (15; 239) \\
        2017/18 & 21,39\% & (35; 561) & - &  & 12,36\% & (11; 259) &      ** &           & 16,51\% &  (9; 109) \\
        2018/19 & 11,82\% &  (8; 330) & - &  & 11,49\% & (14; 348) &      ** &           & 17,97\% & (11; 217) \\
        2019/20 & 12,50\% & (10; 192) & - &  & 12,73\% & (13; 330) &      ** &           & 13,04\% & (14; 514) \\
        \hline
    \end{tabular}
\end{table}
\begin{table}[H]
    \centering
    \caption{Oberösterreich - Jahresvergleich der Verlustraten in den Bezirken. Verlustrate in \%, (TeilnehmerInnen; eingewinterte Völker). -: weniger als fünf TeilnehmerInnen.}
    \scriptsize
    \setlength{\tabcolsep}{0.5em} % for the horizontal padding
    \label{tab:u:district-oberoesterreich}
    \begin{tabular}{|c|*{5}{rr|}}
        \hline
        \multicolumn{11}{|c|}{Oberösterreich - Verlustrate \% (Teilnehmer, eingewinterte Völker)} \\    
        \hline
        \makecell{Jahre} & 
        \multicolumn{2}{c|}{Braunau am Inn}    & 
        \multicolumn{2}{c|}{Eferding}    & 
        \multicolumn{2}{c|}{Freistadt} & 
        \multicolumn{2}{c|}{Gmunden}  &  
        \multicolumn{2}{c|}{Grieskirchen} 
        \\
        \hline
        2013/14 & 10,31\% & (11; 151) &       - &          &  3,97\% & (11; 151) &  5,21\% &  (6; 96)  &  6,85\% &  (5; 219) \\
        2014/15 & 13,72\% & (19; 277) &       - &          & 34,72\% & (10; 144) & 32,14\% & (10; 168) & 44,79\% &   (8; 96) \\
        2015/16 & 10,68\% & (22; 468) &       - &          &  3,79\% & (13; 211) &  5,88\% &  (7; 85)  &  5,77\% &  (8; 104) \\
        2016/17 & 13,21\% & (24; 613) & 19,64\% & (5; 168) & 27,13\% & (21; 328) & 16,88\% & (17; 154) & 34,85\% &   (7; 66) \\
        2017/18 &  7,17\% & (20; 502) & 13,33\% & (6; 75)  &  8,11\% & (19; 296) &  6,80\% &  (8; 103) & 10,30\% &  (9; 165) \\
        2018/19 &  8,09\% & (30; 618) &       - &          &  7,28\% & (18; 261) & 16,13\% & (12; 155) & 23,95\% &  (8; 167) \\
        2019/20 &  7,08\% & (21; 466) & 17,86\% & (9; 112) &  9,59\% & (17; 271) & 10,89\% & (21; 202) & 13,33\% & (13; 300) \\
        \hline
        \makecell{Jahre} & 
        \multicolumn{2}{c|}{Kirchdorf an der Krems}    & 
        \multicolumn{2}{c|}{Linz}    & 
        \multicolumn{2}{c|}{Linz-Land} & 
        \multicolumn{2}{c|}{Perg}  & 
        \multicolumn{2}{c|}{Ried  im Innkreis}
        \\
        \hline
        2013/14 &       - &           &       - &           & 13,31\% & (24; 248) &  7,29\% &   (8; 96) &       - &           \\
        2014/15 & 34,74\% &   (7; 95) &       - &           & 25,37\% & (12; 205) & 39,39\% &   (7; 66) & 21,43\% &  (6; 182) \\
        2015/16 &  6,94\% & (10; 620) &  5,26\% &  ( 5; 38) &  8,04\% & (24; 311) &  4,75\% & (14; 316) &  5,80\% & (10; 207) \\
        2016/17 & 18,27\% & (10; 646) & 21,93\% & (12; 114) & 23,72\% & (21; 253) & 15,79\% & (17; 288) &  8,99\% & (11; 278) \\
        2017/18 &  9,16\% &  (9; 262) & 11,81\% & (13; 127) &  7,56\% & (26; 344) &  9,00\% & (14; 289) &  7,24\% &  (7; 152) \\
        2018/19 & 36,36\% &  (8; 627) & 10,10\% &   (7; 99) & 20,60\% & (21; 267) & 18,08\% & (22; 448) & 10,86\% & (13; 359) \\
        2019/20 & 10,40\% &  (9; 202) & 11,54\% &   (6; 52) & 19,09\% & (25; 309) & 14,38\% & (18; 452) & 12,63\% & (11; 293) \\
        \hline
        \makecell{Jahre} & 
        \multicolumn{2}{c|}{Rohrbach}    & 
        \multicolumn{2}{c|}{Schärding}    & 
        \multicolumn{2}{c|}{Steyr} & 
        \multicolumn{2}{c|}{Steyr-Land}  & 
        \multicolumn{2}{c|}{Urfahr-Umgebung}
        \\
        \hline
        2013/14 & 10,16\% & (23; 256) & 15,84\% & (13; 202) & - &  &  8,33\% & (20; 252) & 26,29\% & (18; 251) \\
        2014/15 &         &         - & 26,44\% & (14; 174) & - &  & 22,56\% & (15; 266) & 14,70\% & (16; 279) \\
        2015/16 &  8,48\% & (16; 165) &  2,48\% & (26; 807) & - &  &  7,73\% & (13; 233) &  5,18\% & (21; 560) \\
        2016/17 & 19,25\% & (10; 187) & 14,95\% & (15; 388) & - &  & 20,13\% & (18; 313) & 19,48\% & (31; 775) \\
        2017/18 &       - &           & 14,21\% & (17; 570) & - &  & 18,22\% & (14; 236) &  9,88\% & (46; 688) \\
        2018/19 &  7,43\% &  (6; 202) & 17,82\% & (28; 606) & - &  & 15,19\% & (13; 283) & 19,40\% & (37; 866) \\
        2019/20 &  7,48\% & (12; 254) &  6,26\% & (22; 591) & - &  & 11,29\% & (16; 248) & 14,37\% & (28; 494) \\
        \hline
        \makecell{Jahre} & 
        \multicolumn{2}{c|}{Vöcklabruck}    & 
        \multicolumn{2}{c|}{Wels}    & 
        \multicolumn{2}{c|}{Wels-Land } & 
        \multicolumn{2}{c|}{}  & 
        \multicolumn{2}{c|}{}
        \\
        \hline
        2013/14 &  8,57\% & (14; 245) & - &  &  9,47\% &  (8; 190) &  &  &  &  \\
        2014/15 & 32,67\% & (23; 300) & - &  & 45,07\% &  (9; 213) &  &  &  &  \\
        2015/16 &  5,68\% & (19; 176) & - &  & 21,14\% & (11; 246) &  &  &  &  \\
        2016/17 & 21,39\% & (34; 631) & - &  & 24,10\% &   (8; 83) &  &  &  &  \\
        2017/18 &  7,14\% & (25; 350) & - &  & 19,80\% & (11; 202) &  &  &  &  \\
        2018/19 & 12,28\% & (31; 505) & - &  & 19,08\% & (15; 325) &  &  &  &  \\
        2019/20 & 11,52\% & (35; 651) & - &  & 21,43\% & (14; 308) &  &  &  &  \\        
        \hline
    \end{tabular}
\end{table}
\begin{table}[H]
    \centering
    \caption{Salzburg - Jahresvergleich der Verlustraten in den Bezirken. Verlustrate in \%, (TeilnehmerInnen; eingewinterte Völker). -: weniger als fünf TeilnehmerInnen.}
    \scriptsize
    \setlength{\tabcolsep}{0.5em} % for the horizontal padding
    \label{tab:u:district-salzburg}
    \begin{tabular}{|c|*{6}{rr|}}
        \hline
        \multicolumn{13}{|c|}{Salzburg - Verlustrate \% (Teilnehmer, eingewinterte Völker)} \\    
        \hline
        \makecell{Jahre} & 
        \multicolumn{2}{c|}{Hallein}    & 
        \multicolumn{2}{c|}{Salzburg}    & 
        \multicolumn{2}{c|}{\makecell{Salzburg-\\Umgebung}} & 
        \multicolumn{2}{c|}{\makecell{Sankt Johann \\ im Pongau}}  &
        \multicolumn{2}{c|}{Tamsweg} &  
        \multicolumn{2}{c|}{Zell am See} 
        \\
        \hline
        2013/14 &       - &          &       - &          & 24,62\% & (12; 260) & 17,48\% & (15; 143) &  6,35\% &   (5; 63) & 11,89\% & (11; 143) \\
        2014/15 & 55,77\% & (6; 407) & 13,64\% &  (5; 44) & 24,51\% & (17; 408) & 37,80\% & (12; 127) & 24,00\% &  (6; 100) & 17,46\% & (18; 252) \\
        2015/16 &       - &          &       - &          & 13,52\% & (16; 244) &  6,07\% & (15; 428) &  2,55\% & (10; 157) &  2,74\% & (22; 402) \\
        2016/17 &  8,01\% & (6; 287) &       - &          & 32,89\% & (20; 152) & 31,31\% & (18; 198) & 18,64\% &  (7; 118) &  9,98\% & (23; 601) \\
        2017/18 & 10,16\% & (5; 256) &       - &          &  8,61\% & (20; 267) & 30,26\% &  (9; 228) &  9,43\% &   (6; 53) &  6,09\% & (15; 345) \\
        2018/19 &  5,16\% & (5; 252) & 48,94\% & (7; 235) & 12,53\% & (20; 415) &  8,18\% & (12; 159) &  0,00\% &   (6; 75) & 12,26\% & (23; 367) \\
        2019/20 & 13,10\% & (6;  84) &  5,88\% & (5;  68) & 16,06\% & (18; 330) & 12,84\% & (14; 257) & 10,74\% & (10; 121) &  8,03\% & (23; 361) \\
        \hline
    \end{tabular}
\end{table}
\begin{table}[H]
    \centering
    \caption{Steiermark - Jahresvergleich der Verlustraten in den Bezirken. Verlustrate in \%, (TeilnehmerInnen; eingewinterte Völker). -: weniger als fünf TeilnehmerInnen. *: Bezirksfusionen in der Steiermark 2013 (Bruck und Mürzzuschlag -> Bruck-Mürzzuschlag, Fürstenfeld und Hartberg -> Hartberg-Fürstenfeld, Feldbach und Radkersburg -> Südoststeiermark).}
    \scriptsize
    \setlength{\tabcolsep}{0.5em} % for the horizontal padding
    \label{tab:u:district-steiermark}
    \begin{tabular}{|c|*{5}{rr|}}
        \hline
        \multicolumn{11}{|c|}{Steiermark - Verlustrate \% (Teilnehmer, eingewinterte Völker)} \\    
        \hline
        \makecell{Jahre} & 
        \multicolumn{2}{c|}{Bruck}    & 
        \multicolumn{2}{c|}{Bruck-Mürzzuschlag}    & 
        \multicolumn{2}{c|}{Deutschlandsberg} & 
        \multicolumn{2}{c|}{Feldbach}  &  
        \multicolumn{2}{c|}{Fürstenfeld} 
        \\
        \hline
        2013/14 & 3,97\% & (12; 126) &       * &           & 13,46\% &   (5; 52) & 7,57\% & (12; 383) & - & - \\
        2014/15 &      * &           & 21,23\% & (25; 405) & 14,15\% &  (9; 205) &      * &           & * &   \\
        2015/16 &      * &           & 12,93\% & (21; 263) &  9,09\% &  (8; 154) &      * &           & * &   \\
        2016/17 &      * &           & 24,94\% & (23; 405) & 24,70\% & (12; 247) &      * &           & * &   \\
        2017/18 &      * &           & 10,95\% & (20; 210) &  4,99\% & (13; 341) &      * &           & * &   \\
        2018/19 &      * &           & 13,59\% & (19; 390) & 17,60\% & (20; 392) &      * &           & * &   \\
        2019/20 &      * &           & 11,63\% & (25; 301) &  9,47\% & (10; 169) &      * &           & * &   \\
        \hline
        \makecell{Jahre} & 
        \multicolumn{2}{c|}{Graz}    & 
        \multicolumn{2}{c|}{Graz-Umgebung}    & 
        \multicolumn{2}{c|}{Hartberg} & 
        \multicolumn{2}{c|}{Hartberg-Fürstenfeld}  & 
        \multicolumn{2}{c|}{Leibnitz}
        \\
        \hline
        2013/14 & 23,81\% &   (8; 42) & 10,06\% & (19; 318) & 10,44\% & (6; 249) &       * &           & 10,18\% & (14; 285) \\
        2014/15 & 18,97\% & (11; 195) & 29,59\% & (22; 365) &       * &          & 43,97\% & (11; 614) & 27,04\% & (18; 196) \\
        2015/16 & 22,41\% &  (11; 58) &  6,61\% & (28; 363) &       * &          &  5,92\% & (16; 608) & 11,28\% & (23; 390) \\
        2016/17 & 20,69\% & (13; 145) & 21,73\% & (41; 543) &       * &          & 13,51\% & (13; 259) & 17,52\% & (21; 314) \\
        2017/18 & 10,61\% & (16; 179) &  9,47\% & (32; 486) &       * &          &  8,33\% & (12; 396) & 10,93\% & (24; 549) \\
        2018/19 &  5,34\% &  (6; 131) &  6,65\% & (34; 722) &       * &          &  6,42\% & (12; 654) & 14,85\% & (17; 303) \\
        2019/20 & 16,67\% & (12; 168) & 14,69\% & (32; 708) &       * &          &  5,96\% & (16; 923) &  9,14\% & (21; 339) \\ 
        \hline
        \makecell{Jahre} & 
        \multicolumn{2}{c|}{Leoben}    & 
        \multicolumn{2}{c|}{Liezen}    & 
        \multicolumn{2}{c|}{Murau} & 
        \multicolumn{2}{c|}{Murtal}  & 
        \multicolumn{2}{c|}{Mürzzuschlag}
        \\
        \hline
        2013/14 &       - &           & 16,30\% &  (7; 184) &  6,19\% & (17; 452) &       - &           & 5,48\% & (6; 73) \\
        2014/15 &       - &           & 10,59\% &  (9; 255) & 10,36\% &  (8; 193) &  8,40\% & (10; 119) &      * &         \\
        2015/16 &       - &           &  9,41\% & (18; 372) &  5,96\% & (10; 235) &  6,25\% &   (6; 64) &      * &         \\
        2016/17 & 26,98\% &  (8; 441) & 16,45\% & (24; 614) & 13,14\% &  (8; 312) &  8,82\% & (11; 170) &      * &         \\
        2017/18 &  5,09\% &  (7; 216) &  7,69\% & (16; 351) &  6,50\% &  (9; 323) & 13,07\% & (11; 176) &      * &         \\
        2018/19 & 18,84\% & (10; 207) & 14,37\% & (21; 508) & 22,48\% & (14; 347) & 22,90\% &  (5; 131) &      * &         \\
        2019/20 & 12,09\% &  (7; 273) & 10,86\% & (26; 534) & 11,16\% &  (7; 251) &  9,77\% &  (9; 133) &      * &         \\
        \hline
        \makecell{Jahre} & 
        \multicolumn{2}{c|}{Radkersburg}    & 
        \multicolumn{2}{c|}{Südoststmk.}    & 
        \multicolumn{2}{c|}{Voitsberg} & 
        \multicolumn{2}{c|}{Weiz}  & 
        \multicolumn{2}{c|}{}
        \\
        \hline
        2013/14 & - &  &       * &           &       - &           &  7,47\% & (17; 522) &&\\
        2014/15 & * &  & 19,60\% & (17; 352) &       - &           & 28,42\% & (15; 366) &&\\
        2015/16 & * &  & 15,71\% & (18; 350) &       - &           &  3,89\% & (13; 386) &&\\
        2016/17 & * &  & 12,95\% & (23; 448) & 38,97\% & (10; 195) & 13,65\% & (18; 740) &&\\
        2017/18 & * &  &  8,22\% & (15; 304) & 12,00\% &  (7; 150) &  7,50\% & (19; 533) &&\\
        2018/19 & * &  & 12,04\% & (20; 382) & 10,00\% & (10; 190) &  8,56\% & (23; 841) &&\\
        2019/20 & * &  & 16,63\% & (24; 457) &  6,92\% & (10; 159) & 11,68\% & (21; 334) &&\\
        \hline
    \end{tabular}
\end{table}
\begin{table}[H]
    \centering
    \caption{Tirol - Jahresvergleich der Verlustraten in den Bezirken. Verlustrate in \%, (TeilnehmerInnen; eingewinterte Völker). -: weniger als fünf TeilnehmerInnen.}
    \scriptsize
    \setlength{\tabcolsep}{0.5em} % for the horizontal padding
    \label{tab:u:district-tirol}
    \begin{tabular}{|c|*{5}{rr|}}
        \hline
        \multicolumn{11}{|c|}{Tirol - Verlustrate \% (Teilnehmer, eingewinterte Völker)} \\    
        \hline
        \makecell{Jahre} & 
        \multicolumn{2}{c|}{Imst}    & 
        \multicolumn{2}{c|}{Innsbruck}    & 
        \multicolumn{2}{c|}{Innsbruck Land} & 
        \multicolumn{2}{c|}{Kitzbühel}  &  
        \multicolumn{2}{c|}{Kufstein} 
        \\
        \hline
        2013/14 &       - &           & 17,24\% &   (5; 29) &  7,81\% & (20; 320) &  5,76\% &  (9; 243) & 22,26\% & (27; 539) \\
        2014/15 &       - &           & 24,53\% &   (7; 53) & 28,07\% & (17; 171) & 24,00\% &    (5;75) & 40,30\% & (26; 335) \\
        2015/16 &  5,43\% & (10; 184) &  5,07\% & (16; 296) &  6,10\% & (31; 426) &  2,88\% & (14; 208) &  3,85\% & (14; 260) \\
        2016/17 & 40,58\% &  (9, 313) &       - &           & 17,85\% & (33; 521) & 10,26\% & (18; 273) & 31,85\% & (12; 248) \\
        2017/18 &  9,84\% &  (9; 244) &  6,29\% &  (7; 159) & 12,27\% & (35; 481) &  6,55\% & (13; 168) &  5,58\% & (15; 215) \\
        2018/19 &  4,87\% &  (7; 226) &  7,79\% &   (8; 77) & 13,67\% & (34; 490) &  8,15\% & (21; 270) & 13,81\% & (20; 572) \\
        2019/20 &  9,52\% & (13; 420) &  8,54\% &  (11; 82) & 10,17\% & (38; 885) &  7,16\% & (16; 447) & 17,99\% & (22; 289) \\
        \hline
        \makecell{Jahre} & 
        \multicolumn{2}{c|}{Landeck}    & 
        \multicolumn{2}{c|}{Lienz}    & 
        \multicolumn{2}{c|}{Reutte} & 
        \multicolumn{2}{c|}{Schwaz}  & &  \\
        \hline
        2013/14 &       - &           &  3,05\% &  (7; 262) &       - &           & 21,07\% &  (7; 261) &&\\
        2014/15 & 20,62\% &   (7; 97) & 19,56\% & (12; 409) &       - &           & 32,10\% & (17; 486) &&\\
        2015/16 &  5,08\% & (12; 177) &  4,62\% &  (9; 238) &  9,56\% & (20; 272) &  3,80\% & (22; 526) &&\\
        2016/17 & 11,43\% & (10; 175) &  9,42\% & (12; 276) & 23,29\% & (13; 249) & 46,85\% & (18; 444) &&\\
        2017/18 & 18,87\% &  (8; 106) & 15,98\% &  (8; 338) & 14,38\% & (14; 313) & 10,36\% & (11; 251) &&\\
        2018/19 & 19,69\% & (10; 127) & 11,33\% & (11; 450) &  8,62\% & (14; 290) & 10,86\% & (15; 442) &&\\
        2019/20 & 15,83\% &  (6; 120) & 23,48\% &  (9; 328) & 15,03\% & (20; 386) & 11,05\% & (18; 742) &&\\
        \hline
    \end{tabular}
\end{table}
\begin{table}[H]
    \centering
    \caption{Vorarlberg - Jahresvergleich der Verlustraten in den Bezirken. Verlustrate in \%, (TeilnehmerInnen; eingewinterte Völker). -: weniger als fünf TeilnehmerInnen.}
    \scriptsize
    \setlength{\tabcolsep}{0.5em} % for the horizontal padding
    \label{tab:u:district-vorarlberg}
    \begin{tabular}{|c|*{4}{rr|}}
        \hline
        \multicolumn{9}{|c|}{Vorarlberg - Verlustrate \% (Teilnehmer, eingewinterte Völker)} \\    
        \hline
        \makecell{Jahre} & 
        \multicolumn{2}{c|}{Bludenz}    & 
        \multicolumn{2}{c|}{Bregenz}    & 
        \multicolumn{2}{c|}{Dornbirn} & 
        \multicolumn{2}{c|}{Feldkirch}
        \\
        \hline
        2013/14 &  9,42\% &  (9; 138) & 16,16\% & (20; 359) & 31,52\% &   (6; 92) & 23,44\% & (14; 128) \\
        2014/15 & 20,65\% & (12; 155) & 20,35\% & (27; 285) & 39,62\% &  (9; 106) & 40,37\% & (19; 161) \\
        2015/16 &  6,80\% & (16; 147) &  4,86\% & (14; 288) &  3,39\% &   (8; 59) &  8,57\% & (12; 105) \\
        2016/17 & 30,13\% & (62; 707) & 22,01\% & (69; 977) & 61,92\% & (23; 239) & 48,07\% & (52; 491) \\
        2017/18 &  4,24\% & (29; 377) & 11,54\% & (38; 797) & 12,10\% & (14; 124) & 12,86\% & (24; 280) \\
        2018/19 & 16,39\% & (49; 659) & 17,86\% & (69; 980) & 23,20\% & (22; 250) & 16,71\% & (37; 431) \\
        2019/20 & 10,73\% & (52; 578) &  8,35\% & (39; 491) & 10,74\% & (13; 121) &  9,39\% & (30; 309) \\
        \hline
    \end{tabular}
\end{table}

\begin{table}[H]
    \centering
    \caption{Wien - Jahresvergleich der Verlustraten in den Bezirken. Verlustrate in \%, (TeilnehmerInnen; eingewinterte Völker). -: weniger als fünf TeilnehmerInnen.}
    \scriptsize
    \setlength{\tabcolsep}{0.5em} % for the horizontal padding
    \label{tab:u:district-wien}
    \begin{tabular}{|c|*{1}{rr|}}
        \hline
        \makecell{Jahre} & 
        \multicolumn{2}{|c|}{\makecell{Wien - Verlustrate \% \\ (Teilnehmer, eingewinterte Völker)}} \\    
        \hline
        2013/14 & 19,18\% &   (32; 318) \\
        2014/15 & 51,53\% &   (66; 458) \\
        2015/16 & 11,48\% &   (41; 479) \\
        2016/17 & 24,76\% &   (70; 832) \\
        2017/18 & 12,59\% &   (59; 945) \\
        2018/19 & 19,58\% & (78; 1.083) \\
        2019/20 & 20,07\% & (92; 1.196)\\
        \hline
    \end{tabular}
\end{table}



\myfig{project-U-wintersterblichkeit/figures/plot_factor_yield_map} % Pfad
{width=0.8\textwidth} % Größe Relativ zu Text Breite
{Die Karte zeigt die grobe Position der Hauptüberwinterungsstände der ImkerInnen mit der jeweiligen Tracht (ohne WanderimkerInnen). Hintergrundfarbe ist die mittlere Verlustrate der Bezirke, weiße Bezirke haben weniger als 5 Meldungen.} % Text unterhalb der Grafik
{Optionaler Kurz Titel} % Optional Kurz Überschrift
{fig:u:factor:yield_map} % Label zum Verweisen im Text