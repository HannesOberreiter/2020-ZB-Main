\section{Methodik}

\subsection{Datenerhebung}

Die Erhebungen der Winterverluste von Bienenvölkern in Österreich wird jährlich zwischen Februar und Mai mittels eines von COLOSS festgelegten Fragebogens durchgeführt. Der Fragebogen, der auch Fragen zur Betriebsweise inkludiert, ist so aufgebaut, dass ein breites Spektrum an Informationen mit einem für die ImkerInnen relativ geringen zeitlichen Aufwand abgefragt wird. Es werden außerdem von jeder Imkerei nur einfach festzustellende Fakten abgefragt, die ohne technische Hilfsmittel beantwortet werden können. Im Fragebogen wird die Anzahl an eingewinterten und die Anzahl an verlorenen Bienenstöcken in drei Kategorien abgefragt (verloren (tote Völker, leere Beuten), verloren durch Elementarschaden, weisellos oder drohnenbrütig), sowie damit in Zusammenhang stehende mögliche Risikofaktoren. Aus diesen Angaben können dann die  Winterverlustraten ermittelt werden.
\newline
Die Teilnahme kann entweder anonym oder nicht-anonym, durch das Hinterlassen von Kontaktdaten, erfolgen. Die Papierfragebögen werden per Post an Imkervereine versendet und bei Veranstaltungen verteilt. Die Umfrage ist online auf \url{www.bienenstand.at} verfügbar sowie, in einer Kurzversion, in der April-Ausgabe der Zeitschrift \enquote{Biene Aktuell} veröffentlicht. Dadurch sollte eine möglichst große Reichweite erzielt werden und auch ImkerInnen ohne Internetzugang haben somit die Chance zur Teilnahme \citep{vanderzee2013}.
\newline
Die im Fragebogen gestellten Fragen beziehen sich auf Standort und Größe der Imkerei, die Anzahl der verlorenen Völker, sowie auch den möglichen Transport der Völker, die Betriebsweise, bestimmte Nahrungsquellen der Bienen und auch die Behandlung gegen die Varroamilbe. Die Fragen beinhalteten: Anzahl an eingewinterten Völkern mit junger Königin (begattet 2019), beobachtete Königinnen, Probleme in der Sammelsaison im Vergleich zu bisherigen Erfahrung, Offener Gitterboden im Winter, Isolierte Beuten im Winter, Kunststoff Beuten, Zertifizierte Bioimkerei, Bienen aus Zuchtprogramm für Varroatoleranz, Kleine Brutzellen (5,1mm oder weniger), Naturwabenbau (ohne Mittelwand), Wachskauf (kein eigener Wachskreislauf), Anteil an Erneuerten Brutwaben (in relativen Prozentgruppen), die Häufigkeit der Beobachtung von verkrüppelte Bienen in der Sammelsaison (häufig, selten, keine, weiß nicht) und Zusammenlegung von schwachen Völkern vorm Winter. Die Auswahl der Fragen erfolgte durch Vorschläge von ImkerInnen aus vorherigen Umfragen oder Diskussionsgruppen und wurden durch COLOSS zum international gebraucht akzeptiert.
\newline
Bei Winterverlusten wurde zwischen Völkerverlusten und Völkern mit Königinnen-Problemen unterschieden. Für die Gesamtverlustrate wurden beide addiert. Des weiteren wurden auch in Bezug auf die verlorenen Völker leicht erkennbare Symptome abgefragt. Dabei konnten die ImkerInnen berichten, ob sie viele tote Bienen im oder vor dem Volk hatten, ob sie keine oder nur wenige tote Bienen im oder vor dem Volk hatten, ob tote Bienen in Zellen gefunden wurden, ob die Bienen kein Futter im Stock hatten (verhungert sind), verhungert sind, obwohl genug Futter im Volk war (Futterabriss), ob sie keines der oben genannten oder unbekannte Symptome hatten und letztlich ob sie, unabhängig vom Schadbild, aufgrund von Elementarschäden (Flut, Vandalismus, Specht, etc.) Völker verloren hatten. Die Anzahl an Völkern die durch Elemtarschäden verloren gingen wurde in weiterer Folge nicht in die Verlustraten und Risiko Analyse aufgenommen, da es sich hierbei nicht strikt um biologische (zum Beispiel Alter der Königin) oder Risikofaktoren durch Betriebsweisen handelt. Seit der Umfrage 2017/18 werden die TeilnehmerInnen außerdem zum Auftreten von verkrüppelten Flügeln befragt, einem möglichen Hinweis auf eine Infektion mit dem Flügeldeformationsvirus (DWV).
\newline
Da es zwei Versionen von Fragebögen gab, wurden bei der längeren Version noch Fragen zur Erneuerung der Waben und zum Alter der Königinnen gestellt beziehungsweise wollten wir wissen, welche Auswirkungen diese Faktoren nach Einschätzung der ImkerInnen auf die Völkerverluste hatte. Zur Auswahl standen folgende Antwortmöglichkeiten: gleich, besser, schlechter oder weiß nicht. Zusätzliche Fragen bezogen sich auf Trachtquellen (Sonnenblumen, Waldtracht, Spätblühende Zwischenfrüchte, Waldtracht mit Melezitose, Raps, Mais) und die Varroabehandlungsmethode sowie deren Zeitraum und ob Varroa-Befall kontrolliert wurde und auch hier gegebenenfalls der Zeitraum. 
\newline
Unser Ziel war es, so viele Daten wie möglich zu sammeln um eine gute Repräsentation der Situation in Österreich zu bekommen. Alle beteiligten ImkerInnen haben an der Umfrage freiwillig teilgenommen. Eine Verpflichtung bestand lediglich für jene Imkereien, die (freiwillig) am Österreichischen Bienengesundheitsprogramm (ÖBGP) teilnahmen. Alle Imkereien unabhängig von der Größe ihrer Betriebe konnten mitmachen. In die Auswertung wurden nur jene Fragebögen aufgenommen, aus denen die Winterverlustrate berechnet werden konnte. Fehlerhafte oder unvollständige Fragebögen wurden nicht ausgewertet. Dazu zählten beispielsweise Fragebögen, in denen die Angaben zum Standort fehlten und/oder keine Angaben über die Zahl der eingewinterten oder überlebenden Völker gemacht wurden.  Offensichtliche Duplikate wurden nach eingehender Prüfung ebenfalls entfernt.
\newline
Zur Wahrung der persönlichen Daten der TeilnehmerInnen, wurde eine Datenschutzerklärung zwischen den an der Auswertung beteiligten Personen erstellt. Alle persönlichen Daten wurden bei der Auswertung entfernt. Wenn eine Kontaktmöglichkeit angegeben wurde, wurde diese nur zur Nachfrage verwendet um fehlende oder fehlerhafte Daten zu korrigieren.

\subsection{Datenvalidierung und Fehlerkontrolle}

Wenn ImkerInnen über einen Papierfragebogen teilnahmen, wurden die Daten manuell in eine Microsoft Excel-Datei übertragen, in welcher alle Umfragedaten gesammelt wurden. Die automatische Überprüfungen der Datenqualität erfolgte mit Formeln in Excel, um Verarbeitungsfehler zu minimieren und mögliche ungültige Antworten hervorzuheben, zum Beispiel mehr Kolonien verloren als existent, wie in \cite{vanderzee2013, brodschneider2013} beschrieben. Diese widersprüchlichen Einträge oder Mehrfacheinträge desselben Imkers/derselben Imkerin wurden entfernt. Fehlende Antworten und eine geringere Anzahl an Fragen im Fragebogen der Zeitschrift \enquote{Biene Aktuell} führten zu einer verringerten Anzahl von Antworten auf manche Fragen. Wenn die Anzahl an Antworten für einen Faktor für eine statistische Analyse nicht ausreichte, wurden die Daten nicht verwendet.
\newline
Die TeilnehmerInnen gaben nur die grobe Position des Hauptbienenstandes zur Überwinterung bekannt, d.h. mindestens Bezirk und Postleitzahl. Die Geolokalisierungen für die Koordinaten wurden mit Python und dem Google Webservice \emph{Geocoding} erstellt und, falls dies fehlschlug, durch eine manuelle Suche ergänzt. Um falsche Geolokalisierungen zu minimieren wurden die resultierenden Standorte  auf einer Bezirkskarte aufgezeichnet und auf korrekte Zuordnung zu den Bezirken getestet, wie von den TeilnehmerInnen der Umfrage angegeben. Die Höhenschätzung für die Standorte wurde mit dem Topografiemodell SRTM-3v4 über einen Webdienst durchgeführt \citep{geonames}.

\subsection{Statistik}

Alle Antworten wurden in einer Excel-Datenbank zusammengeführt, um die Auswertung der gesamten Daten, das heißt sowohl jener aus den Online-Fragebögen als auch der von den Papierfragebögen, durchführen zu können. Die statistische Software R \citep{rcoreteam2020} und das Paket tidyverse \citep{wickham2019}, wurden für die Datenanalyse und die Erstellung von Grafiken verwendet. Der Code, inklusive einer Auflistung der restlichen hier nicht erwähnten Pakete, ist auf GitHub unter einer MIT-Lizenz verfügbar\footnote{\url{https://github.com/HannesOberreiter/coloss_honey_bee_colony_losses_austria} und archiviert \cite{HannesOberreiter2019}}. Die Versionskontrolle der einzelnen Pakete ist in mit einem renv.lockfile gesichert. Eine einfache Onlineversion wurde auch öffentlich zugänglich gemacht: \url{http://bienenstand.at/uncategorized/confidence/}.
\newline
Schätzung der Verlustraten basiert auf der jeweiligen Gesamtzahl der verloren Völkern und wird nicht auf Betriebsebene berechnet sondern immer Anhand der verglichenen Gruppen. Die Konfidenzintervalle (KI) wurden mit Hilfe eines Generalisierten Linearen Modells (GZLM) mit quasi-binomialer Verteilung und der Linkfunktion \enquote{logit} berechnet. Die Erstellung erfolgte nach wissenschaftlich etablierten Methoden \citep{vanderzee2013}. Die so berechnete Verlustschätzung ist als umrahmter Fehlerbalken dargestellt, wobei die Box den 95\% CI darstellt. Die Verluste von Bienenvölkern durch Elementarschaden wurde in der Auswertung nicht inkludiert, außer wenn explizit im Text erwähnt.
\newline
Der Großteil der Analyse wurde als einzelner Faktor mit zwei möglichen unterschiedlichen Gruppen durchgeführt, d.h. \enquote{Ja} und \enquote{Nein} Fragen. Um signifikante Unterschiede zu identifizieren wurden die Konfidenzintervalle zwischen den Variablen (Faktoren) verglichen. Wenn sich die Konfidenzintervalle nicht überlappten, zählten wir dies als signifikanten Unterschied. Wenn die Überlappung sehr gering und es ein Vergleich zwischen zwei Gruppen war, wurde die Null Abweichung abzüglich der Abweichung der Residuen im Modell getestet. Wenn dieser Wert signifikant von Null abwich, hatte der analysierte Faktor im Modell einen signifikanten Einfluss auf das Überleben der Bienenvölker (ANOVA mit $\chi^{2}$ Abweichung, \textit{p}<0,05).
\newline
In unserer Auswertung wird keine Korrektur für die Kumulierung des Alphafehlers vorgenommen, da in unserem Fragebogen verschiedene Hypothesen getestet werden und mögliche Überschneidungen der Hypothesen nur schwer einzuteilen sind. Deswegen versuchen wir über mehrere Jahre die gleichen Fragen beizubehalten um Irrtumswahrscheinlichkeiten in der Statistik zu vermindern.
\newline
Für die Analyse der Kombination verschiedener Varroa Behandlungsmethoden wurde jeweils ein Histogramm erstellt und in Frühling, Sommer und Winter gruppiert. Danach wurde ein Vektor aller möglichen gruppierten Methoden erstellt. Um statistisch relevante Ergebnisse zu erhalten wurden nur Methoden mit mindestens 15 Antworten verwendet und die Methode \enquote{Entfernung der Drohnenbrut} wurde nicht inkludiert. Um alle möglichen Kombinationen zu erstellen wurde die R Funktion \enquote{combn} in Anlehnung an den Algorithmus von \cite{nijenhuis1978} verwendet. Das Maximum der Matrix wurde mit drei verschiedenen Methoden pro Zeile festgelegt. Dann wurde jede Reihe mit weniger als 15 TeilnehmerInnen für die jeweilige Kombination entfernt, bevor die Verlustrate wie zuvor beschrieben berechnet wurde.
\newline
Die Standortkarten wurden mit \cite{rcoreteam2020} und Shapefiles unter einer \enquote{kreativen gemeinsamen Lizenz} erstellt\footnote{\url{https://www.data.gv.at/katalog/dataset/bev_verwaltungsgrenzenstichtagsdaten150000} 01.04.2019, Creative Commons Namensnennung 4.0 International }. Cluster der Bienenstandorte auf der Karte wurden mit der k-Means-Cluster-Suchmethode erstellt. Um die Genauigkeit in Gebieten mit geringerer Dichte zu verbessern wurden die resultierenden Cluster die nur einen einzigen Bienenstand oder eine geringe Anzahl an Bienenständen und hoher Quadratsummen im Cluster aufwiesen entfernt und die ursprüngliche Geolokalisierung für die Karten verwendet.
