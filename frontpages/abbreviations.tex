\phantomsection
%\addtocontents{toc}{\vspace{\normalbaselineskip}}
\addcontentsline{toc}{chapter}{Abkürzungsverzeichnis}
\subsection*{Abkürzungsverzeichnis}
\label{cha:abbr}

\begin{description}
    \item[ABCDS]
    Die fünf Viren ABPV, BQCV, CBPV, DWV und SBV zusammengefasst
    \item[ABPV]
    Akute Bienenparalyse Virus
    \item[AGES]
    Agentur für Gesundheit und Ernährungssicherheit
    \item[BIEN]
    Abteilung für Bienenkunde und Bienenschutz der AGES
    \item[BQCV]
    Schwarzes Königinnenzellen-Virus
    \item[BMASGK]
    Bundesministerium für Arbeit, Soziales, Gesundheit und Konsumentenschutz \\ (Stand 31.10.2017)
    \item[CBPV]
    Chronische Bienenparalyse-Virus
    \item[CI]
    Konfidenzintervall
    \item[COLOSS]
    prevention of honey bee COlony LOSSes
    \item[DWV-A/-B]
    Flügeldeformationsvirus Typ A bzw. Typ B
    \item[EU-RL]
    EU-Referenzlabor
    \item[GLZM]
    Generalisiertes Lineares Modell
    \item[IAPV]
    Israelisches Akute Bienenparalyse-Virus
    \item[KBV]
    Kashmir-Bienenvirus
    \item[LOD und LOQ]
    Limit of Detection, Limit of Quantifikation
    \item[MOBI]
    Abteilung für Molekularbiologie des Institutes für veterinärmedizinische Untersuchungen Mödling der AGES
    \item[Q1]
    Unteres Quartil, 25\% der Werte eines Datensatzes liegen darunter
    \item[Q3]
    Oberes Quartil, 25\% der Werte eines Datensatzes liegen darüber
    \item[RT-qPCR]
    Reverse Transkriptase quantitative PCR
    \item[SBV]
    Sackbrutvirus
    \item[SOP]
    Standard operating procedure
    \item[VDV-1]
    \textit{Varroa destructor} Virus-1 (=DWV-B)
\end{description}