\phantomsection
%\addtocontents{toc}{\vspace{\normalbaselineskip}}
\addcontentsline{toc}{chapter}{Veröffentlichungen und Vortragstätigkeit im Zeitraum der Projektdauer (01.09.2018 bis 29.08.2019)}
\subsection*{Veröffentlichungen und Vortragstätigkeit im Zeitraum der Projektdauer (01. September 2018 bis 29. August 2019)}
\label{cha:publications}

\underline{Projektbezogene Publikationen:}


\begin{itemize}
   
    \item 
    \fullcite{morawetzHealthStatusHoney2019}

\end{itemize}

\underline{Projektbezogene Kongressbeiträge (Poster und Vorträge):}

\begin{itemize}

    \item 
    Morawetz L, Köglberger H, Derakhshifar I, Mayr J, Moosbeckhofer R, Crailsheim, K. Health status and factors identified for winter losses of honey bee colonies. Vortrag: Eurbee 8, Gent 2018
    
    \item 
    Morawetz L, Griesbacher A, Kuchling S, Mayr J, Brodschneider R, Crailsheim K, Moosbeckhofer R. Österreichisches Bienenbrot-Monitoring auf Pestizidbelastung in unterschiedlichen Landnutzungstypen (Projekt Zukunft Biene). Poster: 66. Jahrestagung der Arbeitsgemeinschaft der Institute für Bienenforschung e.V; Frankfurt 2019

    \item 
    Morawetz L, Steinrigl A, Köglberger H, Derakhshifar I, Griesbacher A, Moosbeckhofer R, Crailsheim, K. Bienen und ihre Viren - Gesundheitsmonitoring in Österreichs Bienenvölkern („Zukunft Biene 2“). Poster: 5. Österreichische Citizen Science Konferenz, Obergurgl 2019

    \item 
    Seitz K, Rümenapf T, Dinhopl N., Plevka P., Dikunová A., Lamp B. (2019) First molecular clone of Chronic Bee Paralysis Virus (CBPV); Poster: 29. Jahrestagung der Gesellschaft für Virologie (GfV), Düsseldorf 2019

    \item 
    Seitz K, Power K, Rümenapf T., Buczolich K., Dinhopl N., Lamp B. (2019) First molecular clone of Chronic Bee Paralysis Virus (CBPV); Vortrag; Honeybee Health Symposium Apimondia 2019, Rom

\end{itemize}


\underline{Projektbezogene Vorträge Science to Stakeholders}

\begin{itemize}
    \item 
    Morawetz Linde „Beobachtungsstudie des Projekts Zukunft Biene: Ursachenforschung zur Wintersterblichkeit“ Wanderlehrerfortbildungstagung des Österr. Imkerbundes, Graz, 27.10.2018

    \item 
    Morawetz L „Beobachtungsstudie und Posthoc Untersuchungen: Einflussfaktoren auf den Überwinterungserfolg“ Runder Tisch Zukunft Pflanzenbau, AGES, Wien, 8.11.2018

    \item 
    Moosbeckhofer R „Ergebnisse der Beobachtungs- und Post hoc-Studie zu Einflussfaktoren auf den Überwinterungserfolg von Bienenvölkern“ Fachtagung Österreichischer Erwerbsimkerbund, Premstätten 24.02.2019

    \item 
    Morawetz L „Zukunft Biene 2 – Virenmonitoring“ Gesundheitsreferententagung des Österreichischen Imkerbundes, AGES, Wien, 09.05.2019

\end{itemize}

\underline{Projektbezogene Beiträge in den Medien}

\begin{itemize}

    \item
    Crailsheim K, Brodschneider R (2019) Ursachen der Verluste an Honigbienen in Österreich. Der Pflanzenarzt, 5: 27-29.

    \item
    Brodschneider R, Moosbeckhofer R, Crailsheim K (2019) Winterverluste 2017/18 und Aufruf zur Teilnahme an der Untersuchung 2019. Bienen aktuell, 4: 20-23.

    \item
    Lamp B, Seitz K (2019) Ursachen und Folgen des Bienensterbens. Reportage. ServusTV. \url{https://www.servus.com/tv/ursachen-und-folgen-des-bienensterbens}

\end{itemize}

\underline{Projektbezogene Nennungen in den Medien}

\begin{itemize}

    \item
    studium.at. 13.06.2019. Bienensterben: Winterverluste laut Studie an Uni Graz im Mittelwert.

    \item
    orf.at. 13.06.2019. 15 Prozent der Bienenvölker starben im Winter. \url{https://orf.at/stories/3126641/}

    \item
    orf.at. 13.06.2019. Bienensterben: Winterverluste im Mittelwert. \url{https://steiermark.orf.at/stories/3000281/}

    \item
    citizen-science.at. 13.06.2019. Virenmonitoring. \url{https://www.citizen-science.at/citizen-scienceprojekte/item/448-virenmonitoring}

    \item
    derstandard.at. 13.06.2019. So schlimm steht es wohl doch nicht um unser liebstes Insekt. \url{https://www.derstandard.at/story/2000104810957/so-schlimm-steht-es-um-unser-liebstes-insekt-wohl-doch}

    \item
    krone.at. 13.06.2019. Bienensterben im Winter heuer im Mittelwert. \url{https://www.krone.at/1941068}

    \item
    wienerzeitung.at. 13.06.2019. Bienenverluste im Mittelwert. \url{https://www.wienerzeitung.at/nachrichten/wissen/natur/2013822-Bienenverluste-im-Mittelwert.html}

    \item
    Kronen Zeitung. 14.06.2019. Unseren Bienen geht es besser.

    \item
    blickinsland.at. 14.06.2019. Bienenverluste im Winter durchschnittlich. \url{https://blickinsland.at/bienenverlusteim-winter-durchschnittlich/}

    \item
    Der Standard. 14.06.2019. 15 Prozent Winterverlust bei Bienenvölkern.

    \item
    Tiroler Tageszeitung. 14.06.2019. Weniger Verluste bei Bienen.

    \item
    TT Kompakt. 14.06.2019. Harter Winter für Bienen.

    \item
    Wiener Zeitung. 14.06.2019. Verluste an Bienen im mittleren Bereich.

    \item
    Kleine Zeitung. 14.06.2019. Durchschnittliche Bienenverluste im Winter

\end{itemize}