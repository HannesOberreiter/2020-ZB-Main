\phantomsection
%\addtocontents{toc}{\vspace{\normalbaselineskip}}
\addcontentsline{toc}{chapter}{Veröffentlichungen und Vortragstätigkeit im Zeitraum der Projektdauer (22.12.2017 bis 20.07.2020)}
\subsection*{Veröffentlichungen und Vortragstätigkeit im Zeitraum der Projektdauer (22. Dezember 2017 bis 20. Juli 2020)}
\label{cha:publications}

\underline{Projektbezogene Publikationen:}


\begin{itemize}
   
    \item 
    \fullcite{morawetz2019}
    \item
    \fullcite{brodschneider2019a}    
    \item
    \fullcite{oberreiter2020}

\end{itemize}

\underline{Projektbezogene Kongressbeiträge (Poster und Vorträge):}

\begin{itemize}

    \item 
    Morawetz L, Köglberger H, Derakhshifar I, Mayr J, Moosbeckhofer R, Crailsheim, K. Health status and factors identified for winter losses of honey bee colonies. Vortrag: Eurbee 8, Gent 2018
    
    \item 
    Morawetz L, Griesbacher A, Kuchling S, Mayr J, Brodschneider R, Crailsheim K, Moosbeckhofer R. Österreichisches Bienenbrot-Monitoring auf Pestizidbelastung in unterschiedlichen Landnutzungstypen (Projekt Zukunft Biene). Poster: 66. Jahrestagung der Arbeitsgemeinschaft der Institute für Bienenforschung e.V; Frankfurt 2019

    \item 
    Morawetz L, Steinrigl A, Köglberger H, Derakhshifar I, Griesbacher A, Moosbeckhofer R, Crailsheim, K. Bienen und ihre Viren - Gesundheitsmonitoring in Österreichs Bienenvölkern („Zukunft Biene 2“). Poster: 5. Österreichische Citizen Science Konferenz, Obergurgl 2019

    \item 
    Seitz K, Rümenapf T, Dinhopl N., Plevka P., Dikunová A., Lamp B. (2019) First molecular clone of Chronic Bee Paralysis Virus (CBPV); Poster: 29. Jahrestagung der Gesellschaft für Virologie (GfV), Düsseldorf 2019

    \item 
    Seitz K, Power K, Rümenapf T., Buczolich K., Dinhopl N., Lamp B. (2019) First molecular clone of Chronic Bee Paralysis Virus (CBPV); Vortrag; Honeybee Health Symposium Apimondia 2019, Rom
    
    \item
    Morawetz L, Steinrigl A, Köglberger H, Griesbacher A, Mayr J, Brodschneider R, Crailsheim K, Moosbeckhofer R (2019); Winter colony losses 2015/16 in Austria: are there correlations with pests, pathogens and pesticide residuals?; Poster; 08/SEP - 12/SEP/2019; Montréal, Kanada; 46th Apimondia – International Apicultural Congress
    
    \item
    Morawetz L, Steinrigl A, Köglberger H, Derakhshifar I, Griesbacher A, Moosbeckhofer R, Crailsheim K (2019); Virus monitoring of Austrian honey bee colonies: Are virus titers correlating with symptoms in the field?; Poster; 08/SEP - 12/SEP/2019; Montréal, Kanada; 46th Apimondia – International Apicultural Congress

\end{itemize}


\underline{Projektbezogene Vorträge Science to Stakeholders}

\begin{itemize}
    \item 
    Morawetz Linde „Beobachtungsstudie des Projekts Zukunft Biene: Ursachenforschung zur Wintersterblichkeit“ Wanderlehrerfortbildungstagung des Österr. Imkerbundes, Graz, 27.10.2018

    \item 
    Morawetz L „Beobachtungsstudie und Posthoc Untersuchungen: Einflussfaktoren auf den Überwinterungserfolg“ Runder Tisch Zukunft Pflanzenbau, AGES, Wien, 8.11.2018

    \item 
    Moosbeckhofer R „Ergebnisse der Beobachtungs- und Post hoc-Studie zu Einflussfaktoren auf den Überwinterungserfolg von Bienenvölkern“ Fachtagung Österreichischer Erwerbsimkerbund, Premstätten 24.02.2019

    \item 
    Morawetz L „Zukunft Biene 2 – Virenmonitoring“ Gesundheitsreferententagung des Österreichischen Imkerbundes, AGES, Wien, 09.05.2019
    
    \item 
    Brodschneider R, Auer W, Crailsheim K, Danihlik J, Gratzer K, Heigl H, Moosbeckhofer R, Omar E. Der Pollenspeiseplan unserer Bienen – Vielfalt für gesunde Bienen. Völkermarkt, 12.10.2019.
    
    \item 
    Brodschneider R. Pollen diversity as a factor for honey bee development. Agdermotet, Norwegen, 9.11.2019.
    
    \item
    Morawetz L, Sandén T, Dörler D, Heigl F (2019); Wissenschaft trifft Öffentlichkeit: Citizen Science in AGES Projekten; Vortrag; 31/OKT/2019; AGES WSP, Wien, Österreich; Lunchtime Learning

    \item
    Morawetz L (2019); Zukunft Biene 2 - Zwischenergebnisse des Virenmonitorings; Vortrag; 19/OKT/2019; Altlengbach in NÖ, Österreich; Wanderlehrerfortbildungstagung

\end{itemize}

\underline{Projektbezogene Beiträge in den Medien}

\begin{itemize}

    \item
    Crailsheim K, Brodschneider R (2019) Ursachen der Verluste an Honigbienen in Österreich. Der Pflanzenarzt, 5: 27-29.

    \item
    Brodschneider R, Moosbeckhofer R, Crailsheim K (2019) Winterverluste 2017/18 und Aufruf zur Teilnahme an der Untersuchung 2019. Bienen aktuell, 4: 20-23.

    \item
    Lamp B, Seitz K (2019) Ursachen und Folgen des Bienensterbens. Reportage. ServusTV. \url{https://www.servus.com/tv/ursachen-und-folgen-des-bienensterbens}
    
    \item
    Brodschneider R (2020) Gut übern Winter. Aircampus. Podcasts der Grazer Universitäten. \url{https://www.aircampus-graz.at/podcasts/bienensterblichkeit/}
    
    \item
    Brodschneider R, Moosbeckhofer R, Crailsheim K (2020) Winterverluste 2018/19 und Aufruf zur Teilnahme an der Untersuchung 2020. Bienen aktuell, 4: 21-23.
    
    \item
    Morawetz L, Seitz K, Köglberger H, Rümenapf T, Steinrigl A (2020) Bienenviren und ihre Erforschung in Österreich (Projekt Zukunft Biene 2); JUL/2020; Bienenaktuell; 

\end{itemize}

\underline{Projektbezogene Nennungen in den Medien}

\begin{itemize}

    \item
    studium.at. 13.06.2019. Bienensterben: Winterverluste laut Studie an Uni Graz im Mittelwert.

    \item
    orf.at. 13.06.2019. 15 Prozent der Bienenvölker starben im Winter. \url{https://orf.at/stories/3126641/}

    \item
    orf.at. 13.06.2019. Bienensterben: Winterverluste im Mittelwert. \url{https://steiermark.orf.at/stories/3000281/}

    \item
    citizen-science.at. 13.06.2019. Virenmonitoring. \href{https://www.citizen-science.at/citizen-scienceprojekte/item/448-virenmonitoring}{citizen-science.at}

    \item
    derstandard.at. 13.06.2019. So schlimm steht es wohl doch nicht um unser liebstes Insekt. \href{https://www.derstandard.at/story/2000104810957/so-schlimm-steht-es-um-unser-liebstes-insekt-wohl-doch}{derstandard.at}

    \item
    krone.at. 13.06.2019. Bienensterben im Winter heuer im Mittelwert. \url{https://www.krone.at/1941068}

    \item
    wienerzeitung.at. 13.06.2019. Bienenverluste im Mittelwert. \url{https://www.wienerzeitung.at/nachrichten/wissen/natur/2013822-Bienenverluste-im-Mittelwert.html}

    \item
    Kronen Zeitung. 14.06.2019. Unseren Bienen geht es besser.

    \item
    blickinsland.at. 14.06.2019. Bienenverluste im Winter durchschnittlich. \href{https://blickinsland.at/bienenverlusteim-winter-durchschnittlich/}{blickinsland.at}

    \item
    Der Standard. 14.06.2019. 15 Prozent Winterverlust bei Bienenvölkern.

    \item
    Tiroler Tageszeitung. 14.06.2019. Weniger Verluste bei Bienen.

    \item
    TT Kompakt. 14.06.2019. Harter Winter für Bienen.

    \item
    Wiener Zeitung. 14.06.2019. Verluste an Bienen im mittleren Bereich.

    \item
    Kleine Zeitung. 14.06.2019. Durchschnittliche Bienenverluste im Winter
    
    \item
    wienerzeitung.at. 20.05.2020. Hohe Pollenvielfalt für Bienen, aber nur kurze Zeit. \url{https://www.wienerzeitung.at/nachrichten/wissen/natur/2061316-Hohe-Pollenvielfalt-fuer-Bienen-aber-nur-kurze-Zeit.html}
    
    \item
    news.uni-graz.at. 20.05.2020. Auf die Bäume. 
    \url{https://news.uni-graz.at/de/detail/article/auf-die-baeume/}
    
    \item
    Die Presse. 23.05.2020. Pollenanalyse zieht Bienen die „Höschen“ aus.
    
    \item
    krone.at. 28.05.2020. Heimische Bienen haben Winter gut überstanden. 
    \url{https://www.krone.at/2162778}
    
    \item
    suedtirolnews.it. 28.05.2020. Nur geringe Winterverluste bei Bienen in Österreich. \href{https://www.suedtirolnews.it/wirtschaft/nur-geringe-winterverluste-bei-bienen-in-oesterreich}{suedtirolnews.it}
    
    \item
    arf.at. 28.05.2020. Gut überwintert: Uni Graz meldet geringe Bienensterblichkeit. \href{https://www.arf.at/2020/05/28/gut-ueberwintert-uni-graz-meldet-geringe-bienensterblichkeit/}{arf.at}
    
    \item
    sn.at. 28.05.2020. Nur geringe Winterverluste bei Bienen in Österreich. \href{https://www.sn.at/panorama/oesterreich/nur-geringe-winterverluste-bei-bienen-in-oesterreich-88150126}{sn.at}
    
    \item
    wienerzeitung.at. 28.05.2020. Bienen im Aufwind. \href{https://www.wienerzeitung.at/nachrichten/wissen/natur/2062201-Bienen-im-Aufwind.html}{wienerzeitung.at}
    
    \item
    puls24.at. 28.05.2020. Nur geringe Winterverluste bei Bienen in Österreich. \href{https://www.puls24.at/news/chronik/nur-geringe-winterverluste-bei-bienen-in-oesterreich/205520}{puls24.at}
    
    \item
    science.apa.at. 28.05.2020. Zoologen melden geringe Winterverluste bei Bienen in Österreich. \href{https://science.apa.at/rubrik/natur_und_technik/Zoologen_melden_geringe_Winterverluste_bei_Bienen_in_Oesterreich/SCI_20200528_SCI39391351454789954}{science.apa.at}
    
    \item
    science.orf.at. 28.05.2020. Relativ guter Winter für Bienen in Österreich. \url{https://science.orf.at/stories/3200843/}
    
    \item
    drei.at. 28.05.2020. Nur geringe Winterverluste bei Bienen in Österreich.  
    
    \item
    bvz.at. 28.05.2020. Nur geringe Winterverluste bei Bienen in Österreich. \href{https://www.bvz.at/in-ausland/uni-graz-erforschte-nur-geringe-winterverluste-bei-bienen-in-oesterreich-oesterreich-agrar-tierkrankheiten-zoologie-oesterreich-207676823}{bvz.at}
    
    \item
    volksblatt.at. 28.05.2020. Nur geringe Winterverluste bei Bienen in Österreich. \url{https://volksblatt.at/nur-geringe-winterverluste-bei-bienen-in-oesterreich/}
    
    \item
    vienna.at. 28.05.2020. Nur geringe Winterverluste bei Bienen in Österreich. 
    
    \item
    kleinezeitung.at. 28.05.2020. Nur geringe Winterverluste bei Bienen in Österreich. \href{https://www.kleinezeitung.at/international/tiere/5819585/Imker-zufrieden_Geringe-Winterverluste-bei-unseren-Bienen}{kleinezeitung.at}
    
    \item
    steiermark.orf.at. 28.05.2020. Geringe Winterverluste bei Bienen. \url{https://steiermark.orf.at/stories/3050684/ }
    
    \item
    orf.at. 28.05.2020. Relativ guter Winter für Bienen in Österreich. 
    \url{https://orf.at/stories/3167474/}
    
    \item
    vol.at. 28.05.2020. Nur geringe Winterverluste bei Bienen in Österreich. \href{https://www.vol.at/nur-geringe-winterverluste-bei-bienen-in-oesterreich/6630935#:~:text=Eine%20gute%20Nachricht%20von%20%C3%96sterreichs,Winter%202019%2F20%20nicht%20%C3%BCberlebt.}{vol.at}
    
    \item
    noen.at. 28.05.2020. Nur geringe Winterverluste bei Bienen in Österreich. \href{https://www.noen.at/in-ausland/uni-graz-erforschte-nur-geringe-winterverluste-bei-bienen-in-oesterreich-oesterreich-agrar-tierkrankheiten-zoologie-oesterreich-207676823}{noen.at}
    
    \item
    studium.at. 28.05.2020. Zoologen melden geringe Winterverluste bei Bienen in Österreich. \href{https://www.studium.at/zoologen-melden-geringe-winterverluste-bei-bienen-oesterreich}{studium.at}
    
    \item
    kurier.at. 28.05.2020. Bienensterben in Österreich im Winter vergleichsweise gering. \href{https://kurier.at/chronik/oesterreich/bienensterben-in-oesterreich-im-winter-vergleichsweise-gering/400854374}{kurier.at}
    
    \item
    vn.at. 28.05.2020. Nur geringe Winterverluste bei Bienen in Österreich. \href{https://www.vn.at/newsticker/nur-geringe-winterverluste-bei-bienen-in-oesterreich/1852978}{vn.at}
    
    \item
    Wiener Zeitung. 29.05.2020. Bienen im Aufwind.
    
    \item
    tt.com. 29.05.2020. Nur geringe Winterverluste bei Bienen in Österreich. \url{https://www.tt.com/artikel/16992370/nur-geringe-winterverluste-bei-bienen-in-oesterreich}
    
    \item
    blickinsland.at. 02.06.2020. Bienen überstanden Winter einigermaßen gut. \url{https://blickinsland.at/bienen-ueberstanden-winter-eingermassen-gut/}
    
    \item
    Bauern Zeitung. 04.06.2020. Heimische Imker: Es war ein guter Winter. 
    
    \item
    stmk.lko.at. 15.06.2020. Rückgang bei Bienensterblichkeit.
    \href{https://stmk.lko.at/landwirtschaftliche-mitteilungen-vom-15-juni-2020+2500+3228498}{stmk.lko.at}
    
    \item
    Landwirtschaftlichen Mitteilungen. 15.06.2020. Rückgang bei Bienensterblichkeit.


\end{itemize}