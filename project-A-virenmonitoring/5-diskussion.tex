\section{Diskussion}

\subsection{Repräsentativität der StudienteilnehmerInnen}

Im Vergleich zum ersten Versuchsjahr kam es zu einer geringfügigen Änderung der in der Gruppe der TeilnehmerInnen. Zehn TeilnehmerInnen aus dem ersten Versuchsjahr fielen aus. Fünf TeilnehmerInnen kamen als Ersatz für die ausgefallene TeilneherInnen, die schon frühzeitig abgesagt hatten, neu hinzu. 

Vorrangiges Ziel des Virenmonitorings ist es, eine für Österreich repräsentative Aussage über die Prävalenz von Bienenviren zu machen. Deswegen wurde bei der Auswahl Sorge getragen, dass eine geographische Repräsentativität der einzelnen Bundesländer gewährleistet ist (\cref{tab:b:probenanzahl}). Dies wurde erreicht, indem die TeilnehmerInnen anhand der Bundesländer-Zugehörigkeit ihrer Bienenstände aus den freiwilligen Meldungen ausgewählt wurden. Es kam zu geringfügigen Änderung in der Bundesländer-Verteilung der Bienenstände zwischen der Planung und den tatsächlich eingesandten Proben. Diese Verschiebung kam teilweise durch den Ausfall von sieben Bienenständen zustande. Zusätzlich wählten einige ImkerInnen Bienenstände in anderen Bundesländern als anfangs angegeben zur Teilnahme aus (Verschiebung von NÖ nach Wien).

Im vorliegenden Versuchsjahr lagen die Winterverluste der Virenmonitoring-TeilnehmerInnen 4\% über den Winterverlusten der COLOSS-Studie (= Ergebnisse des Moduls U des vorliegenden Berichts). Der Unterschied zwischen den beiden Winterverlust-Berechnungen wird auch anhand der nicht überlappenden Konfidenzintervalle ersichtlich. An der COLOSS-Studie hatten 1.539 Imkereibetriebe mit 30.724 Völkern teilgenommen und es wurde ein Winterverlust von \confi{12,6}{95}{11,9}{13,3} gemeldet (KAPITEL). Im Vergleich dazu wurde auf den Monitoringständen des Virenmonitorings ein Winterverlust von \confi{16,6}{95}{14,2}{19,3} gemeldet (tote + weisellose Völker, Details siehe \cref{tab:l:winterverluste}).

Dieser Unterschied spielt jedoch aufgrund der unterschiedlichen Zielsetzungen der beiden Projektmodule nur eine untergeordnete Rolle. Ziel der COLOSS-Studie ist es, die Winterverluste in Österreich im Detail zu erfassen. Die Eingabe der Daten erfolgt durch eine möglichst große Anzahl an zum Teil anonymen TeilnehmerInnen (Winter 2019/20: 1.539 Betriebe mit 30.124 Völker). Im Virenmonitoring werden die Winterverluste in erster Linie dazu erhoben, um Rückschlüsse über die Effekte der erhobenenen Viren auf den Überwinterungserfolg der Völker zu untersuchen. Die Daten werden daher von einer vergleichsweise geringen Anzahl an uns bekannten TeilnehmerInnen über mehrere Jahre erhoben (Winter 2019/20: 193 Bienenstände mit 2.931 Völker). Es ist nicht verwunderlich, dass die geringe TeilnehmerInnenzahl und größere Teilnahmehürde zu verschobenen Ergebnissen in den Winterverlusten führt. Es ist vielmehr erfreulich, dass diese Unterschiede erst im zweiten Untersuchungsjahr vorkamen und gering ausfielen.

Die Repräsentativität der Stichprobe ist für die Kenngrößen Imkererfahrung, Betriebsgröße und Bio-Zertifizierung vermutlich nicht voll gegeben. Leider liegen uns für diese Kenngrößen keine Statistik für die Österreichische Imkerschaft vor, wir können sie jedoch teilweise mit den Daten der COLOSS-Studie vergleichen, die eine wesentlich größere Stichprobe aufweist. So sind wahrscheinlich die jungen unerfahrenen ImkerInnen in der Stichprobe des Virenmonitoring überrepräsentiert, da etwa die Hälfte aller teilnehmenden ImkerInnen weniger als elf Jahre Imkererfahrung aufweisen. Ebenso sind die Imkerbetriebe mit weniger als 20 Völkern im Vergleich zu COLOSS-Studien unterrepräsentiert (Virenmonitoring: 43,5\% der TeilnehmerInnen; COLOSS-Studien: >70\%, \cite{brodschneider2018b, brodschneider2019b}, KAPITEL). Der Anteil an zertifizierten Bio-Imkereien scheint mit 24,9\% der TeilnehmerInnen ebenso über dem österreichweiten Schnitt zu liegen. In den COLOSS-Studien lag der Anteil der zertifizierten Bio-Imkereien bei etwa 10\% (\cite{brodschneider2018b, brodschneider2019b}, KAPITEL).

Die beschriebenen Abweichungen der drei Kenngrößen kommen vermutlich aufgrund der Teil\-nehmerInnen-Aquisition mittels freiwilliger Meldung zustande. Bei Gesprächen oder in Emails nannten die TeilnehmerInnen vorrangig drei Gründe für die Teilnahme an der Studie: die Bereitschaft, die wissenschaftliche Forschung zum Thema Bienenviren zu unterstützen, die Möglichkeit, exakte Daten über Virusinfektionen in den eigenen Völkern zu erhalten und das Interesse, sich Wissen über das Thema Bienenviren anzueignen. Aus dieser Motivation heraus war eine Teilnahme an dem Projekt und damit eine freiwillige Meldung vor allem für JungimkerInnen und NeueinsteigerInnen interessant, die die Chance wahrnahmen ihr Wissen zu erweitern. Ebenso ist das vermehrte Interesse von Bioimkereien erklärbar, die aufgrund ihrer Betriebsweise einen besonderen Fokus auf Prävention und damit auf das sichere Erkennen von Krankheiten haben. Die geringe Beteiligung von kleinen Imkereibetrieben lag zum Teil an der Projektanforderung von minimal fünf Probenvölkern, die einen Teil der ImkerInnen automatisch ausschloss. Um die beschriebenen Abweichungen zumindest teilweise zu vermeiden, hätte man eine Zufallsauswahl aus dem österreichweiten Bienenstandsregister (VIS) durchführen müssen. Dies war leider aus rechtlichen Gründen nicht möglich.

\subsection{Prävalenzlevel}

Die errechneten Prävalenzwerte können als Prävalenzwerte auf Bienenstand-Niveau betrachtet werden. Der Bienenstand ist eine epidemiologische Einheit, da aufgrund der gruppenweisen Aufstellung auf engem Raum (Blockaufstellung, große Zahl von Bienenvölkern auf einem Bienenstand), imkerlicher Eingriffe (gleiches Werkzeug für alle Bienenvölker, Wechsel von Beutenteilen, Rähmchen, Futter- und Brutwaben zwischen Völkern, etc.) und Verhalten der Bienen (Verflug der Arbeiterinnen und Drohnen, Räuberei in anderen Standvölkern) ein ständiger Austausch von viruskontaminiertem Material zwischen den Völkern zu erwarten ist. Deswegen haben wir in der vorliegenden Studie die fünf eingesandten Bienenproben eines Standes zu jeweils einer Sammelprobe zusammengefasst und untersucht. Mit einer Sammelprobe aus mehreren Völkern steigt die Wahrscheinlichkeit, dass alle häufig auf einem Stand vorkommenden Bienenviren in der Sammelprobe enthalten sind und das gesamte Virusspektrum auf dem Bienenstand damit beschrieben werden kann \citep{tentcheva2004,mouret2013}.

Die im September 2019 untersuchten Bienenproben enthielten sechs Bienenviren (ABPV, BQCV, CBPV, DWV-A, DWV-B, SBV; \cref{fig:e:virusQUAL}). Zwei weitere getestete Viren wurden nicht entdeckt (IAPV, KBV). Das gefundene Virenspektrum stimmt damit sowohl mit den Ergebnissen des ersten Untersuchungsjahr als auch mit den Ergebnissen des Vorprojekts \enquote{Zukunft Biene} überein \citep{brodschneider2018b,brodschneider2019b}. Das negative Untersuchungsergebnis für IAPV und KBV stimmt auch mit früheren österreichischen Untersuchungen überein, in denen diese Viren entweder gar nicht oder nur in Einzelfällen detektiert wurden \citep{köglberger2009,girsch2012,moosbeckhofer2014}. Doch auch diese Einzelfunde sind inzwischen anzuzweifeln. Im Zuge der Methodenetablierung von IAPV und KBv wurden diese positiven Proben nochmals untersucht. Dabei konnte die Anwesenheit der beiden Viren nicht verifiziert werden (\cref{chap:etablierung_iapv_kbv}).

Das Prävalenzlevel der Viren blieb großteils über die beiden Untersuchungsjahre 2018 und 2019 konstant \citep{brodschneider2019b,morawetz2019a}. In beiden Jahren war BQCV mit einer Prävalenz knapp unter 100\% das häufigste Virus, gefolgt von DWV-B, das in beiden Jahren in über 85\% der Proben vorkam. Auch die seltenen Viren CBPV und DWV-A traten in beiden Jahren mit einer ähnlichen Häufigkeit auf, bei beiden blieb die Prävalenz unter 10\%. Interessanterweise kam es jedoch zu Unterschieden in der Prävalenz der Viren ABPV und SBV. ABPV kam im Untersuchungsjahr 2019 um 30\% seltener vor als im Jahr zuvor (Prävalenz 2018 \confi{54}{95}{47}{60}). SBV hingegen kam im Jahr 2019 um 30\% häufiger vor als im Vorjahr (Prävalenz 2018: \confi{62}{95}{55}{69}).

Schwankungen in der Häufigkeit lassen sich auch im Vergleich zu anderen österreichischen Studien beobachten. In der Vorgängerstudie „Zukunft Biene“ wurden 210 Bienenproben, die im September 2015 genommen wurden, auf ABPV, CBPV, DWV-A und DWV-B untersucht \citep{morawetz2018}. ABPV ist damals signifikant häufiger in Bienenproben gefunden worden als die anderen drei Virusspezies und die Häufigkeit von DWV-A und DWV-B unterschied sich nicht signifikant. Auch in einer früheren österreichischen Studie, die Proben von 43 asymptomatischen Völkern im Jahr 2009 auf sechs Viren untersuchte, war ABPV gemeinsam mit SBV das am häufigsten nachgewiesene Virus, gefolgt von BQCV und DWV \citep{köglberger2009}.  Ein Vergleich der absoluten Häufigkeiten mit den Vorgängerstudien ist nicht sinnvoll, da diese das Virusauftreten in Einzelvölkern gemessen haben und damit nicht mit den Daten aus Sammelproben des vorliegenden Projektes verlgeichbar sind.  Auch sind die Ergebnisse zwischen den aufgrund der unterschiedlichen angewandten PCR-Methoden nicht uneingeschränkt vergleichbar.

Die drei Viren ABPV, DWV-B, SBV unterschieden sich in der Prävalenz zwischen den neun Bundesländern (\cref{fig:h:virusQUAL_BL,tab:j:praevalenz_BL}). Dabei waren ABPV und DWV-B in Tirol mit der geringsten Prävalenz vertreten und SBV in Kärnten. In Wien und dem Burgenland wurden hingegen bei allen drei Viren sehr hohe Prävalenzwerte gemessen. Ähnliche Muster wurde schon im Vorjahr beobachtet: im Jahr 2018 waren die Prävalenzwerte die Viren ABPV, BQCV, DWV-B und SBV für Stände in Tirol und Kärnten im Vergleich niedrig, während für Wiener und Burgenländische Stände sehr hohe Prävalenzen gemessen wurden \citep{brodschneider2019b}. Diese Unterschiede stehen wahrscheinlich in Zusammenhang mit der Seehöhe, da die teilnehmenden Bienenstände in Tirol und Kärnten zusammen mit den Bienenständen in Salzburg am höchsten gelegen sind, während die Bienenstände in Wien und Burgenland im Flach- und Hügelland lagen (\cref{fig:j:COR_BL_Seehoehe}). Tatsächlich war für die Viren ABPV, DWV-B und SBV die Prävalenz auf Ständen in größerer Seehöhe signifikant niedriger als auf tiefer gelegenen Ständen (\cref{fig:k:virusQUAL_Seehoehe,tab:k:praevalenz_Seehoehe}).


Der Zusammenhang zwischen der Virus-Prävalenz und der Seehöhe konnte nun in zwei aufeinander folgenden Versuchsjahren für Österreich gezeigt werden. Er ist vermutlich durch klimatische und ökologische Faktoren bedingt, die sowohl Einfluss auf die Vegetation und die Art der Landnutzung als auch auf die Entwicklung der Bienenvölker haben. So hat die durch die klimatischen Bedingungen bedingte kürzere Brutzeit der Völker in höheren Lagen einen hemmenden Einfluss auf die Entwicklung der Varroa-Population und damit auf die Verbreitung von Varroa-assoziierten Viren wie DWV oder SBV im Bienenvolk \citep{mcmenamin2015,sumpter2004}. Es ist außerdem zu vermuten, dass die Dichte der Bienenvölker in höheren Regionen geringer ist als in niedrigeren Lagen. Mit einer geringen Völkerdichte sinkt die Wahrscheinlichkeit, dass Varroamilben und die von ihnen verbreiteten Viren zwischen Bienenvölkern durch Räuberei, Bienen-Verflug oder Kontakte bei Blütenbesuchen übertragen werden \citep{forfert2015,forfert2016,nolan2016,peck2016}. Ebenso ist es zu erwarten, dass sich die Höhenlagen durch Unterschiede im Futterpflanzenspektrum und der Pflanzendiversität unterscheiden. Da die Ernährung einen wichtigen Einfluss auf die Gesundheit und Abwehrkräfte eines Volkes besitzt, können ungünstige Ernährungssituationen mitverantwortlich für die Ausbreitung von Viren in Bienenvölkern sein \citep{alaux2011,degrandi-hoffman2015}.

Es wurde kein Zusammenhang zwischen den von den ImkerInnen beobachteten Virussymptomen und der Prävalenz des entsprechenden Virus gefunden (\cref{chap:egebnisse:symptome}). Auch im Vorjahr wurde nur bei einem Symptom ein signifikanter Zusammenhang mit der Prävalenz hergestellt (DWV-B und \enquote{Bienen mit verkrüppelten Flügeln}). Es besteht also kein oder kaum ein Zusammenhang zwischen der reinen Nachweisbarkeit des Virus und den sichtbaren Symptomen der dem Virus entsprechenden Virose in den beprobten Völkern. Das Phänomen, dass ein Virus ohne Krankheitssymptome nachweisbar ist, ist bei Bienenviren schon länger bekannt. Mann spricht hier von verdeckten (covert) und offenen (overt) Infektionen \citep{yue2007}. Bei verdeckten Infektionen ist der Virus in geringer Konzentration nachweisbar, doch die Biene zeigt keine Krankheitssymptome, während bei offenen Infektionen hohe Virustiter nachweisbar sind und die Biene klare Symptome der entsprechenden Virose zeigt \citep{schurr2019}.


\subsection{Höhe Virustiter}

\textbf{fehlt noch}

\subsection{Winterverluste}

In allen gerechneten Modellen hat sich die Höhe des DWV-B Titers als wichtigster Einflussfaktor auf die Wahrscheinlichkeit eines Winterverlustes erwiesen. Zusätzlich hatte das aussagekräftigste Modell, der Random Forest, den ABPV Titer als wichtigen Einflussfaktor identifiziert (\cref{fig:q:WV_RF_significance,tab:n:Mcomparison}). Die logistische Regression, ein etwas weniger aussagekräftiges Modell, enthält zusätzlich noch die Faktoren \enquote{Varroamilben auf Bienen} und Anzahl der Völker des Betriebs (\cref{tab:m:regression}).

\subsubsection{DWV-B}

Der Zusammenhang zwischen dem Auftreten von DWV und Winterverlusten ist gut belegt, auch für Österreich und die umliegenden Ländern (Deutschland: \cite{genersch2010}, Schweiz: \cite{dainat2012}, Österreich: \cite{morawetz2018}). Je höher die Konzentration von DWV in den untersuchten Proben ist, desto wahrscheinlicher ist das Volk krank und in Gefahr abzusterben \citep{amiri2015,berthoud2010,barroso-arevalo2019,dainat2012}. Unklar ist nach wie vor, welcher Virustiter als gefährdend einzuschätzen ist. Die in der Literatur genannten Schwellen bezüglich DWV bewegen sich zwischen $10^6$ und $10^7$ RNA-Kopien/Biene \citep{amiri2015,barroso-arevalo2019,möckel2011,schurr2019}. Aufgrund der Modellierungen der Daten der letzten beiden Jahre würden wir die Gefährdungsschwelle von DWV-B in Östereich bei etwa $10^8 - 10^9$ RNA-Kopien/\si{\milli\liter} Homogenat ansetzen (etwa 3x$10^7$ - 3x$10^8$ RNA-Kopien/Biene; \cref{fig:r:WV_RF_predictors}). Unsere Daten würden die Schwelle daher um zumindest eine 10er Potenz höher einschätzen als die bisherige Literatur. Es bleibt jedoch noch abzuwarten, ob die Daten des dritten Probenjahres noch Änderungen in der Einschätzung hevorrufen oder die derzeitige Datenlage bestätigen.

Der Zusammenhang zwischen hohem DWV-B Titer und Winterverlusten bezieht indirekt die Varroamilbe aus wichtiger Faktor in das Modell mit ein. Uns liegen für die Probenvölker keine Informationen über die Varroabelastung vor. Doch das Auftreten von DWV ist eng mit dem Auftreten der Varroamilbe verknüpft \citep{bowen-walker1999,dainat2013,gisder2009}. Je höher der Titer von DWV ist, desto höher ist auch die Milbenbelastung im Volk \citep{barroso-arevalo2019}. Dieser Zusammenhang wird auch durch unsere Daten bestätigt:  die abgefragten Symptome zur Varroose, \enquote{Varroamilben auf Bienen} und \enquote{Bienen mit verkrüppelten Flügeln}, waren mit signifikant erhöhtem DWV-B Titer verbunden (\cref{fig:m:DWV_Varroa,fig:n:DWV_Fluegel}). Im logistischen Modell ist der Faktor \enquote{Varroamilben auf Bienen} sogar ein zusätzlicher Faktor, dessen Auftreten eine erhöhte Wintersterblichkeit vorhersagt.

\subsubsection{ABPV}

Ein hoher ABPV Titer führte auf den Monitoringständen zu einer erhöhten Wahrscheinlichkeit an Winterverlusten (\cref{fig:r:WV_RF_predictors}). Der Zusammenhang zwischen Winterverlust und ABPV ist bereits für umliegende Ländern gezeigt worden \citep{genersch2010,berthoud2010}. Im Vorgängerprojekt \enquote{Zukunft Biene} konnte jedoch kein Zusammenhang zwischen ABPV und Winterverlusten für Österreich festgestellt werden \citep{morawetz2018}. Höhere Titerwerte in der vorliegenden Studie und neue Modellierungsansätze lassen zwei Vermutungen zu, wie es zu diesen unterschiedlichen Ergebnissen gekommen ist. Entweder war im Untersuchungsjahr der Vorläuferstudie ABPV wirklich aufgrund der geringen Titerwerte kein Einflussfaktor auf die Winterverluste. Oder das neue Modell Random Forest ist in der Lage Muster zu erkennen, die dem logistischen Modell verborgen bleiben. Die zweite Annahme ist auch durch die Tatsache gestützt, dass auch das logistische Modell in der aktuellen Untersuchung kein Effekt von ABPV entdeckt (\cref{tab:m:regression}).

Der Virustiter, ab dem ABPV als Winteverlust-gefährdend anzusehen ist, ist schwerer zu bestimmen als für DWV-B. Die Kurve des ABPV Titers im Random Forest Modell  steigt kontinuierlich an und zeigt, anders als bei DWV-B, keine deutlichen Sprung (\cref{fig:r:WV_RF_predictors}). Gleichzeitig sagt das Modell auch eine hohe Wahrscheinlichkeit voraus im Winter abzusterben, ohne dass ABPV gefunden wurde. Dies weist darauf hin, dass viele der Winterverluste durch andere (bekannte oder unbekannte) Faktoren zustande kommen. Beide Gegebenheiten mögen auch die Gründe dafür sein, dass der ABPV-Titer nicht als Einflussfaktor im logistischen Modell dargestellt wurde. Es bleibt zu warten, ob die Erhöhung des Datensatzes im dritten Untersuchungsjahr die Datenlage verbessert und eine klare Aussage zu einer Gefährdungsschwelle ermöglicht.

\subsubsection{Imkererfahrung}

Das logistische Modell benennt die gesamte Anzahl an Völkern im Imkerbetrieb als Einflussfaktor für Winterverluste. Je mehr Völker in einem Betrieb vorhanden sind, desto geringer ist die Wahrscheinlichkeit, dass die Probenvölker absterben. Dieses Phänomen ist aus anderen internationalen Studien, aber auch aus der österreichischen COLOSS-Studie bekannt (\cite{brodschneider2016,lee2015,jacques2017,oberreiter2020,vanderzee2014}, KAPITEL). Die Größe des Betriebes ist eine Kennzahl für die Professionalität des Imkereibetriebs \citep{jacques2017}. Andere Kennzahlen der Professionalität, die mit Verbesserung der Überwinterungsrate einhergehen, sind die imkerlichen Erfahrungsjahre und das Alter des Imkers/der Imkerin \citep{jacques2017,morawetz2019}. Die Professionalität steigt dabei mit den Erfahrungsjahren, sinkt jedoch einem gewissen Lebensalter wieder ab. Diese Ergebnisse betonen den positiven Einfluss, den eine gute Schulung der Imkerschaft und die dadurch entstehende Professionalität auf die Überwinterung von Bienenvölkern hat.

\subsubsection{Vollständigkeit der Modelle}

In der Kreuzvalidierung der gerechneten Modelle zeigte sich, dass die verschiedenen Modelle nur einen gewissen Anteil der Winterverlust-Wahrscheinlichkeit erklären (\cref{chap:vorhersage.modelle}). Das heißt, in den Modellen gab es zahlreiche Fälle, in denen keiner der vorliegenden Faktoren eine Erklärung für das Absterben der Völker darstellt. Dieses Ergebnis kann im wesentlichen aus zwei Gründen zustande kommen. Einerseits ist es wahrscheinlich, dass wesentliche Faktoren im Modell fehlen, da sich die präsentierte Modellierungen auf die Effekte von Viren konzentriert hat. Bekannte Faktoren für Winterverluste, die wir nicht in das Modell integriert haben, sind zum Beispiel Klimafaktoren, das Alter der Königin, nicht-virale Krankheiten wie Amerikanische Faulbrut oder Nosema oder Effekte der imkerlichen Betriebsführung \citep{brodschneider2018,chauzat2016,morawetz2019,switanek2017}. Daher sollten die hier präsentierten Modelle nie als alleinige Erklärung von Winterverlusten verwendet werden, sondern auch das Wissen um andere Faktoren in Interpretation und Maßnahmensetzung mit einbezogen werden.
Andererseits könnten die in der Modellierung verwendeten Messungen die Gesundheit des Volkes ungenügend darstellen. So kann eine Bienenprobe aus 10 Bienen pro Volk nie den Gesamtzustand eines Volkes aus zehntausenden Bienen beschreiben. Es kann daher nicht ausgeschlossen werden, dass das Analyseergebnis zwar negative oder geringe Titerwerte ergibt, der generelle Virustiter im anderen Bienen des Volkes jedoch wesentlich höher liegt. Eine Erhöhung der Stichprobe ist jedoch sinnlos, da es sowohl hoch aufwendig als kontraproduktiv für die Praxis ist, einen Großteil der Bienen eines Volkes für eine  Virustestung zu opfern. Das Ziel für die Zukunft ist daher, verlässliche und gering invasive Messmethoden für Erkrankungsanzeichen eines Volkes zu entwickeln (Zitate: B-Good etc.).

\subsection{Zusammenfassung und Ausblick}

\textbf{fehlt noch}