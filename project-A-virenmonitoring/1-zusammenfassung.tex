Im Modul A wird ein österreichweites Monitoring von Bienenviren durchgeführt. Trotz der Bedeutung der Bienenviren für die Bienengesundheit ist über das Vorkommen von Viren in Österreichs Honigbienenvölkern bisher nur begrenztes Wissen vorhanden, das keine gesicherten Aussagen zur generellen Prävalenz der Bienenviren in Österreich erlaubt. Daher wird die Prävalenz von acht Bienenviren auf Bienenstandniveau über drei Jahre erhoben. Diese Viren umfassen das Akute Bienenparalyse-Virus (ABPV), das Schwarze Königinnenzellen-Virus (BQCV), das Chronische Bienenparalyse-Virus (CBPV), das Flügeldeformationsvirus (getrennt in Typ A [DWV-A] und Typ B [DWV-B]), das Israelische Akute Paralyse-Virus (IAPV), das Kashmir-Bienenvirus (KBV) und das Sackbrutvirus (SBV). 
\newline
Mit dem dritten Zwischenbericht liegen die Ergebnisse für die zweite Probenahme im Herbst 2019 vor, an der 193 ImkerInnen aus ganz Österreich teilnahmen. In den Bienenproben wurden sechs der acht untersuchten Viren gefunden, die Viren IAPV und KBV wurden in keiner Probe nachgewiesen. Am häufigsten nachgewiesenen wurden BQCV in 191 der 193 Proben (Prävalenz: 99,0\%; 95\%~CI:~96,3-99,7\%) und DWV-B in 171 der 193 Proben (88,6\%; 95\%~CI:~83,3-92,4\%). SBV wurde mit einer Prävalenz von \confi{80,8}{95}{74,7}{85,8} am dritthäufigsten gefunden (156 Proben positiv). ABPV war in 65 Proben nachweisbar (33,7\%; 95\%~CI:~27,4-40,6\%). CBPV und DWV-A wurden selten nachgewiesen: CBPV in 14 Proben (7,3\%; 95\%~CI:~4,4-11,8\%) und DWV-A in einer Probe (0,5\%; 95\%~CI:~0,1-2,9\%). Die beiden nicht nachgewiesenen Viren KBV und IAPV hatten die gleiche Prävalenz von \confi{0,0}{95}{0,0}{2,0}.
\newline
Der Virustiter der positiven Proben variierte bei allen nachgewiesenen Viren um mehrere Zehnerpotenzen (minimaler Titer: $10^4$ - $10^8$ RNA-Kopien/\si{\milli\liter} Homogenat, maximaler Titer: $10^7$ - $10^{11}$ RNA-Kopien/\si{\milli\liter} Homogenat). Die drei Viren ABPV, BQCV und SBV hatten die geringsten Titer (Median unter $10^6$). Bei CBPV lag der Median um eine Zehnerpotenz höher bei 8,8x$10^6$ RNA-Kopien/\si{\milli\liter} Homogenat. Der mediane Titer von DWV-B lag bei etwa $10^8$ RNA-Kopien/\si{\milli\liter} Homogenat und war mit Abstand am höchsten. DWV-A wurde nur in einer Probe gefunden (3,3x$10^7$ RNA-Kopien/\si{\milli\liter}).
\newline
Die Prävalenz von ABPV, DWV-B und SBV unterschied sich zwischen den verschiedenen Bundesländern und Seehöhen. Die drei Viren traten in Wien und dem Burgenland besonders häufig auf, in Tirol sehr selten. Dies mag an der unterschiedlichen Seehöhe der Bienenstände in den verschiedenen Bundesländern liegen. Denn die Viren kamen besonders häufig in niederen Lagen und seltener in höheren Lagen vor. Auch der DWV-B Titer stand in negativen Zusammenhang mit der Seehöhe. Es ist zu vermuten, dass eine verkürzte Brutzeit durch die kühleren klimatischen Bedingungen in größeren Höhen eine Hauptursache für eine verringerte Virusreproduktion auf diesen Ständen ist.
\newline
 Um den Zusammenhang zwischen Winterverlust und Bienenviren zu beschreiben wurden vier verschiedene Modellierungsansätze gerechnet. Zusätzlich zu den Daten der Virustiter wurden acht weitere potentielle Einflussfaktoren zu den Eigenschaften des Betriebes und der Völker in die Modellierungen aufgenommen. Alle Modelle bestätigen den Zusammenhang zwischen einem hohen DWV-B Titer und einer hohen Wahrscheinlichkeit von Winterverlusten. Einflussfaktoren, die nur in einzelnen Modellen vorkamen, waren ein hoher ABPV-Titer, das Symptom \enquote{Varroamilben auf Bienen} und die Betriebsgröße. Die beiden letzten Faktoren stehen für den negativen Einfluss der Varroamilbe und den positiven Einfluss von imkerlicher Professionalität auf die Überlebenswahrscheinlichkeit der Völker.
 Es wird sich zeigen, ob durch die Auswertungen im Folgejahr und der damit einhergehenden größeren Datenmenge die derzeitigen Aussagen bestätigt werden können.