Im Modul A wird ein österreichweites Monitoring von Bienenviren durchgeführt. Trotz der Bedeutung der Bienenviren für die Bienengesundheit ist über das Vorkommen von Viren in Österreichs Bienenvölkern bisher nur begrenztes Wissen vorhanden, das keine gesicherten Aussagen zur generellen Prävalenz der Bienenviren in Österreich erlaubt. Daher wird die Prävalenz von acht Bienenviren auf Bienenstandniveau über drei Jahre erhoben. Diese Viren umfassen das Akute Bienenparalyse-Virus (ABPV), das Schwarze Königinnenzellen-Virus (BQCV), das Chronische Bienenparalyse-Virus (CBPV), das Flügeldeformationsvirus (getrennt in Typ A (DWV-A) und Typ B (DWV-B)), das Israelische Akute Paralyse-Virus (IAPV), das Kashmir-Bienenvirus (KBV) und das Sackbrutvirus (SBV). 
\newline
Mit dem 3. Zwischenbericht liegen die Ergebnisse für die zweite Probenahme im Herbst 2019 vor, an der 193 ImkerInnen aus ganz Österreich teilnahmen. In den Bienenproben wurden sechs der acht untersuchten Viren gefunden, die Viren IAPV und KBV wurden in keiner Probe nachgewiesen. Die beiden am häufigsten nachgewiesenen Viren, BQCV und DWV-B, waren in fast allen Bienenproben nachweisbar. BQCV war in 191 der 193 Proben (Prävalenz: 99,0\%; 95\%~CI:~96,8-99,8\%) und DWV-B in 171 der Proben nachweisbar (88,6\%; 95\%~CI:~83,6-92,6\%). SBV wurde mit einer Prävalenz von  \confi{80,8}{95}{74,9}{86,0} am dritthäufigsten gefunden (156 Proben positiv). ABPV war in 65 Proben nachweisbar (33,7\%; 95\%~CI:~27,2-40,5\%). CBPV und DWV-A wurden selten nachgewiesen: CBPV in 14 Proben (7,3\%; 95\%~CI:~4,1-11,5\%) und DWV-A in 1 Probe (0,5\%; 95\%~CI:~0,0-2,3\%). 
\newline
Der Virustiter der positiven Proben variierte bei allen nachgewiesenen Viren um mehrere Zehnerpotenzen. Der minimal gemessene Titer lag zwischen $10^5$ und $10^9$ RNA-Kopien/\si{\milli\liter} Homogenat, der maximal gemessene Wert zwischen $10^7$ und 10\textsuperscript{11} RNA-Kopien/\si{\milli\liter} Homogenat. Die drei Viren ABPV, BQCV und SBV hatten die geringsten Titer (Median unter $10^6$). Bei CBPV lag der Median um eine Zehnerpotenz höher bei 8,8x$10^6$ RNA-Kopien/\si{\milli\liter} Homogenat. Der mediane Virustiter von DWV-B lag bei etwa $10^8$ RNA-Kopien/\si{\milli\liter} Homogenat und war mit Abstand an höchsten. DWV-A wurde nur in einer Probe gefunden (3,3x$10^7$ RNA-Kopien/\si{\milli\liter}).
\newline
Für das Monitoringjahr 2019 wurden bisher die Virusprävalenz und die Virustiter der acht
gemessenen Bienenviren für Gesamtösterreich und die neun Bundesländer ausgewertet. Die
weiteren Analysen über die Korrelation zwischen Virusauftreten und Seehöhe, Winterverlusten
und Virussymptomatik in den Völkern werden in den nächsten Monaten durchgeführt. Sie
werden im nächsten Zwischenbericht im Herbst 2020 behandelt.