\begin{table}[htp]
    \caption{VIS-Angaben zur Verteilung der Bienenstände (Bienenst.) über die österreichischen Bundesländer (Quelle: BMASGK, Stichtag: 31.10.2017), sowie die Anzahl der freiwilligen Meldungen zur Projektteilnahme (Interessenten), der teilnehmenden Bienenstände und erhaltenen Proben im Monitoringjahr 2019. Für die Berechnungen anhand der VIS-Daten wurde nur auf Bienenstände zurückgegriffen, die mit mindestens einem Volk im VIS registriert waren.}
    \centering
    \begin{tabular}{|l|c|c|c|c|}
    \hline
    Bundesländer    & Bienenst. Österreich &  Interessenten & Bienenst. 2019        & Proben 2019 \\ [0.5ex]
                    & Prozent                &  Anzahl       &   Anzahl (Prozent)    & Anzahl (Prozent)\\
    \hline
    Burgenland      &   3\%     &   13   & 6 (3,0\%)    & 6 (3,1\%)     \\
    Kärnten         &   12\%    &   38   & 23 (11.5\%)  & 22 (11,4\%)       \\
    NÖ              &   20\%    &   64   & 39 (19,5\%)  & 35 (18,1\%)       \\
    OÖ              &   22\%    &   60   & 43 (21,5\%)  & 42 (21,8\%)       \\
    Salzburg        &   7\%     &   21   & 14 (7,0\%)   & 14 (7,3\%)       \\
    Steiermark      &  18\%     &   37   & 37 (18,5\%)  & 35 (18,1\%)       \\
    Tirol           &   10\%    &   30   & 20 (10,0\%)  & 20 (10,4\%)       \\
    Vorarlberg      &   6\%     &   16   & 11 (5,5\%)   & 10 (5,2\%)       \\
    Wien            &   3\%     &   18   & 7 (3,5\%)    & 9 (4,7\%)       \\
    \hline
    Gesamt          &   100\%   &   297  & 200 (100\%)  & 193 (100\%)       \\
    \hline
    \end{tabular}
    \label{tab:b:probenanzahl}
\end{table}
