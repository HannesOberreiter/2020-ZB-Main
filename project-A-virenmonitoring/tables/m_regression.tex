\begin{table}[htb]
    \caption{Multivariates Modell (GLM mit quasibinomialer Verteilung): Einfluss auf die Wintersterblichkeit von Bienenvölkern. Drei Prädiktoren enthalten: Virustiter von DWV-B im September vor der Einwinterung (Vergleichskategorie: negativ, K = Konzentration), Symptom \enquote{Varroamilben auf Bienen} im September gesehen (Vergleichskategorie: Symptom nicht gesehen) und Gesamtanzahl der Völker des Imkereibetriebes (Anzahl Völker). n=390 Proben.}
    \centering
    %\setlength{\tabcolsep}{0.2em} % for the horizontal padding
    \label{tab:m:regression}
    \begin{tabular}{l|rrr}
        \toprule
        Prädiktor & Schätzer & Standardfehler & P-Wert \\
        \midrule
        DWV-B geringe K     & -0,039  & 0,360    & 0,913  \\ 
        DWV-B mittlere K    & 0,502   & 0,337    & 0,137\\ 
        DWV-B hohe K        & 1.204   & 0,355    & <0,001\\ 
        \midrule
        Varroa - gesehen        & 0,680 & 0,212 & 0,001 \\
        Varroa - keine Angabe   & 0,057 & 0,317 & 0,858 \\
        \midrule
        Anzahl Völker           & -0,003    & 0,002     & 0,050 \\
        \bottomrule
    \end{tabular}

\end{table}