\section{Material und Methoden}

\subsection{Zeitablauf}

Im Modul A – „Virenmonitoring“ wird im September die Prävalenz von acht Bienenviren in drei aufeinander folgenden Jahren erhoben (2018-2020). Der September wurde als Beprobungsmonat gewählt, weil die meisten Viren zu diesem Zeitpunkt die höchste Prävalenz aufweisen \citep{demiranda2013}. Die dreijährige Laufzeit erlaubt es, die Prävalenz der Viren zwischen den Jahren zu vergleichen.
Vor Beginn des ersten Versuchsjahres wurden interessierte Imkerinnen und Imker für eine Teilnahme gewonnen (Februar - Mai 2018, \cref{chap:werbung}). Aus diesen wurden mittels einer stratifizierten Zufallsauswahl die 200 Teilnehmerinnen und Teilnehmer für die Studie ausgesucht und ihre Teilnahme fixiert (\cref{fig:a:projektverlauf}, \cref{chap:auswahl}). Es ist geplant, dass diese über den gesamten Zeitraum an dem Projekt teilnehmen. Eventuelle Ausfälle werden im Frühsommer des jeweiligen Jahres durch Interessentinnen und Interessenten von der Warteliste ersetzt.

Das Virenmonitoring läuft in allen drei Jahren ident ab (\cref{fig:a:projektverlauf}). Ende August werden die Materialien zur Durchführung der Probenahme von der Abteilung Bienenkunde und Bienenschutz (BIEN) an die Teilnehmerinnen und Teilnehmer verschickt. In den ersten Septemberwochen führen die Teilnehmerinnen und Teilnehmer die Probenahme gemäß der beigelegten Arbeitsanleitung durch (siehe \cref{chap:probenahme}) und verschicken die Bienen danach lebend in Königinnenversandkäfigen an die Abteilung BIEN. Sofort nach Eintreffen werden die Proben bei -18°C tiefgekühlt und gelagert (siehe \cref{chap:probenbearbeitung}). Im weiteren Verlauf werden die Proben dann für die Virusanalytik vorbereitet.

Die Probenanalyse auf die acht zu untersuchenden Bienenviren wird an der AGES, Abteilung Molekularbiologie (MOBI) des Instituts für veterinärmedizinische Untersuchungen Mödling (IVet-Mödling), mittels quantitativer real time RT-PCR im jeweils der Probenahme folgenden Winter durchgeführt (siehe \cref{chap:virusanalytik}). Die Teilnehmerinnen und Teilnehmer werden jedes Jahr voraussichtlich im auf die Probenahme folgenden Februar über die Ergebnisse der Virusanalyse informiert. Im darauffolgenden Frühling melden die teilnehmenden Imkerinnen und Imker die Überwinterungsergebnisse für die Probenvölker sowie für den gesamten beprobten Bienenstand an die Abteilung BIEN zurück (siehe \cref{chap:erhebung_winterv}). Diese Daten werden für die Prüfung möglicher statistischer Zusammenhänge zwischen der Virusprävalenz / der Höhe der Virusbelastung (Virustiter) und Winterausfällen verwendet.

\myfig{project-A-virenmonitoring/figures/MM_Schema_Versuchsjahre}
%% filename in given folder
{width=\textwidth,height=\textheight}
%% maximum width/height, aspect ratio will be kept
{Schematischer Ablauf des Virenmonitorings pro Versuchsjahr.}%% caption
{}%% optional (short) caption for table of figures
{fig:a:projektverlauf}%% label


\subsection{Werbung und Auswahl der TeilnehmerInnen}\label{chap:werbung}

Eine Stichprobe sollte repräsentativ für die Grundgesamtheit sein, damit ihre Ergebnisse deren Eigenschaften wahrheitsgemäß wiedergeben. Eine repräsentative Stichprobe lässt sich unter anderem dadurch erstellen, dass die Elemente der Stichprobe zufällig aus der Grundgesamtheit ausgewählt werden \citep{vanderzee2013}. Dies hieße im Falle des Virenmonitorings, dass die teilnehmenden Bienenstände zufällig aus einer Liste aller in Österreich vorhandenen Bienenstände gezogen werden. Ein solches Bienenstandregister liegt in Österreich seit dem Jahr 2017 im Rahmen des Veterinärinformationssystems (VIS) vor. Diese VIS-Aufzeichnungen dürfen jedoch ausschließlich zum Zwecke der Überwachung und Bekämpfung von Tierseuchen und der Überwachung der Lebensmittelsicherheit verwendet werden (TSG § 8 Abs. 6). Daher war es nicht möglich, diese Daten zur Auswahl der TeilnehmerInnen des Virenmonitorings heranzuziehen. Stattdessen wurde in der Imkerschaft für die Teilnahme am „Virenmonitoring“ geworben. 

\myfig{project-A-virenmonitoring/figures/MM_ZuBi2_Facebook}
  {width=\textwidth} % Größe Relativ zu Text Breite
  {Beispiele der Werbung für das Virenmonitoring auf Facebook und Twitter.} % Text unterhalb der Grafik
  {} % Optional Kurz Überschrift
  {fig:b:facebook} % Label zum Verweisen im Text

Bei der Teilnehmerwerbung wurde eine Vielzahl an Informationskanälen verwendet, um eine möglichst breite Bandbreite an Imkerinnen und Imkern zu erreichen und einen systematischen Auswahlfehler zu vermeiden. Wir informierten die Imkerinnen und Imker über die Möglichkeit zur Teilnahme sowohl über die Verbandstrukturen (Werbung auf Jahreshauptversammlungen, Email-Aufruf an die Landesverbände, etc.) und die österreichische Imkerzeitschrift „Bienen aktuell“ als auch über alternative Kanäle wie den AGES Facebook- und Twitter-Account (\cref{tab:a:werbung}, \cref{fig:b:facebook}). Dabei wurden zur Verstärkung der Botschaft „Wir machen den Viren-Check“ auch Imker und Imkerinnen, die sich dazu bereit erklärt hatten, als aktive Vorbilder präsentiert.
Außerdem wurden Teilnehmerinnen und Teilnehmer aus der Beobachtungsstudie und der COLOSS-Studie des Vorprojektes „Zukunft Biene“, die einer weiteren Kontaktaufnahme schriftlich zugestimmt hatten, gezielt angeschrieben. Die Anmeldung zur Teilnahme erfolgte sowohl über ein Online-Formular als auch per Papierformular, das in der Fachzeitschrift „Bienen aktuell“ abgedruckt war, sowie auf Imkerveranstaltungen ausgelegt wurde.

% Hannes: Habe euren Table richtig gestellt, es ist vielleicht eine gute Idee die Tabellen in einem Ordner zu sammeln und im Dokument über  einzubinden. Wenn viele Tabellen sind behält man so den Überblick besser. 

\begin{table}[htp]
    \caption{Maßnahmen zur TeilnehmerInnenwerbung für das Virenmonitoring im Zeitraum Februar bis Mai 2018}
    \centering
    \begin{tabular}{l|p{13cm}}
    \multicolumn{2}{l}{\textbf{Internet}} \\
    \hline
    Februar 2018  & Aufruf auf \url{www.zukunft-biene.at} \\
    Februar 2018  & Aufruf auf \url{www.imkerbund.at} \\ 
    Februar 2018  & Aufruf auf \url{www.biene-oesterreich.at} \\ 
    Mai 2018      & Facebook und Twitter Kampagne (6 Postings) \\ 
    
    \multicolumn{2}{l}{\textbf{}} \\
    \multicolumn{2}{l}{\textbf{Bienen aktuell}} \\
    \hline
    März 2018     & Artikel in „Bienen aktuell“ \\ 
    Mai 2018      & Erinnerung in „Bienen aktuell“ \\ 
    
    \multicolumn{2}{l}{\textbf{}} \\
    \multicolumn{2}{l}{\textbf{Vortrag/Werbung}} \\
    \hline
    27.Jänner 2018    & Jahreshauptversammlung des Imkervereins Oberes Feistritztal \\
    17.Feburar 2018   & Jahreshauptversammlung des Wiener Landesverbandes für Bienenzucht \\ 
    24.Feburar 2018   & Österreichische Erwerbsimkertagung \\
    24.März 2018      & Jahreshauptversammlung des NÖ Landesverbandes für Bienenzucht \\
    21.April 2018     & Bundesversammlung des österreichischen Imkerbundes \\
    25.April 2018     & Gesundheitsreferententagung \\
    
    \multicolumn{2}{l}{\textbf{}} \\
    \multicolumn{2}{l}{\textbf{Email-Aufrufe}} \\
    \hline
    
    März, April 2018 & Email-Aufrufe an die Landesverbände mit der Bitte, auf den Generalversammlungen für die Teilnahme zu werben (sofern kein Vortrag über das Projekt Zukunft Biene 2 stattfand) \\
    Februar 2018 & Email-Aufruf an 91 TeilnehmerInnen der Beobachtungsstudie von Zukunft Biene, die einer weiteren Kontaktaufnahme zugestimmt hatten\\
    März 2018 & Email-Aufrufe an ca. 1500 österreichische ImkerInnen der COLOSS-Studie, die einer weiteren Kontaktaufnahme zugestimmt hatten\\
    April, Mai 2018 & Zweiter Emailaufruf mit der Bitte um TeilnehmerInnen-Werbung an die Landesverbände der Bundesländer Kärnten, Salzburg, Oberösterreich und Vorarlberg aufgrund unterdurchschnittlicher Meldungsfrequenz in diesen Bundesländern \\
    
    \end{tabular}
    \label{tab:a:werbung}
\end{table}


\subsection{Stichprobengröße und Auswahl der TeilnehmerInnen} \label{chap:auswahl}

Als Stichprobengröße für das Virenmonitoring wurde die Anzahl von 200 Ständen gewählt. Bei einer Stichprobe dieser Höhe können wir davon ausgehen, dass die Prävalenz eines Virusaufkommens innerhalb eines 95\%igen Konfidenzintervalls angegeben werden kann, das in etwa ±7\% Schwankungsbreite besitzt (\cref{fig:c:samplesize}). Dies heißt, dass Prävalenzunterschiede von 14\% zwischen den Versuchsjahren bei einer Stichprobengröße von 200 als signifikant erkannt werden können. Um die Sensitivität des Monitorings derart zu erhöhen um Prävalenzunterschiede zwischen den Versuchsjahren von 10\% erkennen zu können (Schwankungsbreite ±5\%), müsste die Stichprobengröße auf fast das Doppelte erhöht werden (\cref{fig:c:samplesize}) und wäre daher mit einem unverhältnismäßigen Aufwand an Kosten verbunden. Im vorliegenden Monitoring wird pro Stand eine Sammelprobe von fünf Völkern ausgewertet. Das erhöht die Wahrscheinlichkeit eines positiven Virusnachweises, da nur eines von fünf Völkern infiziert sein muss, um ein positives Ergebnis zu erhalten. Es ist daher zu erwarten, dass auf Standniveau tendenziell höhere Prävalenzen auftreten werden als dies bei der Auswertung von Einzelvölkern der Fall wäre.

\myfig{project-A-virenmonitoring/figures/MM_Probenvolumen}
  {width=\textwidth} % Größe Relativ zu Text Breite
  {Berechnung der notwendigen Stichprobengröße, um bei einer binomialen Verteilung der unterschiedlichen zu erwartenden Prävalenzen ein 95\%iges Konfidenzintervall von ±5\%, ±6\% und ±7\% zu erreichen. Basis der Berechnung ist das Wilson-Konfidenzintervall, Berechnung durchgeführt in R mit der Version 3.4.1 \citep{rcoreteam2020} mit dem package „binomSamSize“ (Höhle, 2017). Die gewählte Stichprobengröße von 200 Ständen ist als rote Linie eingezeichnet. Überschreitet der graue Balken die rote Linie, wäre eine höhere Anzahl an Stichproben als die gewählten 200 notwendig um die gewünschte Schwankungsbreite des Konfidenzintervalls zu erreichen (fast immer bei ±5\%, nie bei ±7\%).} % Text unterhalb der Grafik
  {} % Optional Kurz Überschrift
  {fig:c:samplesize} % Label zum Verweisen im Text

Vor Durchführung der Auswahl wurde der Interessenten-Datensatz von Mehrfachmeldungen bereinigt (insgesamt 297 Interessenten meldeten 339 Standorte). Dies gewährleistete, dass Imkerinnen und Imker mit Einfachmeldungen eine gleich große Chance hatten ausgewählt zu werden, wie jene, die mit mehreren Standorten im Datensatz vertreten waren. Die Auswahl der TeilnehmerInnen erfolgte nach einer stratifizierten Zufallsauswahl. Die Stratifizierung betraf die geografische Verteilung der ausgewählten Bienenstände über Österreich. Das heißt jedes Bundesland war anteilig in dem Ausmaß vertreten, der dem Anteil der im VIS gemeldeten Bienenstände des Bundeslandes an der Gesamtanzahl der österreichischen Bienenstände entsprach \cref{tab:b:probenanzahl}.

\begin{table}[htp]
    \caption{VIS-Angaben zur Verteilung der Bienenstände (Bienenst.) über die österreichischen Bundesländer (Quelle: BMASGK, Stichtag: 31.10.2017), sowie die Anzahl der freiwilligen Meldungen zur Projektteilnahme (Interessenten), der teilnehmenden Bienenstände und erhaltenen Proben im Monitoringjahr 2019. Für die Berechnungen anhand der VIS-Daten wurde nur auf Bienenstände zurückgegriffen, die mit mindestens einem Volk im VIS registriert waren.}
    \centering
    \begin{tabular}{|l|c|c|c|c|}
    \hline
    Bundesländer    & Bienenst. Österreich &  Interessenten & Bienenst. 2019        & Proben 2019 \\ [0.5ex]
                    & Prozent                &  Anzahl       &   Anzahl (Prozent)    & Anzahl (Prozent)\\
    \hline
    Burgenland      &   3\%     &   13   & 6 (3,0\%)    & 6 (3,1\%)     \\
    Kärnten         &   12\%    &   38   & 23 (11.5\%)  & 22 (11,4\%)       \\
    NÖ              &   20\%    &   64   & 39 (19,5\%)  & 35 (18,1\%)       \\
    OÖ              &   22\%    &   60   & 43 (21,5\%)  & 42 (21,8\%)       \\
    Salzburg        &   7\%     &   21   & 14 (7,0\%)   & 14 (7,3\%)       \\
    Steiermark      &  18\%     &   37   & 37 (18,5\%)  & 35 (18,1\%)       \\
    Tirol           &   10\%    &   30   & 20 (10,0\%)  & 20 (10,4\%)       \\
    Vorarlberg      &   6\%     &   16   & 11 (5,5\%)   & 10 (5,2\%)       \\
    Wien            &   3\%     &   18   & 7 (3,5\%)    & 9 (4,7\%)       \\
    \hline
    Gesamt          &   100\%   &   297  & 200 (100\%)  & 193 (100\%)       \\
    \hline
    \end{tabular}
    \label{tab:b:probenanzahl}
\end{table}


\subsection{Durchführung der Probenahme} \label{chap:probenahme}

Die Probenahme wird jeweils im September der drei Versuchsjahre von den Imkerinnen und Imkern selbst durchgeführt (\cref{fig:d:sampling}, \cref{chap:anhang_Anleitung}). Dafür werden sie von der Abteilung BIEN per Post mit allen notwendigen Materialien zur Durchführung ausgestattet (eine ausführliche bebilderte Arbeitsanleitung, Fragebogen, Plastiketiketten, Königinnenkäfige samt Futterteig, einen Fragebogen, frankiertes Rücksendekuvert). Die Arbeitsanleitung zur Probenahme ist im Anhang I (\cref{chap:anhang_Anleitung}) zu finden, der Fragebogen im Anhang II (\cref{chap:anhang_Fragebogen}).

Im ersten Versuchsjahr wählten die Imkerinnen und Imker die fünf Versuchsvölker nach einem vorgegebenen Schema aus und markierten sie mit Plastiketiketten. Sie wurden angewiesen, diese Etiketten die ganze Versuchsdauer an den jeweiligen Völkern zu belassen, um die Völker zweifelsfrei während der gesamten Studiendauer identifizieren zu können. Sollte im Laufe der drei Studienjahre ein Volk ausfallen, wird es durch ein anderes Volk des Bienenstandes ersetzt.

Die Bienenproben werden aus jener Zarge entnommen, in der sich die Brut befindet oder in der Zarge des Bienensitzes, falls keine Brut vorhanden ist. Sie wird auf jener brutfreien Wabe genommen, die an die äußerste Brutwabe anschließt (\cref{fig:d:sampling}a). Falls keine Brut vorhanden ist, wird die Probe von einer äußeren Wabe des Bienensitzes genommen. Pro Probenvolk wird ein Königinnenversandkäfig mit zehn Bienen gefüllt (\cref{fig:d:sampling}b). Als optionale Zusatzaufgabe bitten wir die Imkerinnen und Imker ihre Völker auf folgende fünf Krankheitssymptome durchzusehen, die mit Virenbefall in Verbindung stehen können: erhöhter Bienentotenfall vor dem Volk, Varroamilben auf Bienen, Bienen mit verkrüppelten Flügeln, schwarz glänzende Bienen, Sackbrutsymptome in der Brut. 
Diese Aufgaben sind optional, da zur Durchführung die bienenbesetzen Waben des gesamten Bienenvolks überprüft werden müssen, was sehr zeitaufwändig ist.


\myfig{project-A-virenmonitoring/figures/MM_Probenahme}
  {width=\textwidth} % Größe Relativ zu Text Breite
  {Ablauf der Probenahme. Die teilnehmenden Imkerinnen und Imker (a) identifizieren die Wabe zur Probenahme (hier: Wabe angrenzend an das Brutnest) und (b) füllen einen Königinnenversandkäfig mit zehn Bienen. (c) Käfig und Fragebogen werden in ein vorfrankiertes und adressiertes Kuvert gefüllt und schnellstmöglich an die AGES geschickt. Fotos aus der bebilderten Arbeitsanleitung.} % Text unterhalb der Grafik
  {} % Optional Kurz Überschrift
  {fig:d:sampling} % Label zum Verweisen im Text


Die fünf gefüllten Königinnenversandkäfige werden gemeinsam mit dem ausgefüllten Fragebogen in das vorfrankierte Rücksende-Kuvert gegeben (\cref{fig:d:sampling}c). Das Kuvert ist entsprechend dem PRIO-Tarif der österreichischen Post für Päckchen frankiert, der einen raschen Versand mit Zustellung am nächsten Tag verspricht. Das Kuvert wird noch am selben oder am nächsten Tag bei der Post aufgegeben. Wir ersuchen die Imkerinnen und Imker das Kuvert zwischen Montag und Mittwoch aufzugeben, um zu verhindern, dass der Bienenversand über das Wochenende abläuft und damit unnötig verlängert wird. Sofort nach Eintreffen der Kuverts in der AGES werden diese auf -18°C gekühlt und die Bienen somit schnell abgetötet.

Zusätzlich werden die Teilnehmerinnen und Teilnehmer ersucht den beigelegten Fragebogen aus zehn Fragen zu beantworten (\cref{chap:anhang_Fragebogen}). Dieser enthält allgemeine Fragen zu Imkerbetrieb und Bienenstand, sowie Fragen zu Hygienemaßnahmen und Volksgesundheit. Zusätzlich wird abgefragt, bei wie vielen der Probenvölker die definierten Krankheitssymptome beobachtet wurden (optional).


\subsection{Probenbearbeitung} \label{chap:probenbearbeitung}

Nachdem die geschlossenen Proben-Kuverts mit den Bienenproben mindestens 24 Stunden bei \SI{-18}{\degreeCelsius} gelagert worden waren, wurden sie weiterbearbeitet. Der Fragebogen wurde entnommen, auf Angabe des Probenahmedatums überprüft und mit LISA-Etikett des Laborinformationssystems versehen. Aus jedem der fünf Bienenkäfige wurden zehn Arbeiterinnen in ein „Extracting Bag“ (BIOREBA AG) überführt und \SI{15}{\milli\liter} DEPC-behandeltem Wasser (Ambion) dazupipettiert (\SI{3}{\milli\liter} pro 10 Bienen). Eventuell vorhandene Drohnen wurden aussortiert. Im Jahr 2019 waren bei 39 Proben zu wenige Arbeiterinnen in den Käfigen (durchschnittliche Anzahl Bienen: 49,5 ± 1,4 SD: minimale Anzahl Bienen: 39). Dies wurde vermerkt und die Menge an DEPC-Wasser angepasst. Vier Einsendungen enthielten vier anstatt fünf Käfige; in diesen Fällen wurden 40 Bienen entnommen und ebenfalls die Menge des DEPC-Wassers angepasst. Die Bienen wurden mit Hilfe eines Homogenisators (HOMEX 6, Bioreba) homogenisiert. Das Homogenat wurde bis zur Weiterverwendung bei \SI{-18}{\degreeCelsius} gelagert.

\subsection{Abfrage Winterverluste} \label{chap:erhebung_winterv}

Es wurden die Winterverluste sowohl für die fünf Probenvölker als auch für den gesamten Bienenstand, auf dem das Monitoring durchgeführt wurde, abgefragt. Es wurde die Anzahl der eingewinterten Völker sowie die Anzahl der Völker, die am Ende des Winters abgestorben waren (tote Völker, leere Beuten), erhoben. Zusätzlich wurde abgefragt, wie viele der bei der Auswinterung lebenden Völker weisellos oder drohnenbrütig waren. Derartige Völker werden in der COLOSS-Studie ebenfalls zu den Winterverlusten gezählt und wurden erfasst, um eine Vergleichbarkeit zu erzielen \citep{brodschneider2016,brodschneider2018}. Die Winterverluste wurden bei den meisten TeilnehmerInnen online durch einen an die TeilnehmerInnen per Email verschickten Link abgefragt. Auf Wunsch wurde auch ein Fragebogen auf Papier ausgeschickt. In Ausnahmefällen erfolgte die Abfrage per Telefon.

\subsection{Virusanalytik} \label{chap:virusanalytik}

Die Bienenproben wurden an der AGES, Abteilung für Molekularbiologie (MOBI), des Instituts für veterinärmedizinische Untersuchungen Mödling (IVet-Mödling), mittels quantitativer real time RT-PCR (RT-qPCR) auf acht Bienenviren untersucht (ABPV, BQCV, CBPV, DWV-A, DWV-B, IAPV, KBV, SBV). 
Für die Analyse wurden die Bienen aller fünf Bienenproben eines Standes zu einer Sammelprobe vereint. Dies hat den Vorteil, dass mit einer einzigen Probe kostengünstig ein Überblick über die vorkommenden Viren und die Virusbelastung am Bienenstand möglich ist. Allerdings ist als Folge der Sammelprobenbildung eine statistische Prüfung auf mögliche Korrelationen mit Winterverlusten ausschließlich auf der Ebene der Sammelprobe und nicht auf der Ebene des Einzelvolks möglich.

\subsubsection{Plasmide und Bienenhomogenate zur Methodenetablierung und Validierung} \label{chap:plasmide_homogenate}

Vom EU-Referenzlabor (EU-RL) für Bienengesundheit in Frankreich (Anses Sophia Antipolis) wurden die in der \cref{tab:c:plasmide} aufgelisteten Plasmide als Standards zur Methodenetablierung und zur absoluten Quantifizierung der entsprechenden Bienenviren zur Verfügung gestellt. Die fünf Viren ABPV, BQCV, CBPV, DWV und SBV werde in Folge mit ABCDS abgekürzt, wobei bei DWV die beiden Genotypen DWV-A und DWV-B, letzteres auch bekannt als Varroa destructor virus-1 – VDV-1 –, unterschieden werden. Alle Plasmide standen nur in sehr begrenzter Menge zur Verfügung, jeweils \SI{12}{\micro\liter} der G9 Verdünnung (0,2×$10^9$ Kopien/\si{\micro\liter}). Zudem wurden sechzehn Bienenhomogenate vom EU-RL zur Methodenetablierung und im Rahmen eines Ringversuches für CBPV zur Verfügung gestellt (\cref{tab:d:homogenate}).

\begin{table}
    \centering
    \caption{Vom EU-RL im Jahr 2017 zu Etablierungs- und Validierungszwecken zum Nachweis von ABPV, BQCV, CBPV, DWV-A, DWV-B und SBV zur Verfügung gestellte Plasmide.}
    \label{tab:c:plasmide}
    \begin{tabular}{|l|c|c|}
        \hline
        Virus   &   Plasmid, Bezeichnung EU-RL & Größe in Basenpaaren (bp)\\
        \hline
        ABPV            & pB2       & 4371\\
        BQCV            & pNC14     & 3716\\
        CBPV            & pAb1      & 3815\\
        DWV-A           & pC1       & 4393\\
        DWV-B (VDV-1)   & pFab1     & 4651\\
        SBV             & pD1       & 4442\\
        \hline
    \end{tabular}
\end{table}

\begin{table}
    \centering
    \caption{Vom EU-RL zu Validierungs- und Ringversuchszwecken zum Nachweis von ABPV, BQCV, CBPV, DWV-A und SBV (ABCDS) zur Verfügung gestellte Bienenhomogenate.}
    \label{tab:d:homogenate}
    \begin{tabular}{lll}
        \toprule
        \multirow{2}{*}{
            \shortstack{Probennummer MOBI \\ (Jahr des Probeneingangs)}
            } &
        \multirow{2}{*}{
            \shortstack{Probenbezeichnung \\ EU-RL}
            } &
        \multirow{2}{*}{
            \makecell{Virus}
            } \\
        & & \\
        %Probennummer MOBI           &   Probenbezeichnung EU-RL     & Virus\\
        %(Jahr des Probeneingangs)   &                            & \\
        \midrule
        3155-5 (2017)*                  & ABCDS Method adoption         & ABPV, BQCV, CBPV,\\ 
                                        & sample                        & DWV, SBV\\
        3155-6 (2017)*                  & ABCDS Method adoption         & ABPV, BQCV, CBPV,\\ 
                                        & sample                        & DWV, SBV\\
        3156-1 bis -10 (2017) (n=10)    & ILPT (CBPV) samples           & CBPV\\
        913-1 (2018)                    & CBPV-5.58                     & CBPV\\
        913-2 (2018)                    & CBPV-6.79                     & CBPV\\
        913-3 (2018)                    & CBPV-8.02                     & CBPV\\
        913-4 (2018)                    & CBPV-9.24                     & CBPV\\
        \bottomrule
        \multicolumn{3}{l}{*Bei diesen Proben handelt es sich laut Auskunft vom EU-RL um Replikate}
    \end{tabular}
\end{table}

Von der Abteilung BIEN wurden zehn Proben (Bienenhomogenate, Nukleinsäureextrakte, cDNA bzw. Plasmide) für die Etablierung der Reverse Transkriptase quantitative PCR (RT-qPCR) für IAPV und KBV zur Verfügung gestellt (\cref{tab:e:validierung}).

\begin{table}
    \centering
    \caption{: Von der Abteilung BIEN zu Etablierungs-, und Validierungszwecken zum Nachweis von IAPV und KBV zur Verfügung gestellte Proben, Virusnachweis erfolgte in der Abteilung BIEN mittels quantitativer RT-PCR. Zitate geben über die Arbeiten Auskunft, im Rahmen derer die jeweiligen Proben gewonnen worden waren.}
    \label{tab:e:validierung}
    \begin{tabular}{*{4}{l}}
        \toprule
        \multirow{2}{*}{
        \shortstack{Probennr. \\ MOBI}
        } &
        \multirow{2}{*}{
        \shortstack{Bezeichnung \\ BIEN}
        } &
        \multirow{2}{*}{
        \shortstack{Probenart}
        } &
        \multirow{2}{*}{
        \shortstack{Virus-Nachweis \\ Abteilung BIEN}
        } \\
        &&& \\
        \midrule
        %Probennr.       &   Bezeichung          & Probenart                 & Virus- Nachweis\\
        %MOBI            &   BIEN                &                           & Abteilung BIEN\\
        %\hline
        856-1 (2018)    & IT-1/730              & RNA-Extrakt (verdünnt     & SBV, IAPV\\ 
                        &                       & 1:30), \SI{20}{\micro\liter}  &\\
        %\hline
        856-2 (2018)    & IT-2/730              & Bienenhomogenat, \SI{500}{\micro\liter} & SBV, IAPV\\ 
        %\hline
        856-3 (2018)    & IT-3/733              & Bienenhomogenat, \SI{500}{\micro\liter} & BQCV, DWV, SBV,\\
                        &                       &                           & IAPV\\
        %\hline
        856-4 (2018)    & IT-4/IAPV-            & Plasmidischer Klon pTB-   & IAPV\\
                        & Plasmid               & 47 der publizierten       &\\
                        &                       & Sequenz EF219380 aus      &\\
                        &                        & Israel; ca. \SI{2}{\milli\liter} &\\
        %\hline
        856-5 (2018)    & IT-5/ IAPV-           & Plasmidischer Klon A5     & IAPV\\
                        & Plasmid               & \citep{blanchard2008}     &\\
                        &                       & ca. \SI{100}{\micro\liter} &\\ 
        %\hline
        856-6 (2018)    & IT- IAPV- cDNA        & cDNA aus Isolat 57-2      & IAPV\\
                        &                       & \citep{blanchard2008}     &\\
                        &                       & ca. \SI{40}{\micro\liter} &\\
        %\hline
        856-7 (2018)    & KT-1/1125             & RNA-Extrakt, \SI{20}{\micro\liter} & BQCV, SBV, KBV\\
        %\hline
        856-8 (2018)    & KT-2/1125             & Bienenhomogenat, \SI{500}{\micro\liter} & BQCV, SBV, KBV\\
        %\hline
        856-9 (2018)    & KT-3/1126             & Bienenhomogenat, \SI{500}{\micro\liter} & BQCV, SBV, KBV\\
        %\hline
        856-10 (2018)   & KT-4/KBV-cDNA         & cDNA \citep{siede2005}    & KBV\\
                        &                       & aus Deutschland, ca. \SI{100}{\micro\liter} &\\
        \bottomrule
    \end{tabular}
\end{table}

Da die vom EU-RL bereitgestellten ABCDS Plasmide nur in sehr begrenztem Umfang zur Verfügung standen, wurden diese nach Rücksprache mit dem EU-RL vermehrt. Dazu wurden Verdünnungen (1:10 in Wasser) dieser Plasmide in chemisch kompetente \textit{Escherichia coli} Bakterien transformiert (MAX Efficiency\textsuperscript{\textregistered}
 DH5$\alpha$\textsuperscript{TM} Competent Cells; Invitrogen), auf Agarplatten mit Ampicillin (imMedia\textsuperscript{TM} Growth Medium, agar, ampicillin, X-gal/IPTG; ThermoFisher) ausplattiert und entsprechende Kolonien selektiert und in Liquid Broth plus Ampicillin (\SI{50}{\micro\gram}/\si{\milli\liter}) kultiviert. Aus diesen Flüssigkulturen wurden danach die in \textit{E. coli} vermehrten Plasmide mit dem GeneJET Plasmid Miniprep Kit (ThermoFisher) aufgereinigt und deren Reinheit und Menge mithilfe eines Photometers (SmartSpec\textsuperscript{TM} 3000 Spectrophotometer, BIORAD) bestimmt. Zusätzlich wurde die DNA-Menge noch mit Fluorimetrie (Qubit 3 Fluorometer und Qubit dsDNA BR Assay Kit; ThermoFisher) bestimmt. Sämtliche Plasmide zeigten eine hohe Reinheit (OD260/280 Ratio >1,7). Die beiden Quantifizierungsmethoden stimmten sehr gut überein (maximale Abweichung 25\%). Die Kopienanzahl wurde aus der bekannten Plasmidgröße (\cref{tab:c:plasmide}) und der mittels Fluorimetrie bestimmten DNA-Konzentration der Plasmidlösung ($c_{\text{Plasmidlösung}}$) mit folgender Formel berechnet:

\begin{equation}
    c_{\text{Plasmidlösung}} = \frac{\text{DNA-Konzentration}}{\text{Molekulargewicht des Plasmids}}
\end{equation}

\begin{equation}
    \frac{\text{Anzahl Kopien}}{\si{\micro\liter}} = c_{\text{Plasmidlösung}}*6,02*10^{23}
\end{equation}

Entsprechende Verdünnungsreihen im Konzentrationsbereich von $10^1$–$10^8$ Kopien/\SI{5}{\micro\liter} wurden für alle sechs Plasmide (\cref{tab:c:plasmide}) in nukleasefreiem Wasser unter Zusatz von \SI{30}{\nano\gram}/\si{\micro\liter} tRNA angefertigt. Ein mittels qPCR durchgeführter Vergleich der Ct-Werte aus Verdünnungsreihen, die einerseits anhand der in der Abteilung MOBI klonierten Plasmide, andererseits anhand der vom EU-RL zur Verfügung gestellten Plasmide hergestellt worden waren, ergab mit Ausnahme von CBPV eine nahezu perfekte Übereinstimmung der erhaltenen Ct-Werte. Bei CBPV ergab sich ein deutlicher Unterschied zwischen den Plasmiden, die einerseits von der Abteilung MOBI, andererseits vom EU-RL hergestellt, bzw. quantifiziert worden waren. Nachdem die von der Abteilung MOBI erhaltenen Werte auch durch Wiederholung und Nachtestung mit einem weiteren CBPV-Klon reproduzierbar waren und alle Plasmide auf die gleiche Art hergestellt, quantifiziert und deren Konzentration berechnet worden war, wird davon ausgegangen, dass die seitens MOBI berechneten Werte korrekt sind und es wurden in der Folge sämtliche CBPV-Quantifizierungen der Projektproben basierend auf dem von der MOBI hergestellten CBPV-Plasmid durchgeführt. Der Unterschied zwischen dem CBPV-Plasmid der Abteilung MOBI und dem vom EU-RL wirkt sich in der absoluten Quantifizierung in einem etwa zehnfachen Unterschied in der berechneten Viruslast aus. Die anhand des EU-RL Plasmides berechneten Viruslasten wären etwa zehnmal höher. Eine vergleichende Quantifizierung der EU-RL ABCDS-Referenzproben, sowohl mit den vom EU-RL bereitgestellten Standardplasmiden als auch den an der Abteilung MOBI propagierten Plasmiden, ergab weitgehend übereinstimmende Werte für alle sechs Bienenviren. Dementsprechend konnten die neu hergestellten Plasmide, die nun in ausreichender Menge vorlagen, für die Quantifizierung von ABPV, BQCV, CBPV, DWV-A, DWV-B und SBV im Rahmen dieses Projektes verwendet werden.

\subsubsection{Nukleinsäureextraktion aus Bienenhomogenaten} 
\label{chap:extraktion}

Die Nukleinsäureextraktion aus Bienenhomogenat-Proben erfolgte semi-automatisiert mit dem LSI MagVet\textsuperscript{TM} Universal Isolation Kit mit dem Protokoll „RNA purification from total blood \& serum” auf dem KingFisher\textsuperscript{TM} Flex (beides ThermoFisher). Die Extraktion erfolgte aus \SI{100}{\micro\liter} Homogenat und die gereinigte Nukleinsäure wurde in \SI{80}{\micro\liter} Puffer eluiert und bis zur Analyse bei \SI{-20}{\degreeCelsius} (Langzeitlagerung bei \SI{-80}{\degreeCelsius}) gelagert. Zu Vergleichszwecken wurden ausgewählte Bienenhomogenatproben manuell mit dem QIAamp Viral RNA Mini Kit (Qiagen), sowie automatisiert mit dem NucleoSpin\textsuperscript{\textregistered} 96 Virus Core Kit (Macherey-Nagel) auf der Freedom EVO\textsuperscript{\textregistered} 150 Plattform (Tecan) extrahiert. Die Extraktion erfolgte im Fall des QIAamp Viral RNA Mini Kit aus \SI{140}{\micro\liter} Homogenat und die gereinigte Nukleinsäure wurde in \SI{60}{\micro\liter} Puffer eluiert. Beim NucleoSpin\textsuperscript{\textregistered} 96 Virus Core Kit erfolgte die Extraktion aus \SI{100}{\micro\liter} Homogenat und die Elution in \SI{100}{\micro\liter} Puffer. Diese unterschiedlichen Verhältnisse im Proben- zu Eluatvolumen wurden bei der Berechnung der Viruskopienanzahl/\si{\milli\liter} berücksichtigt.

\subsubsection{Methodenetablierung und -validierung zum Nachweis und zur Quantifizierung von ABPV, BQCV, CBPV, DWV-A, DWV-B und SBV (ABCDS)}
\label{chap:methoden_abcds}

Für den Nachweis und die Quantifizierung von ABPV, BQCV, DWV-A, DWV-B und SBV mittels RT-qPCR existiert eine Standard operating procedure (SOP), die vom EU-Referenzlabor (EU-RL) für Bienengesundheit in Frankreich (Anses Sophia Antipolis) zur Verfügung gestellt wurde. Die in der SOP beschriebene ABPV RT-qPCR basiert auf der Quantifizierung der Kapsidprotein-Gensequenz \citep{jamnikar2012}; die BQCV RT-qPCR auf dem C-terminalen Bereich der Polyprotein Gen-Sequenz  \citep{chantawannakul2006}; die DWV-A und DWV-B RT-qPCR Methoden auf der Quantifizierung der jeweiligen VP3-Kodiersequenz \citep{schurr2019} und der SBV-Nachweis auf der Quantifizierung des N-terminalen Bereich der Polyprotein Gensequenz \citep{blanchard2014}. Für den Nachweis und die Quantifizierung von CBPV stand eine weitere SOP zur Verfügung. Die Methode beruht auf der Quantifizierung der RNA-abhängigen RNA-Polymerase (Rd-Rp) Gensequenz des CBPV-Genoms \citep{blanchard2007}. Sämtliche Primer- und Sondensequenzen wurden anhand der oben zitierten Literatur bestellt (ThermoFisher bzw. Eurofins), mit der Ausnahme, dass die Sonden für ABPV, BQCV, DWV-A, DWV-B und CBPV – anstelle des vom in den SOPs empfohlenen TAMRA – mit dem Black Hole Quencher 1 (BHQ-1) versehen wurden.

In den beiden oben genannten SOPs erfolgt die Detektion und Quantifizierung der Bienenvirus RNAs als two-step RT-qPCR (=Reverse Transkription-qPCR) mit getrennter Reverser Transkription und darauffolgender qPCR Amplifikation der im Reverser Transkriptions-Schritt erzeugten cDNA. Demgegenüber erfolgte die RT-qPCR im Rahmen dieses Arbeitspaketes als one-step RT-qPCR, wobei RT und qPCR hintereinander im selben Reaktionsgefäß stattfinden. Hauptvorteil der one-step RT-qPCR ist, dass die Anzahl an Pipettierschritten und damit mögliche Fehlerquellen reduziert werden. Zudem wird weniger Zeit für das Probenhandling benötigt, was die one-step RT-qPCR besonders für einen höheren Probendurchsatz, wie er in diesem Arbeitspaket gegeben ist, attraktiv macht. Ein weiterer Unterschied zu den in den beiden EU-RL SOPs angegebenen Protokollen besteht darin, dass hier keine parallele Amplifikation von interner Kontroll-DNA durchgeführt wurde. Im Gegensatz dazu wurde die Abwesenheit von PCR-Inhibitoren im Rahmen dieses Projektes durch den gesondert durchgeführten Nachweis der Apis-Actin mRNA in allen Proben bestätigt.

Die \SI{25}{\micro\liter} Reaktionsmixes für die ABCDS one-step RT-qPCRs bestanden aus 12,5 \si{\micro\liter} 2x RT-PCR buffer, \SI{1}{\micro\liter} 25x RT-PCR enzyme mix (AgPath-ID One-step RT-PCR kit, ThermoFisher), \SI{5}{\micro\liter} Nukleinsäureextrakt und den jeweils spezifischen Primern und fluoreszenzmarkierten TaqMan-Sonden, sowie nukleasefreiem Wasser. Die Konzentrationen der verwendeten Primer und Sonden entsprachen den in den beiden EU-RL SOPs angegebenen Werten. Demnach wurden für die ABPV RT-qPCR die Primer ABPV1 und ABPVRn (je \SI{800}{\nano\mol}), sowie die Sonde ABPVnTaq (\SI{100}{\nano\mol}) verwendet \citep{jamnikar2012}. Für die BQCV RT-qPCR wurden die Primer BQV8195F und BQV8265R (je \SI{320}{\nano\mol}), sowie die Sonde BQCV8217T (\SI{200}{\nano\mol}) eingesetzt \citep{chantawannakul2006}. Für die CBPV RT-qPCR wurden die Primer qCBPV 9 und qCBPV 10 (je \SI{500}{\nano\mol}), sowie die Sonde CBPV 2 probe (\SI{200}{\nano\mol}) verwendet (Blanchard et al., 2007). Für die DWV-A RT-qPCR wurden die Primer F-DWV\_4250 und R-DWV\_4321 (je \SI{400}{\nano\mol}), sowie die Sonde Pr-DWV\_4293 (\SI{100}{\nano\mol}) verwendet, für die DWV-B RT-qPCR wurden die Primer F-VDV1\_4218 und R-VDV1\_4290 (je \SI{1200}{\nano\mol}), sowie die Sonde Pr-VDV1\_4266 (\SI{400}{\nano\mol}) eingesetzt \citep{schurr2019}. Für die SBV RT-qPCR wurden die Primer SBV-F434 und SBV-R503 (je \SI{320}{\nano\mol}), sowie die Sonde SBV-P460 (\SI{200}{\nano\mol}) eingesetzt \citep{blanchard2014}.

Zum Nachweis der erfolgreichen Nukleinsäureextraktion aus den Bienenhomogenaten, respektive zum Nachweis der Abwesenheit PCR-inhibitorischer Substanzen, wurde – wie schon im Vorgängerprojekt „Zukunft Biene“ – die Apis-Actin mRNA in den Bienenhomogenatproben semiquantitativ (d.h. ohne entsprechender Eichkurve) bestimmt \citep{morawetz2018}. Für die Apis-Actin RT-qPCR wurden die Primer Apis-$\beta$-actin-F und Apis-$\beta$-actin-R (je \SI{400}{\nano\mol}), sowie die Sonde Apis-$\beta$-actin-Probe (\SI{200}{\nano\mol}) verwendet \citep{chen2005}.

Das Temperaturprofil für die ABCDS-, sowie die Apis-Actin RT-qPCRs bestand aus einem RT-Schritt bei \SI{45}{\degreeCelsius}/\SI{10}{\minute}, gefolgt von \SI{95}{\degreeCelsius}/\SI{10}{\minute} zur Inaktivierung der Reversen Transkriptase und Aktivierung der Taq-Polymerase und 42 Zyklen mit jeweils Denaturierung bei \SI{95}{\degreeCelsius}/\SI{10}{\second} und Annealing/Extension bei \SI{60}{\degreeCelsius}/\SI{1}{\minute}. Die RT-qPCRs wurden auf dem 7500Fast Real-time PCR System (ThermoFisher) oder auf dem Mx3005P (Agilent) durchgeführt, die Auswertung erfolgte mit der 7500 Software v2.3 (ThermoFisher), bzw. mit der MxPro – Mx3005P v4.10 Software (Agilent).

Die eigentliche Methodenetablierung lief in zwei Schritten ab. Im ersten Schritt wurden die RT-qPCR Parameter, wie Limit of detection (LOD\textsubscript{PCR}), Linearität der Quantifizierung, PCR-Effizienz und Limit of quantification (LOQ\textsubscript{PCR}), erhoben. Im zweiten Schritt wurde die komplette Methode von der Nukleinsäureextraktion bis zur RNA-Quantifizierung anhand der Erhebung von LOD\textsubscript{Method} und LOQ\textsubscript{Method} evaluiert. Für beide Schritte hatte das EU-RL anhand einer Französischen Norm (AFNOR, 2015) Referenzwerte definiert, anhand derer die Konformität der Methode mit den EU-RL Standards überprüft werden konnte.

LOD\textsubscript{PCR}, Linearität, PCR-Effizienz und LOQ\textsubscript{PCR} waren bei ABPV, BQCV, CBPV, DWV-A und SBV unter Verwendung von vom EU-RL zur Verfügung gestellten Berechnungsvorlagen konform mit den EU-RL Standards: die G2-Verdünnungsstufe (100 Kopien/Reaktion) wurde in allen fünf Tests positiv detektiert, die PCR-Effizienz war bei allen fünf Tests im Bereich von 95-104\%, die Linearität der Quantifizierung wurde bei allen Tests im Bereich von $10^2$-$10^8$ Kopien/Reaktion bestätigt. Bei Anwendung derselben Vorlage für DWV-B wurde die Konformität der verwendeten RT-qPCR Methode auch für diesen Test bestätigt, hier lag die PCR-Effizienz etwas niedriger (87-90\%).

Bei der Überprüfung der kompletten Methode von der Nukleinsäureextraktion bis zur RNA-Quantifizierung konnten hinsichtlich DWV-B keine Aussagen getroffen werden, da kein entsprechender Standard vom EU-RL verfügbar war. Die übrigen fünf Methoden waren im Hinblick auf das LOD\textsubscript{Method} ebenfalls konform mit den EU-RL Standards. Beim LOQ\textsubscript{Method} schreibt das EU-RL eine maximale Abweichung der erhobenen Viruslast von ± einer log10-Stufe zum Erreichen der Konformität mit dem EU-RL Standard vor. Diese Konformität wurde für CBPV, DWV-A und SBV erreicht. Bei ABPV und BQCV lagen die Viruslasten/Biene aber um einen Faktor von durchschnittlich 1,78 (ABPV) bzw. 2,14 log10 (BQCV) über den EU-RL Referenzwerten. Diese Überschätzung der Viruslast bei den beiden genannten Bienenviren zeigte sich bei beiden verwendeten Extraktionsmethoden (zwei Methoden wurden hier anhand derselben Proben verglichen). Sie war jedoch bei Verwendung der semiautomatisierten Extraktion mit dem KingFisher\textsuperscript{TM} Flex etwas niedriger, sodass diese Methodik hier bevorzugt wurde. Interessanterweise hat das EU-RL bei eigenen Untersuchungen eine systematische Überschätzung der ABPV und BQCV Viruslasten beobachtet und in der Folge die Werte rechnerisch korrigiert \citep{schurr2019}. Eine derartige Korrektur wurde hier nicht durchgeführt. Die an derselben Stelle vom EU-RL beschriebene und rechnerisch korrigierte systematische Unterschätzung der DWV-B und SBV Viruslasten wurde bei unseren Untersuchungen nicht gesehen.


\subsubsection{CBPV-Ringversuch und Quantifizierung weiterer CBPV Referenzproben} \label{chap:ringversuch_cbpv}

Im Dezember 2017 nahm MOBI an einem Ringversuch des EU-RL für Bienengesundheit teil. Alle zehn Ringversuchsproben wurden qualitativ richtig erkannt. Bei der quantitativen Analyse der drei CBPV-positiven Ringtestproben wurden vergleichsweise niedrige Viruslasten erzielt (im Bereich von 6,31 bis 63,1-fach unter dem robusten Mittelwert/der robusten Standardabweichung aller Teilnehmer), jedoch waren die Ergebnisse im Hinblick auf Sensitivität, Spezifität, Präzision und Richtigkeit konform mit dem EU-RL Standard. Der Ringversuch wurde somit erfolgreich absolviert.

Vier weitere CBPV-Referenzproben (913-1 bis 913-4, \cref{tab:d:homogenate}) mit unterschiedlicher CBPV-Last wurden ebenfalls vom EU-RL bezogen und mit der beschriebenen Methodik quantifiziert. Bei Verwendung der Extraktion mit dem KingFisher\textsuperscript{TM} Flex wurden alle vier Proben mit einer Abweichung von unter einer log10 Stufe richtig quantifiziert.


\subsubsection{Methodenetablierung und -validierung zum Nachweis und zur Quantifizierung von IAPV und KBV} \label{chap:etablierung_iapv_kbv}

Für diese beiden RNA-Bienenviren standen keine Empfehlungen oder SOPs vom EU-RL für Bienengesundheit zur Verfügung. Die Methodenauswahl wurde daher anhand verfügbarer Literaturstellen \citep{coxfoster2007,demiranda2010,maori2007,palacios2008,stoltz1995,chantawannakul2006} oder von in der Abteilung BIEN vorhandenen Erfahrungen getroffen. Basierend auf \citep{demiranda2010} wurden verschiedene IAPV und KBV-spezifische Primer und Sonden in silico evaluiert. Dazu wurden mithilfe des Basic Local Alignment Search Tool (BLAST) des National Center for Biotechnology Information (\url{https://blast.ncbi.nlm.nih.gov/Blast.cgi}) Sequenzen mit entsprechender Homologie zu den von den jeweiligen Primern amplifizierten Genabschnitten gesucht und in der Folge mithilfe der Software BioEdit \citep{hall1999} die Anzahl an nicht-passenden Basenpaarungen zwischen den Primern und den entsprechenden Zielsequenzen analysiert. Basierend auf dieser Analyse erschienen die Primer KBV 6639F/6801R und IAPV 6627F/6792R \citep{demiranda2010} als am vielversprechendsten. Daher wurden diese Primer für eine SYBR-Green RT-qPCR synthetisiert (ThermoFisher). Zusätzlich wurden auch die Primer KBV 5425F/5800R \citep{stoltz1995} und IAPV 8880F/9336R \citep{maori2007} synthetisiert, da diese bereits in der Abt. BIEN verwendet worden waren und die Hintergrundinformation in \cref{tab:e:validierung} teilweise auf der Verwendung dieser beiden Primersysteme basiert.

In Vorversuchen zeigte sich, dass die KBV und IAPV Primer nach \cite{demiranda2010} jedoch bei den aus Österreich stammenden Proben 856-1 bis -3, sowie 856-7 bis -9 (2018) (\cref{tab:e:validierung}) kein positives Ergebnis erbrachten, sodass das in Österreich vorkommende IAPV und KBV damit möglicherweise nicht erfasst würden. Im Gegensatz wurden in konventionellen RT-PCRs mit den KBV und IAPV Primern nach \cite{stoltz1995} und \cite{maori2007} auch in den österreichischen Bienenproben (\cref{tab:e:validierung}) Amplifikate in den erwarteten Größen erhalten, die sich bei Sequenzierung (BigDye Terminator v3.1 Cycle Sequencing Kit und 3130xl Genetic Analyzer; Thermo Fisher) und anschließender BLAST-Analyse zuerst als IAPV und KBV bestätigen ließen. In der Folge wurden zwei aus österreichischen Bienenproben stammende IAPV- und KBV-Amplifikate (aus den Proben 856-2 und 856-8; \cref{tab:e:validierung}) mithilfe des TOPO TA Cloning\textsuperscript{\textregistered} Kit (Invitrogen) in ein Plasmid-Backbone inseriert und mit der in \cref{chap:plasmide_homogenate} bereits beschriebenen Methodik in \textit{E. coli} vermehrt, die Plasmid-DNA daraus präpariert und quantifiziert. Zusätzlich wurde das Vorhandensein des IAPV- bzw. KBV Amplifikates in der gereinigten Plasmid-DNA mittels Sequenzierung bestätigt. Allerdings stellte sich bei weiteren Untersuchungen heraus, dass die mittels RT-PCR nach \cite{maori2007} aus österreichischen Proben erhaltenen IAPV Sequenzen möglicherweise als Resultat einer Kreuzkontamination zu betrachten sind, während es sich bei den mittels RT-PCR nach \cite{stoltz1995} aus österreichischen Proben erhaltenen Sequenzen nach phylogenetischer Analyse nicht um KBV, sondern um ein verwandtes, allerdings weder ABPV, IAPV oder KBV zuzuordnendes Virus handeln dürfte. In derselben phylogenetischen Analyse wurde gezeigt, dass die Proben 856-6 (2018) und 856-10 (2018) tatsächlich als IAPV bzw. KBV anzusprechen sind \cref{tab:e:validierung}. Beide Proben wurden aus dem Ausland (Frankreich bzw. Deutschland) bezogen \citep{siede2005,blanchard2008}. Somit bestätigte sich die Darstellung von \cite{demiranda2010}, wonach die RT-PCR nach \cite{stoltz1995} als nicht spezifisch für KBV anzusehen ist. In der Folge wurde daher beschlossen, in Abwesenheit anderer als geeignet erscheinender RT-PCR Protokolle für die weiteren Untersuchungen mit den Primern KBV 6639F/6801R und IAPV 6627F/6792R \citep{demiranda2010} weiterzuarbeiten. Bei den Reverse Primern KBV 6801R und IAPV 6792R handelt es sich um denselben Primer, der zu Vereinfachungszwecken hier auf KBV-IAPV-R umbenannt wurde.

Für die KBV, beziehungsweise die IAPV RT-qPCR, wurden somit jeweils die Primer KBV 6639F/ KBV-IAPV-R, beziehungsweise die Primer IAPV 6627F/KBV-IAPV-R (je \SI{450}{\nano\mol}), zusammen mit 1x POWER SYBR Green Master Mix und 1x RT Enzyme Mix (ThermoFisher), sowie nukleasefreiem Wasser und \SI{5}{\micro\liter} Probe in einem Reaktionsvolumen von \SI{20}{\micro\liter} angesetzt. Das Temperaturprofil bestand aus einem RT-Schritt bei \SI{48}{\degreeCelsius}/\SI{30}{\minute}, gefolgt von \SI{95}{\degreeCelsius}/\SI{10}{\minute} zur Inaktivierung der Reversen Transkriptase und Aktivierung der Taq-Polymerase und 40 Zyklen mit jeweils Denaturierung bei \SI{95}{\degreeCelsius}/\SI{10}{\second}, Annealing bei \SI{58}{\degreeCelsius}/\SI{30}{\second} und Extension bei \SI{72}{\degreeCelsius}/\SI{30}{\second}. Auf die Amplifizierung folgte hier noch eine Schmelzkurvenanalyse, bei der die Schmelztemperatur (TM) der gebildeten RT-qPCR Produkte bestimmt wurde. Zur absoluten Quantifizierung wurden die aus den Proben 856-6 und 856-10 (2018) amplifizierten RT-qPCR Produkte auf einem Agarosegel elektrophoretisch aufgetrennt und nach Ausschneiden der Banden und Aufreinigung der DNA (QIAquick Gel Extraction Kit; Qiagen) fluorimetrisch quantifiziert (Qubit 3 Fluorometer und Qubit dsDNA BR Assay Kit; ThermoFisher). Die Berechnung der Kopienanzahl erfolgte nach der unter \cref{chap:plasmide_homogenate} angegebenen Formel. RT-qPCRs wurden auf dem 7500Fast Real-time PCR System (ThermoFisher) oder auf dem CFX96 Touch Real-Time PCR System (Bio-Rad) durchgeführt, die Auswertung erfolgte mit der 7500 Software v2.3 (ThermoFisher), bzw. CFX Maestro 1.1 Software (Bio-Rad). Auf beiden Geräten lagen die Schmelztemperaturen für die IAPV bzw. KBV Amplifikationsprodukte bei ca. 79-80\si{\degreeCelsius}.


\subsubsection{Testung und Quantifizierung der Projektproben} 
\label{chap:quantifizierung}

Zusätzlich zu den 193 Projektproben wurden während der Homogenisierung der Proben in der Abteilung BIEN elf Wasserproben als Prozesskontrollen mitgeführt. Zudem wurde bei der Nukleinsäureextraktion in jeder zweiten Spalte einer 96-well Platte (also pro 14 Proben) eine PBS-Negativextraktionskontrolle zur Erkennung von Kreuzkontaminationen während der Extraktion mitgeführt. Alle Proben, Prozesskontrollen und Extraktionskontrollen wurden wie unter \cref{chap:extraktion} beschrieben mit dem LSI MagVet\textsuperscript{TM} Universal Isolation Kit auf dem KingFisher\textsuperscript{TM} Flex (beides ThermoFisher) extrahiert und in der Folge im Einzelansatz auf Apis-Actin mRNA, sowie auf ABPV, BQCV, CBPV, DWV-A, DWV-B, SBV, IAPV und KBV RNA getestet. Alle im Erstansatz positiven Proben wurden dann erneut extrahiert und unter Zuhilfenahme von externen Standardverdünnungsreihen mit bekannter Konzentration im Doppelansatz quantifiziert. Proben, die nach der Erstuntersuchung positiv, in der Wiederholungstestung im Doppelansatz aber negativ getestet wurden, wurden als negativ beurteilt. Ergab sich nach Wiederholungsextraktion und RT-qPCR Ansatz ein qualitativ und/oder semiquantitativ deutlich abweichendes Ergebnis wurde zur Verifikation erneut extrahiert und im Doppelansatz quantifiziert. Die Angabe der finalen Kopienanzahl für jedes Virus pro \si{\milli\liter} Bienenhomogenat ergab sich aus der Formel: Kopienanzahl/Reaktion (Mittelwert aus zwei Replikaten) x 160. Während der Extraktion ist es teilweise zu geringgradigen Verschleppungen von viruspositivem Material gekommen (positive Ergebnisse bei den Negativextraktionskontrollen, siehe \cref{chap:negativkontrollen}). In allen Fällen blieben diese Nachweise in einem sehr niedrigen Konzentrationsbereich, unterhalb des vom EU-RL definierten LOD\textsubscript{PCR}/LOQ\textsubscript{PCR} von 100 Kopien/Reaktion (= 1,6x$10^4$ Kopien/\si{\milli\liter} Homogenat). Daher wurde dieses als Grenze zwischen letztendlich als positiv bzw. negativ bewerteten Proben definiert, um falsch-positive Ergebnisse aufgrund von geringen Verunreinigungen weitgehend auszuschließen.

\subsection{Statistik}

\subsubsection{Zusammenhang zwischen Prävalenz bzw. Virustiter und Standort- bzw.
Volksfaktoren}

Die gesamte Auswertung wurde mit dem Statistikprogramm R mit der Version 3.5.2. durchgeführt \citep{rcoreteam2020}. Alle Abbildungen wurden mit R erstellt (package „ggplot2“: \citep{wickham2016}).
Für alle Virus-Prävalenzen wurde das 95\% Konfidenzintervall mittels eines General Linear Models mit einer quasibinomialen Verteilung berechnet und angegeben \citep{vanderzee2013}. Zusätzlich wurde der Zusammenhang zwischen der Virusprävalenz und der Variablen Bundesland mit der gleichen Modellierung berechnet. Die Signifikanz dieser Modelle wurde mit Hilfe eines log-likelyhood Chi²-Tests berechnet (package „car“: \citep{fox2011}).  Als Posthoc-Test wurde ein Tukey Test durchgeführt (package „multcomp“: \citep{hothorn2008}). 
Zur Auswertung möglicher Zusammenhänge zwischen den genannten Faktoren und dem Virustiter wurden die nicht-parametrischen statistischen Methoden Wilcoxon-Mann-Whitney-Test und Kruskal-Wallis-Test angewandt. Aufgrund der fehlenden Normalverteilung in der Variable Virustiter wurden diese mittels Median beschrieben. Der Median stellt jenen Wert dar, unter dem die eine Hälfte aller Werte liegt, während die andere Hälfte der Werte darüber liegt. Um die Streuung darzustellen, wurde das untere Quartil (1Q, 25\% der Werte liegen darunter) und das obere Quartil (3Q, 75\% der Werte liegen darunter) angegeben.
