\begin{otherlanguage}{english}

Content of module A is a monitoring for bee viruses in Austria. Despite the impact of bee viruses onto bee health, our knowledge of the viruses' occurence in Austrian bee colonies is limited and does not allow an estimation of the viruses' prevalence in Austria. Thus, data about the prevalence of eight bee viruses is being collected during three years. The viruses analysed are the Acute Bee Paralysis Virus (ABPV), the Black Cell Queen Virus (BQCV), the Chronic Bee Paralysis Virus (CBPV), the Deformed Wing Virus (separated in type A (DWV-A) and type B (DWV-B)), the Israeli Acute Paralysis Virus (IAPV), the Kashmir Bee Virus (KBV) and the Sacbrood Virus (SBV).
\newline
In this third progress report the results of the second sample taking in autumn 2019, in which 193 beekeepers from all over Austria took part, are being reported. Six out of eight analysed bee viruses were detected in the bee samples. BQCV and DWV-B were the most commonly detected viruses, they were present in nearly every sample. BQCV was detected in 191 out of 193 samples (prevalence: 99,0\%; 95\%~CI:~96,3-99,7\%) and DWV-B was detected in 171 samples (88,6\%; 95\%~CI:~83,3-92,4\%). SBV was detect third most frequent with a prevalence of \confi{80,8}{95}{74,7}{85,8} (156 samples positive). ABPV was detectable in 65 samples (33,7\%; 95\%~CI:~27,4-40,6\%). CBPV and DWV-A were  seldom detected: CBPV in 14 samples (7,3\%; 95\%~CI:~4,4-11,8\%) and DWV-A in one sample (0,5\%; 95\%~CI:~0,1-2,9\%). 
\newline
In all detected viruses the positive samples' titer ranged between several decimal powers. The minimal titer value ranged between $10^5$ and $10^9$ RNA-copies/\si{\milli\liter} homogenate, the maximal value between $10^7$ and $10^{11}$ RNA-copies/\si{\milli\liter} homogenate. The three viruses ABPV, BQCV und SBV showed the lowest titers (median below $10^6$). With CBPV the median was one decimal power higher (8,8x$10^6$ RNA-copies/\si{\milli\liter} homogenate). The median titer value of DWV-B  with $10^8$ RNA-copies/\si{\milli\liter} homogenate was by far the highest one. DWV-A was detected in only one sample (3,3x$10^7$ RNA-copiea/\si{\milli\liter} homogenate).
\newline
The prevalence of ABPV, DWV-B and SBV differed significantly between the different federal provinces and sea levels. The viruses' highest prevalence was measured in Vienna and the Burgenland. In Tyrol the lowest prevalences were measured. This patterns may be caused by differences in the apiaries' sea levels in the different federal provinces. The viruses were most prevalent at lower sea levels and least prevalent at higher sea levels. Additionally, the DWV titer was negatively correlated with the sea level. It is to be assumed that the brood period is shortened by the cooler climate at higher sea levels, which may be the main reason for a decreased virus reproduction in these apiaries.
\newline
In order to correlate winter loss and bee viruses four different modell types were calculated. For modelling the titer values of the six viruses and eight further influencing factors concerning beekeeper operation and colony characteristics were used. All models highlighted a positive correlation between the DWV-B titer and high winter losses. Further influencing factors, which appeared only in some of the models, were a high ABPV-titer, the symptom \enquote{Varroa mites on bees} and the operation's total number of colonies. The two latter factors stress the negative impact of the Varroa mite and the positive impact of high professionalism onto the colonies' survival. It remains to be seen, if the next year's data collection and the accompaning increase in data will verify the current statements


\end{otherlanguage}
