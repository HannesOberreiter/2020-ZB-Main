\section{Ergebnisse}

\subsection{Projektfortschritt}
\subsubsection{Kontakt mit den ProjektteilnehmerInnen}

\textit{Probenahme 2019}

Im Jahr 2019 schickten 193 teilnehmende Imkerinnen und Imker ihre Proben ein (97\% der 200 ausgeschickten Probensets, \cref{tab:b:probenanzahl}). Die Gründe der Ausfälle waren unterschiedlich: Krankheitsfälle (2 ImkerInnen), unerwartet auftretende Bienengiftallergie (1 ImkerIn), keine Bienen mehr (1 ImkerIn) und unerklärtes Ausbleiben der Proben (3 ImkerInnen). Wie in 2018 kam es zu einer geringfügigen Verschiebung der Anteile der Bundesländer, da einige Imkerinnen und Imker einen Bienenstand in einem anderen Bundesland als dem ursprünglich angegebenen Bundesland gewählt hatten (\cref{tab:b:probenanzahl}: Vergleich „Anzahl ausgewählter Bienenstände“, „Anzahl erhaltener Proben“).

Die Proben wurden zwischen 28. August 2019 und 13. Oktober 2019 genommen, wobei 165 der TeilnehmerInnen (85\%) die Proben im anvisierten Zeitfenster zwischen 31. August und 15. September 2019 genommen haben. Weitere 25 Proben trafen in den restlichen zwei Septemberwochen ein und zwei Proben im Oktober 2019.

Vier der ImkerInnen haben nur von vier Völkern - statt wie geplant von fünf Völkern - Proben eingeschickt. Dies lag daran, dass alle genau fünf Völker an dem Probenahme-Bienenstand aufgestellt hatten und eines dieser Völker kurz vor der Probenahme abgestorben war. Wir haben entschieden, diese vier Proben in die Analyse aufzunehmen, um die Messreihe der drei Jahre nicht zu unterbrechen.

Die Fragebögen wurden von allen Teilnehmerinnen und Teilnehmern ausgefüllt. Bei etwa einem Drittel der Fragebögen wurde aufgrund unlogischer oder fehlender Angaben seitens der TeilnehmerInnen noch bei den TeilnehmerInnen rückgefragt und die Angaben gegebenenfalls korrigiert. Die optionale Zusatzaufgabe einer Durchsicht auf ausgewählte Krankheitssymptome wurde von 161 der Imkerinnen und Imker vollständig durchgeführt (83,4\% von 193). Weitere 10 ImkerInnen gaben zu einem Teil der Symptome Beobachtungen an. 22 ImkerInnen führten zum Probenahmezeitpunkt keine Völkerdurchsicht auf die ausgewählten Krankheitssymptome durch.

\textit{Mitteilung der Ergebnisse der Probenahme 2019}

Die teilnehmenden ImkerInnen erhielten Anfang März 2019 die individualisierten Ergebnisse der Virusuntersuchungen ihrer Sammelprobe per E-Mail oder per Brief (\cref{chap:anhang_Ergebnis}). Die Ergebnismitteilung enthielt folgende Information über die acht untersuchten Viren: den Nachweis (positiv/negativ), den Virustiter und dessen Beurteilung (niedrig, mittel, hoch). Die Beurteilung erfolgte als relative Einschätzung im Vergleich zu den in der gesamten Stichprobe gemessenen Titerwerten des jeweiligen Virus (niedrig: Titer im Bereich der niedrigsten 25\%; mittel: Titer liegt zwischen 25\% und 75\%, hoch: Titer liegt im Bereich der 25\% höchsten Werte).

Allen fünf ImkerInnen, die im Jahr 2019 das erste Mal an der Probenahme teilgenommen haben, wurde das Informationsdokument vom Vorjahr mit Basiswissen zu den untersuchten Bienenviren zugeschickt (\cref{chap:anhang_FAQ}).


\textit{Abfrage Winterverluste 2019/20}

Die Umfrage zu den Winterverlusten 2019/20 wurde allen 193 ImkerInnen, die im September 2019 Proben eingeschickt hatten, Anfang April 2020 per E-Mail-Link oder Brief zugeschickt. ImkerInnen, die noch keine Winterverluste gemeldet hatten, wurden im Monats-Rhythmus per E-Mail oder Telefon daran erinnert, uns die Daten zukommen zu lassen. Mit 13. Juli 2020 hatten 192 TeilnehmerInnen ihre Überwinterungsergebnisse gemeldet. Ein Imker/einer Imkerin hat keine Daten zu den Winterverlusten geliefert. Da diese Person nicht mehr per Telefon oder E-Mail erreichbar war, wurde sie aus dem Projekt ausgeschieden.

\subsubsection{Datenauswertung}

Für das Monitoringjahr 2019 wurden die Virusprävalenz und die Virustiter der acht gemessenen Bienenviren für Gesamtösterreich und die neun Bundesländer ausgewertet. Zusätzlich wurden Korrelation zwischen Virusauftreten und Seehöhe bzw. Virussymptomatik erstellt. Der Zusammenhang zwischen den Winterverlusten der Probenvölker und den Virustitern sowie acht weiteren Risikofaktoren wurde berechnet. 

\subsection{Kennwerte der teilnehmenden Imkereibetriebe}

Die Betriebsgröße der im Jahr 2019 teilnehmenden Imkereibetriebe bewegte sich zwischen fünf und 350 Völkern pro Betrieb. Im Mittel besaß ein teilnehmender Betrieb 38,1 Völker (Standardabweichung: ±50,2 Völker). Die Stichprobe bestand aus 84 Betrieben mit bis zu 20 Völkern, 79 Betrieben mit 21 bis 50 Völkern und 30 Betrieben mit über 50 Völkern (43,5\%; 40,9\% und 15,6\% der 193 teilnehmenden Betriebe).

Die im Jahr 2019 teilnehmenden ImkerInnen setzten sich sowohl aus sehr erfahrenen ImkerInnen mit maximal 58 Jahren Imkereierfahrung als auch aus NeueinsteigerInnen mit einem Jahr Imkereierfahrung zusammen. Der Mittelwert der Stichprobe betrug 16,6 Jahre Imkereierfahrung (Standardabweichung: ±15,1 Jahre). Der Anteil der ImkerInnen mit wenigen Erfahrungsjahren überwog: 54,4\% der ImkerInnen hatten bis zu 10 Erfahrungsjahre (105 von 193 TeilnehmerInnen), weitere 26,9\% zwischen 11 und 30 Jahre (52 von 193) und die restlichen 18,7\% der TeilnehmerInnen imkerten schon über 30 Jahre (36 von 193). 

Die TeilnehmerInnen der Probenahme 2019 betrieben zu 71,5\% konventionelle Imkerei (138 Betriebe) und zu 24,9\% imkerten sie mit einem Bio-Zertifikat (48 Betriebe). 3,6\% der konventionellen Betriebe wurden unter \enquote{in Umstellung} geführt (7 Betriebe), da sie angaben in Umstellung zum Biobetrieb zu sein. Sie könnten daher eventuell im nächsten Jahr in die Gruppe der bio-zertifizierten Betriebe eingereiht werden.

\subsection{Ergebnisse Virusdiagnostik}

\subsubsection{Prozesskontrollen}
\label{chap:prozesskontrollen}

Im Zuge der Probenaufarbeitung wurden elf Wasserproben als Prozesskontrollen mitgeführt. Alle Prozesskontrollen waren negativ auf Apis-Actin mRNA, sowie auf alle getesteten viralen Erreger. Diese Ergebnisse weisen darauf hin, dass es während der Homogenisation der Proben zu keinen Verschleppungen viruspositiven Materials zwischen den Proben gekommen ist.

\subsubsection{Negativextraktionskontrollen}
\label{chap:negativkontrollen}

Bei einem gewissen Anteil der Negativextraktionskontrollen (n=14) wurden je nach Virus geringe Anteile an positiven Virusnachweisen beobachtet. Diese betrafen bei ABPV 14\% und bei DWV-B 7\% der Negativextraktionskontrollen. In allen Fällen blieben diese Nachweise in einem sehr niedrigen Konzentrationsbereich, unterhalb des LOQ\textsubscript{PCR}. Die anderen getesteten Viren (BQCV, CBPV, DWV-A, SBV, IAPV, KBV) waren in keiner der Negativextraktionskontrollen nachweisbar. Diese Ergebnisse lassen den Schluss zu, dass es im Zuge der Extraktion, bzw. im Zuge der darauffolgenden wiederholten PCR-Untersuchungen ausgehend von denselben Extrakten nicht zu nennenswerten Kreuzkontaminationen der Proben gekommen ist. Ein geringes Ausmaß an Kreuzkontaminationen ist trotz sorgfältiger Aufarbeitung bei den im Rahmen dieses Projektes beobachteten Anteilen an viruspositiven Proben und den teilweise sehr hohen Viruslasten zu erwarten.

\subsubsection{Semi-quantitativer Nachweis der Apis-Actin mRNA}

Die Apis-Actin mRNA war in allen 193 Projektproben eindeutig mit Ct-Werten von 16 bis 24 nachweisbar, was auf eine erfolgreiche Nukleinsäureextraktion und die Abwesenheit nennenswerter Anteile an inhibitorischen Substanzen in den Nukleinsäureextrakten hinweist.

\subsection{Virusprävalenz und Viruskonzentration}

Im September 2019 wurden sechs der acht untersuchten Viren in den 193 untersuchten Proben gefunden (\cref{fig:e:virusQUAL}). IAPV und KBV wurden in keiner Probe gefunden. Drei Viren traten in über 80\% der Proben auf (BQCV, DWV-B, SBV) und waren damit sehr häufig. Die anderen drei Viren traten in weniger als einem Drittel der Proben auf und waren damit eher selten (ABPV, CBPV, DWV-A).

BQCV war am häufigsten und trat mit einer Prävalenz von \confi{99,0}{95}{96,3}{99,7} auf (\cref{fig:e:virusQUAL}). DWV-B hatte mit \confi{88,6}{95}{83,3}{92,4} das zweithöchste und SBV mit \confi{80,8}{95}{74,7}{85,8} das dritthöchste Auftreten. ABPV kam mit einer Prävalenz von \confi{33,7}{95}{27,4}{40,6} deutlich seltener vor. CBPV hatte eine Prävalenz von \confi{7,3}{95}{4,4}{11,8}. DWV-A kam in nur einer von 193 Proben vor und hatte daher eine Prävalenz von \confi{0,5}{95}{0,1}{2,9}. Die beiden nicht nachgewiesenen Viren KBV und IAPV hatten die idente Prävalenz von \confi{0,0}{95}{0,0}{2,0}.

\myfig{project-A-virenmonitoring/figures/Erg_virus19_QUAL}
  {width=\textwidth} % Größe Relativ zu Text Breite
  {Prävalenz der acht untersuchten Viren in Bienenproben vom Herbst 2019 (±95\%CI). Stichprobenumfang: 193 Proben.} % Text unterhalb der Grafik
  {} % Optional Kurz Überschrift
  {fig:e:virusQUAL} % Label zum Verweisen im Text

In 99\% der untersuchten Proben wurde zumindest ein Virus gefunden (191 von 193 Proben). Die maximale Anzahl an gemessenen Viren pro Probe waren fünf (2\% der Proben). In den meisten Proben wurden entweder drei Viren (49\% der Proben) oder vier Viren (30\%) nachgewiesen. In 17\% der Proben wurden zwei Viren nachgewiesen, sehr selten nur ein einziger Virus (2\% der Proben). Im Mittel wurden 3,1 Viren (Standardabweichung: ± 0,85 SD) in einer Probe gefunden.

\begin{table}
    \centering
    \caption{Virustiter der sechs in den 193 untersuchten Proben nachgewiesenen Viren im Herbst 2019. Anzahl~= ~Anzahl positiver Proben.}
    \label{tab:f:virusQUANT}
    \begin{tabular}{l*{6}{r}}
        \toprule
        \makecell{Virus}      &   
        \makecell{Anzahl}     & 
        \makecell{Median}     & 
        \makecell{1. Quartil} & 
        \makecell{3. Quartil} & 
        \makecell{Minimum}    & 
        \makecell{Maximum}    \\
        \midrule
        ABPV    & 65        & 4,46 x $10^{5}$    & 3,43 x $10^{4}$  & 2,24 x $10^{8}$   & 1,63 x $10^{4}$   & 2,73 x $10^{10}$\\ 
        BQCV    & 191       & 6,30 x $10^{5}$    & 2,72 x $10^{5}$  & 2,23 x $10^{6}$   & 1,80 x $10^{4}$   & 3,09 x $10^{9}$\\
        CBPV    & 14        & 8,81 x $10^{6}$    & 1,40 x $10^{5}$  & 9,27 x $10^{8}$   & 2,46 x $10^{4}$   & 2,56 x $10^{10}$\\
        DWV-A   & 1         & 3,30 x $10^{7}$    & 3,30 x $10^{7}$  & 3,30 x $10^{7}$   & 3,30 x $10^{7}$   & 3,30 x $10^{7}$\\
        DWV-B   & 171       & 1,03 x $10^{8}$    & 4,76 x $10^{6}$  & 5,81 x $10^{8}$   & 1,76 x $10^{4}$   & 2,18 x $10^{10}$\\
        SBV     & 156       & 4,19 x $10^{5}$    & 7,34 x $10^{4}$  & 3,58 x $10^{7}$   & 1,61 x $10^{4}$   & 2,54 x $10^{10}$\\
        \bottomrule
    \end{tabular}
\end{table}

Der Virustiter der positiven Proben variierte bei allen gemessenen Viren um mehrere Zehnerpotenzen. Der minimal gemessene Titer lag bei allen Viren zwischen $10^4$ und $10^8$ RNA-Kopien/\si{\milli\liter} Homogenat. Der maximal gemessene Wert lag zwischen $10^7$ und $10^{11}$ RNA-Kopien/\si{\milli\liter} Homogenat (\cref{tab:f:virusQUANT}, \cref{fig:e:virusQUAL}). Anders ausgedrückt, wurden in den Proben zwischen Zehntausenden und Hundert Millionen RNA-Kopien/\si{\milli\liter} Homogenat gemessen. Die drei Viren ABPV, BQCV und SBV hatten die geringsten Titer (Median unter $10^6$). Bei diesen drei Virusarten wurden jedoch auch Werte von über $10^9$ RNA-Kopien/\si{\milli\liter} Homogenat gemessen. Bei CBPV lag der Median um eine Zehnerpotenz höher bei 8,8 x $10^6$ RNA-Kopien/\si{\milli\liter} Homogenat. Der mediane Virustiter von DWV-B lag bei etwa $10^8$ RNA-Kopien/\si{\milli\liter} Homogenat und war mit Abstand am höchsten. Entsprechend hatten auch etwa 25\% der positiven Proben von CBPV und DWV-B einen Virustiter über $10^9$ RNA-Kopien/\si{\milli\liter} Homogenat. Die statistischen Angaben zum Virus DWV-A sind aufgrund der geringen Stichprobenanzahl von nur einer positiven Proben nicht aussagekräftig.


\myfig{project-A-virenmonitoring/figures/Erg_virus19_QUANT}
  {width=\textwidth} % Größe Relativ zu Text Breite
  {Virustiter der sechs nachgewiesenen Viren im Herbst 2019. Es sind nur die Werte der positiven Proben gezeigt (N= Anzahl der positiven Proben). „Mill.“ Million (=$10^6$); „Mrd.“ Milliarde (=$10^9$), „Bill.“ Billion (=$10^{12}$). Darstellung als Boxplot Diagramm: dicker Mittelstrich: Median; untere/obere Grenze der Box: unteres/oberes Quartil; Antennen: maximal 1,5-facher Interquartilsabstand.} % Text unterhalb der Grafik
  {} % Optional Kurz Überschrift
  {fig:e:virusQUANT} % Label zum Verweisen im Text


\subsection{Geographische Varianz in der Virusprävalenz und Viruskonzentration}
\subsubsection{Bundesländer}

Die Virusprävalenz von drei der gefundenen Viren unterschied sich signifikant zwischen den einzelnen Bundesländern (ABPV, DWV-B, SBV; \cref{fig:h:virusQUAL_BL,tab:j:praevalenz_BL}). Für BQCV und CBPV wurde kein Zusammenhang zwischen Virusprävalenz und Bundesland gemessen. DWV-A wurde aufgrund der geringen Stichprobe von einem positiven Wert nicht statistisch ausgewertet. Bei ABPV und DWV-B variierte auch der Virustiter der positiven Proben zwischen den Bundesländern (\cref{fig:i:virusQUANT_BL}). 

Die Virusprävalenz und der Virustiter von ABPV unterschieden sich signifikant zwischen den neun Bundesländern (Prävalenz: $\chi^2$=21,91; df=8; P=0,005; Titer: H=15,81; df=7; P=0,027). ABPV ist in Wien mit \confi{66,7}{95}{35,4}{87,9} am häufigsten aufgetreten. Der höchste Virustiter wurde im Burgenland gemessen mit einem Median von 7,53x$10^8$ RNA-Kopien/\si{\milli\liter} Homogenat (Q1-Q3: 3,77x$10^8$-1,05x$10^9$ RNA-Kopien/\si{\milli\liter} Homogenat) gemessen. Die minimale Prävalenz wurde in Tirol gemessen, hier betrug die Prävalenz \confi{0,0}{95}{0,0}{16,1}. Der geringste Virustiter wurde in Salzburg gemessen und betrug im Median 2,34x$10^4$ RNA-Kopien/\si{\milli\liter} Homogenat (Q1-Q3: 2,25x$10^4$-5,60x$10^4$ RNA-Kopien/\si{\milli\liter} Homogenat).

Weder Prävalenz noch Titer von BQCV unterschieden sich zwischen den neun Bundesländern (Prävalenz: $\chi^2$=5,64; df=8; P=0,687; Titer: H=11,26; df=8; P=0,187). In sieben Bundesländern waren alle Proben positiv auf BQCV (Prävalenz 100\%). Nur in Niederösterreich und Tirol waren einige der Proben negativ.

Die Prävalenz von CBPV unterschied sich nicht signifikant zwischen den Bundesländern (Prävalenz: $\chi^2$=9,67; df=8; P=0,289). Die maximale Prävalenz wurde in Vorarlberg mit \confi{20,0}{95}{5,7}{51,0} gemessen. Am geringsten war die Prävalenz im Burgenland und Niederösterreich. In beiden Bundesländern wurden keine positiven Proben gefunden (Prävalenz Bgl. \confi{0,0}{95}{0,0}{39,0}; Prävalenz NÖ.: \confi{0,0}{95}{0,0}{9,9}). Aufgrund der geringen positiven Probenanzahl in den einzelnen Bundesländern wurde keine Auswertung des Virustiters durchgeführt (siehe Stichprobenzahlen in \cref{fig:h:virusQUAL_BL}).

DWV-A wurde aufgrund der geringen Stichprobe von einem positiven Wert nicht statistisch ausgewertet.

Bei DWV-B unterschieden sich Prävalenz und Virustiter signifikant zwischen den Bundesländern (Prävalenz: $\chi^2$=25,79; df=8; P=0,001; Titer: H=18,48; df=8; P=0,018). Die höchste Prävalenz wurde im Burgenland und in Wien mit 100\% erreicht (Burgenland: \confi{100,0}{95}{61,0}{100,0}; Wien: \confi{100,0}{95}{70,1}{100,0}). Die geringste Prävalenz wurde in Salzburg mit \confi{64,3}{95}{38,8}{83,7} gemessen. In Wien wurde der höchste mediane Virustiter gemessen (1,57x$10^9$ RNA-Kopien/\si{\milli\liter} Homogenat; Q1-Q3: 5,51x$10^7$-2,68x$10^9$ RNA-Kopien/\si{\milli\liter} Homogenat). Der geringste mediane Titer wurde in Tirol gemessen und betrug 4,86x$10^6$ RNA-Kopien/\si{\milli\liter} Homogenat (Q1-Q3: 1,05x$10^5$-8,61x$10^7$ RNA-Kopien/\si{\milli\liter} Homogenat).


Die Prävalenz von SBV unterschied sich signifikant zwischen den neun Bundesländern ($\chi^2$=31,48; df=8; P<0,001). Burgenland und Vorarlberg wiesen die höchste Prävalenz auf; alle Proben aus diesen beiden Bundesländern waren positiv auf SBV (Burgenland: \confi{100,0}{95}{61,0}{100,0}; Vorarlberg: \confi{100,0}{95}{72,2}{100,0}). Die geringste Prävalenz hatten die Proben aus Kärnten mit einer Prävalenz von \confi{40,9}{95}{23,3}{61,3}. Der Virustiter unterschied sich nicht signifikant zwischen den Proben aus den unterschiedlichen Bundesländern (H=8,57; df=8; P=0,380).

\myfig{project-A-virenmonitoring/figures/Erg_virus19_QUAL_BL}
  {width=\textwidth} % Größe Relativ zu Text Breite
  {Virusprävalenz in den neun Bundesländen (±95\%CI). Anzahl Stände pro Bundesland: Bgl. N=6; Ktn. N=22; NÖ. N=35;.OÖ. N=42; Sbg. N=14; Stm. N=35; Tirol N=20; Vbg. N=10; Wien N=9.} % Text unterhalb der Grafik
  {} % Optional Kurz Überschrift
  {fig:h:virusQUAL_BL} % Label zum Verweisen im Text
  
  
 \myfig{project-A-virenmonitoring/figures/Erg_virus19_QUANT_BL}
  {width=\textwidth} % Größe Relativ zu Text Breite
  {Virustiter der sechs nachgewiesenen Viren in den positiven Proben in den neun Bundesländern (N=Anzahl der positiven Proben). „Mill.“ Million (=$10^6$); „Mrd.“ Milliarde (=$10^9$), „Bill.“ Billion (=$10^{12}$). Statistik in der Grafik: Kruskal-Wallis-Test. Darstellung als Boxplot Diagramm: dicker Mittelstrich: Median; untere/obere Grenze der Box: unteres/oberes Quartil; Antennen: maximal 1,5-facher Interquartilsabstand.} % Text unterhalb der Grafik
  {} % Optional Kurz Überschrift
  {fig:i:virusQUANT_BL} % Label zum Verweisen im Text
  
 \input{project-A-virenmonitoring/tables/j_prävalenz_BL}
 
 
 \subsubsection{Seehöhe}
 
 Aufgrund der orographischen Gegebenheiten in Österreich weisen die Bienenstände signifikante Unterschiede in der Seehöhe zwischen den unterschiedlichen Bundesländern auf (H=92,05; df=8; P<0,001; \cref{fig:j:COR_BL_Seehoehe}). Die beschriebenen Unterschiede im Virusauftreten zwischen den Bundesländern könnten auch mit den Unterschieden in der Seehöhe zusammenhängen. Daher wurde diese als zusätzlicher Faktor ausgewertet.
 
 Bei drei der sechs gemessenen Bienenviren war ein Zusammenhang zwischen der Prävalenz des Virus und der Seehöhe des entsprechenden Bienenstandes nachweisbar (ABPV, DWV-B, SBV; \cref{tab:k:praevalenz_Seehoehe,fig:k:virusQUAL_Seehoehe}). Dabei sank die Prävalenz umso mehr, je höher der Bienenstand gelegen war. Bei BQCV und CBPV wurde kein Zusammenhang zwischen dem Auftreten des Virus und der Seehöhe gefunden. DWV-A wurde aufgrund der geringen Stichprobe nicht ausgewertet. Bei DWV-B war zusätzlich ein Zusammenhang zwischen der Höhe des Virustiters und der Seehöhe des Bienenstandes nachweisbar (\cref{fig:l:virusQUANT_Seehoehe}).
 
\myfig{project-A-virenmonitoring/figures/Erg_virus19_Seehoehe_BL}
  {width=\textwidth} % Größe Relativ zu Text Breite
  {Seehöhe der Bienenstände nach Bundesländern. N=Anzahl der Proben. Darstellung als Boxplot Diagramm: dicker Mittelstrich: Median; untere/obere Grenze der Box: unteres/oberes Quartil; Antennen: maximal 1,5-facher Interquartilsabstand.} % Text unterhalb der Grafik
  {}% Optional Kurz Überschrift
  {fig:j:COR_BL_Seehoehe} % Label zum Verweisen im Text
 
 ABPV trat umso seltener auf, je höher der Bienenstand gelegen war ($\chi^2$=22,99; df=4; P<0,001). Auf Bienenständen der Gruppe $\leq \SI{200}{\meter}$ war die Prävalenz mit \confi{69,2}{95}{42,4}{87,3} am höchsten. Die geringste Prävalenz wurde in der Gruppe 601-800\si{\meter} gemessen und betrug \confi{9,8}{95}{3,9}{22,5}. Es war kein signifikanter Zusammenhang zwischen der Seehöhe des Bienenstandes und dem Virustiter von ABPV-positiven Proben feststellbar (H=9,46; df=4, P=0,051). Der P-Wert liegt nur knapp über der Signifikanzgrenze. Es könnte daher sein, dass ein möglicher vorhandener Effekt aufgrund der niedrigen Stichprobenzahlen für die Gruppe $\leq \SI{200}{\meter}$ nicht gezeigt werden kann (siehe Stichprobenzahl in \cref{fig:l:virusQUANT_Seehoehe}).

Für BQCV wurde weder ein Zusammenhang zwischen der Seehöhe und der Prävalenz des Virus noch der Seehöhe und der Höhe des Virustiters für die BQCV-positiven Proben festgestellt (Prävalenz: $\chi^2$=4,11; df=4; P=0,392; Titer: H=6,78; df=4; P=0,148).

Für CBPV wurden keine Unterschiede in der Virusprävalenz zwischen den verschiedenen Seehöhen festgestellt ($\chi^2$=0,86; df=4; P=0,931). Aufgrund der geringen positiven Probenanzahl in den einzelnen Seehöhe-Gruppen wurde keine Auswertung des Virustiters durchgeführt (siehe Stichprobenzahlen in \cref{fig:l:virusQUANT_Seehoehe}).

DWV-A wurde aufgrund der geringen Stichprobe von einem positiven Wert nicht statistisch ausgewertet.

Für DWV-B wurde ein Zusammenhang zwischen der Seehöhe des Bienenstandes und der Virusprävalenz sowie des Virustiters festgestellt (Prävalenz: $\chi^2$=41,16; df=4; P<0,001; Titer: H=16,76; df=4; P=0,002). Dabei sank sowohl die Prävalenz als auch der Titer umso mehr ab, desto höher der beprobte Bienenstand gelegen war. In allen Proben der Gruppe $\leq \SI{200}{\meter}$ war DWV-B nachweisbar, die Prävalenz lag daher bei \confi{100,0}{95}{77,2}{100,0}. Ebenso war der Virustiter in dieser Gruppe im Median am höchsten (Median 1,57x$10^9$ RNA-Kopien/\si{\milli\liter} Homogenat; Q1-Q3: 1,36x$10^8$-2,76x$10^9$ RNA-Kopien/\si{\milli\liter} Homogenat). Die niedrigste Prävalenz wurde in der Gruppe $> \SI{800}{\meter}$ gemessen und betrug \confi{57,7}{95}{38,9}{74,5}. In der selben Gruppe war auch der mediane Titer am niedrigsten und betrug 1,46x$10^7$ RNA-Kopien/\si{\milli\liter} Homogenat (Q1-Q3: 1,81x$10^6$-1,18x$10^8$ RNA-Kopien/\si{\milli\liter} Homogenat).

Die Wahrscheinlichkeit des Auftretens von SBV nahm ebenfalls umso mehr ab, je höher der Bienenstand gelegen war ($\chi^2$= 12,74; df=4; P=0,013). Die maximale Prävalenz wurde für die Gruppe 201-400\si{\meter} gemessen und betrug \confi{93,3}{95}{84,1}{97,4}. Die geringste Prävalenz wurde in der Gruppe 601-800\si{\meter} gemessen und betrug \confi{68,3}{95}{53,0}{80,4}. Für SBV wurde zwischen den verschiedenen Seehöhen kein Unterschied im Virustiter festgestellt (H=6,13; df=4; P=0,190).

 \input{project-A-virenmonitoring/tables/k_prävalenz_seehoehe}
 
 \myfig{project-A-virenmonitoring/figures/Erg_virus19_QUAL_Seehoehe}
  {width=\textwidth} % Größe Relativ zu Text Breite
  {Virusprävalenz der Bienenstände auf unterschiedlichen Seehöhen (±95CI). Anzahl Stände pro Kategorie: $\leq200\si{\m}$ N=13; 201-400\si{\m} N=16; 401-600\si{\m} N=53; $\leq800\si{\m}$ N=41; $>800\si{\m}$ N=26} % Text unterhalb der Grafik
  {} % Optional Kurz Überschrift
  {fig:k:virusQUAL_Seehoehe} % Label zum Verweisen im Text
 
  
 \myfig{project-A-virenmonitoring/figures/Erg_virus19_QUANT_Seehoehe}
  {width=\textwidth} % Größe Relativ zu Text Breite
  {Virustiter der sechs nachgewiesenen Viren in den positiven Proben in den fünf Kategorien für Seehöhe (N=Anzahl der positiven Proben). „Mill.“ Million (=$10^6$); „Mrd.“ Milliarde (=$10^9$), „Bill.“ Billion (=$10^{12}$). Statistik in der Grafik: Kruskal-Wallis-Test. Darstellung als Boxplot Diagramm: dicker Mittelstrich: Median; untere/obere Grenze der Box: unteres/oberes Quartil; Antennen: maximal 1,5-facher Interquartilsabstand.} % Text unterhalb der Grafik
  {} % Optional Kurz Überschrift
  {fig:l:virusQUANT_Seehoehe} % Label zum Verweisen im Text
  
  
  
 \subsection{Zusammenhang Viruskonzentration und Winterverluste}
  
Die Winterverluste (tote Völker) unter den Probenvölkern betrugen \confi{13,6}{95}{10,7}{16,8} und unterschieden sich nicht von den Winterverlusten der Monitoringstände (=Stände, auf dem die Probenahme durchgeführt wurde; \cref{tab:l:winterverluste}). Zur besseren Vergleichbarkeit mit den Ergebnissen von Modul U sind die Winterverluste auch als „tote + weisellose Völker“ dargestellt.
  
\begin{table}[htp]
    \caption{Winterverluste unter den Probenvölkern (4-5 pro Stand) und auf den Monitoringständen \protect\linebreak  (n=192~Stände).}
    \centering
     \scriptsize
    \begin{tabular}{llrrrr}
    \toprule
    \multirow{2}{*}{
        \shortstack{Art der Berechnung \\ der Winterverluste}
    } &
    \multirow{2}{*}{
        \shortstack{Gruppe}
    } &
    \multirow{2}{*}{
        \shortstack{Eingewintert \\ (Anzahl Völker)}
    } &
    \multirow{2}{*}{
        \shortstack{Winterverluste \\ (Anzahl Völker)}
    } &
    \multicolumn{2}{c}{
        \multirow{2}{*}{
        \makecell{Winterverluste (95\% CI)}
        }
    } \\
    & & & & \\
    
    
    %Art der Berechnung  & Gruppe &  Eingewintert & Winterverluste  &   Winterverluste & (95\% CI) \\ [0.5ex]
    %der Winterverluste  &       &   [Anzahl Völker] &   [Anzahl Völker] & &\\
    \midrule
    \textbf{nur tote Völker}             & Probenvölker      &  956   & 130    & 13,6\%  & (10,7-16,8\%)\\                                            & Monitoringstand     &  2931  &338     & 11,5\%  & (9,3-14,1\%)\\
    \midrule
    \textbf{tote + weisellose}    &  Probenvölker     &  956   & 172    & 18,0\%  & (14,8-21,5\%)\\
    \textbf{Völker}                  & Monitoringstand         & 2931   & 487    & 16,6\%  & (14,2-19,3\%)\\
    \bottomrule
    \end{tabular}
    \label{tab:l:winterverluste}
\end{table}


%Nachricht an Linde: eingefügt unter 6-Anhang
  
\subsubsection{Logistische Regression}

Bei der logistischen Regression erwiesen sich die Faktoren DWV-B, das Symptom „Varroamilben auf Bienen“ und die Gesamtanzahl der Völker des Imkereibetriebes als statistisch signifikant (\cref{tab:m:regression}). Völker, in deren Proben DWV-B mit einer hohen Konzentration gefunden wurde, hatten eine signifikant höhere Wahrscheinlichkeit über den Winter abzusterben, als die Völker, in deren Proben kein DWV-B nachgewiesen werden konnte. Eine geringe oder mittlere Konzentration von DWV-B führte jedoch nicht zu einer signifikant erhöhten Sterblichkeit der Bienenvölker. Wenn bei mindestens einem Probenvolk im Herbst das Symptom \enquote{Varroamilben auf Bienen} beobachtet wurde, stieg die Wahrscheinlichkeit eines Absterbens der Völker signifikant im Vergleich zu Sammelproben, in deren Völkern das Symptom nicht beobachtet wurde. Die Wahrscheinlichkeit der Wintersterblichkeit unterschied sich nicht zwischen Sammelproben ohne Symptome und Sammelproben, von deren Völkern keine Angaben zu den Symptomen gemacht wurden. Die Wahrscheinlichkeit eines Absterbens der Monitoringvölker sank mit zunehmender Völkerzahl des Imkereibetriebes. 

\begin{table}[htb]
    \caption{Multivariates Modell (GLM mit quasibinomialer Verteilung): Einfluss auf die Wintersterblichkeit von Bienenvölkern. Drei Prädiktoren enthalten: Virustiter von DWV-B im September vor der Einwinterung (Vergleichskategorie: negativ, K = Konzentration), Symptom \enquote{Varroamilben auf Bienen} im September gesehen (Vergleichskategorie: Symptom nicht gesehen) und Gesamtanzahl der Völker des Imkereibetriebes (Anzahl Völker). n=390 Proben.}
    \centering
    %\setlength{\tabcolsep}{0.2em} % for the horizontal padding
    \label{tab:m:regression}
    \begin{tabular}{l|rrr}
        \toprule
        Prädiktor & Schätzer & Standardfehler & P-Wert \\
        \midrule
        DWV-B geringe K     & -0,039  & 0,360    & 0,913  \\ 
        DWV-B mittlere K    & 0,502   & 0,337    & 0,137\\ 
        DWV-B hohe K        & 1.204   & 0,355    & <0,001\\ 
        \midrule
        Varroa - gesehen        & 0,680 & 0,212 & 0,001 \\
        Varroa - keine Angabe   & 0,057 & 0,317 & 0,858 \\
        \midrule
        Anzahl Völker           & -0,003    & 0,002     & 0,050 \\
        \bottomrule
    \end{tabular}

\end{table}

Die folgenden Faktoren wurden nicht in das Modell eingebaut, da sie die Vorhersage des Modells nicht verbesserten: Virustiter von ABPV, BQCV, CBPV, SBV, das Auftreten des Symptoms \enquote{deformierte Flügel bei Bienen}, das Untersuchungsjahr, die Anzahl der schwachen Monitoringvölker, die Seehöhe des Bienenstandes, die Jahre an imkerlicher Erfahrung und das Vorhandensein eines Bio-Zertifikats.

\subsubsection{Regression Tree}

In der vorliegenden Analyse wiesen die Regression Trees eine hohe Varianz auf. Das heißt, die verschiedenen erstellten Bäume und daher auch ihre Aussagen unterschieden sich stark voneinander. Deshalb wurde dieser Ansatz nicht bis ins Detail weiter verfolgt und es wurde stattdessen auf die varianzreduzierende Methode des Random Forest-Modells zurückgegriffen. Beispielhaft ist in der Abbildung \ref{fig:p:WV_RegressionTree} einer der berechneten Regression Trees dargestellt. Als relevantestes Feature wurde DWV-B identifiziert. Höhere Werte des Virustiters gehen tendenziell mit einer höheren Wintersterblichkeit einher. Weiters wurden ABPV, BQCV, SBV, Seehöhe und die Jahre an imkerlicher Erfahrung für die Durchführung der Splits verwendet


 \myfig{project-A-virenmonitoring/figures/Erg_Virus19_WV_RegressionTree_small}
  {width=\textwidth} % Größe Relativ zu Text Breite
  {Exemplarischer Regression Tree zu den Einflussfaktoren auf die Winterverluste der Monitoringvölker (n=390 Proben). In der Mitte jedes Splits steht die entsprechende Bedingung. Die linke Abzweigung stellt die Untergruppe dar, auf die die Bedingung zutrifft („yes“) und die rechte Abzweigung stellt die Untergruppe dar, auf die die jeweilige Bedingung nicht zutrifft („no“). Die Ergebniskästchen geben in der oberen Zeile die mittlere Höhe der Winterverluste in der entsprechenden Gruppe (0.00 = 0\% Winterverlust; 1.00 = 100\% Winterverlust) und in der unteren Zeile den Anteil der Stichproben, die in diese Gruppe fallen, an.} % Text unterhalb der Grafik
  {} % Optional Kurz Überschrift
  {fig:p:WV_RegressionTree} % Label zum Verweisen im Text


\subsubsection{Random Forest-Modell}

 Ein Random Forest-Modell stellt eine Mittlung der Ergebnisse von tausenden Regression Trees dar. Er wird daher nicht als Baum dargestellt. Stattdessen werden die Faktoren nach ihrer Relevanz gereiht. Die zwei Kennwerte hierfür sind \enquote{Total Decrease in Node Impurity} und \enquote{Mean Decrease Accuracy}. Beide Kennwerte rechnen DWV-B den höchsten Einfluss auf die Winterverluste zu, zweitgereiht wird ABPV geführt (\cref{fig:q:WV_RF_significance}). Nach diesen zwei Prädiktoren bricht die Importance stark ein, die weiteren Prädiktoren tauchen weit seltener als wichtige Faktoren in den Regression Trees auf. Die Reihung der restlichen Prädiktoren unterscheidet sich zwischen den beiden Kennwerten. Auch dies lässt auf eine unklare Datenlage schließen.

 
\myfig{project-A-virenmonitoring/figures/Erg_Virus19_WV_RF_significance}
  {width=\textwidth} % Größe Relativ zu Text Breite
  {Einflussfaktoren auf die Winterverluste: Ergebnisse des Random Forest-Modells.} % Text unterhalb der Grafik
  {} % Optional Kurz Überschrift
  {fig:q:WV_RF_significance} % Label zum Verweisen im Text
  
  Für die zwei wichtigsten Faktoren wurde die Vorhersage des Modells im Detail dargestellt (\cref{fig:r:WV_RF_predictors}). Bei beiden Viren stieg die Wahrscheinlichkeit eines Ausfalls mit der Erhöhung des Virustiters, der Kurvenverlauf unterschied sich jedoch stark. Bei DWV-B war bei niederem Virustiter kein erhöhter Winterverlust zu erwarten. Erst ab einem Titer von $10^8$ RNA-Kopien/\si{\milli\liter} Homogenat stieg die vorhergesagte Winterverlustrate steil nach oben und erreichte etwa 21\% vorhergesagte Wintersterblichkeit bei $10^9$ RNA-Kopien/\si{\milli\liter} Homogenat. Im Gegensatz stieg bei ABPV die Wahrscheinlichkeit eines Winterverlustes kontinuierlich mit Erhöhung des Virustiters. Zu beachten ist auch die Streuung der vorhergesagten Wintersterblichkeit bei einem negativen Virustiter. Vorhersagen erreichten hier Maximalwerte von 18\% bei DWV-B und 32\% bei ABPV.
  
  
  \myfig{project-A-virenmonitoring/figures/Erg_Virus19_WV_RF_predictors}
  {width=\textwidth} % Größe Relativ zu Text Breite
  {Vorhersage der Wintersterblichkeit durch das Random Forest-Modell. Dargestellt sind die zwei einflussreichsten Prädiktoren des Modells: der Virustiter von DWV-B (links) und ABPV (rechts) im September vor der Auswinterung (n=390). „Mill.“ Million (=$10^6$); „Mrd.“ Milliarde (=$10^9$), „Bill.“ Billion (=$10^{12}$).} % Text unterhalb der Grafik
  {} % Optional Kurz Überschrift
  {fig:r:WV_RF_predictors} % Label zum Verweisen im Text
  
 
\subsubsection{Stacking}

Für die Stacking Modelle wurden Vorhersagen anhand einer Kombination des Random Forest-Modells und des logistischen Regressionsmodells erstellt. Die Meta-Learner ermittelten für jedes der Modelle einen Gewichtungskoeffizienten. Je höher der Koeffizient eines Modells ist, desto stärker wurden dessen Vorhersagen in der endgültigen Vorhersage berücksichtigt. Wie in \cref{tab:o:stacking} zu sehen ist, beziehen beide Meta-Learner die Logistische Regression etwas stärker in das Stacking Modell ein als das Random Forest-Modell. Die Unterschiede in den Koeffizienten sind jedoch gering. Dies heißt, dass beide Stacking Modelle empfehlen, die Ergebnisse des Random Forest-Modell und der Logistischen Regression in der Interpretation zu kombinieren und als gleichwertig zu betrachten.

\begin{table}[H]
    \caption{Stacking Modelle: Einfluss auf die Wintersterblichkeit von Bienenvölkern. Es wurden das Random Forest-Modell (\cref{fig:q:WV_RF_significance}) und die logistische Regression (\cref{tab:m:regression}) gestackt.}
    \centering
    %\setlength{\tabcolsep}{0.2em} % for the horizontal padding
    \label{tab:o:stacking}
    \begin{tabular}{l|rr}
    \toprule
        \multirow{2}{*}{\shortstack{Meta-Learner}} & 
        \multirow{2}{*}{\shortstack{Koeffizient \\ Random Forest}} & 
        \multirow{2}{*}{\shortstack{Koeffizient \\ Log. Regression}} \\
        & & \\
        %& Koeffizient   & Koeffizient \\
        %Meta-Learner  & Random Forest & Log. Regression \\
    \midrule
        Breiman             & 0,499     & 0,501 \\ 
        Ridge Regression    & 0,440     & 0,480 \\
    \bottomrule
    \end{tabular}
\end{table}

 
\subsubsection{Vergleich der Vorhersagekraft der Winterverlust-Modelle} \label{chap:vorhersage.modelle}
 
 Um die Güte der unterschiedlichen Modelle zu bewerten, wurde eine Kreuzvalidierung durchgeführt (\cref{tab:n:Mcomparison}). Dabei wurde jedes Modell zuerst mit einem Trainingsdatensatz angepasst und danach mit einem Test-Datensatz getestet. Der RMSE-Wert gibt an, wie stark die Vorhersage des Modells von der Wirklichkeit abweicht. Je höher der RMSE-Wert ist, desto häufiger hat das Modell Fehler in der Vorhersage gemacht.
 
 Das Basismodell wurde ohne Prädiktoren gerechnet und hat daher auch die höchsten RMSE-Werte (\cref{tab:n:Mcomparison}). Den niedrigsten RMSE-Wert im Trainingsdatensatz hatte der Regression Tree, die Anwendung auf den Testdatensatz zeigte hingegen keine Verbesserung gegenüber dem Basismodell. Insgesamt schneidet das Random Forest-Modell am besten ab, da es sowohl im Traingslauf als auch im Testlauf einen sehr geringen RMSE-Wert hatte. Am zweitbesten wurden die beiden Stacking Modelle bewertet. Generell ist die Vorhersagegüte auch des besten Modells nicht viel besser als die Güte des Basismodells. Dies kann dadurch erklärt werden, dass einige Völker trotz hohem DWV-B Titer überlebt haben und dieser Umstand nicht durch die restlichen Prädiktoren erklärt werden konnte. Es dürften also noch weitere Faktoren eine wichtige Rolle spielen, die in der vorliegenden Analyse nicht berücksichtigt wurden.
 
\begin{table}[H]
    \caption{Vergleich der unterschiedlichen Modelle hinsichtlich ihrer Vorhersagekraft mittels Kreuzvalidierung. Der Trainingsdatensatz bestand aus 75\% der Daten, der Testdatensatz aus 25\% der Daten. RMSE =  Root-Mean-Square-Error (~durchschnittlicher Vorhersagefehler).}
    \centering
    %\setlength{\tabcolsep}{0.2em} % for the horizontal padding
    \label{tab:n:Mcomparison}
    \begin{tabular}{l|rr}
        \toprule
        Modell  & Trainings-RMSE & Test-RMSE \\
        \midrule
        Basismodell                     & 0,209     & 0,240 \\ 
        Regression Tree                 & 0,168     & 0,240\\ 
        Random Forest                   & 0,174     & 0,214\\ 
        Logistische Regression          & 0,194     & 0,228\\ 
        Stacking nach Breiman           & 0,198     & 0,219\\ 
        Stacking mit Ridge Regression   & 0,198     & 0,220\\ 
        \bottomrule
    \end{tabular}
\end{table}
 
 Es zeigt sich, dass durch das Stacking keine weitere Verbesserung in der Vorhersage der Wintersterblichkeit erreicht werden konnte. Das Einzelmodell des Random Forest zeigt die beste Vorhersagegüte. Die Variable-Importance-Plots legen nahe, dass vor allem DWV-B einen Einfluss auf die Wintersterblichkeit der Bienenvölker hat. Dies wird auch in allen anderen Modellen bestätigt. Sowohl im Random Forest-Modell als auch im logistischen Regressionsmodell sieht man, dass vor allem ein DWV-B Titer über $10^9$ RNA-Kopien/\si{\milli\liter} Homogenat die Wintersterblichkeit erhöht (\cref{tab:m:regression,fig:r:WV_RF_predictors}). Das Random Forest-Modell zeigt zusätzlich einen Zusammenhang zwischen einem erhöhtem ABPV Titer und einem Ansteigen der Wintersterblichkeit. Weitere mögliche Einflussfaktoren sind laut logistischer Regression das Auftreten des Symptoms \enquote{Varroamilben auf Bienen} und die Größe des Imkereibetriebes.
 
 
\subsection{Zusammenhang Virusprävalenz und Viruskonzentration mit den berichteten Symptomen der Völker} \label{chap:egebnisse:symptome}

171 Imker und Imkerinnen sahen im September 2019 ihre Probenvölker auf Symptome von Viruserkrankungen durch (88,6\% von 193 TeilnehmerInnen). Dabei gaben 161 ImkerInnen Informationen zu allen fünf abgefragten Symptomen und 10 ImkerInnen Informationen zu einem Teil der abgefragten Symptome. In 50,3\% aller Rückmeldungen wurden keine Symptome beobachtet (86 von 171 ImkerInnen), in 35,7\% der Rückmeldungen wurde ein Symptom beobachtet (61 ImkerInnen), in 10,5\% der Rückmeldungen wurden zwei Symptome beobachtet (18 ImkerInnen) und in 3,5\% drei Symptome beobachtet (6 ImkerInnen). Auf keinem Bienenstand wurden mehr als drei Symptome beobachtet.
Dabei wurden \enquote{schwarz-glänzende Bienen} bei weitem am häufigsten beobachtet (37,5\% der Meldungen positiv). Die Symptome \enquote{Varroamilben auf Bienen} und \enquote{Bienen mit verkrüppelten Flügeln} waren von etwa einen Zehntel aller TeilnehmerInnen beobachtet worden (13,7\% und 10,7\% der Meldungen positiv). Die Symptome \enquote{Totenfall vor dem Bienenvolk} und \enquote{Sackbrutsymptome} wurden sehr selten beobachtet (3,5\% und 3,0\% der Meldungen positiv).

\subsubsection{Totenfall vor dem Bienenvolk}

Es haben 88\% der teilnehmenden ImkerInnen Angaben zu dem Symptom \enquote{Totenfall vor dem Bienenvolk} gemacht (170 ImkerInnen von 193). Dieses Symptom wurde auf 6 Monitoringständen beobachtet und ist daher auf 3,5\% der Stände aufgetreten. Ein erhöhter Totenfall vor dem Bienenvolk kann viele verschiedene Ursachen haben und kann auch ein Hinweis auf eine CBPV Infektion sein. Aufgrund des seltenen Auftretens der Symptomatik war eine Auswertung dieses Merkmals auf Zusammenhang mit CBPV nicht möglich.

\subsubsection{Varroamilben auf Bienen}

Zu dem Symptom \enquote{Varroamilben auf Bienen} haben 87,0\% ImkerInnen Angaben gemacht (168 von 193 TeilnehmerInnen). 13,7\% der Meldungen berichteten von \enquote{Varroamilben auf Bienen} in mindestens einem der Völker (23 von 168). \enquote{Varroamilben auf Bienen} sind ein Hinweis auf einen hohen Varroabefall der Bienen und damit auf eine akute Infektion mit DWV. Daher wurde ein Zusammenhang zwischen den Beobachtungen und dem Auftreten von DWV-B untersucht. DWV-A wurde nur in einer Probe gefunden und daher ist die Stichprobe hier zu klein für eine statistische Untersuchung.

Es bestand kein signifikanter Zusammenhang zwischen dem Auftreten von DWV-B und der Beobachtung von \enquote{Varroamilben auf Bienen} ($\chi^2$=1,85; df=1; P=0,174; \cref{fig:m:DWV_Varroa}). In den 23 Fällen mit Beobachtungen von \enquote{Varroamilben auf Bienen} wurde auch in jeder zugehörigen Probe DWV-B festgestellt, die Prävalenz betrug daher \confi{100,0}{95}{85,7}{100,0}. Wurden keine \enquote{Varroamilben auf Bienen} beobachtet, lag die Prävalenz von DWV-B bei \confi{88,3}{95}{82,0}{92,5}.

Der Titer von DWV-B war doppelt so hoch, wenn in den Probenvölkern \enquote{Varroamilben auf Bienen} beobachtet wurden, wie wenn das Symptom nicht beobachtet wurde (\cref{fig:m:DWV_Varroa}). Dieser Unterschied ist signifikant (W=1242; P=0,050). In Proben von Völkern mit dem Symptom wurde im Median ein DWV-B Titer von 2,11x$10^8$ RNA-Kopien/\si{\milli\liter} Homogenat (Q1-Q3: 1,85x$10^7$-1,35x$10^9$ RNA-Kopien/\si{\milli\liter} Homogenat) gemessen. In Proben von Völkern, in denen keine \enquote{Varroamilben auf Bienen} beobachtet wurden, wurde im Median ein DWV-B Titer von 1,04x$10^8$ RNA-Kopien/\si{\milli\liter} Homogenat (Q1-Q3: 4,55x$10^6$-5,73x$10^8$ RNA-Kopien/\si{\milli\liter} Homogenat) gemessen.


\myfig{project-A-virenmonitoring/figures/Erg_virus19_Varroa}
  {width=\textwidth} % Größe Relativ zu Text Breite
  {Zusammenhang zwischen der (A) Prävalenz von DWV-B bzw. (B) dem Titer der DWV-B positiven Proben mit dem Auftreten des Symptoms „Varroamilben auf Bienen“. „ja“: in zumindest einem der fünf Probenvölker Symptom beobachtet; „nein“ in keinem der fünf Probenvölker Symptom beobachtet; „n“ Probenanzahl; „Mill.“ Million (=$10^6$); „Mrd.“ Milliarde (=$10^9$), „Bill.“ Billion (=$10^{12}$). Statistik in der Grafik: (A) $\chi^2$-Test; (B) Kruskal-Wallis-Test. Boxplot Diagramm: dicker Mittelstrich: Median; untere/obere Grenze der Box: unteres/oberes Quartil; Antennen: maximal 1,5-facher Interquartilsabstand.} % Text unterhalb der Grafik
  {} % Optional Kurz Überschrift
  {fig:m:DWV_Varroa} % Label zum Verweisen im Text

\subsubsection{Bienen mit verkrüppelten Flügeln}

87,6\% ImkerInnen machten Angaben zum Symptom \enquote{Bienen mit verkrüppelten Flügeln} (169 von 193 TeilnehmerInnen). Davon gaben 10,7\% der ImkerInnen an, dass sie das Symptom bei mindestens einem Volk beobachtet hatten. \enquote{Bienen mit verkrüppelten Flügeln} sind ein Hinweis auf eine akute DWV-Infektion. Daher wurde der Zusammenhang zwischen dem Auftreten von DWV-B und dem Auftreten des Symptoms untersucht. DWV-A wurde nur in einer Probe gefunden und wurde daher nicht statistisch ausgewertet.

Die Prävalenz von DWV-B unterschied sich nicht signifikant zwischen Proben, die aus Völkern mit \enquote{Bienen mit verkrüppelten Flügeln} entnommen wurden, und Proben aus symptomfreien Völkern ($\chi^2$=1,18; df=1; P=0,277; \cref{fig:n:DWV_Fluegel}). Das Virus war in allen Proben aus Völkern mit \enquote{Bienen mit verkrüppelten Flügeln} vorhanden und hatte eine Prävalenz von \confi{100,0}{95}{82,4}{100,0}. In der Gruppe ohne Anzeichen von \enquote{Bienen mit verkrüppelten Flügeln} lag die Prävalenz von DWV-B bei \confi{88,7}{95}{82,7}{92,9}.

Der DWV-B Titer unterschied sich signifikant zwischen den Proben aus Völkern mit \enquote{Bienen mit verkrüppelten Flügeln} und den Proben aus Völkern ohne Symptom (W=776; P=0,003; \cref{fig:n:DWV_Fluegel}). Dabei lag der Titer von DWV-B bei Proben aus Völkern mit Symptom im Median bei 5,11x$10^8$ RNA-Kopien/\si{\milli\liter} Homogenat (Q1-Q3: 3,73x$10^7$-2,03x$10^9$ RNA-Kopien/\si{\milli\liter} Homogenat). In Proben von Völkern ohne \enquote{Bienen mit verkrüppelten Flügeln} wurde im Median ein DWV-B Titer von 9,55x$10^7$ RNA-Kopien/\si{\milli\liter} Homogenat (Q1-Q3: 3,07x$10^6$-4,96x$10^8$ RNA-Kopien/\si{\milli\liter} Homogenat) gemessen.

 
 \myfig{project-A-virenmonitoring/figures/Erg_virus19_Fluegel}
  {width=\textwidth} % Größe Relativ zu Text Breite
  {Zusammenhang zwischen der (A) Prävalenz von DWV-B bzw. (B) dem Titer der DWV-B positiven Proben mit dem Auftreten des Symptoms „Bienen mit verkrüppelten Flügel“. „ja“: Symptom in zumindest einem der fünf Probenvölker beobachtet; „nein“ Symptom in keinem der fünf Probenvölker beobachtet; „n“ Probenanzahl; „Mill.“ Million (=$10^6$); „Mrd.“ Milliarde (=$10^9$), „Bill.“ Billion (=$10^{12}$). Statistik in der Grafik: (A) $\chi^2$-Test; (B) Kruskal-Wallis-Test. Boxplot Diagramm: dicker Mittelstrich: Median; untere/obere Grenze der Box: unteres/oberes Quartil; Antennen: maximal 1,5-facher Interquartilsabstand.} % Text unterhalb der Grafik
  {} % Optional Kurz Überschrift
  {fig:n:DWV_Fluegel} % Label zum Verweisen im Text

\subsubsection{Schwarz-glänzende Bienen}

87,0\% der teilnehmenden ImkerInnen machten Angaben zum Symptom \enquote{schwarz-glänzende Bienen} (168 von 193 TeilnehmerInnen). Dabei wurde in 37,5\% der Fälle angegeben, dass das Symptom in zumindest einem der fünf Probenvölker beobachtet wurde (63 von 168). Da schwarz-glänzende Bienen ein Symptom für CBPV sein können, wurden die Daten auf Korrelation zwischen CBPV und der beobachteten Symptomatik ausgewertet.
 
 Es gab keinen signifikanten Zusammenhang zwischen dem Auftreten von \enquote{schwarz-glänzenden Bienen} und der Prävalenz von CBPV($\chi^2$=1,53; df=1; P=0,216; \cref{fig:o:CBPV_schwarz}). Die Prävalenz betrug bei Proben aus Völkern mit \enquote{schwarz-glänzenden Bienen} \confi{11,1}{95}{5,5}{21,2} und bei Völkern ohne beobachtetem Symptom \confi{4,8}{95}{2,1}{10,7}. Daher war die Prävalenz von CBPV zwar doppelt so hoch, wenn das Symptom beobachtet wurde, als wenn es nicht beobachtet wurde. Doch die deutlich überlappenden Balken des 95\% Konvidenzintervalls zeigen, dass dieser Unterschied auch durch Zufall zustande gekommen sein kann.
 
 Es gab keinen signifikanten Zusammenhang zwischen dem Auftreten von \enquote{schwarz-glänzenden Bienen} und der Höhe des Virustiters bei CBPV-positiven Proben (W=27; P=0,149; \cref{fig:o:CBPV_schwarz}). Für die Proben aus Völkern mit dem Symptom betrug der CBPV-Titer im Median 1,34x$10^5$ RNA-Kopien/\si{\milli\liter} Homogenat (Q1-Q3: 4,59x$10^4$-1,83x$10^8$ RNA-Kopien/\si{\milli\liter} Homogenat) und für die Proben aus symptomfreien Völkern betrug er im Median 1,64x$10^7$ RNA-Kopien/\si{\milli\liter} Homogenat (Q1-Q3: 1,20x$6^7$-4,50x$10^8$ RNA-Kopien/\si{\milli\liter} Homogenat).
 
 
  \myfig{project-A-virenmonitoring/figures/Erg_virus19_schwarz}
  {width=\textwidth} % Größe Relativ zu Text Breite
  {Zusammenhang zwischen der (A) Prävalenz von CBPV bzw. (B) dem Titer der CBPV positiven Proben mit dem Auftreten des Symptoms „schwarz-glänzende Bienen“. „ja“: Symptom in zumindest einem der fünf Probenvölker beobachtet; „nein“ Symptom in keinem der fünf Probenvölker beobachtet; „n“ Probenanzahl; „Mill.“ Million (=$10^6$); „Mrd.“ Milliarde (=$10^9$), „Bill.“ Billion (=$10^{12}$). Statistik in der Grafik: (A) $\chi^2$-Test; (B) Kruskal-Wallis-Test. Boxplot Diagramm: dicker Mittelstrich: Median; untere/obere Grenze der Box: unteres/oberes Quartil; Antennen: maximal 1,5-facher Interquartilsabstand.} % Text unterhalb der Grafik
  {} % Optional Kurz Überschrift
  {fig:o:CBPV_schwarz} % Label zum Verweisen im Text

\subsubsection{Sackbrutsymptome}

Insgesamt haben 87\% der TeilnehmerInnen Angaben zu Sackbrut-Symptomen gemacht (168 von 193 TeilnehmerInnen). Sackbrutsymptome wurden auf fünf Ständen in mindestens einem Probenvolk beobachtet und sind damit auf 3,0\% der Stände aufgetreten. Aufgrund des seltenen Auftretens des Symptoms war eine Auswertung auf Zusammenhang mit SBV nicht möglich.