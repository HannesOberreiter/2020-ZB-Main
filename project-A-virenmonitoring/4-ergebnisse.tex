\section{Ergebnisse}

\subsection{Projektfortschritt}
\subsubsection{Kontakt mit den ProjektteilnehmerInnen}

\textit{Probenahme 2019}

Im Jahr 2019 schickten 193 teilnehmenden Imkerinnen und Imker ihre Proben ein (97\% der 200 zugeschickten Probensets, \cref{tab:b:probenanzahl}). Die Gründe der Ausfälle waren unterschiedlich: Krankheitsfälle (2 ImkerInnen), unerwartet auftretende Bienenallergie (1 ImkerIn), keine Bienen mehr (1 ImkerIn) und unerklärtes Ausbleiben der Proben (3 ImkerInnen). Wie in 2018, kam es zu einer geringfügigen Verschiebung der Anteile der Bundesländer, da einige Imkerinnen und Imker einen Bienenstand in einem anderen Bundesland als dem ursprünglich angegebenen Bundesland gewählt hatten (\cref{tab:b:probenanzahl}: Vergleich „Anzahl ausgewählter Bienenstände“, „Anzahl erhaltener Proben“).

Die Proben wurden zwischen 28. August 2019 und 13. Oktober 2019 genommen, wobei 165 der TeilnehmerInnen (85\%) die Proben im anvisierten Zeitfenster zwischen 31. August und 15. September 2019 genommen haben. Weitere 25 Proben traffen in den restlichen zwei Septemberwochen ein und zwei Proben im Oktober 2019.

Vier der ImkerInnen haben nur von vier Völkern - statt wie geplant von fünf Völkern - Proben eingeschickt. Dies lag daran, dass alle genau fünf Völker an dem Probenahme-Bienenstand aufgestellt hatten und eines ihrer Völker kurz vor der Probenahme abgestorben war. Wir haben entschieden, diese vier Proben in die Analyse aufzunehmen, um die Messreihe der drei Jahre nicht zu unterbrechen.

Die Fragebögen wurden von allen Teilnehmerinnen und Teilnehmern ausgefüllt. Bei etwa einem Drittel der Fragebögen wurde aufgrund unlogischer oder fehlender Angaben seitens der TeilnehmerInnen noch bei den TeilnehmerInnen rückgefragt und die Angaben gegebenenfalls korrigiert. Die optionale Zusatzaufgabe einer Durchsicht auf ausgewählte Krankheitssymptome wurde von 161 der Imkerinnen und Imker vollständig durchgeführt (83,4\% von 193). Weitere 10 ImkerInnen gaben zu einem Teil der Symptome Beobachtungen an. 22 ImkerInnen führten keine Völkerdurchsicht durch.

\textit{Mitteilung der Ergebnisse der Probenahme 2019}

Die teilnehmenden ImkerInnen erhielten Anfang März 2019 die individualisierten Ergebnisse der Virusuntersuchungen ihrer Sammelprobe per E-Mail oder per Brief (\cref{chap:anhang_Ergebnis}). Die Ergebnismittelung enthielt folgende Information über die acht untersuchten Viren: den Nachweis (positiv/negativ), den Virustiter und dessen Beurteilung (niedrig, mittel, hoch). Die Beurteilung erfolgte als relative Einschätzung im Vergleich zu den in der gesamten Stichprobe gemessenen Titerwerten des jeweiligen Virus (niedrig: Titer im Bereich der niedrigsten 25\%; mittel: Titer liegt zwischen 25\% und 75\%, hoch: Titer liegt im Bereich der 25\% höchsten Werte).

Allen fünf ImkerInnen, die im Jahr 2019 das erste Mal an der Probenahme teilgenommen haben, wurde das Informationsdokument vom Vorjahr mit Basiswissen zu den untersuchten Bienenviren  zugeschickt (\cref{chap:anhang_FAQ}).


\textit{Abfrage Winterverluste 2019/20}

Die Umfrage zu den Winterverlusten 2019/20 wurde allen 193 ImkerInnen, die im September 2019 Proben eingeschickt hatten, Anfang April 2020 per E-Mail-Link oder Brief zugeschickt. ImkerInnen, die noch keine Winterverluste gemeldet hatte, wurden im Monats-Rhythmus per E-Mail oder Telefon daran erinnert, uns die Daten zukommen zu lassen. Mit 13. Juli 2020 hatten 192 TeilnehmerInnen ihre Überwinterungsergebnisse gemeldet. Daten von einem Imker/einer Imkerin fehlen noch. Da diese Person nicht mehr per Telefon oder Email erreichbar war, wurde sie aus dem Projekt ausgeschieden.

\subsubsection{Datenauswertung}

Für das Monitoringjahr 2019 wurden bisher die Virusprävalenz und die Virustiter der acht gemessenen Bienenviren für Gesamtösterreich und die neun Bundesländer ausgewertet. Die weiteren Analysen über die Korrelation zwischen Virusauftreten und Seehöhe, Winterverlusten und Virussymptomatik in den Völkern werden in den nächsten Monaten durchgeführt. Sie werden im nächsten Zwischenbericht im Herbst 2020 behandelt. 

\subsection{Kennwerte der teilnehmenden Imkereibetriebe}

Die Betriebsgröße der im Jahr 2019 teilnehmenden Imkereibetriebe bewegte sich zwischen fünf und 350 Völkern pro Betrieb. Im Mittel besaß ein teilnehmender Betrieb 38,1 Völker (Standardabweichung: ±50,2 Völker). Die Stichprobe bestand aus 84 Betrieben mit bis zu 20 Völkern, 79 Betrieben mit 21 bis 50 Völkern und 30 Betrieben mit über 50 Völkern (43,5\%; 40,9\% und 15,6\% der 193 teilnehmenden Betriebe).

Die im Jahr 2019 teilnehmenden ImkerInnen setzten sich sowohl aus sehr erfahrenen ImkerInnen mit maximal 58 Jahren Imkereierfahrung als auch aus NeueinsteigerInnen mit einem Jahr Imkereierfahrung zusammen. Der Mittelwert der Stichprobe betrug 16,6 Jahre Imkereierfahrung (Standardabweichung: ±15,1 Jahre). Der Anteil der ImkerInnen mit wenigen Erfahrungsjahren überwog: 54,4\% der ImkerInnen hatten bis zu 10 Erfahrungsjahre (105 von 193 TeilnehmerInnen), weitere 26,9\% zwischen 11 und 30 Jahre (52 von 193) und die restlichen 18,7\% der TeilnehmerInnen imkerten schon über 30 Jahre (36 von 193). 

Die TeilnehmerInnen der Probenahme 2019 betrieben zu 71,5\% konventionelle Imkerei (138 Betriebe) und zu 24,9\% imkerten sie mit einem Biozertifikat (48 Betriebe). 3,6\% der konventionellen Betriebe wurden unter "in Umstellung" geführt (7 Betriebe), da sie angaben in Umstellung zum zum Biobetrieb zu sein. Sie könnten daher eventuell in den nächsten Jahren in die Gruppe der Biobetriebe eingereiht werden.

\subsection{Ergebnisse Virusdiagnostik}

\subsubsection{Prozesskontrollen}
\label{chap:prozesskontrollen}

Im Zuge der Probenaufarbeitung wurden elf Wasserproben als Prozesskontrollen mitgeführt. Alle Pozesskontrollen waren negativ auf Apis-Actin mRNA, sowie auf alle getesteten viralen Erreger. Diese Ergebnisse weisen darauf hin, dass es während der Homogenisation der Proben zu keinen Verschleppungen viruspositiven Materials zwischen den Proben gekommen ist.

\subsubsection{Negativextraktionskontrollen}
\label{chap:negativkontrollen}

Bei einem gewissen Anteil der Negativextraktionskontrollen (n=14) wurden je nach Virus geringe Anteile an positiven Virusnachweisen beobachtet. Diese betrafen bei ABPV 14\% und bei DWV-B 7\% der Negativextraktionskontrollen. In allen Fällen blieben diese Nachweise in einem sehr niedrigen Konzentrationsbereich, unterhalb des LOQ\textsubscript{PCR}. Die anderen getesteten Viren (BQCV, CBPV, DWV-A, SBV, IAPV, KBV) waren in keiner der Negativextraktionskontrollen nachweisbar. Diese Ergebnisse lassen den Schluss zu, dass es im Zuge der Extraktion, bzw. im Zuge der darauffolgenden wiederholten PCR-Untersuchungen ausgehend von denselben Extrakten nicht zu nennenswerten Kreuzkontaminationen der Proben gekommen ist. Ein geringes Ausmaß an Kreuzkontaminationen ist trotz sorgfältiger Aufarbeitung bei den im Rahmen dieses Projektes beobachteten Anteilen an viruspositiven Proben und den teilweise sehr hohen Viruslasten zu erwarten.

\subsubsection{Semi-quantitativer Nachweis der Apis-Actin mRNA}

Die Apis-Actin mRNA war in allen 193 Projektproben eindeutig mit Ct-Werten von 16 bis 24 nachweisbar, was auf eine erfolgreiche Nukleinsäureextraktion und die Abwesenheit nennenswerter Anteile an inhibitorischen Substanzen in den Nukleinsäureextrakten hinweist.

\subsection{Virusprävalenz und Viruskonzentration}

Im September 2019 wurden sechs der untersuchten acht Viren in den 193 untersuchten Proben gefunden (\cref{fig:e:virusQUAL}). IAPV und KBV wurden in keiner Probe gefunden. Drei Viren traten in über 80\% der Proben auf (BQCV, DWV-B, SBV) und waren damit sehr häufig. Die anderen drei Viren traten in weniger als einem Drittel der Proben auf und waren damit eher selten (ABPV, CBPV, DWV-A).

BQCV war am häufigsten und trat mit einer Prävalenz von \confi{99,0}{95}{96,8}{99,8} auf (\cref{fig:e:virusQUAL}). DWV-B hatte mit \confi{88,6}{95}{83,6}{92,6} das zweithöchste und SBV mit \confi{80,8}{95}{74,9}{86,0} das dritthöchste Auftreten. ABPV kam mit einer Prävalenz von \confi{33,7}{95}{27,2}{40,5} deutlich seltener vor. CBPV hatte eine Prävalenz von \confi{7,3}{95}{4,1}{11,5}. DWV-A kam in nur einer von 193 Proben vor und hatte daher eine Prävalenz von \confi{0,5}{95}{0,0}{2,3}.

\myfig{project-A-virenmonitoring/figures/Erg_virus19_QUAL}
  {width=\textwidth} % Größe Relativ zu Text Breite
  {Prävalenz der acht untersuchten Viren in Bienenproben vom Herbst 2019 (±95\%KI). Stichprobenumfang: 193 Proben.} % Text unterhalb der Grafik
  {} % Optional Kurz Überschrift
  {fig:e:virusQUAL} % Label zum Verweisen im Text

In 99\% der untersuchten Proben wurde zumindest ein Virus gefunden (191 von 193 Proben). Die maximale Anzahl an gemessenen Viren pro Probe waren fünf (2\% der Proben). In den meisten Proben wurden entweder drei Viren (49\% der Proben) oder vier Viren (30\%) nachgewiesen. In 17\% der Proben wurden zwei Viren nachgewiesen, sehr selten nur ein einziger Virus (2\% der Proben). Im Mittel wurden 3,1 Viren (± 0,85 SD) in einer Probe gefunden.

\begin{table}
    \centering
    \caption{Virustiter der sechs in den 193 untersuchten Proben nachgewiesenen Viren im Herbst 2019. Anzahl~= ~Anzahl positiver Proben.}
    \label{tab:f:virusQUANT}
    \begin{tabular}{l*{6}{r}}
        \toprule
        \makecell{Virus}      &   
        \makecell{Anzahl}     & 
        \makecell{Median}     & 
        \makecell{1. Quartil} & 
        \makecell{3. Quartil} & 
        \makecell{Minimum}    & 
        \makecell{Maximum}    \\
        \midrule
        ABPV    & 65        & 4,46 x $10^{5}$    & 3,43 x $10^{4}$  & 2,24 x $10^{8}$   & 1,63 x $10^{4}$   & 2,73 x $10^{10}$\\ 
        BQCV    & 191       & 6,30 x $10^{5}$    & 2,72 x $10^{5}$  & 2,23 x $10^{6}$   & 1,80 x $10^{4}$   & 3,09 x $10^{9}$\\
        CBPV    & 14        & 8,81 x $10^{6}$    & 1,40 x $10^{5}$  & 9,27 x $10^{8}$   & 2,46 x $10^{4}$   & 2,56 x $10^{10}$\\
        DWV-A   & 1         & 3,30 x $10^{7}$    & 3,30 x $10^{7}$  & 3,30 x $10^{7}$   & 3,30 x $10^{7}$   & 3,30 x $10^{7}$\\
        DWV-B   & 171       & 1,03 x $10^{8}$    & 4,76 x $10^{6}$  & 5,81 x $10^{8}$   & 1,76 x $10^{4}$   & 2,18 x $10^{10}$\\
        SBV     & 156       & 4,19 x $10^{5}$    & 7,34 x $10^{4}$  & 3,58 x $10^{7}$   & 1,61 x $10^{4}$   & 2,54 x $10^{10}$\\
        \bottomrule
    \end{tabular}
\end{table}

Der Virustiter der positiven Proben variierte bei allen gemessenen Viren um mehrere Zehnerpotenzen. Der minimal gemessene Titer lag bei allen Viren zwischen $10^5$ und $10^9$ RNA-Kopien/\si{\milli\liter} Homogenat. Der maximal gemessene Wert lag zwischen $10^7$ und $10^{11}$ RNA-Kopien/\si{\milli\liter} Homogenat (\cref{tab:f:virusQUANT}, \cref{fig:e:virusQUAL}). Anders ausgedrückt, wurden in den Proben zwischen Hunderttausenden und Hundert Millionen RNA-Kopien/\si{\milli\liter} Homogenat gemessen. Die drei Viren ABPV, BQCV und SBV hatten die geringsten Titer (Median unter $10^6$). Bei diesen drei Virusarten wurden jedoch auch Werte von über $10^9$ RNA-Kopien/\si{\milli\liter} Homogenat gemessen. Bei CBPV lag der Median um eine Zehnerpotenz höher bei 8,8 x $10^6$ RNA-Kopien/\si{\milli\liter} Homogenat. Der mediane Virustiter von DWV-B lag bei etwa $10^8$ RNA-Kopien/\si{\milli\liter} Homogenat und war mit Abstand an höchsten. Entsprechend hatten auch etwa 25\% der positiven Proben von CBPV und DWV-B einen Virustiter über $10^9$ RNA-Kopien/\si{\milli\liter} Homogenat. Die statistischen Angaben zum Virus DWV-A sind aufgrund der geringen Stichprobenanzahl von nur einer positiven Proben nicht aussagekräftig.


\myfig{project-A-virenmonitoring/figures/Erg_virus19_QUANT}
  {width=\textwidth} % Größe Relativ zu Text Breite
  {Virustiter der sechs nachgewiesenen Viren im Herbst 2019. Es sind nur die Werte der positiven Proben gezeigt (N= Anzahl der positiven Proben). „Mill.“ Million (=10$10^6$); „Mrd.“ Milliarde (=$10^9$), „Bill.“ Billion (=$10^{12}$). Darstellung als Boxplot Diagramm: dicker Mittelstrich: Median; untere/obere Grenze der Box: unteres/oberes Quartil; Antennen: maximal 1,5-facher Interquartilsabstand.} % Text unterhalb der Grafik
  {} % Optional Kurz Überschrift
  {fig:e:virusQUANT} % Label zum Verweisen im Text


\subsection{Geographische Varianz in der Virusprävalenz und Viruskonzentration}
\subsubsection{Bundesländer}

Die Virusprävalenz von vier der gefundenen Viren unterschied sich signifikant zwischen den einzelnen Bundesländern (ABPV, BQCV, DWV-B, SBV; \cref{fig:h:virusQUAL_BL}). 
 Für CBPV wurde kein Zusammenhang zwischen Virusprävalenz und Bundesland gemessen. DWV-A wurde aufgrund der geringen Stichprobe von einem positiven Werten nicht statistisch ausgewertet. Bei ABPV und DWV-B variierte auch der Virustiter der positiven Proben zwischen den Bundesländern (\cref{fig:i:virusQUANT_BL}). 

Die Virusprävalenz und der Virustiter von ABPV unterschied sich signifikant zwischen den neun Bundesländern (Prävalenz: $\chi^2$=29,38; df=8; P<0,001; Titer: H=15,81; df=7; P=0,027). ABPV ist im Burgenland mit \confi{50,0}{95}{16,4}{83,6} am häufigsten aufgetreten. Gleichzeitig wurde dort auch der höchste Virustiter mit einem Median von 7,58x$10^8$ RNA-Kopien/\si{\milli\liter} Homogenat (Q1-Q3: 3,77x$10^8$-1,05x$10^9$ RNA-Kopien/\si{\milli\liter} Homogenat) gemessen. Die minimale Prävalenz wurde in Tirol gemessen, hier war keine Probe positiv (Prävalenz 0\%). Der geringste Virustiter wurde in Salzburg gemessen und betrug im Median 2,34x$10^4$ RNA-Kopien/\si{\milli\liter} Homogenat (Q1-Q3: 2,25x$10^4$-5,60x$10^4$ RNA-Kopien/\si{\milli\liter} Homogenat).

Die Prävalenz von BQCV unterschied sich signifikant zwischen den neun Bundesländern ($\chi ^2$=17,51; df=8; P=0,025). In sieben Bundesländern waren alle Proben positiv auf BQCV (Prävalenz 100\%). Nur in Niederösterreich und Tirol waren einige der Proben negativ. Dabei war die Prävalenz in Tirol mit \confi{95,0}{95}{88,0}{98,6} am geringsten. Es konnte kein signifikanter Unterschied in der Höhe des Virustiters zwischen den Bundesländern festgestellt werden (H=11,26; df=8; P=0,187).

Die Prävalenz von CBPV unterschied sich nicht signifikant zwischen den Bundesländern, es ist jedoch eine Tendenz zu Unterschieden zu erkennen ($\chi^2$=14,25; df=8; P=0,076). Die maximale Prävalenz wurde in Vorarlberg mit \confi{20,0}{95}{4,4}{47,0} gemessen. Am geringsten war die Prävalenz mit im Burgenland und Niederösterreich. In beiden Bundesländern wurden keine positiven Proben gefunden (Prävalenz 0\%). Der Virustiter unterschied sich nicht signifikant zwischen den neun Bundesländern (H=7,42; df=6; P=0,284).

DWV-A wurde aufgrund der geringen Stichprobe von einem positiven Werten nicht statistisch ausgewertet.

Bei DWV-B unterschieden sich Prävalenz und Virustiter signifikant zwischen den Bundesländern (Prävalenz: $\chi^2$=26,37; df=8; P<0,001; Titer: H=18,48; df=8; P=0,018). Die höchste Prävalenz wurde im Burgenland und in Wien mit 100\% erreicht. Die geringste Prävalenz wurde in Salzburg mit \confi{64,3}{95}{38,8}{85,2} gemessen. In Wien wurde der höchste mediane Virustiter gemessen (1,57x$10^9$ RNA-Kopien/\si{\milli\liter} Homogenat; Q1-Q3: 5,51x$10^7$-2,68x$10^9$ RNA-Kopien/\si{\milli\liter} Homogenat). Der gerinste mediane Titer wurde in Tirol gemessen und betrug 4,86x$10^6$ RNA-Kopien/\si{\milli\liter} Homogenat (Q1-Q3: 1,05x$10^5$-8,61x$10^7$ RNA-Kopien/\si{\milli\liter} Homogenat).


Die Prävalenz von SBV unterschied sich signifikant zwischen den neun Bundesländern ($\chi^2$=30,79; df=8; P<0,001). Burgenland und Vorarlberg wiesen die höchste Prävalenz auf; alle Proben aus diesen beiden Bundesländern waren positiv auf SBV (Prävalenz 100\%). Die geringste Prävalenz hatten die Proben aus Kärnten mit einer Prävalenz von \confi{40,9}{95}{22,5}{61,2}. Der Virustiter unterschied sich nicht signifikant zwischen den Proben aus den unterschiedlichen Bundesländern (H=8,57; df=8; P=0,380).

\myfig{project-A-virenmonitoring/figures/Erg_virus19_QUAL_BL}
  {width=\textwidth} % Größe Relativ zu Text Breite
  {Virusprävalenz in den neun Bundesländen (±95\%KI). Anzahl Stände pro Bundesland: Bgl. N=6; Ktn. N=22; NÖ. N=35;.OÖ. N=42; Sbg. N=14; Stm. N=35; Tirol N=20; Vbg. N=10; Wien N=9.} % Text unterhalb der Grafik
  {} % Optional Kurz Überschrift
  {fig:h:virusQUAL_BL} % Label zum Verweisen im Text
  
  
 \myfig{project-A-virenmonitoring/figures/Erg_virus19_QUANT_BL}
  {width=\textwidth} % Größe Relativ zu Text Breite
  {Virustiter der sechs nachgewiesenen Viren in den positiven Proben in den neun Bundesländern (N=Anzahl der positiven Proben). „Mill.“ Million (=10$10^6$); „Mrd.“ Milliarde (=$10^9$), „Bill.“ Billion (=$10^{12}$). Statistik in der Grafik: Kruskal-Wallis-Test. Darstellung als Boxplot Diagramm: dicker Mittelstrich: Median; untere/obere Grenze der Box: unteres/oberes Quartil; Antennen: maximal 1,5-facher Interquartilsabstand.} % Text unterhalb der Grafik
  {} % Optional Kurz Überschrift
  {fig:i:virusQUANT_BL} % Label zum Verweisen im Text






