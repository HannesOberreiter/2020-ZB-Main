\section{Einleitung}

Bienenviren gelten gemeinsam mit der Varroamilbe (\textit{Varroa destructor}) als wichtige Faktoren für das Absterben von Bienenvölkern über den Winter \citep{carreck2010,genersch2010,dainat2012}. Vor allem die Viren ABPV (Akute Bienenparalyse-Virus), DWV (Flügeldeformationsvirus) und IAPV (Israelisches Akute Paralyse-Virus) stehen im Verdacht, Winterverluste zu verursachen \citep{coxfoster2007,berthoud2010,genersch2010}. Schwere Infektionen mit ABPV oder DWV gehen meist mit einem schweren Befall mit der Varroamilbe einher. Diese fungiert als Vektor und überträgt die Viren auf Bienen und Bienenbrut \citep{bowen-walker1999,chen2004}. Dadurch werden die Bienen doppelt geschädigt; sowohl direkt durch die Saugtätigkeit und den Nährstoffentzug durch die Milbe und ihre Nachkommen als auch durch die Funktion als Virusvektor \citep{amdam2004,highfield2009}. Auch während des Jahres können Virusinfektionen zu Ausfällen und Schwächungen von Bienenvölkern führen. In diesem Zusammenhang sind vor allem CBPV (Chronische Bienenparalyse-Virus), BQCV (Schwarzes Königinnenzellen-Virus) oder SBV (Sackbrutvirus) zu nennen \citep{chen2007,ribiere2010,roy2015}.

Zahlreiche internationale Studien zeigen, dass es bezüglich der Prävalenz von Bienenviren zwischen den Regionen und den Erhebungsjahren beträchtliche Unterschiede gibt \citep{tentcheva2004,genersch2010,traynor2016}. Es sind daher entsprechende eigene Untersuchungen erforderlich, um Informationen zur Situation in Österreich zu erhalten. 

Trotz der Bedeutung der Bienenviren für die Bienengesundheit ist über das Vorkommen von Viren in Österreichs Bienenvölkern bisher nur begrenztes Wissen vorhanden. Dieses stammt aus Vorläuferprojekten; meist von Bienen- und Brutproben aus abgestorbenen, kranken und zusammenbrechenden Völkern und von Völkern mit Vergiftungssymptomen \citep{berenyi2006,köglberger2009,girsch2012,moosbeckhofer2014}. In toten, geschwächten und erkrankten Völkern ist jedoch mit einem anderen Virenspektrum zu rechnen als in gesunden Völkern \citep{amiri2015,morawetz2018}. Die Ergebnisse der Vorprojekte erlauben somit keine gesicherten Aussagen zur generellen Prävalenz der untersuchten Bienenviren in Österreich. Auch kann mit den vorhandenen Daten nicht unterschieden werden, welche Viren allgemein häufig in Bienenvölkern auftreten und welche tendenziell bei Völkern mit Problemen zu finden sind. 

Im vorliegenden Projekt wird daher die Prävalenz von sieben Bienenviren in Österreich über mehrere Jahre erhoben (ABPV, BQCV, CBPV, DWV, IAPV, KBV [Kashmir-Bienenvirus], SBV). Vom DWV-Virus wurden in den letzten Jahren mehrere Typen beschrieben \citep{martin2012,mordecai2016}. Im vorliegenden Projekt werden bei DWV die zwei Typen A und B unterschieden \citep{martin2012}. DWV Typ B wird in der Literatur auch als Varroa destructor virus-1 (= VDV-1) bezeichnet. Daher werden im Projekt de facto acht Bienenviren erfasst und es wird im weiteren Bericht auf acht Bienenviren Bezug genommen.

Die Bienenviren ABPV, BQCV, CBPV, DWV und SBV wurden regelmässig in österreichischen Bienenviren nachgewiesen \citep{berenyi2006,girsch2012,köglberger2009,morawetz2018}. IAPV und KBV wurden nur in wenigen Einzelvölkern nachgewiesen \citep{girsch2012}. 

Primäres Ziel des Virenmonitorings im Projekt „Zukunft Biene 2“ ist die Klärung folgender Fragen bezüglich der Virenprävalenz: 

\begin{itemize}
    \item Wie hoch ist die Prävalenz der acht genannten Bienenviren in Österreich? 
    \item Gibt es Schwankungen in der Virusprävalenz zwischen den drei Untersuchungsjahren?
\end{itemize}

Das sekundäre Ziel ist es, den möglichen Einfluss der in den Bienenvölkern nachgewiesenen Viren in Bezug auf Winterverluste zu untersuchen. Dabei können mit dem zu erwartenden Datensatz folgende Fragen behandelt werden: 

\begin{itemize}
    \item Gibt es eine Korrelation zwischen dem Auftreten von einzelnen Bienenviren vor der Einwinterung und den Winterverlusten bei den Probenvölkern/dem beprobten Bienenstand im darauffolgenden Winter? 
    \item Gibt es eine Korrelation zwischen der Höhe des Virustiters der unterschiedlichen Bienenviren vor der Einwinterung und den Winterverlusten bei den Probenvölkern/dem beprobten Bienenstand im darauffolgenden Winter?
\end{itemize}

Die zu erwartenden Gesamtergebnisse werden es möglich machen abzuschätzen, welche der untersuchten Bienenviren für die Bienengesundheit – und damit für den Bienenbestand und die Imkereiwirtschaft in Österreich – von hoher Relevanz sind. Gleichzeitig werden Viren und deren kritsche Titerwerte identifiziert, die im Untersuchungszeitraum einen negativen Einfluss auf die Überwinterung der Bienenvölker haben.
