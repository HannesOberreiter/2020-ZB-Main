\section{Anhang}

%\subsection{Anhang I: Abürzungsverzeichnis}

%\begin{table}[h!]
    \centering
    \caption{Abkürzungen}
    \label{tab:g:abkürzungen}
    \begin{tabular}{l|l}
        \toprule
        Abkürzung   &   Beschreibung\\
        \midrule
        ABPV        & Akute Bienenparalyse Virus\\
        BQCV        & Schwarzes Königinnenzellen Virus\\
        CBPV        & Chronische Bienenparalyse Virus\\
        DWV         & Flügeldeformationsvirus\\
        SBV         & Sackbrutvirus\\
        \bottomrule
    \end{tabular}
\end{table}

\subsection{Anhang I: Anleitung zur Probenahme}
\label{chap:anhang_Anleitung}

\myfig{project-A-virenmonitoring/figures/Anhang/Anhang_Probenahme_2019_S1}
  {width=0.87\textwidth} % Größe Relativ zu Text Breite
  {Arbeitsanleitung für Probenahme - Seite 1} % Text unterhalb der Grafik
  {} % Optional Kurz Überschrift
  {fig:a1:anleitung1} % Label zum Verweisen im Text
  
\myfig{project-A-virenmonitoring/figures/Anhang/Anhang_Probenahme_2019_S2}
  {width=0.93\textwidth} % Größe Relativ zu Text Breite
  {Arbeitsanleitung für Probenahme - Seite 2} % Text unterhalb der Grafik
  {} % Optional Kurz Überschrift
  {fig:a1:anleitung2} % Label zum Verweisen im Text
  
\myfig{project-A-virenmonitoring/figures/Anhang/Anhang_Probenahme_2019_S3}
  {width=0.93\textwidth} % Größe Relativ zu Text Breite
  {Arbeitsanleitung für Probenahme - Seite 3} % Text unterhalb der Grafik
  {} % Optional Kurz Überschrift
  {fig:a1:anleitung3} % Label zum Verweisen im Text
  
 \myfig{project-A-virenmonitoring/figures/Anhang/Anhang_Probenahme_2019_S4}
  {width=0.93\textwidth} % Größe Relativ zu Text Breite
  {Arbeitsanleitung für Probenahme - Seite 4} % Text unterhalb der Grafik
  {} % Optional Kurz Überschrift
  {fig:a1:anleitung4} % Label zum Verweisen im Text

\subsection{Anhang II: Fragebogen Probenahme 2019}
\label{chap:anhang_Fragebogen}
\myfig{project-A-virenmonitoring/figures/Anhang/Anhang_Fragebogen1}
  {width=0.87\textwidth} % Größe Relativ zu Text Breite
  {Beispiel Fragebogen bei Probenahme - Seite 1} % Text unterhalb der Grafik
  {} % Optional Kurz Überschrift
  {fig:a:fragebogen1} % Label zum Verweisen im Text
  
  \myfig{project-A-virenmonitoring/figures/Anhang/Anhang_Fragebogen2}
  {width=0.93\textwidth} % Größe Relativ zu Text Breite
  {Beispiel Fragebogen bei Probenahme - Seite 2} % Text unterhalb der Grafik
  {} % Optional Kurz Überschrift
  {fig:a:fragebogen2} % Label zum Verweisen im Text

\subsection{Anhang III: Ergebnis Virenanalyse}
\label{chap:anhang_Ergebnis}

\textbf{Ergebnis Virenanalyse Probenahme Herbst 2019}

\textbf{ID Bienenstand}: VIM***\\
\textbf{Datum Probenahme}: **.09.2019\\
\textbf{Ort Probenahme}: Gemeinde, Bundesland\\\

Probenahme:\\
Es wurden von Ihnen Bienenproben aus 5 Bienenvölkern entnommen (etwa zehn Bienen pro
Volk) und lebend in Königinnenversandkäfigen an die AGES, Abteilung Bienenkunde und
Bienenschutz, geschickt. Dort wurden die Bienen umgehend auf -20°C tiefgefroren und damit
abgetötet. Aus den Bienen Ihrer Einsendung wurde eine Sammelprobe von 50 Arbeiterinnen
erstellt und diese mittels einer molekularbiologischen Analyse (RT-PCR) auf insgesamt acht
Bienenviren untersucht:

\begin{itemize}
    \item Akute Bienenparalyse Virus (=ABPV)
    \item Schwarzes Königinnenzellen Virus (=BQCV)
    \item Chronische Bienenparalyse Virus (=CBPV)
    \item Flügeldeformationsvirus Typ A (=DWV A)
    \item Flügeldeformationsvirus Typ B (=DWV B)
    \item Israelisches Akute Paralyse Virus (=IAPV)
    \item Kashmir-Bienenvirus (=KBV)
    \item Sackbrutvirus (=SBV)
\end{itemize}

Die Ergebnisse des Bienenstandes VIM*** sind auf den folgenden zwei Seiten sowohl in \cref{tab:i:bsp_ergebnis,fig:a3:bsp_ergebnis} dargestellt.

\textbf{Ergebnisse Tabelle (ID Bienenstand VIM***)}\\
In der Tabelle werden die Ergebnisse aller Viren wie folgt dargestellt:
\begin{itemize}
    \item Virus (ABPV,…., SBV)
    \item Nachweis (positiv/negativ)
    \item Viruskonzentration (RNA-Kopien pro ml Homogenat)
\end{itemize}


\begin{table}[htp]
    \caption{Ergebnisse Bienenstand VIM***}
    \centering
    \begin{tabular}{|l|l|c|c|}
    \hline
    Virus   & Nachweis &  Viruskonzentration    & Beurteilung \\
    \hline
    ABPV    &  negativ  &   ---                 & ---\\
    \textbf{BQCV}    &  \textbf{positiv}  &   \textbf{11,43 Mill.}         & \textbf{hoch}\\
    CBPV    &  negativ  &   ---                 & ---\\
    DWV A   &  negativ  &   ---                 & ---\\
    \textbf{DWV B}   &  \textbf{positiv}  &   \textbf{2,55 Mill.}          & \textbf{hoch}\\
    IAPV    &  negativ  &   ---                 & ---\\
    KBV     &  negativ  &   ---                 & ---\\
    SBV     &  positiv  &   1,15 Mill.          & mittel\\
    \hline
    \end{tabular}
    \label{tab:i:bsp_ergebnis}
\end{table}


Legende:\\
positiv: Virus-RNA wurde in der Sammelprobe festgestellt
Konzentration: gibt an wie viele RNA-Kopien pro Milliliter untersuchter Lösung (=Homogenat)
gemessen wurden
Beurteilung: setzt die jeweilige Viruskonzentration Ihres Standes in Beziehung zu den positiv
gemessenen Werten der anderen teilnehmenden Stände:
\begin{itemize}
    \item niedrig: Konzentration der Probe liegt im Bereich der 25\% niedrigsten Werte
    \item mittel: Konzentration der Probe liegt im mittleren Bereich aller Werte
    \item hoch: Konzentration der Probe liegt im Bereich der 25\% höchsten Werte
    \item keine Beurtl.: zu wenige positive Proben für eine Beurteilung der Viruskonzentration
    \item - - -: nicht nachweisbar
\end{itemize}

\textbf{Ergebnisse Grafik (ID Bienenstand VIM***)}\\
Die \cref{fig:a3:bsp_ergebnis} vergleicht das Messergebnis Ihres Standes mit den Messergebnissen der anderen
teilnehmenden Bienenstände, die positiv auf das entsprechende Virus getestet wurden. Dabei stellt die Grafik Ihren Messwert (=schwarzer Diamant), den niedrigsten und den höchsten
gemessenen Wert der Vergleichsstände (unterer und oberer Balken), sowie die auf der vorigen
Seite definierten Bereiche “hoch”, “mittel” und “niedrig” dar. Wenn kein Messwert (schwarzer
Diamant) dargestellt ist, war dieses Virus in Ihrer Probe nicht nachweisbar. Bitte beachten Sie
dass die Skala logarithmisch ist, das heißt die Darstellung der Konzentration erfolgt in 10er
Potenzen (0,1 Millionen, 1 Million, 10 Millionen etc.). Für das Bienenvirus DWV-A ist nur ein
positiver Fall vorhanden - daher die abweichende Darstellung


\myfig{project-A-virenmonitoring/figures/Anhang/Anhang_Bsp_Ergebnis}
  {width=1\textwidth} % Größe Relativ zu Text Breite
  {Ergebnis Bienenstand VIM***} % Text unterhalb der Grafik
  {} % Optional Kurz Überschrift
  {fig:a3:bsp_ergebnis} % Label zum Verweisen im Text
  
\myfig{project-A-virenmonitoring/figures/Anhang/Anhang_Bsp_Legende}
  {width=1\textwidth} % Größe Relativ zu Text Breite
  {Legende zum Ergebnis} % Text unterhalb der Grafik
  {} % Optional Kurz Überschrift
  {fig:a3:bsp_legende} % Label zum Verweisen im Text



\subsection{Anhang IV: Informationsblatt zu unseren Ergebnismitteilungen} \label{chap:anhang_FAQ}


Nach Übermittlung der Virusergebnisse gab es erfreulicherweise zahlreiche Rückmeldungen und natürlich auch Fragen der am Virenmonitoring teilnehmenden Imker und Imkerinnen. Die am häufigsten gestellten Fragen und unsere Antworten darauf haben wir nun in Form eines „Frage und Antwort Kataloges“ zusammengestellt. Dieser wird an alle Teilnehmerinnen und Teilnehmer verschickt, unabhängig davon, ob sie eine Frage gestellt haben oder nicht. Somit können alle den größtmöglichen Nutzen aus der Projektteilnahme ziehen.

%\begin{table}[h!]
    \centering
    \caption{Abkürzungen}
    \label{tab:g:abkürzungen}
    \begin{tabular}{l|l}
        \toprule
        Abkürzung   &   Beschreibung\\
        \midrule
        ABPV        & Akute Bienenparalyse Virus\\
        BQCV        & Schwarzes Königinnenzellen Virus\\
        CBPV        & Chronische Bienenparalyse Virus\\
        DWV         & Flügeldeformationsvirus\\
        SBV         & Sackbrutvirus\\
        \bottomrule
    \end{tabular}
\end{table}

\subsubsection{„Frage und Antwort“ Katalog}
\textbf{Interpretation der Ergebnisse}

\textit{Kann ich von den Untersuchungsergebnissen auch Rückschlüsse auf Einzelvölker ziehen?}

Es liegt uns nur eine Information über das Vorhandensein von Viren am Stand vor (repräsentiert durch die fünf Probenvölker), jedoch nicht für das einzelne Volk. Aufgrund der Logistik ist generell bei dem Projekt keine Einzelvolkuntersuchung möglich (nicht-sterile Probenahme, gemeinsamer Versand der Probenkäfige in einem Kuvert etc.). Da bekannt ist, dass zwischen den Völkern eines Standes immer mit Verflug von Bienen zu rechnen ist, ist es sinnvoll, Virusinfektionen auf Standebene zu betrachten.


\textit{Erklärung zur Grafik bei der Ergebnismitteilung}

Die Grafik zeigt Ihre Ergebnisse (schwarzer „Diamant“), die auch in der Tabelle verbal und als Zahlenwert angegeben sind. Um Ihnen einen Vergleich Ihrer Ergebnisse zu den anderen teilnehmenden Ständen zu ermöglichen, weist die Grafik zusätzlich auch Informationen über die anderen teilnehmenden Stände aus (minimal gemessene Konzentration, maximal gemessen Konzentration etc.).

\textbf{Allgemeines zu Bienenviren}

\textit{Wo finden sich Viren im Bienenvolk?}

Viren können im gesamten Bienenvolk vorkommen. In erster Linie können die verschiedenen Bienenstadien (erwachsene Bienen, Puppen, Larven, Eier) Träger von Viren sein. Viren können aber auch in verschiedenen Materialien (Honig, Pollen, Bienenbrot, Wachs, Waben, Beuten) vorhanden sein.

\textit{Was ist eine Virose?}

Viren im Volk sind nicht gleichbedeutend mit einer Viruserkrankung (=Virose). Auch in gesunden Völkern können viele Viren latent vorhanden sein. Ein latenter Befall bedeutet, dass die Bienen keinerlei Anzeichen einer Krankheit (=Symptome) zeigen. Durch verschiedene Faktoren – wie Stress, hoher Varroabefall oder Futtermangel – können die Bienen dann Krankheitssymptome entwickeln. Erst dann spricht man von einer Virose. In manchen Fällen können derartige Symptome auch wieder ganz oder zeitweilig verschwinden.

\textit{Wie kann ich einem Ausbruch von Virosen vorbeugen?}

Da der Ausbruch einer Virose verschiedene Ursachen haben kann, gibt es auch verschiedene Möglichkeiten zur Vorbeugung. Hier die wichtigsten:

\begin{itemize}
    \item Varroamilben bekämpfen
    
    Varroamilben übertragen Viren im Volk von infizierten zu nicht-infizierten Bienen und auf die Brut. Außerdem können sich gewisse Viren in den Milben vermehren und werden in Milben aktiviert. Überdies schwächen Varroamilben die Bienen und machen sie damit empfänglicher für Virosen. Die Varroabehandlung reduziert primär den Varroabefall, ein Rückgang virusinfizierter Bienen erfolgt aber erst im Verlauf der nächsten Bienengenerationen.
    
    Als Faustregel gilt:
    \begin{itemize}
        \item Beginn der Varroabekämpfung nach Trachtschluss so zeitig als möglich, damit gesunde Winterbienen gebildet werden
        \item Anwendung geeigneter zugelassener Mittel zum richtigen Zeitpunkt
        \item nicht mehr Anwendungen als nötig
        \item nach dem Prinzip „so wenig als möglich, mit geeigneten, zugelassenen Mitteln“
    \end{itemize}
    \item Gute Stand- und Nahrungsbedingungen schaffen
    \begin{itemize}
        \item bienengenehmer Standort mit optimalem Kleinklima
        \item Völker sollten während der ganzen Bienensaison ausreichend Nektar und Pollen im Nahbereich vorfinden
        \item nicht zu viele Völker auf einem Stand stellen, angepasst an die dort vorhandene Trachtsituation
    \end{itemize}
    \item Räuberei und Verflug vermeiden
    \begin{itemize}
        \item Aufstellung der Völker nicht in einer Reihe, sondern blockweise mit Flugbrett in unterschiedliche Himmelsrichtungen, um Verflug zu verringern
        \item Vermeidung von Räuberei
    \end{itemize}
    \item Hygienische Maßnahmen treffen
    \begin{itemize}
        \item Keine Wiederverwendung von Waben (Leer-, Futter-, Pollenwaben) aus abgestorbenen Völkern in anderen Völkern oder beim Aufbau von Jungvölkern
        \item Waben toter Völker einschmelzen
        \item Beuten gründlich reinigen, Rähmchen erneuern oder ebenfalls gründlich reinigen
    \end{itemize}
\end{itemize}

\textit{Was ist der Unterschied zwischen DWV Typ A und DWV Typ B?}

DWV ist ein sehr variables Virus und die Forschung über die Bedeutung der Typen ist voll im Gange. Mit Hilfe moderner molekularbiologischer Nachweismethoden kann DWV nun in die Typen A und B unterschieden werden. Im Jahr 2018 war DWV Typ A in den Monitoring-Proben im Rahmen des Projektes sehr selten zu finden, DWV Typ B hingegen sehr häufig. DWV Typ B wird derzeit als die schädlichere Variante angesehen, die auch besonders stark mit der Varroamilbe assoziiert ist. Die Forschung dazu ist aber noch im Fluss.

\textit{Welche Viren stehen im Zusammenhang mit Varroabefall?}

Ein besonders enger Zusammenhang mit Varroabefall ist bei den Viren ABPV und DWV gegeben (Übertragung durch die Varroamilbe, teilweise Vermehrung in der Varroamilbe). Auch für die Viren BQCV  und SBV ist ein gewisser Zusammenhang bekannt.

\textit{Was kann ich tun, wenn die Ergebnisse einen hohen Virenbefall nachgewiesen haben? Muss ich etwas tun?}

Grundsätzlich leitet sich bei einem Nachweis von Viren noch KEIN Handlungsbedarf ab, selbst wenn der Virusbefall hoch ist. Wenn Sie jedoch gleichzeitig Auffälligkeiten bei den Völkern beobachten, sollten Sie die hohe Virusbelastung als eine mögliche Ursache in Betracht ziehen. In diesem Fall sollten Sie Maßnahmen ergreifen – wir haben einige davon in dem Abschnitt „Bekämpfung der Viruserkrankungen“ angeführt. In \cref{tab:h:symptome_viruserkrankungen} sehen Sie den Zusammenhang zwischen Symptomen mit den untersuchten Viren.

\begin{table}
\centering
\caption{Symptome}
\label{tab:h:symptome_viruserkrankungen}
\begin{tabular}{l|ccccc}
\toprule
\textbf{Symptome} & \textbf{ABPV} & \textbf{BQCV} & \textbf{CBPV} & \textbf{DWV} & \textbf{SBV} \\
\midrule
\makecell[l]{Völkerzusammenbrüche, \\ massiver Totenfall} 
& \Large{$+$} & \Large{$+$} & \Large{$+$} & \Large{$+$} & \Large{$+$} \\
\midrule
\makecell[l]{Verhaltensänderung, \\ Krabbler, Zittern} & \Large{$+$} & & \Large{$+$} & & \\
\midrule
\makecell[l]{Verstümmelte Flügel \\ bei Bienen}
& & & & \Large{$++$} &\\
\midrule
\makecell[l]{Körpergröße verkleinert}
& & & & \Large{$+$} & \\
\midrule
\makecell[l]{aufgeblähter Hinterleib}
& & & \Large{$+$} & & \\
\midrule
\makecell[l]{Haarlosigkeit \\ („schwarzsüchtig“)}
& & & \Large{$++$} & & \\
\midrule
\makecell[l]{Dysenterie \\ („Durchfall“)}
& & & \Large{$+$} & & \\
\midrule
\makecell[l]{Brutschäden}
& \Large{$++$} & \Large{$++$} & & \Large{$+$} & \Large{$++$} \\
\bottomrule
\multicolumn{6}{l}{$+$: Zusammenhang gegeben}\\
\multicolumn{6}{l}{$++$: starker Zusammenhang gegeben}\\
\end{tabular}\\
\end{table}

\textit{Was bedeutet ein Ergebnis „hoch“ oder „niedrig“ im Vergleich zu den Ergebnissen anderer Teilnehmer?}

Nach derzeitigem Wissensstand gibt es keine konkreten Angaben über Schadschwellen für bestimmte Bienenviren ab denen mit einer akuten Viruserkrankung zu rechnen ist. Diese Frage zu beleuchten ist auch eine Aufgabe des Projektes und wir hoffen, dass wir in drei Jahren mehr wissen. Der Vergleich zu den Ergebnissen der anderen Teilnehmerinnen und Teilnehmer wurde vorgenommen, um Ihnen die Einschätzung Ihrer positiven Virenergebnisse trotzdem etwas zu erleichtern.

\textbf{Abgestorbene Völker}

\textit{Was können die Ursachen für tote Völker im Herbst oder Winter sein?}

Wenn Bienenvölker während oder vor der Überwinterung zu Grunde gehen, kann das unterschiedliche Ursachen haben. In vielen Fällen sind mehrere schädliche Faktoren gemeinsam ursächlich an dem Völkerzusammenbruch beteiligt. Wenn im Herbst ein hoher Virusbefall festgestellt wurde, ist dies ein möglicher Faktor für den Winterverlust. Weitere Faktoren können hoher Varroabefall, Königinnenverlust, Futtermangel, ungeeignetes Futter, zu geringe Volksstärke bei der Einwinterung, Störungen der Winterruhe etc. sein.

\textit{Was mache ich mit Wabenmaterial aus abgestorbenen Völkern? Kann ich es weiterverwenden?}

Es gibt leider keine wissenschaftlichen Arbeiten zu diesem Problem und so raten wir zur Vorsicht im Umgang mit Wabenmaterial aus abgestorbenen Völkern. Wenn die Völker an einem zu hohen Varroabefall in Herbst oder im Winter zugrunde gegangen sind, bleiben in den meisten Fällen bienenleere Kisten, oft mit einer großen Menge an Honig zurück. Diese Futtervorräte, sowie das andere Wabenmaterial können durch Viren belastet sein. Wir empfehlen daher, das gesamte Wabenmaterial zu entsorgen (einzuschmelzen) und die Beuten gründlich zu reinigen.

\textit{Kann ich die Wintervorräte aus toten Völkern verfüttern?}

Honigwaben aus abgestorbenen Völkern sollten \textbf{NICHT} wieder verfüttert, sondern eingeschmolzen werden.

\subsubsection{Bekämpfung der Viruserkrankungen}

Eine direkte Bekämpfung von Viruserkrankungen ist nicht möglich, da keine Tierarzneimittel verfügbar sind. Neben den oben genannten Vorbeugungsmaßnahmen, die den Ausbruch von Viruserkrankungen verhindern sollen, werden hier Maßnahmen gegen Viruserkrankungen genannt, die als indirekte Bekämpfungsmethoden die Verbreitung der Viren eindämmen oder die Virusbelastung reduzieren und die Selbstheilung des Volkes unterstützen. In vielen Fällen können Symptome auch wieder selbstständig verschwinden (spontane Selbstheilung des Volkes).

\textit{Was mache ich bei Symptomen von ABPV?}

\begin{itemize}
    \item Akute Bienenparalyse ist mit starkem Varroabefall assoziiert, daher konsequente Varroabekämpfung durchführen
    \item Aufstellung der Völker optimieren um Verflug gering zu halten
    \item Bei starkem Auftreten können Brutwaben mit erkrankter Brut entnommen werden, dies unterstützt die Selbstheilung des Volkes.
\end{itemize}

\textit{Was mache ich bei Symptomen von BQCV?}

\begin{itemize}
    \item Keine speziellen Bekämpfungsmaßnahmen bekannt
    \item BQCV ist zwar häufig, es kommt aber selten zu Symptomen (Schwarzfärbung der Innenseite der Weiselzellen)
    \item Hinweis: kann Grund für Ausfall von Zuchtserien sein
\end{itemize}

\textit{Was mache ich bei Symptomen von CBPV?}

\begin{itemize}
    \item Völkermassierungen vermeiden
    \item Hinweis: fluglose Zeiten können zu Ausbruch beitragen
\end{itemize}

\textit{Was mache ich bei Symptomen von DWV?}

\begin{itemize}
    \item DWV-Symptome sind mit starkem Varroabefall assoziiert (=Varroose-Symptome), daher konsequente Varroabekämpfung durchführen
\end{itemize}

\textit{Was mache ich bei Symptomen von SBV?}

\begin{itemize}
    \item Sackbrut ist eine Faktorenkrankheit, die vor allem durch Futtermangel der Larven bei Trachtlücken ausgelöst wird, gute Tracht und Futterversorgung fördern die Gesundung
    \item bei starkem Auftreten können Brutwaben mit erkrankter Brut entnommen werden, dies unterstützt die Selbstheilung des Volkes
\end{itemize}





