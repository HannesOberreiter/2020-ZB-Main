\phantomsection
\addcontentsline{toc}{section}{Vorwort OLD Text!}
\section*{Vorwort OLD Text!}

Die Basis der Projekte Zukunft Biene bildeten die Winterverlusterhebungen des Institutes für Zoologie (jetzt Institut für Biologie) der Universität Graz, die seit nunmehr 12 Jahren durchgeführt werden. Die immer größere Datendichte hatte es im ersten Projekt Zukunft Biene erlaubt profunde Aussagen über verschiedene Zusammenhänge der Winterverluste mit Wetter und Landnutzung zu tätigen. Auch im abgelaufenen Projektjahr, einem Jahr mit durchschnittlichen Winterverlusten aber relativ großen lokalen Unterschieden, konnten Ergebnisse der letzten Jahre bestätigt und somit verfestigt werden. Diese sind insbesondere der positive Effekt der Verwendung von jungen Königinnen und der Anwendung biotechnischer Methoden zur Varroa-Bekämpfung. Auch der Einfluss der Betriebsgröße, der Wanderimkerei und der Tracht konnte wieder gezeigt werden. Eine Brücke zum Viren-Schwerpunkt des Projektes \enquote{Zukunft Biene 2} stellt auch das Ergebnis dar, dass wenn von den Imkereien Bienen mit verkrüppelten Flügen beobachtet wurden, unter anderem ein Symptom von Virenbefall, dies ein Alarmsignal für statistisch signifikant erhöhte Winterverluste
darstellt.

Die immer größere Datendichte dieser Ergebnisse zeigt auch eine gute Übereinstimmung der Ergebnisse der Uni Graz und der AGES hinsichtlich der Wintersterblichkeit. Die beiden Institutionen hatten wohlüberlegt unterschiedliche Konzepte bei teilweise überlappenden Fragestellungen verwendet. Hatte die AGES gezielt definierte Stände beprobt, so verwertete die
Uni Graz die größere Stichprobe der freiwilligen Einsendung der österreichischen ImkerInnen. Einer der großen Vorzüge des Projektes Zukunft Biene 2 ist der lange Untersuchungszeitraum der uns für die Virendetektion in Österreichs Bienenvölkern - untersucht werden 8 unterschiedliche Viren - zur Verfügung steht. Im ersten der drei anberaumten Jahre konnten bereits interessante Befunde über das Auftreten von Bienenviren in Österreich erarbeitet werden. So zeigten sich
sowohl lokale Unterscheide als auch, und das ist neu, eine Abhängigkeit von der Meereshöhe. Manche Viren kamen in nahezu allen untersuchten Völkern vor. Bisher liegen erst die Resultate eines Untersuchungsjahres vor, die Ergebnisse der weiteren Untersuchsjahre werden eine spannende Sache.

Die jüngste Partnerin des Kombinationsprojektes, die Veterinärmedizinische Universität Wien, bemüht sich um eine kostengünstige und einfach durchzuführende Bestimmungsmethode für Infektionen mit den wichtigsten Bienenviren. Diese Methoden sollen in weiterer Folge auch den BienenhalterInnen im Freiland zur Verfügung stehen. Besonders weit und vielversprechend sind dabei die Experimente mit Iflaviren, wie den weit verbreiteten Flügeldeformationsviren und
Sackbrutviren.

\begin{flushright}
    Karl Crailsheim, Projektleiter \\
    Graz, am 22.8.2019
\end{flushright}