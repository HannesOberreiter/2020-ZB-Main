\phantomsection
\addcontentsline{toc}{section}{Vorwort}
\section*{Vorwort}

Im Oktober 2019 durfte ich, wie bereits bei Projektbeginn geplant, die Leitung des Projekts \enquote{Zukunft Biene 2} von Herrn Professor Karl Crailsheim übernehmen. Herr Professor Crailsheim hat das Projekt vom Start im Jahr 2017 an --- obwohl bereits im Ruhestand --- geleitet und wird uns erfreulicherweise auch noch bis Projektende als Mitglied des Projekts erhalten bleiben. Ich möchte mich hiermit bei ihm für die geleistete Arbeit bedanken und festhalten, dass ich in bester Zusammenarbeit ein intaktes Projektkonsortium übergeben bekommen habe, das in allen Vorhaben im Zeitplan war. Ich selbst bin seit Beginn der \enquote{Zukunft Biene} Projektreihe im Jahr 2014 in das Projekt involviert, und habe im Jahr 2008 mit Professor Crailsheim die ersten Untersuchungen der Wintersterblichkeit in Österreich begonnen. Seit damals schätze ich den Austausch mit den PraktikerInnen, und habe in den letzten Jahren daran gearbeitet, Untersuchungen mit Imkerbeteiligung im Sinne von Citizen Science in Österreich voranzutreiben. Das zeigt sich auch in 2 Modulen des laufenden Projektes, in dem ImkerInnen aktiv in die Datengenerierung und Probennahme eingebunden sind. Auch in dem von der Covid-19 Pandemie beherrschten Jahr 2020 haben wir unsere Untersuchungen durchgeführt, und wie geplant Bienenproben auf Viren untersuchen können. Und das, obgleich ProjektmitarbeiterInnen der AGES zeitweise bei der Corona-Hotline
ausgeholfen haben und die KollegInnen von der Veterinärmedizinischen Universität Wien PCR-Tests auf Corona durchgeführt haben.

Zu unserer Freude sind im Jahr 2019 auch noch drei wissenschaftliche Publikationen mit Daten aus dem Projekt \enquote{Zukunft Biene 1} (Laufzeit: 2014-2017) erschienen. Dies zeigt, dass uns die von den Fördergebern gewährte Forschungsförderung Zeit für valide, international geprüfte, und gut aufbereitete wissenschaftliche Erkenntnisse lässt. Die für alle zugänglichen Publikationen finden sich in der Online-Zeitschrift \textit{PlosOne} zur in den Jahren 2014 und 2015 an 1596 Bienenvölkern durchgeführten
Beobachtungsstudie \citep{morawetz2019}, in der Online-Zeitschrift \textit{Scientific Reports} zur Trachtpflanzendiversität von Pollenspendern \citep{brodschneider2019a}, sowie in der Zeitschrift \textit{Agriculture, Ecosystems \& Environment} wo ein mehrjähriger Vergleich zwischen Österreich und unserem Nachbarland Tschechien hinsichtlich Imkerei und Winterverluste veröffentlicht wurde \citep{brodschneider2019}.

Eine aus dem Projekt \enquote{Zukunft Biene 2} entstammende wissenschaftliche Publikation mit einer detaillierten Auswertung der Überwinterungs-Ergebnisse der österreichischen Imkerei unter
Berücksichtigung unterschiedlicher Betriebsweisen ist im Jahr 2020 in der Online-Zeitschrift \textit{Diversity} erschienen \citep{oberreiter2020}. Wir haben der internationalen Wissenschaftsgemeinschaft damit nach einer ersten, 2010 im \textit{Journal of Apicultural Research} erschienenen Analyse, ein weiteres Mal die österreichische Imkerei und die State-of-the-Art Untersuchung von Völkersterblichkeit anhand der
von Imkereien zur Verfügung gestellten Daten darlegen können. Alle diese Publikationen zeugen von der gesteigerten Aufmerksamkeit, die der Honigbiene in unserem Land in den letzten Jahren durch diese Projektförderungen zu Teil wurde, und die sich hoffentlich direkt und möglicherweise auch indirekt auf die Praxis in Form eines gesunden Imkereisektors auswirkt.

Die immer mit Brisanz erwartete Auswinterung der Bienenvölker zeigt seit nunmehr drei Jahren in Folge
durchschnittliche Ergebnisse. Im Winter 2019/20 haben Österreichs Imkereien relativ gut ausgewintert. Nur 12,6\% der Bienenvölker haben den Winter nicht überlebt; ein Wert der akzeptiert und durch Nachschaffung auch wieder kompensiert werden kann. Es gab jedoch auch in dieser Periode wieder lokal oder
vereinzelt inakzeptabel hohe Verlustraten.

Ich möchte mich mit diesem Vorwort bei allen MitarbeiterInnen des Projekts, bei den beteiligten ImkerInnen, sowie natürlich bei den Fördergebern für die hervorragende Zusammenarbeit bedanken.
Einen langjährig gedienten Mitstreiter dürfen wir 2020 in den Ruhestand entlassen. Dr. Rudolf Moosbeckhofer hat als Leiter der Abteilung für Bienenkunde der AGES über Jahrzehnte die Bienenforschung in Österreich mitbestimmt, und kennt den Sektor wie wohl nur wenige. Ich wünsche ihm im Namen aller
ProjektmitarbeiterInnen alles Gute, wünsche mir, dass wir noch oft von seiner Expertise profitieren können, und dem Rudi volle Honigtöpfe und einen Korb voller Pilze!

\begin{flushright}
    Robert Brodschneider, Projektleiter \\
    Graz, am 03.07.2019
\end{flushright}