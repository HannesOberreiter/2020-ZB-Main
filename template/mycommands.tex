%doc% 
%doc% \subsection{\texttt{myfig} --- including graphics made easy}
%doc% 
%doc% The classic: you can easily add graphics to your document with \verb#\myfig#:
%doc% \begin{verbatim}
%doc%  \myfig{flower}%% filename w/o extension in the folder figures
%doc%        {width=0.7\textwidth}%% maximum width/height, aspect ratio will be kept
%doc%        {This flower was photographed at my home town in 2010}%% caption
%doc%        {Home town flower}%% optional (short) caption for list of figures
%doc%        {fig:flower}%% label
%doc% \end{verbatim}
%doc% 
%doc% There are many advantages of this command (compared to manual
%doc% \texttt{figure} environments and \texttt{includegraphics} commands:
%doc% \begin{itemize}
%doc% \item consistent style throughout the whole document
%doc% \item easy to change; for example move caption on top
%doc% \item much less characters to type (faster, error prone)
%doc% \item less visual clutter in the \TeX{}-files
%doc% \end{itemize}
%doc% 
%doc% 
\newcommand{\myfig}[5]{
%% example:
% \myfig{}%% filename in figures folder
%       {width=0.5\textwidth,height=0.5\textheight}%% maximum width/height, aspect ratio will be kept
%       {}%% caption
%       {}%% optional (short) caption for list of figures
%       {}%% label
\begin{figure}[H]
  \centering
  \includegraphics[keepaspectratio,#2]{#1}
  \caption[#4]{#3}
  \label{#5} %% NOTE: always label *after* caption!
\end{figure}
}


%doc% 
%doc% \subsection{\texttt{myclone} --- repeat things!}
%doc% 
%doc% Using \verb#\myclone[42]{foobar}# results the text \enquote{foobar} printed 42 times.
%doc% But you can not only repeat text output with \texttt{myclone}. 
%doc%
%doc% Default argument
%doc% for the optional parameter \enquote{number of times} (like \enquote{42} in the example above) 
%doc% is set to two.
%doc% 
%% \myclone[x]{text}
\newcounter{myclonecnt}
\newcommand{\myclone}[2][2]{%
  \setcounter{myclonecnt}{#1}%
  \whiledo{\value{myclonecnt}>0}{#2\addtocounter{myclonecnt}{-1}}%
}

%old% %d oc% 
%old% %d oc% \subsection{\texttt{fixxme} --- sidemark something as unfinished}
%old% %d oc% 
%old% %d oc% You know it: something has to be fixed and you can not do it right
%old% %d oc% now. In order to \texttt{not} forget about it, you might want to add a
%old% %d oc% note like \verb+\fixxme{check again}+ which inserts a note on the page
%old% %d oc% margin such as this\fixxme{check again} example.
%old% %d oc%
%old% \newcommand{\fixxme}[1]{%%
%old% \textcolor{red}{FIXXME}\marginpar{\textcolor{red}{#1}}%%
%old% }

% Simple Mail Function, generates clickable E-Mail Link, needs hyperref
\newcommand{\mailto}[1]{
  \href{mailto:#1}{#1}
}

%% Sample Text
\newcommand{\sample}[1]{\textit{n}~=~{#1}}

%% 95% CI Intervall
\newcommand{\confi}[4]{{#1}\% ({#2}\%~CI:~{#3}-{#4}\%)}

%% correct euro symbol
\DeclareUnicodeCharacter{20AC}{\euro}
