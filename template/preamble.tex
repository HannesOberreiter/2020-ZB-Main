\documentclass[%
fontsize=\myfontsize,%% size of the main text
paper=\mypapersize,  %% paper format
parskip=\myparskip,  %% vertical space between paragraphs (instead of indenting first par-line)
DIV=16,            %% calculates a good DIV value for type area; 66 characters/line is great
headinclude=true,    %% is header part of margin space or part of page content?
headings=optiontoheadandtoc,
footinclude=false,   %% is footer part of margin space or part of page content?
open=right,          %% "right" or "left": start new chapter on right or left page
appendixprefix=true, %% adds appendix prefix; only for book-classes with \backmatter
bibliography=totoc,  %% adds the bibliography to table of contents (without number)
draft=\mydraft,      %% if true: included graphics are omitted and black boxes
                     %%          mark overfull boxes in margin space
BCOR=\myBCOR,        %% binding correction (depends on how you bind
                     %% the resulting printout.
\mylaterality       %% oneside: document is not printed on left and right sides, only right side
                     %% twoside: document is printed on left and right sides
]{scrbook}  %% article class of KOMA: "scrartcl", "scrreprt", or "scrbook".
            %% CAUTION: If documentclass will be changed, *many* other things
            %%          change as well like heading structure, ...



% FIXXME: adopting class usage:
% from scrbook -> scrartcl OR scrreport:
% - remove appendixprefix from class options
% - remove \frontmatter \mainmatter \backmatter \appendix from main.tex

% FIXXME: adopting language:
% add or modify language parameter of package »babel« and use language switches described in babel-documentation

%doc%
%doc% \subsection{\texttt{inputenc}: UTF8 as input charset}
%doc%
%doc% You are able and should use \myacro{UTF8} character settings for writing these \TeX{}-files.
%doc%
%\usepackage{ucs}             %% UTF8 as input characters; UCS incompatible to biblatex
\usepackage[T1]{fontenc} %% for special characters
\usepackage[utf8]{inputenc} %% UTF8 as input characters

%% Source: http://latex.tugraz.at/latex/tutorial#laden_von_paketen

%doc%
%doc% \subsection{\texttt{textcomp}: Support for Text Companion fonts}
%doc%
%doc% Provides some text symbols such as bullet in \myacro{TS1} encoding.
%doc% Depending on what font or symbols you use, you might not even need this.
%doc% Removing this package will at worst result in a warning.
\usepackage{textcomp}
%% Source: https://www.ctan.org/pkg/textcomp

%doc%
%doc% \subsection{\texttt{babel}: Language settings}
%doc%
%doc% The default setting of the language is American. Please change settings for
%doc% additional or alternative languages used in \texttt{main.tex}.
%doc%
%doc% Please note that the default language of the document is the \emph{last} language
%doc% which is added to the package options.
%doc%
%doc% To set only parts of your document in a different language as the rest, use for example\newline
%doc% \verb+\foreignlanguage{ngerman}{Beispieltext in deutscher Sprache}+\newline
%doc% For using foreign language quotes, please refer to the \verb+\foreignquote+,
%doc% \verb+\foreigntextquote+, or \verb+\foreignblockquote+ provided by
%doc% \texttt{csquotes} (see Section~\ref{sub:csquotes}).
%doc%
\usepackage[\mylanguage]{babel}  %% used languages; default language is *last* language of options

%% improved german hyphen breaks
\usepackage[ngerman=ngerman-x-latest]{hyphsubst}


%doc%
%doc% \subsection{\texttt{scrlayer-scrpage}: Headers and footers}
%doc%
%doc% Since this template is based on \myacro{KOMA} script it uses its great
%doc% \texttt{scrlayer-scrpage} (previously \texttt{scrpage2})
%doc% package for defining header and footer information. Please refer to the \myacro{KOMA}
%doc% script documentation how to use this package.
%doc%
\usepackage{scrlayer-scrpage} %%  advanced page style using KOMA


%doc% \subsubsection{Example citation commands}
%doc%
%doc% This section demonstrates some example citations using the style \texttt{authoryear}.
%doc% You can change the citation style in \texttt{main.tex} (\texttt{mybiblatexstyle}).
%doc%
%doc% \begin{itemize}
%doc% \item cite \cite{Eijkhout2008} and cite \cite{Bringhurst1993, Eijkhout2008}.
%doc% \item citet \citet{Eijkhout2008} and citet \citet{Bringhurst1993, Eijkhout2008}.
%doc% \item autocite \autocite{Eijkhout2008} and autocite \autocite{Bringhurst1993, Eijkhout2008}.
%doc% \item autocites \autocites{Eijkhout2008} and autocites \autocites{Bringhurst1993, Eijkhout2008}.
%doc% \item citeauthor \citeauthor{Eijkhout2008} and citeauthor \citeauthor{Bringhurst1993, Eijkhout2008}.
%doc% \item citetitle \citetitle{Eijkhout2008} and citetitle \citetitle{Bringhurst1993, Eijkhout2008}.
%doc% \item citeyear \citeyear{Eijkhout2008} and citeyear \citeyear{Bringhurst1993, Eijkhout2008}.
%doc% \item textcite \textcite{Eijkhout2008} and textcite \textcite{Bringhurst1993, Eijkhout2008}.
%doc% \item smartcite \smartcite{Eijkhout2008} and smartcite \smartcite{Bringhurst1993, Eijkhout2008}.
%doc% \item footcite \footcite{Eijkhout2008} and footcite \footcite{Bringhurst1993, Eijkhout2008}.
%doc% \item footcite with page \footcite[p.42]{Eijkhout2008} and footcite with page \footcite[compare][p.\,42]{Eijkhout2008}.
%doc% \item fullcite \fullcite{Eijkhout2008} and fullcite \fullcite{Bringhurst1993, Eijkhout2008}.
%doc% \end{itemize}
%doc%
%doc% Please note that the citation style as well as the bibliography style
%doc% can be changed very easily. Refer to the settings in
%doc% \texttt{main.tex} as well as the very good documentation of \texttt{biblatex}.
%doc%

%doc% \subsubsection{Using this template with \myacro{APA} style}
%doc%
%doc% First, you have to have the \myacro{APA} biblatex style
%doc% installed. Modern \LaTeX{} distributions do come with
%doc% \texttt{biblatex} and \myacro{APA} style. If so, you will find the
%doc% files \texttt{biblatex-apa.pdf} (style documentation) and
%doc% \texttt{biblatex-apa-test.pdf} (file with citation examples) on your
%doc% hard disk.
%doc%
%doc% \begin{enumerate}
%doc% \item Change the style according to \verb#\newcommand{\mybiblatexstyle}{apa}#
%doc% \item Add \verb#\DeclareLanguageMapping{american}{american-apa}# or \\
%doc%   \verb#\DeclareLanguageMapping{german}{german-apa}# to your
%doc%   preamble\footnote{You might want to use section \enquote{MISC
%doc%       self-defined commands and settings} for this.}
%doc% \end{enumerate}
%doc%
%doc% These steps change the biblatex style to \myacro{APA} style

%doc%
%doc% \subsubsection{Using this template with \textsc{Bib}\TeX{}}
%doc%
%doc% If you do not want to use \texttt{Biber} and \texttt{biblatex}, you
%doc% have to change several things:
%doc% \begin{itemize}
%doc% \item in \verb#preamble/preamble.tex#
%doc%   \begin{itemize}
%doc%   \item remove the usepackage command of \texttt{biblatex}
%doc%   \item remove the \verb#\addbibresource{...}# command
%doc%   \end{itemize}
%doc% \item in \verb#main.tex#
%doc%   \begin{itemize}
%doc%   \item replace \verb=\printbibliography= with the usual
%doc%     \verb=\bibliographystyle{yourstyle}= and \verb=\bibliography{yourbibfile}=
%doc%   \end{itemize}
%doc% \item if you are using \myacro{GNU} \texttt{make}: modify \verb=Makefile=
%doc%   \begin{itemize}
%doc%   \item replace \verb#BIBTEX_CMD = biber# with \verb#BIBTEX_CMD = bibtex#
%doc%   \end{itemize}
%doc% \item Use the reference file \texttt{references-bibtex.bib}
%doc%   instead of \texttt{references-biblatex.bib}
%doc% \end{itemize}
%doc%
%doc%
\usepackage[backend=biber, %% using "biber" to compile references (instead of "biblatex")
style=\mybiblatexstyle, %% see biblatex documentation
%style=alphabetic, %% see biblatex documentation
dashed=\mybiblatexdashed, %% do *not* replace recurring reference authors with a dash
backref=\mybiblatexbackref, %% create backlings from references to citations
uniquelist=false,
natbib=true, %% offering natbib-compatible commands
hyperref=true, %% using hyperref-package references
sorting=ynt, sortcites %% sort multiple cites by date
]{biblatex}  %% remove, if using BibTeX instead of biblatex

\addbibresource{\mybiblatexfile} %% remove, if using BibTeX instead of biblatex
\addbibresource{\mybiblatexfileA} %% remove, if using BibTeX instead of biblatex
\addbibresource{\mybiblatexfileU} %% remove, if using BibTeX instead of biblatex
\addbibresource{\mybiblatexfileV} %% remove, if using BibTeX instead of biblatex



%doc%
%doc% \subsection{Miscellaneous packages} \label{subsec:miscpackages}
%doc%
%doc% There are several packages included by default. You might want to activate or
%doc% deactivate them according to your requirements:
%doc%
%doc% \begin{enumerate}

%doc% \item[\texttt{\href{http://www.ctan.org/pkg/graphicx}{%%
%doc% graphicx%%
%doc% }}]
%doc% The widely used package to use graphical images within a \LaTeX{} document.
\ifthenelse{\boolean{\mydraft}}{   %% the \mydraft switches between
                                   %% showing rectangles instead of graphics
  \usepackage[pdftex,draft]{graphicx}
}
{
  \usepackage[pdftex]{graphicx}
}

%doc% \item[\texttt{\href{https://secure.wikimedia.org/wikibooks/en/wiki/LaTeX/Formatting\#Other\_symbols}{%%
%doc% pifont%%
%doc% }}]
%doc% For additional special characters available by \verb#\ding{}#
\usepackage{pifont}


%doc% \item[\texttt{\href{http://ctan.org/pkg/ifthen}{%%
%doc% ifthen%%
%doc% }}]
%doc% For using if/then/else statements for example in macros
\usepackage{ifthen}

%% pre-define ifthen-boolean variables:
\newboolean{myaddlistoftodos}
\newboolean{english_affidavit}


%doc% \item[\texttt{\href{http://www.ctan.org/tex-archive/fonts/eurosym}{%%
%doc% eurosym%%
%doc% }}]
%doc% Using the character for Euro with \verb#\officialeuro{}#
\usepackage[left]{eurosym}

%doc% \item[\texttt{\href{http://www.ctan.org/tex-archive/help/Catalogue/entries/xspace.html}{%%
%doc% xspace%%
%doc% }}]
%doc% This package is required for intelligent spacing after commands
\usepackage{xspace}

%doc% \item[\texttt{\href{https://secure.wikimedia.org/wikibooks/en/wiki/LaTeX/Colors}{%%
%doc% xcolor%%
%doc% }}]
%doc% This package defines basic colors. If you want to get rid of colored links and headings
%doc% please change corresponding value in \texttt{main.tex} to \{0,0,0\}.
\usepackage[usenames,dvipsnames]{xcolor}
\definecolor{DispositionColor}{RGB}{\mydispositioncolor} %% used for links and so forth in screen-version

%doc% \item[\texttt{\href{http://www.ctan.org/pkg/ulem}{%%
%doc% ulem%%
%doc% }}]
%doc% This package offers strikethrough command \verb+\sout{foobar}+.
\usepackage[normalem]{ulem}

%doc% \item[\texttt{\href{http://www.ctan.org/pkg/framed}{%%
%doc% framed%%
%doc% }}]
%doc% Create framed, shaded, or differently highlighted regions that can
%doc% break across pages.  The environments defined are
%doc% \begin{itemize}
%doc%   \item framed: ordinary frame box (\verb+\fbox+) with edge at margin
%doc%   \item shaded: shaded background (\verb+\colorbox+) bleeding into margin
%doc%   \item snugshade: similar
%doc%   \item leftbar: thick vertical line in left margin
%doc% \end{itemize}
\usepackage{framed}

%doc% \item[\texttt{\href{http://www.ctan.org/pkg/eso-pic}{%%
%doc% eso-pic%%
%doc% }}]
%doc% For example on title pages you might want to have a logo on the upper right corner of
%doc% the first page (only). The package \texttt{eso-pic} is able to place things on absolute
%doc% and relative positions on the whole page.
\usepackage{eso-pic}

%doc% \item[\texttt{\href{http://ctan.org/pkg/enumitem}{%%
%doc% enumitem%%
%doc% }}]
%doc% This package replaces the built-in definitions for enumerate, itemize and description.
%doc% With \texttt{enumitem} the user has more control over the layout of those environments.
\usepackage{enumitem}

%doc% \item[\texttt{\href{http://www.ctan.org/tex-archive/macros/latex/contrib/todonotes/}{%%
%doc% todonotes%%
%doc% }}]
%doc% This packages is \emph{very} handy to add notes\footnote{\texttt{todonotes} replaced
%doc% the \texttt{fixxme}-command which previously was defined in the
%doc% \texttt{preamble\_mycommands.tex} file.}. Using for example \verb#\todo{check}#
%doc% results in something like this \todo{check} in the document. Do read the
%doc% great package documentation for usage of other very helpful commands such as
%doc% \verb#\missingfigure{}# and \verb#\listoftodos#. The latter one creates an index of all
%doc% open todos which is very useful for getting an overview of open issues.
%doc% The package \texttt{todonotes} require the packages \texttt{ifthen}, \texttt{xkeyval}, \texttt{xcolor},
%doc% \texttt{tikz}, \texttt{calc}, and \texttt{graphicx}. Activate
%doc% and configure \verb#\listoftodos# in \texttt{main.tex}.
%\usepackage{todonotes}
%\usepackage[\mytodonotesoptions]{todonotes}  %% option "disable" removes all todonotes output from resulting document

%disabled% \item[\texttt{\href{http://www.ctan.org/tex-archive/macros/latex/contrib/blindtext}{%%
%disabled% blindtext%%
%disabled% }}]
%disabled% This package is used to generate blind text for demonstration purposes.
%disabled% %% This is undocumented due to problems using american english; author informed
\usepackage{blindtext}  %% provides commands for blind text:
%disabled% %% \blindtext creates some text,
%disabled% %% \Blindtext creates more text.
%disabled% %% \blinddocument creates a small document with sections, lists...
%disabled% %% \Blinddocument creates a large document with sections, lists...
%% 2012-03-10: vk: author published a corrected version which is able to handle "american english" as well. Did not have time to check new package version for this template here.

% Better Tabular Tables \tabularx{}
\usepackage{tabularx}
% inside table to make linebreaks with makecell and also multirows and collumns
\usepackage{multirow, multicol, makecell}

% Allows us to supress floating enviroments 
% https://tex.stackexchange.com/a/9486/167781
\usepackage{float}

% landscape mode
\usepackage{pdflscape}

% special math equations and SI units
\usepackage{amsmath}
\usepackage{siunitx}